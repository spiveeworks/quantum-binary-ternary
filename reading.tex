\documentclass[]{article}

\usepackage{amsmath}
\usepackage{amssymb}

\renewcommand\thesection{}
\renewcommand\thesubsection{}

\newcommand{\bra}[1]{\langle #1 |}
\newcommand{\ket}[1]{| #1 \rangle}
\newcommand{\braket}[2]{\langle #1 | #2 \rangle}

\newcommand{\ord}[1]{\left| #1 \right|}
\newcommand{\norm}[1]{\left\Vert #1 \right\Vert}

\begin{document}

\section{April 10}

Reading J Tolar, "On Clifford groups in quantum computing"

An $N$ state system corresponds to a hilbert space $\mathbb{C}^N$.

"Generalized Pauli Matrices" generate a group, "Weyl-Heisenberg group", semantics.

The "normalizer" of this is called the Clifford group. So I guess WH is not normal in U(N), but in Clifford group it is. That's surprising to me, but I guess it makes sense given the normal property becomes weaker the less group elements you are conjugating against. So Clifford group is the set $\{g\ |\ g^{-1}Xg \in \textit{WH}, \forall x \in \textit{WH}\}$.

"Clifford quotient group" sounds like Clifford group without scalar multiplication, which sounds good to me. $U(N)$ seems so redundant/free I will take every quotient I can get.

"Symmetries of Pauli gradings" of an algebra apparently describe some detail of clifford quotient groups, and this paper will describe something more detailed than that? No idea what a Pauli grading is.

$Q_N\ket{j} = \omega_N^j\ket{j}$, $P_N\ket{j} = \ket{j + 1}$, so in 2d:

\begin{align*}
Q_2 &= \left[\begin{matrix}
1 & 0 \\
0 & -1
\end{matrix}\right] = Z
\\P_2 &= \left[\begin{matrix}
0 & 1 \\
1 & 0
\end{matrix}\right] = X
\end{align*}

These elements along with $\omega_N$ are order $N$, and are nearly commutative.
\[\Pi_N = \{\omega_N^i P_N^j Q_N^k\}\]

This is not $H(N)$ apparently? Do we need a generalized version of $Y$ before this becomes the Weyl Heisenberg group? Or am I missing something.

$\omega_N$ and $Q_N$ clearly have computational basis as their eigenvectors, being diagonal, and $P_N$ will have $\ket{v_i} = \sum_j \omega_N^{ij}\ket{j}$ as eigenvectors, eigenvalues $\omega_N^i$.... transforming into this basis is the discrete fourier transform! Aha! Ok back to the text. I don't know what a configuration space is or what "eigenvector of position means".

Ah yes $\tau_N = \omega_N^{\frac{1}{2}}$ lets us define $Y$.

\[\tau_2P_2Q_2 = i
\left[\begin{matrix}
0 & 1 \\
1 & 0
\end{matrix}\right]
\left[\begin{matrix}
1 & 0 \\
0 & -1
\end{matrix}\right]
=
\left[\begin{matrix}
	0 & -i \\
	i & 0
\end{matrix}\right]
= Y\]

Then $H(N) = \{\tau_N^h\omega_N^i Q_N^j P_N^k\}$, good. $\ord{H(N)}=2\ord{\Pi_N}=2N^3$.

Oh this phase factor is just for even N. Fascinating. Naively that sounds like a novel thing to attack in a mixed level system?

Apparently $\tau_2 = -i$, so the equation is actually $Y = \tau_2 Q_2 P_2$

Then the centralizer is just the set of scalars $\{\tau^i\}$, and since $Q_N$ and $P_N$ commute, up to phase shift $\omega_N$, quotienting by the centralizer gives the abelian group $\mathbb{Z}_N^2$. Easy.

\end{document}
