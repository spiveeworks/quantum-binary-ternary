\documentclass[]{article}

\usepackage{amsmath}

\renewcommand\thesection{}
\renewcommand\thesubsection{}

% Title Page
\title{Honours Diary 2020}
\author{Jarvis Carroll}


\begin{document}
\maketitle

\section{March 13}

Set up TeXstudio and basic document structure.

\subsection{Exercise 2.1}
Linear Dependence, show that $(1,-1)$, $(1,2)$ and $(2,1)$ are linearly dependent.

\begin{align*}
	&(1, -1) + (1, 2) - (2, 1) \\
=\ &(1+1-2, -1+2-1) \\
=\ &(0, 0)
\end{align*}

\subsection{Exercise 2.2}
Matrix representations: Suppose $V$ is a vector space with basis vectors $|0\rangle$ and $|1\rangle$, and $A$ is a linear operator from $V$ to $V$ such that $A|0\rangle=|1\rangle$ and $A|1\rangle=|0\rangle$. Give a matrix representation for $A$, with respect to the input basis $|0\rangle$, $|1\rangle$, and the output basis $|0\rangle$, $|1\rangle$. Find input and output bases which give rise to a different matrix representation of $A$.

Equation 2.12 gives us the defining property of matrix representations:
\[A|v_j\rangle=\sum_i A_{ij}|w_i\rangle\]
This gives us a pair of vector equations:
\[
|1\rangle = A|0\rangle = A_{00}|0\rangle + A_{10}|1\rangle
\]
\[
|0\rangle = A|1\rangle = A_{01}|0\rangle + A_{11}|1\rangle
\]
By linear independence of $|0\rangle$, $|1\rangle$, it follows that
\begin{align*}
A_{00} &= 0 & A_{01} &= 1 \\
A_{10} &= 1 & A_{11} &= 0
\end{align*}
i.e. A has the matrix representation:
\[
A = \left[\begin{matrix}
0 & 1 \\
1 & 0
\end{matrix}\right]
\]

\end{document}          
