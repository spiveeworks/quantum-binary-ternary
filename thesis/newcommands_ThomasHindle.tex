% newcommands.tex (new command definitions)

% Here you would include any additional packages that you want to use.
% You should make sure they don't clash with the above packages that
% are in use in the style file.
% If you want to call in some style files or new packages, put them here
%\usepackage{undertilde}
%\usepackage[left=2cm,right=2cm,top=2cm,bottom=2cm]{geometry}
\usepackage[a4paper]{geometry}
%\usepackage[latin1]{inputenc}
\usepackage{amsmath, latexsym, color, graphicx, amssymb, here}
\usepackage{amsfonts}
\usepackage{epsf, epsfig, pifont,tikz}
\usepackage{graphics, calrsfs}
%\usepackage{tangocolors}
\usepackage{times}
\usepackage{fancybox,calc}
\usepackage{hyperref}
\usepackage{pgfplots}
\usepackage{verbatim}
\usepackage{esint} 
\usepackage{amsthm}


%% environment name, counter, text
\newtheorem{theorem}{Theorem}[section]
\newtheorem{lemma}[theorem]{Lemma}
\newtheorem{definition}[theorem]{Definition}
%% Use like:
% \begin{definition}
% Blah blah blah
% \end{definition}



%% Brackets
\newcommand{\lb}{\left(}
\newcommand{\rb}{\right)}
%% Square Brackets
\newcommand{\lbs}{\left[}
\newcommand{\rbs}{\right]}

%Bolded letters
\newcommand{\A}{{\bf A}}
\newcommand{\B}{{\bf B}}
\newcommand{\T}{{\bf T}}
\newcommand{\C}{{\bf C}}
\newcommand{\N}{{\bf N}}
\newcommand{\n}{{\bf n}}
\newcommand{\bfx}{{\ensuremath{\mathbf{x}}}}
%Bolded letters, which previously had a command
\renewcommand{\v}{{\bf v}}
\renewcommand{\r}{{\bf r}}
\renewcommand{\a}{{\bf a}}

%Set notation. Needs to be in a maths environment (equation or $$)
\newcommand{\R}{{\mathbb R}}
\newcommand{\Z}{{\mathbb Z}}

%Unit vectors. Needs to be in a maths environment (equation or $$)
\newcommand{\uniti}{{\hat{\mbox{\boldmath $\imath$}}}}
\newcommand{\unitj}{{\hat{\mbox{\boldmath $\jmath$}}}}
\newcommand{\unitk}{{\hat{\mbox{\boldmath $\mathit{k}$}}}}
\newcommand{\unitn}{{\hat{\mbox{\boldmath $\mathit{n}$}}}}
\newcommand{\unite}{{\hat{\mbox{\boldmath $\mathit{e}$}}}}
\newcommand{\unitu}{{\hat{\mbox{\boldmath $\mathit{u}$}}}}


%Shorthand for commonly used expressions with cursive font
\newcommand{\ie}{{\em i.e.} \/}
\newcommand{\eg}{{\em e.g.} \/}
\newcommand{\etc}{{\em etc.} \/}
\newcommand{\etal}{{\em et al. }}


%Bolded, cursive strings of characters
\newcommand{\mathbi}[1]{\textbf{\em #1}}


%Bolded dot symbol
\newcommand{\bcdot}{\mbox{\boldmath $\, \cdot \, $}}

%bolds and puts in a cursive underline for a vector
\renewcommand{\vec}[1]{{\mbox{\boldmath $\utilde{\mathit{#1}}$}}}


%Shorthand commands for writing xy,xz,yz planes
\newcommand{\xyplane}{$x$-$y$ plane\/}
\newcommand{\xzplane}{$x$-$z$ plane\/}
\newcommand{\yzplane}{$y$-$z$ plane\/}
\newcommand{\abplane}[2]{${#1}$-${#2}$ plane\/}

%Full derivative of A wrt B. Needs to be in a maths environment (equation or $$)
\newcommand{\FullDif}[2]{\dfrac{d {#1}}{d {#2}}}

%Partial derivative of A wrt B. Needs to be in a maths environment (equation or $$)
\newcommand{\ParDif}[2]{\dfrac{\partial {#1}}{\partial {#2}}}
%Second partial derivative of A wrt B. Needs to be in a maths environment (equation or $$)
\newcommand{\DblParDif}[2]{\frac{\partial^2 #1}{\partial #2 ^2}}

%Material derivative of A wrt B. Needs to be in a maths environment (equation or $$)
\newcommand{\MatDif}[2]{\dfrac{D {#1}}{D {#2}}}


%Writes the cartesian gradient of a function. Leave as {} to write the operator. Needs to be in a maths environment (equation or $$)
\newcommand{\grad}[1]{\ParDif{#1}{x}  \uniti+ \ParDif{#1}{y} \, \unitj + \ParDif{#1}{z} \, \unitk}


%Writes the cartesian Laplacian of a function. Leave as {} to write the operator. Needs to be in a maths environment (equation or $$)
\newcommand{\laplacian}[1]{\DblParDif{{#1}}{x} + \DblParDif{{#1}}{y}+ \DblParDif{{#1}}{z}}

%Writes the cylindrical gradient of a function. Leave as {} to write the operator. Needs to be in a maths environment (equation or $$)
\newcommand{\gradCylindrical}[1]{\ParDif{{#1}}{\rho} \, \unite_\rho \ + \ \dfrac{1}{\rho} \ParDif{{#1}}{\phi} \, \unite_\phi \ + \ \ParDif{{#1}}{z} \, \unite_z}



%Writes the spherical gradient of a function. Leave as {} to write the operator. Needs to be in a maths environment (equation or $$)
\newcommand{\gradSpherical}[1]{\ParDif{{#1}}{r} \, \unite_r \ + \ \dfrac{1}{r} \ParDif{{#1}}{\theta} \, \unite_\theta \ + \ \dfrac{1}{r \sin(\theta)}\ParDif{{#1}}{\phi} \, \unite_\phi}

%Writes the spherical divergence of a function. Leave as {}{}{} to write the operator, otherwise #1 is r component, #2 is theta component and #3 is phi component. Needs to be in a maths environment (equation or $$)
\newcommand{\divSpherical}[3]{\dfrac{1}{R^2}\, \ParDif{}{R}\lb(R^2\, {#1}\rb) \ + \  \dfrac{1}{R \sin(\theta)} \ParDif{}{\theta} \lb(\sin(\theta)\, {#2}\rb) \ + \ \dfrac{1}{R \sin(\theta)}\ParDif{{#3}}{\phi} \  }

%Creates a cartesian position vector, argument is the vector field in question. Needs to be in a maths environment (equation or $$)
\newcommand{\posvect}[1]{{#1}_x \, \uniti + {#1}_y \, \unitj + {#1}_z \, \unitk}

%Creates the cartesian generic position vector r. Needs to be in a maths environment (equation or $$)
\newcommand{\posvectr}{x \, \uniti + y \, \unitj + z \, \unitk}
%Creates the cylindrical generic position vector r. Needs to be in a maths environment (equation or $$)
\newcommand{\posvectcyl}{\rho \, \unite_\rho \, + z \, \unite_z}
%Creates the spherical generic position vector r. Needs to be in a maths environment (equation or $$)
\newcommand{\posvectsph}{r \, \unite_\r}


%Shorthand for a parametrised position vector. Needs to be in a maths environment (equation or $$)
\newcommand{\parametricposvectr}{\vec{r}(t) \ = \ x(t) \, \uniti + y(t) \, \unitj + z(t) \, \unitk}

%Creates the divergence of a specified vector. Needs to be in a maths environment (equation or $$)
\renewcommand{\div}[1]{\nabla \bcdot \vec{#1}}

%Creates the curl of a specified vector. Needs to be in a maths environment (equation or $$)
\newcommand{\curl}[1]{\nabla \times \vec{#1}}

%Creates the magnitude of a specified vector. Needs to be in a maths environment (equation or $$)
\newcommand{\magnitude}[1]{ \| \vec{#1} \| }



%Creates a differential distance in Cartesian. Needs to be in a maths environment (equation or $$)
\newcommand{\drCart}{d x \, \uniti + d y \, \unitj + d z \, \unitk}
%Creates a differential distance in Cylindrical. Needs to be in a maths environment (equation or $$)
\newcommand{\drCyl}{d \rho \, \unite_\rho + \rho \, d \phi \, \unite_\phi + d z \, \unite_z}
%Creates a differential distance in Spherical. Needs to be in a maths environment (equation or $$)
\newcommand{\drSph}{d r \, \unite_r + r \, d \theta \, \unite_\theta + r \, \sin(\theta) \, d \phi \, \unite_\phi }


%Creates arbitrary vectors with 3 arguments. Needs to be in a maths environment (equation or $$)
\newcommand{\coordvect}[3]{{#1} \, \uniti \, + \, {#2} \, \unitj \, + {#3} \, \unitk}
\newcommand{\cylcoordvect}[3]{{#1} \, \unite_\rho \, + \, {#2} \, \unite_\phi \, + {#3} \, \unite_z}
\newcommand{\sphcoordvect}[3]{{#1} \, \unite_\r \, + \, {#2} \, \unite_\theta \, + {#3} \, \unite_\phi}


%Creates a box around a given equation. Must be in a maths environment.
\newsavebox{\fmbox}
\newenvironment{eqnframe}[1]     
{
	\begin{center} 
	\begin{lrbox}{\fmbox}
	\begin{minipage}{#1}
}     
{
	\end{minipage}
	\end{lrbox}\fbox{\usebox{\fmbox}}
	\end{center}
}


%These make differently aligned boxes, with width {#1} and height {#2}
\newcommand\Tpad[2]{\rule[4.5ex]{{#1}pt}{{#2}pt}}
\newcommand\Bpad[2]{\rule[-3.75ex]{{#1}pt}{{#2}pt}}

%These make slightly different shaped zeros.
\renewcommand{\labelenumi}{\textbf{\arabic{enumi}}.}
\renewcommand{\labelenumii}{\textbf{(\roman{enumii})}}
\renewcommand{\labelenumiii}{\textbf{(\alph{enumiii})}}

%Lets you use \cross instead of times. Needs to be in a maths environment (equation or $$)
\newcommand{\cross}{\times}


% define some colors
\definecolor{cBlue}{rgb}{.255,.41,.884} % RoyalBlue of svgnames
\definecolor{cRed}{rgb}{1, 0, 0} % Red of svgnames


%Shorthand for various useful things
%Argument 1, with argument 2 as a subscript
\newcommand{\An}[2]{{#1}_{{#2}}}
%Define the pair of co-ordinates
\newcommand{\AnBm}[4]{(\An{{#1}}{{#2}},\An{{#3}}{{#4}})}

%integral from x0 to x1
\newcommand{\intxoxi}{\int_{\An{x}{0}}^{\An{x}{1}}}

%f(x,y)
\newcommand{\fxy}{f(x,y)}
%g(x,y)
\newcommand{\gxy}{g(x,y)}

%%f(x,y,z)
\newcommand{\fxyz}{f(x,y,z)}
%%g(x,y,z)
\newcommand{\gxyz}{g(x,y,z)}

%% Variant epsilon
\newcommand{\ep}{\varepsilon}

%%Yprime squared
\newcommand{\yp}{(y^\prime)^{2}}

%%Coordinates ~extra style~
\newcommand{\cord}{co$\ddot{\text{o}}$rdinates\ }