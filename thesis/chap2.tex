% chap2.tex (Chapter 2 of the thesis)

\chapter[BACKGROUND AND LITERATURE REVIEW]{Background and Literature Review}
[background already implies a revision/summary of the literature, should I just call this chapter one or the other]

In classical computation data is represented in the voltage levels of conductive materials, and logic is implemented using transistors as digital switches, whose behaviour is more responsive and more consistent to the difference between 0 voltage and high voltage, than to the difference between different levels of high voltage. [multiplication thing is more how amplifiers work than switches, so I've ended up being more vague. Not sure what is necessary here.] This means that hardware for classical computation deals almost exclusively with Boolean algebra and binary arithmetic.

In contrast to this quantum computation inherits its number system directly from the dimension of the Hilbert space of the particles being used for computation. For example the spin of an electron forms a 2-dimensional space, the net spin of a Nitrogen-14 atom forms a 3-dimensional space, and the energy level of a trapped ion can form a multi-dimensional space depending on the number of energy levels allowed. This means that quantum computation has a much easier time implementing logical and numerical systems other than the binary system used in classical contexts, but despite this most of the quantum computation in theoretical and practical contexts is based on binary logic and arithmetic, since this is the simplest, and inherits a lot of theoretical results directly from classical computation.

In practice the problem of implementing higher logic systems than binary is much more tractable in the quantum case, whereas the problem of creating and maintaining a large number of entangled quantum objects over time, a problem that doesn't exist in the classical case, becomes the primary constraint limiting practical quantum computation. By using logic systems that store more information in a smaller number of particles, we might actually have an easier time developing a quantum computer capable of performing any given scale of computation. This means that the step from binary to higher logic systems has a lot of motivation in the quantum case.

While there is a lot of potential in the topic of non-binary quantum computation, and a lot of novelty already described, even this has the implicit assumption that all quantum objects need to have the same dimension. [bring in quantum number when talking about physical things] In theory this isn't necessary either, and it might be possible to have a quantum computer with a mixture of objects of different dimension. Writing algorithms for such a device could allow one to economise on the strengths and weaknesses of each individual number system, potentially requiring even less physical complexity for the same level of computational power. Further, although we won't discuss fault-tolerant computation anywhere else in this thesis, it is worth noting that in order to mitigate noise in real world quantum computers, information-theoretic encoding schemes are used that embed a certain number of `logical' qubits in a larger number of `physical' qubits, so we may one day find quantum computation acting on logically mixed systems, regardless of the quantum number of the physical objects involved.

To these ends we shall review some of the literature that makes up the current understanding of quantum computation in ternary and higher forms of logic, as well as what little is currently known about mixed systems with a combination of qubits, qutrits, and higher objects. Of note also are the textbook ``Quantum Computation and Quantum Information''\cite{textbook} and the general audience paper ``Quantum Algorithm Implementations for Beginners''\cite{algos} which provide extensive discussion of the foundational aspects of qubit computation. Much of non-binary and mixed logic research is a direct generalization of the techniques collected in these two texts. [maybe stop framing it as a multi-valued logic review now, it's just the background content of our own results... a lot of this is good material for an abstract though]

\subsection{On Clifford Groups}
The first paper studying higher forms of computation that we shall look at is the paper ``On Clifford groups in Quantum Computing''\cite{tolar-clifford}, which aims to understand an important group of unitary matrices -- the Clifford group. This group is generally defined as being the normalizer of the Weyl-Heisenberg group. (or sometimes the Pauli group, especially in cases where this is equivalent such as $n=3$)

The first thing that Tolar does is develop a way of proving a known isomorphism between Clifford groups on single objects, and finite groups acting on ring modules. Specifically he analyses conjugation of the Pauli group as a group action of the Clifford group. Since this group action is unaffected by scalar factors, it turns out to be a very algebraically powerful, in particular the Pauli group is Abelian up to scale factors, forming a convenient Abelian subgroup of these Ad-actions.

In any case the technique turns out to be amenable to analysing systems of multiple objects as well, allowing a general understanding of Clifford groups in composite quantum systems in terms of the same kinds of groups that single object Clifford groups are isomorphic to.

In particular, for single object Clifford groups the isomorphism was to the group $(\mathbb{Z}_n\times\mathbb{Z}_n)\rtimes\text{SL}(2,\mathbb{Z}_n)$, which is a finite group generated by less than $4$ group elements; for systems with $k$ objects of the same dimension $n$ the Clifford group was further isomorphic to $\mathbb{Z}_n^{2k}\rtimes\text{SL}(2k, \mathbb{Z}_n)$. For pairs of objects with co-prime dimension the Clifford group was essentially the Clifford group of single objects with the same dimension, but for systems with an object of dimension $ap^k$ and an object of dimension $bp^{k+r}$ for some $a, b, p, k, r$ positive and $p$ prime, no decomposition of this form was found, an interesting place left to explore by this analysis.

This paper thus presents a technique for analyzing Clifford groups of arbitrary and mixed dimension, a presentation of how many Clifford groups are isomorphic to subsets of different ring modules, and finally a novel unexplored possibility within mixed-level objects of particular dimensions.

This is the only paper we found that looks at quantum systems with objects of differing dimension.

\subsection{Universal Computation}
[I should be talking about the difference between gates and circuits MUCH earlier than this, directly in the quantum mechanics section maybe.]

A foundational result in quantum computation is that of universal computation, that certain combinations of quantum gate can be used to implement any quantum algorithm to some accuracy, given sufficient circuit depth. The resulting circuits are generally too long to use in practice, compared to compilation techniques that rely on specific properties of the algorithm in question, but the result is still useful since it proves that it's not impossible, i.e.\ its necessary and sufficient conditions provide a starting point for designing and using quantum computers in practice.

Universal computation for qubit computers is described in full detail in \cite{textbook}, where it is done in 3 steps: Arbitrary unitary matrix decomposed into gates with multiple control bits and a single target bit, then into operations acting on a single object, plus the Toffoli gate, and finally these single object operations into a finite set of gates using the asymptotic techniques given in \cite{universal-qubit}.

When dealing with objects other than qubits the last step of this process becomes much more difficult, simply because the geometry of $U(n)/U(1)$ is more complicated than that of $U(2)/U(1)$. [Have we talked about global phase and $G/U(1)$ yet?] In order to overcome this the paper \cite{multi-valued-logic} uses a different approach, by proposing a device that extends the most common technique for implementing qubits, the linear ion trap, to arbitrary quantum number, along with a description of how such a device could implement a family of objects which is not discrete, but which in the first half of the paper they had already shown to be universal.

[discontinuity -- remainder of this section is old] The first half of the paper is focused on an intuitive scheme for decomposing an arbitrary unitary matrix based on its spectral decomposition, where objects analogous to the generalized $X$ and $Z$ Pauli matrices are used to implement gradually larger classes of unitary operator, until eventually all unitary operators have been implemented using only these $X$ and $Z$ like objects acting on one or two digits at a time.

The $X$ and $Z$ like gates are not really individual gates, but families of gates, so this result is not general to any quantum computer; typically a theoretical proof of arbitrary (or universal) quantum computation relies on a very small set of individual gates, such as $\{H, \text{Toffoli}\}$, whereas this paper relies on families of gates with $2d$ real-valued parameters, and a similar number of degrees of freedom in how each of these families might be chosen, making for a computational basis that is easy to work with, but vague to implement.

The paper gets away with this by proposing a physical implementation of this family of gates, through direct control of physical parameters of the device, relating the implementation of these basic gates to known problems of quantum device control in linear trapped ion systems, where the $n$-dimensional computational basis corresponds to the various excitation levels of the ion.

This paper is a very promising theoretical and practical foundation for working with quantum systems with multiple objects all of one arbitrary dimension.

\subsection{Qudit versions of the qubit ``pi-over-eight'' gate}
\begin{itemize}
	\item A standard basis for asymptotically universal quantum computation in binary contexts is the Hadamard gate, and the ``pi-over-eight'' phase gate, which is a unitary operation that is diagonal in the computational basis
	\item this paper generalizes the latter gate to a family of gates with similar properties, acting on any quantum object with a prime number of basis states.
	\item first the basic properties of the phase gate are described, and solved algebraically, giving a discrete family of diagonal unitary matrices in the same level of the Clifford hierarchy as the phase gate in 2 dimensions.
	\item this family of gates is then understood as a finite Abelian group, and is therefore reduced to the direct product of cyclic groups
	\item the different number of states gave different group structure, depending on whether there were 2, 3, or more states. These 3 cases were each given a small generator of diagonal phase-like gates.
	\item these gates turn out to share important geometric properties with the phase gate, in addition to being diagonal, they are maximally distinct from any Clifford gate, which has implications for how accurate they will be in the presence of noise.
	\item \ [something something magic state distillation]
\end{itemize}

\subsection{Improved Ternary Arithmetics}
\begin{itemize}
	\item Opens with a description of two implementations of ternary addition
	\item First is a "modified ripple-carry adder" which uses only a single ancillary qutrit to compute a sum in-place
	\item Next is a "carry look-ahead adder" which implements a divide-and-conquer algorithm of addition in order to implement addition with many more ancillary qubits, but logarithmic circuit depth through parallelism
	\item both of these algorithms are fairly unsurprising, but their implementation takes advantage of the specifics of qutrits in order to minimize the memory and computational overhead as described
	\item addition is foundational to implementing modular arithmetic, which the paper discusses in the context of Shor's period finding algorithm
	\item modular addition, subtraction, and integer comparison are discussed, including the positive or negative effects they have on the computational cost of the circuit
	\item the carry look-ahead adder also demonstrates a non-numerical variable which takes 3 values, well leveraged in this algorithm, and very interesting in the discussion of different number schemes and their relative trade-offs.
	\item the technology that was used to optimize each algorithm is given, where gates that permute the computational basis are understood as polynomials acting on the digits of this basis, and algebraic manipulation is used to understand which gates are equivalent to which
	\item further this technology was used to construct important gates exactly (as opposed to asymptotically) using a diagonal unitary gate described in ``Qudit versions of the qubit ``pi-over-eight'' gate.''
	\item the technology of this paper is very well attuned to the goal of understanding quantum computation in different number systems, both from algorithmic perspective of efficiently implementing useful computations, and from the more foundational perspective of choosing a basis/generator in which to do these computations.
	\item further the clear communication of which kinds of non-clifford gate are used allows one to apply the algorithms to any other basis, finding a more suitable compilation of these operations. This was done well in that the gates required were all well justified, and their equivalence to each-other was made clear.
\end{itemize}

