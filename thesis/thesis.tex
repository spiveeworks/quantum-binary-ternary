% thesis.tex (starting point of a UTas mathematics thesis)

% Note that the following defaults are also contained in the 'report' class
% (this is not exhaustive...see appropriate references for all options)
% 'oneside' mode...overide with 'twoside' to force output for two-sided printing.
% 'final' mode...overide with 'draft' to see linebreak malfunctioning.
% For example, to use two-sided printing one would declare the 'documentclass'
% as follows...
% '\documentclass[11pt,a4paper,twoside]{report}'
%
% Specific Mathematical options are as follows
% 'leqno' to force equation numbering on the left side of the page.
% 'fleqn' to force formulas to flush left (centered is the default).
% If 'fleqn' is specified then the left indent is controlled with '\mathindent'.
% i.e. '\setlength{\mathindent}{2.5cm}'

% Declare overall type of document (use 11pt report class on A4 paper).
\documentclass[12pt,a4paper]{report}

\usepackage[hidelinks]{hyperref}
\usepackage{listings}

\usepackage{amsmath}
\usepackage{amssymb}
\usepackage{amsthm}

\newtheorem{theorem}{Theorem}[chapter]
\newtheorem{lemma}[theorem]{Lemma}
\newtheorem{prop}[theorem]{Proposition}
\theoremstyle{definition}
\newtheorem{define}{Definition}[chapter]

\renewcommand\qedsymbol{$\square$}

\def\chapterautorefname{Chapter}
\def\sectionautorefname{Section}
\def\subsectionautorefname{Section}
\def\defineautorefname{Definition}
\def\theoremautorefname{Theorem}
\def\propautorefname{Proposition}

\usepackage{tikz}
\usetikzlibrary{quantikz}

%\renewcommand\thesection{}
%\renewcommand\thesubsection{}

% went the whole year without tikz, had to uncomment this when I started using it, worked without any further modification!
%\newcommand{\bra}[1]{\langle #1 |}
%\newcommand{\ket}[1]{| #1 \rangle}
%\newcommand{\braket}[2]{\langle #1 | #2 \rangle}

\newcommand{\ord}[1]{\left| #1 \right|}
\newcommand{\norm}[1]{\left\Vert #1 \right\Vert}

\DeclareMathOperator{\vecspan}{span}

% Include the style file which contains all the required formatting
% information that is set out in the Research Higher Degrees Resource
% Handbook (2003 version). NOTE: This file uses the following packages
% 'graphicx' for graphics manipulation
% 'fancyhdr' for nice headers and footers.
% 'makeidx' for generating the index
% 'tocbibind' for adding table of contents entries for bibliography, index etc.
% 'sectsty' for generating stylised chapter and section headings.
% You will need to make sure your LaTeX installation has these packages
% installed...else it wont work :(
\usepackage{MathPhysHonoursThesis}

%% newcommands.tex (new command definitions)

% Here you would include any additional packages that you want to use.
% You should make sure they don't clash with the above packages that
% are in use in the style file.
% If you want to call in some style files or new packages, put them here
\usepackage{undertilde}
%\usepackage[left=2cm,right=2cm,top=2cm,bottom=2cm]{geometry}
\usepackage[a4paper]{geometry}
\usepackage[latin1]{inputenc}
\usepackage{amsmath, latexsym, color, graphicx, amssymb, here}
\usepackage{amsfonts}
\usepackage{epsf, epsfig, pifont,tikz}
\usepackage{graphics, calrsfs}
%\usepackage{tangocolors}
\usepackage{times}
\usepackage{fancybox,calc}
\usepackage{hyperref}
\usepackage{pgfplots}
\usepackage{verbatim}

% Some examples (yours may be different):
\newtheorem{theorem}{Theorem}[section]
\newtheorem{lemma}[theorem]{Lemma}
\newcommand{\bfx}{{\ensuremath{\mathbf{x}}}}

\newcommand{\A}{{\bf A}}
\newcommand{\B}{{\bf B}}
\newcommand{\T}{{\bf T}}
\newcommand{\C}{{\bf C}}
\newcommand{\N}{{\bf N}}
\newcommand{\R}{{\mathbb R}}
\newcommand{\Z}{{\mathbb Z}}
\newcommand{\n}{{\bf n}}
\renewcommand{\v}{{\bf v}}
\renewcommand{\r}{{\bf r}}
\renewcommand{\a}{{\bf a}}

\newcommand{\uniti}{{\hat{\mbox{\boldmath $\imath$}}}}
\newcommand{\unitj}{{\hat{\mbox{\boldmath $\jmath$}}}}
\newcommand{\unitk}{{\hat{\mbox{\boldmath $\mathit{k}$}}}}
\newcommand{\unitn}{{\hat{\mbox{\boldmath $\mathit{n}$}}}}
\newcommand{\unite}{{\hat{\mbox{\boldmath $\mathit{e}$}}}}
\newcommand{\unitu}{{\hat{\mbox{\boldmath $\mathit{u}$}}}}
\newcommand{\ie}{{\em i.e.} \/}
\newcommand{\eg}{{\em e.g.} \/}
\newcommand{\etc}{{\em etc.} \/}
\newcommand{\etal}{{\em et al. }}
\newcommand{\mathbi}[1]{\textbf{\em #1}}
\newcommand{\bcdot}{\mbox{\boldmath $\, \cdot \, $}}
\newcommand{\vect}[1]{{\mbox{\boldmath $\utilde{\mathit{#1}}$}}}
\newcommand{\xyplane}{$x$-$y$ plane \/}
\newcommand{\xzplane}{$x$-$z$ plane \/}
\newcommand{\yzplane}{$y$-$z$ plane \/}
\newcommand{\dint}{\int \! \! \int}
\newcommand{\tint}{\int \! \! \int \! \! \int}
\newcommand{\doint}{\bigcirc \! \! \! \! \! \! \! \! \int \! \! \! \! \!  \int}
\newcommand{\inlinedoint}{\circ\!\!\!\! \! \int \!\!\!\! \int}
%\newcommand{\deloperator}[3]{{\frac{\partial{#1}}{\partial x} \, \uniti \frac{\partial{#2}}{\partial y} \, \unitj + \frac{\partial{#3}}{\partial z} \, \unitk}}
\newcommand{\deloperator}{{\frac{\partial}{\partial x} \, \uniti + \frac{\partial}{\partial y} \, \unitj + \frac{\partial}{\partial z} \, \unitk}}
\newcommand{\laplaceoperator}{{\frac{\partial^2}{\partial x^2} + \frac{\partial^2}{\partial y^2} + \frac{\partial^2}{\partial z^2}}}
%\newcommand{\grad}[1]{{\frac{\partial {#1}}{\partial x} \, \uniti + \frac{\partial {#1}}{\partial y} \, \unitj + \frac{\partial {#1}}{\partial z} \, \unitk}}
\newcommand{\gradCylindrical}[1]{{\frac{\partial {#1}}{\partial \rho} \, \unite_\rho \ + \ \frac{1}{\rho} \frac{\partial {#1}}{\partial \phi} \, \unite_\phi \ + \ \frac{\partial {#1}}{\partial z} \, \unite_z} }
\newcommand{\gradSpherical}[1]{{\frac{\partial {#1}}{\partial r} \, \unite_r \ + \ \frac{1}{r} \frac{\partial {#1}}{\partial \theta} \, \unite_\theta \ + \ \frac{1}{r \sin(\theta)}\frac{\partial {#1}}{\partial \phi} \, \unite_\phi} }
\newcommand{\divSpherical}[3]{{\frac{1}{R^2}\, \frac{\partial}{\partial R}\left(R^2\, {#1}\right) \ + \  \frac{1}{R \sin(\theta)} \frac{\partial}{\partial \theta} \left(\sin(\theta)\, {#2}\right) \ + \ \frac{1}{R \sin(\theta)}\frac{\partial {#3}}{\partial \phi} \  }}
%\newcommand{\laplacian}[1]{{\frac{\partial^2 {#1}}{\partial x^2} + \frac{\partial^2 {#1}}{\partial y^2} + \frac{\partial^2 {#1}}{\partial z^2}}}
\newcommand{\posvect}[1]{{#1}_1 \, \uniti + {#1}_2 \, \unitj + {#1}_3 \, \unitk}
\newcommand{\posvectr}{x \, \uniti + y \, \unitj + z \, \unitk}
\newcommand{\posvectcyl}{\rho \, \unite_\rho \, + z \, \unite_z}
\newcommand{\posvectsph}{r \, \unite_\r}
\newcommand{\arbvect}[3]{{#1} \, \uniti \, + \, {#2} \, \unitj \, + \, {#3} \, \unitk}
\newcommand{\genvect}[5]{{#1} \, \uniti \, {#2}\,  {#3} \, \unitj \, {#4} \, {#5} \, \unitk}
\newcommand{\parametricposvectr}{\vect{r}(t) \ = \ x(t) \, \uniti + y(t) \, \unitj + z(t) \, \unitk}
\newcommand{\divergence}[1]{\nabla \bcdot \vect{#1}}
%\newcommand{\curl}[1]{\nabla \times \vect{#1}}
\newcommand{\magnitude}[1]{ \| \vect{#1} \| }
\newcommand{\drCart}{d x \, \uniti + d y \, \unitj + d z \, \unitk}
\newcommand{\drCyl}{d \rho \, \unite_\rho + \rho \, d \phi \, \unite_\phi + d z \, \unite_z}
\newcommand{\drSph}{d r \, \unite_r + r \, d \theta \, \unite_\theta + r \, \sin(\theta) \, d \phi \, \unite_\phi }
\newcommand{\coordvect}[3]{{#1} \, \uniti \, + \, {#2} \, \unitj \, + {#3} \, \unitk}
\newcommand{\cylcoordvect}[3]{{#1} \, \unite_\rho \, + \, {#2} \, \unite_\phi \, + {#3} \, \unite_z}
\newcommand{\sphcoordvect}[3]{{#1} \, \unite_\r \, + \, {#2} \, \unite_\theta \, + {#3} \, \unite_\phi}

\newcommand{\ul}[1]{\underline{#1}}

\newsavebox{\fmbox}
\newenvironment{eqnframe}[1]     
{
	\begin{center} 
	\begin{lrbox}{\fmbox}
	\begin{minipage}{#1}
}     
{
	\end{minipage}
	\end{lrbox}\fbox{\usebox{\fmbox}}
	\end{center}
}

\newcommand\Tpad{\rule[4.5ex]{0pt}{0pt}}
\newcommand\Bpad{\rule[-3.75ex]{0pt}{0pt}}

\renewcommand{\labelenumi}{\textbf{\arabic{enumi}}.}
\renewcommand{\labelenumii}{\textbf{(\roman{enumii})}}
\renewcommand{\labelenumiii}{\textbf{(\alph{enumiii})}}

\newcommand{\parD}[2]{\frac{\partial #1}{\partial #2}}
\newcommand{\parDD}[2]{\frac{\partial^2 #1}{\partial #2 ^2}}
\newcommand{\laplacian}{\Delta}
\renewcommand{\div}{\nabla\cdot}
\newcommand{\grad}{\nabla}
\newcommand{\divp}{\nabla^\prime\cdot}
\newcommand{\gradp}{\nabla^\prime}
\newcommand{\curl}{\nabla\times}
\newcommand{\cross}{\times}
\renewcommand{\dot}{\cdot}
% define some colors
\definecolor{cBlue}{rgb}{.255,.41,.884} % RoyalBlue of svgnames
\definecolor{cRed}{rgb}{1, 0, 0} % Red of svgnames


%% newcommands.tex (new command definitions)

% Here you would include any additional packages that you want to use.
% You should make sure they don't clash with the above packages that
% are in use in the style file.
% If you want to call in some style files or new packages, put them here
%\usepackage{undertilde}
%\usepackage[left=2cm,right=2cm,top=2cm,bottom=2cm]{geometry}
\usepackage[a4paper]{geometry}
%\usepackage[latin1]{inputenc}
\usepackage{amsmath, latexsym, color, graphicx, amssymb, here}
\usepackage{amsfonts}
\usepackage{epsf, epsfig, pifont,tikz}
\usepackage{graphics, calrsfs}
%\usepackage{tangocolors}
\usepackage{times}
\usepackage{fancybox,calc}
\usepackage{hyperref}
\usepackage{pgfplots}
\usepackage{verbatim}
\usepackage{esint} 
\usepackage{amsthm}


%% environment name, counter, text
\newtheorem{theorem}{Theorem}[section]
\newtheorem{lemma}[theorem]{Lemma}
\newtheorem{definition}[theorem]{Definition}
%% Use like:
% \begin{definition}
% Blah blah blah
% \end{definition}



%% Brackets
\newcommand{\lb}{\left(}
\newcommand{\rb}{\right)}
%% Square Brackets
\newcommand{\lbs}{\left[}
\newcommand{\rbs}{\right]}

%Bolded letters
\newcommand{\A}{{\bf A}}
\newcommand{\B}{{\bf B}}
\newcommand{\T}{{\bf T}}
\newcommand{\C}{{\bf C}}
\newcommand{\N}{{\bf N}}
\newcommand{\n}{{\bf n}}
\newcommand{\bfx}{{\ensuremath{\mathbf{x}}}}
%Bolded letters, which previously had a command
\renewcommand{\v}{{\bf v}}
\renewcommand{\r}{{\bf r}}
\renewcommand{\a}{{\bf a}}

%Set notation. Needs to be in a maths environment (equation or $$)
\newcommand{\R}{{\mathbb R}}
\newcommand{\Z}{{\mathbb Z}}

%Unit vectors. Needs to be in a maths environment (equation or $$)
\newcommand{\uniti}{{\hat{\mbox{\boldmath $\imath$}}}}
\newcommand{\unitj}{{\hat{\mbox{\boldmath $\jmath$}}}}
\newcommand{\unitk}{{\hat{\mbox{\boldmath $\mathit{k}$}}}}
\newcommand{\unitn}{{\hat{\mbox{\boldmath $\mathit{n}$}}}}
\newcommand{\unite}{{\hat{\mbox{\boldmath $\mathit{e}$}}}}
\newcommand{\unitu}{{\hat{\mbox{\boldmath $\mathit{u}$}}}}


%Shorthand for commonly used expressions with cursive font
\newcommand{\ie}{{\em i.e.} \/}
\newcommand{\eg}{{\em e.g.} \/}
\newcommand{\etc}{{\em etc.} \/}
\newcommand{\etal}{{\em et al. }}


%Bolded, cursive strings of characters
\newcommand{\mathbi}[1]{\textbf{\em #1}}


%Bolded dot symbol
\newcommand{\bcdot}{\mbox{\boldmath $\, \cdot \, $}}

%bolds and puts in a cursive underline for a vector
\renewcommand{\vec}[1]{{\mbox{\boldmath $\utilde{\mathit{#1}}$}}}


%Shorthand commands for writing xy,xz,yz planes
\newcommand{\xyplane}{$x$-$y$ plane\/}
\newcommand{\xzplane}{$x$-$z$ plane\/}
\newcommand{\yzplane}{$y$-$z$ plane\/}
\newcommand{\abplane}[2]{${#1}$-${#2}$ plane\/}

%Full derivative of A wrt B. Needs to be in a maths environment (equation or $$)
\newcommand{\FullDif}[2]{\dfrac{d {#1}}{d {#2}}}

%Partial derivative of A wrt B. Needs to be in a maths environment (equation or $$)
\newcommand{\ParDif}[2]{\dfrac{\partial {#1}}{\partial {#2}}}
%Second partial derivative of A wrt B. Needs to be in a maths environment (equation or $$)
\newcommand{\DblParDif}[2]{\frac{\partial^2 #1}{\partial #2 ^2}}

%Material derivative of A wrt B. Needs to be in a maths environment (equation or $$)
\newcommand{\MatDif}[2]{\dfrac{D {#1}}{D {#2}}}


%Writes the cartesian gradient of a function. Leave as {} to write the operator. Needs to be in a maths environment (equation or $$)
\newcommand{\grad}[1]{\ParDif{#1}{x}  \uniti+ \ParDif{#1}{y} \, \unitj + \ParDif{#1}{z} \, \unitk}


%Writes the cartesian Laplacian of a function. Leave as {} to write the operator. Needs to be in a maths environment (equation or $$)
\newcommand{\laplacian}[1]{\DblParDif{{#1}}{x} + \DblParDif{{#1}}{y}+ \DblParDif{{#1}}{z}}

%Writes the cylindrical gradient of a function. Leave as {} to write the operator. Needs to be in a maths environment (equation or $$)
\newcommand{\gradCylindrical}[1]{\ParDif{{#1}}{\rho} \, \unite_\rho \ + \ \dfrac{1}{\rho} \ParDif{{#1}}{\phi} \, \unite_\phi \ + \ \ParDif{{#1}}{z} \, \unite_z}



%Writes the spherical gradient of a function. Leave as {} to write the operator. Needs to be in a maths environment (equation or $$)
\newcommand{\gradSpherical}[1]{\ParDif{{#1}}{r} \, \unite_r \ + \ \dfrac{1}{r} \ParDif{{#1}}{\theta} \, \unite_\theta \ + \ \dfrac{1}{r \sin(\theta)}\ParDif{{#1}}{\phi} \, \unite_\phi}

%Writes the spherical divergence of a function. Leave as {}{}{} to write the operator, otherwise #1 is r component, #2 is theta component and #3 is phi component. Needs to be in a maths environment (equation or $$)
\newcommand{\divSpherical}[3]{\dfrac{1}{R^2}\, \ParDif{}{R}\lb(R^2\, {#1}\rb) \ + \  \dfrac{1}{R \sin(\theta)} \ParDif{}{\theta} \lb(\sin(\theta)\, {#2}\rb) \ + \ \dfrac{1}{R \sin(\theta)}\ParDif{{#3}}{\phi} \  }

%Creates a cartesian position vector, argument is the vector field in question. Needs to be in a maths environment (equation or $$)
\newcommand{\posvect}[1]{{#1}_x \, \uniti + {#1}_y \, \unitj + {#1}_z \, \unitk}

%Creates the cartesian generic position vector r. Needs to be in a maths environment (equation or $$)
\newcommand{\posvectr}{x \, \uniti + y \, \unitj + z \, \unitk}
%Creates the cylindrical generic position vector r. Needs to be in a maths environment (equation or $$)
\newcommand{\posvectcyl}{\rho \, \unite_\rho \, + z \, \unite_z}
%Creates the spherical generic position vector r. Needs to be in a maths environment (equation or $$)
\newcommand{\posvectsph}{r \, \unite_\r}


%Shorthand for a parametrised position vector. Needs to be in a maths environment (equation or $$)
\newcommand{\parametricposvectr}{\vec{r}(t) \ = \ x(t) \, \uniti + y(t) \, \unitj + z(t) \, \unitk}

%Creates the divergence of a specified vector. Needs to be in a maths environment (equation or $$)
\renewcommand{\div}[1]{\nabla \bcdot \vec{#1}}

%Creates the curl of a specified vector. Needs to be in a maths environment (equation or $$)
\newcommand{\curl}[1]{\nabla \times \vec{#1}}

%Creates the magnitude of a specified vector. Needs to be in a maths environment (equation or $$)
\newcommand{\magnitude}[1]{ \| \vec{#1} \| }



%Creates a differential distance in Cartesian. Needs to be in a maths environment (equation or $$)
\newcommand{\drCart}{d x \, \uniti + d y \, \unitj + d z \, \unitk}
%Creates a differential distance in Cylindrical. Needs to be in a maths environment (equation or $$)
\newcommand{\drCyl}{d \rho \, \unite_\rho + \rho \, d \phi \, \unite_\phi + d z \, \unite_z}
%Creates a differential distance in Spherical. Needs to be in a maths environment (equation or $$)
\newcommand{\drSph}{d r \, \unite_r + r \, d \theta \, \unite_\theta + r \, \sin(\theta) \, d \phi \, \unite_\phi }


%Creates arbitrary vectors with 3 arguments. Needs to be in a maths environment (equation or $$)
\newcommand{\coordvect}[3]{{#1} \, \uniti \, + \, {#2} \, \unitj \, + {#3} \, \unitk}
\newcommand{\cylcoordvect}[3]{{#1} \, \unite_\rho \, + \, {#2} \, \unite_\phi \, + {#3} \, \unite_z}
\newcommand{\sphcoordvect}[3]{{#1} \, \unite_\r \, + \, {#2} \, \unite_\theta \, + {#3} \, \unite_\phi}


%Creates a box around a given equation. Must be in a maths environment.
\newsavebox{\fmbox}
\newenvironment{eqnframe}[1]     
{
	\begin{center} 
	\begin{lrbox}{\fmbox}
	\begin{minipage}{#1}
}     
{
	\end{minipage}
	\end{lrbox}\fbox{\usebox{\fmbox}}
	\end{center}
}


%These make differently aligned boxes, with width {#1} and height {#2}
\newcommand\Tpad[2]{\rule[4.5ex]{{#1}pt}{{#2}pt}}
\newcommand\Bpad[2]{\rule[-3.75ex]{{#1}pt}{{#2}pt}}

%These make slightly different shaped zeros.
\renewcommand{\labelenumi}{\textbf{\arabic{enumi}}.}
\renewcommand{\labelenumii}{\textbf{(\roman{enumii})}}
\renewcommand{\labelenumiii}{\textbf{(\alph{enumiii})}}

%Lets you use \cross instead of times. Needs to be in a maths environment (equation or $$)
\newcommand{\cross}{\times}


% define some colors
\definecolor{cBlue}{rgb}{.255,.41,.884} % RoyalBlue of svgnames
\definecolor{cRed}{rgb}{1, 0, 0} % Red of svgnames


%Shorthand for various useful things
%Argument 1, with argument 2 as a subscript
\newcommand{\An}[2]{{#1}_{{#2}}}
%Define the pair of co-ordinates
\newcommand{\AnBm}[4]{(\An{{#1}}{{#2}},\An{{#3}}{{#4}})}

%integral from x0 to x1
\newcommand{\intxoxi}{\int_{\An{x}{0}}^{\An{x}{1}}}

%f(x,y)
\newcommand{\fxy}{f(x,y)}
%g(x,y)
\newcommand{\gxy}{g(x,y)}

%%f(x,y,z)
\newcommand{\fxyz}{f(x,y,z)}
%%g(x,y,z)
\newcommand{\gxyz}{g(x,y,z)}

%% Variant epsilon
\newcommand{\ep}{\varepsilon}

%%Yprime squared
\newcommand{\yp}{(y^\prime)^{2}}

%%Coordinates ~extra style~
\newcommand{\cord}{co$\ddot{\text{o}}$rdinates\ }


% If you want to generate an index you should include the following command
% which puts a makeindex command in the preamble. Additionally, you need to un-comment
% the file 'index' in the '\includeonly' command below and also include the 'index'
% file in the main document.
\makeindex

% During thesis writing you might only want to see the pdf of the current chapter rather than the entire 
% thesis. However, you will want to ensure page numbers, chapter numbers, and referencing is maintained 
% for the entire thesis. 
% In the main section below, between \begin{document} and \end{document} you can use the \include{}
% command to specify all of the components (other .tex files) of the thesis. 
% Prior to \begin{document} you can use \includeonly{} to specify which of those components you want to 
% see. You can comment out files that you will include later or have already finished to speed
% up TeX processing. Comment out the \includeonly{} command to see the entire thesis.

%\includeonly{prelude % Contains all the relevant candidate information (name, degrees, abstract etc)
%,chap1  % The first chapter
%,chap2  % The second chapter
%,chap3  % The third chapter
%,chap4  % The fourth chapter
%%,app0   % Needed to switch to appendix mode
%,app1   % The first appendix
%,app2   % The second appendix
%,biby   % Makes the bibliography from the BibTeX database
%,index  % Places the index in the thesis
%}

% Begin the thesis
\begin{document}

% Include all the pieces of your thesis in here.

% prelude.tex (specification of which features in `mathphdthesis.sty' you
% are using, your personal information, and your title & abstract)

% Specify features of `mathphdthesis.sty' you want to use:
\titlepgtrue % main title page (required)
\signaturepagetrue % page for declaration of originality (required)
\copyrighttrue % copyright page (required)
\abswithesisfalse % abstract to be bound with thesis (optional)
\ackfalse % acknowledgments page (optional)
\tablecontentstrue % table of contents page (required)
\tablespagefalse % table of contents page for tables (required only if you have tables)
\figurespagefalse % table of contents page for figures (required only if you have figures)

\title{PLACE THESIS TITLE HERE} % use all capital letters
\author{Your Name} % use mixed upper & lower case
\prevdegrees{B.Sc. (Qld)} % Used to specify your previous degrees...use mixed upper & lower case
\advisor{Dr Ask Me} % example: Professor Lawrence K. Forbes 
\dept{Mathematics} % your academic department
\submitdate{October 2018} % month & year of your thesis submission

\newcommand{\abstextwithesis}
{
The basic abstract goes here. You can use paragraphs and normal \LaTeX~commands.

For example, I looked at the problem $a^n+b^n=c^n$ and found some
integer solutions for $n>2$! Take that Fermat!
}

\newcommand{\acknowledgement}
{
Thank all your helpers here.

This Honours thesis template for \LaTeX~exists primarily because of the efforts of past students Tim Callaghan, Mike Sumner, and Melissa Humphries in producing templates for their respective PhD theses.

Thank you also to Thomas Hindle for tidying newcommands.tex. This file is filled with shortcuts to simplify the writing of repetitive equations, expressions, and environments. Thomas has removed redundant definitions, improved existing definitions, introduced some new ones, and added descriptions to each of the shortcuts.
}

% Take care of things in `mathphdthesis.sty' or 'MathPhysHonoursThesis.sty'
% behind the scenes.
% Basically just does a check of all the fields that have been activated
% above and fills out the appropriate pages and adds them to the thesis.
\beforepreface
\afterpreface


% Set page numbering to arabic the first time we commence a chapter.
% This is required to get the page numbering font correct.
\pagenumbering{arabic}

% chap1.tex (Chapter 1 of the thesis)

% Note that the text in the [] brackets is the one that will
% appear in the table of contents, whilst the text in the {}
% brackets will appear in the main thesis.

%[script style reduce fraction size]

%[quantikz]

%\section{Quantum Mechanics}

\chapter[PRELIMINARIES]{Preliminaries}
\section{Algebra of Unitary Matrices}
In quantum mechanics it will turn out crucial to have a strong theory of unitary operations acting linearly on complex-valued objects, so to that end we shall define these concepts and their notation here. [And sketch some theorems?]
\subsection{Hilbert Spaces}
Cartesian coordinates provide a powerful abstract way of reasoning about physical space as the combination of 3 variables, or conversely a way of visualizing combinations of variables as planes or volumes within a physical space. Hilbert spaces are a description which abstracts the Cartesian coordinate system even further, describing a much larger class of mathematical objects with similar geometric properties, including the space of possible states that a quantum object can take.

The abstract definition of a Hilbert space describes the following mathematical objects:
\begin{itemize}
	\item A set of `scalars', either $\mathbb{R}$ or $\mathbb{C}$
	\item A (nonempty) set of `vectors' $\mathbb{H}$
	\item A binary operation of vector addition: $\mathbb{H} \times \mathbb{H} \to \mathbb{H}$, written $u + v$ where $u$, $v$ are the vectors being added.
	\item An operation that `scales' vectors by a scalar: $\mathbb{K} \times \mathbb{H} \to \mathbb{H}$, simply written $av$ where $a$ is the scalar and $v$ is the vector
	\item An operation called the inner product, yielding a scalar for each pair of vectors: $\mathbb{H} \times \mathbb{H} \to \mathbb{K}$, written for now as $(u, v)$ where $u$, $v$ are the vectors involved.
	\item Well defined limits of any Cauchy sequence in $\mathbb{H}$, written $\lim_{n \to \infty} v_n$ where $(v_n)_{n=1}^\infty$ is the Cauchy sequence
\end{itemize}
Note that since $\mathbb{H}$ is non-empty, it must contain a `zero vector' acquired by scaling any element of $\mathbb{H}$ by 0. This vector is itself written as 0. When these objects are available, we say that $\mathbb{H}$ is a Hilbert space if it further satisfies the following algebraic properties, for any $a, b\in \mathbb{K}$, $u,v,w \in \mathbb{H}$:
\begin{itemize}
	\item vector associativity: $u + (v + w) = (u + v) + w$
	\item vector commutativity: $u + v = v + u$
	\item scalar identity: $1v = v$
	\item scalar associativity: $a(bv) = (ab)v$
	\item vector distribution: $a(u + v) = au + av$
	\item scalar distribution: $(a+b)v = av + bv$
	\item conjugate symmetry of inner products: $(u, v) = (v, u)^*$ (where $a^*$ is the complex conjugate of $a$)
	\item right linearity: $(u, v+w) = (u, v) + (u, w)$
	\item positive definite: $(v, v)$ real, and strictly positive whenever $v \neq 0$
\end{itemize}
Additionally Cauchy sequences must actually converge to their limits, a property whose formal definition and applications are beyond the scope of this thesis.

The primary Hilbert spaces discussed in this thesis are the sets of complex valued column vectors:
\[\mathbb{C}^n = \left\{\left[\begin{matrix} a_0\\a_1\\\vdots\\a_n\end{matrix}\right]\ |\ a_0, a_1, \dots a_n \in \mathbb{C}\right\}\]
Adding and scaling of vectors of course take the usual definitions:
\[
c_0\left[\begin{matrix} a_0\\a_1\\\vdots\\a_n\end{matrix}\right]
+
c_1\left[\begin{matrix} b_0\\b_1\\\vdots\\b_n\end{matrix}\right]
=
\left[\begin{matrix} c_0a_0+c_1b_0\\c_0a_1+c_1b_1\\\vdots\\c_0a_n+c_1b_n\end{matrix}\right]
\]
Inner products take a fairly natural definition:
\[
\left(
\left[\begin{matrix} a_0\\a_1\\\vdots\\a_n\end{matrix}\right]
,
\left[\begin{matrix} b_0\\b_1\\\vdots\\b_n\end{matrix}\right]
\right)
=
\left[\begin{matrix} a_0^*&a_1^*&\cdots&a_n^*\end{matrix}\right]
\left[\begin{matrix} b_0\\b_1\\\vdots\\b_n\end{matrix}\right]
= \sum_{i=0}^{n-1} a_i^*b_i
\]
Inner products introduce a concept of orthogonality to a space of vectors, we say two vectors $u, v$ are orthogonal when $(u, v) = 0$.
\[
\left(\left[\begin{matrix}1\\0\end{matrix}\right],\left[\begin{matrix}0\\1\end{matrix}\right]\right) = 0
\]
In general if two column vectors $v_i$ and $v_j$ have coordinates 0 except for their $i$th and $j$th coordinate respectively, then $(v_i, v_j) = \delta_{ij}$, 1 when $i = j$ and 0 otherwise. This condition is what it means for the set $\{v_0 \dots v_{n-1}\}$ to be an orthonormal basis of the vector space $\mathbb{C}^n$. It is a fundamental result of Hilbert spaces that if a countable set of vectors $v_i \in \mathbb{H}$ span the whole space, that is if every vector $w \in \mathbb{H}$ is an infinite linear combination of the vectors $v_i$, then $\mathbb{H}$ has an orthonormal basis. If this orthonormal basis is finite then we say that $\mathbb{H}$ is finite dimensional, and can observe any vector $v \in \mathbb{H}$ as analogous to the following column vector in $\mathbb{C}^n$:
\[\left[\begin{matrix}
	(v_0, v)\\
	(v_1, v)\\
	\cdots\\
	(v_n, v)
\end{matrix}\right]\]
Notice that the column vectors corresponding to the basis vectors $v_i$ have coordinate 0 everywhere except for their $i$th co-ordinate which is 1. Further the co-ordinates of these column vectors are exactly the co-ordinates of $v$ in $\mathbb{H}$:
\[\left(v_i, \sum_{j=0}^{n-1} a_jv_j\right) = \sum_{j=0}^{n-1} a_j\left(v_i, v_j\right) = a_i\]
Finally the inner products of vectors $u, v \in \mathbb{H}$ are exactly the inner products of the corresponding column vectors:
\begin{align*}
	(u, v) &= \left(\sum_{i=0}a_iv_i, \sum_{j=0} b_jv_j\right)
	\\&= \sum_{j=0}b_j \left(\sum_{i=0}a_iv_i, v_j\right)
	\\&= \sum_{j=0}b_j \left(v_j, \sum_{i=0}a_iv_i\right)^*
	\\&= \sum_{i,j=0}a_i^*b_j \left(v_j, v_i\right)^*
	\\&= \sum_{i,j=0}a_i^*b_j \delta_{ij}
	\\&= \sum_{i=0}a_i^*b_i
	\\&= \sum_{i=0}(v_i, u)^*(v_i, v)
	\\&= \left(
	\left[\begin{matrix}
		(v_0, u)\\
		(v_1, u)\\
		\cdots\\
		(v_n, u)
	\end{matrix}\right]
	,
	\left[\begin{matrix}
		(v_0, v)\\
		(v_1, v)\\
		\cdots\\
		(v_n, v)
	\end{matrix}\right]
	\right)
\end{align*}
[could prove conjugate linearity earlier, or stop proving linear algebra results altogether!!]

In this sense any finite dimensional Hilbert space is geometrically equivalent to the complex vector space $\mathbb{C}^n$ where $n$ is the size of any orthonormal basis of $\mathbb{H}$.

As we shall describe in Chapter 1, quantum computation most commonly acts on Hilbert spaces that are finite dimensional, and while the specific space being considered can be important in practical contexts, this thesis is concerned with the theoretical and therefore only needs to understand the complex vector space $\mathbb{C}^n$.

One caveat to this equivalence is the concept of basis-independence, that many concepts in linear algebra are made useful by the fact that they give the same result regardless of which basis is used for the space in question. To show that something is basis independent, it is sufficient and often convenient to define it directly in terms of the inner product and vector operations, rather than the coordinate system induced by a given basis. As an example the $\ell^2$ norm in $\mathbb{C}^n$ does not depend on the basis used:
\[\norm{v}_2 = \sqrt{(v, v)} = \sqrt{\sum_{i=0}^{n-1}\ord{v_i}^2}\]
This cannot be done for any other $\ell^p$ norm.

\subsection{Dirac Notation}
There are a number of notations available for reasoning about linear algebra and vector spaces, but we shall see [do we?] that in order to reason about quantum algorithms we will rely heavily on the following techniques:
\begin{itemize}
	\item linear operators defined in terms of inner products
	\item change of basis via unitary operators
	\item change of index set when summing vectors
\end{itemize}
For this collection of techniques the Dirac notation for vectors and linear operators is particularly well suited.

The main feature of Dirac notation is the ket, where a bar and angle bracket are used to distinguish a symbol as representing a vector: $\ket{v}$, $\ket{*}$, $\ket{0}$, $\ket{i}$, etc.

The most common vectors used are the basis vectors, and in Dirac notation it is natural to use the index of each basis vector as the symbol \textit{for the vector itself}:
\begin{align*}
	\ket{0} = \left[\begin{matrix}
		1\\
		0
	\end{matrix}\right]
	&&&
	\ket{1} = \left[\begin{matrix}
		0\\
		1
	\end{matrix}\right]
\end{align*}

The inner product of two vectors $(\ket{u}, \ket{v})$ is normally written $\braket{u}{v}$, called a `bra-ket'. This is inspired by the notation $\langle u, v\rangle$, used before Dirac, but allows an elegant representation of co-vectors. Co-vectors map a vector to a scalar, and in an inner product space every vector has a natural co-vector defined using the inner product. In Dirac notation we can simply omit the ket from an inner product to notate a co-vector:
\begin{align*}
	\bra{u} &: \mathbb{C}^n \to \mathbb{C}
	\\\bra{u} &\equiv \ket{v} \mapsto \braket{u}{v}
\end{align*}

\subsection{Linear Operators}
If $U$ and $V$ are Hilbert spaces, (or generally vector spaces) then a map $A: U \to V$ is a linear operator if it satisfies the following in general:
\[A(a\ket{u} + b\ket{v}) = aA\ket{u} + bA\ket{v}\]
If we have vectors $\ket{u} \in U$ and $\ket{v} \in V$ then we can define a simple linear operator using the inner product in $U$:
\[\ket{v}\bra{u} \equiv \bra{u'} \mapsto \braket{u}{u'}\ket{v}\]
The fact that this is linear is a consequence of the inner product being linear in its second argument, one of the defining properties of Hilbert spaces.

Note that if we reverse the notation for scaling a vector we get the expression $\ket{v}\braket{u}{u'}$, which looks very similar to the application $\ket{v}\bra{u}(\ket{u'})$, again equivocating the map with the object it produces, just as was done in defining co-vectors.

As we have discussed, when dealing with any finite dimensional Hilbert space we can assume the space is equivalent to some complex vector space $\mathbb{C}^n$, in which case matrix representations of a linear operator become available. We derive matrix representation explicitly, since the relevant technology is readily available from Dirac notation. First observe that linear operators $A$ say, are determined uniquely by the image of each basis vector $\ket{j}$, say $A\ket{j} = \ket{v_j}$. First evaluate an arbitrary application of $A$:
\begin{align*}
	A\left(\sum_{j=0}^{n-1}u_j\ket{j}\right)
	&= \sum_{j=0}^{n-1}u_jA\ket{j}
	\\&= \sum_{j=0}^{n-1}u_j\ket{v_j}
\end{align*}
Then this turns out to be the same result that we get from applying the following:
\begin{align*}
	\left(\sum_{j=0}^{n-1}\ket{v_j}\bra{j}\right)\left(\sum_{i=0}^{n-1}u_i\ket{i}\right)
	&= \sum_{i=0}^{n-1}\sum_{j=0}^{n-1}u_i\braket{j}{i}\ket{v_j}
	\\&= \sum_{j=0}^{n-1}u_j\ket{v_j}
\end{align*}
Since these two maps are equal on their whole domain, they are the same:
\[A = \sum_{j=0}^{n-1} \ket{v_j}\bra{j}\]
Further if the codomain is also finite dimensional then we can repeat this process. First note that the outer product notation $\ket{v}\bra{u}$ is linear on $\ket{v}$, regardless of which vector $\ket{u'}$ it is applied to:
\[(a\ket{v_1}+b\ket{v_2})\braket{u}{u'} = a\ket{v_1}\braket{u}{u'}+b\ket{v_2}\braket{u}{u'}\]
Now suppose the image vectors $\ket{v_j}$ are in $\mathbb{C}^m$ and have coordinates themselves, $\ket{v_j} = \sum_i^{m-1} a_{ij}\ket{i}$, then:
\begin{align*}
	A = \sum_{i,j} a_{ij}\ket{i}\bra{j}
\end{align*}
That is, all linear operators are a linear combination of outer products between basis vectors, making these outer products a basis for the (Hilbert) space of linear operators between finite dimensional Hilbert spaces. We have derived a way of representing linear operators $\mathbb{C}^n \to \mathbb{C}^m$ as an array of $n \times m$ scalar coordinates, which is the exact structure we want from a matrix representation, and matrix multiplication appears directly from composing the corresponding maps:
\[\left(\sum_{i,j}a_{ij}\ket{i}\bra{j}\right)\left(\sum_{j',k}b_{j'k}\ket{j'}\bra{k}\right) = \sum_{i,k}\left(\sum_j a_{ij}b_{jk}\right)\ket{i}\bra{k}\]

For example any linear operator $A: \mathbb{C}^2 \to \mathbb{C}^2$ can be written as follows:
\begin{align*}
	A = \left[\begin{matrix}
		a&b\\
		c&d
	\end{matrix}\right]
	=&\ 
	a\left[\begin{matrix}
		1&0\\
		0&0
	\end{matrix}\right]
	+
	b\left[\begin{matrix}
		0&1\\
		0&0
	\end{matrix}\right]
	+
	c\left[\begin{matrix}
		0&0\\
		1&0
	\end{matrix}\right]
	+
	d\left[\begin{matrix}
		0&0\\
		0&1
	\end{matrix}\right]
	\\=&\  a\ket{0}\bra{0}+b\ket{0}\bra{1}+c\ket{1}\bra{0}+d\ket{1}\bra{1}
\end{align*}

When defining a linear operator $L: \mathbb{C}^n \to \mathbb{C}^n$ we can define it as the `linear extension' of a more succinct map $\phi: \{\ket{i}\ |\ 0 \leq i < n\} \to \mathbb{C}^n$, by first defining $\ket{v_j} = \phi(\ket{j})$ and then setting $L = \sum_{j} \ket{v_j}\bra{j}$. This is simply a matrix whose columns are the vectors $\ket{v_j}$.
\subsection{The Hermitian Adjoint and Diagonalizable Matrices}
An aspect of linear algebra essential to quantum computation is diagonalization. We shall describe a number of properties that a square matrix can have, in terms of both its Hermitian Adjoint, and its eigenvalues, but first we define these concepts.

The Hermitian Adjoint of a linear operator $A$ (square or otherwise) is the unique linear operator $A^\dagger$ that satisfies $\braket{u}{v'} = \braket{u'}{v}$ whenever $\ket{v'} = A\ket{v}$, and $\ket{u'} = A^\dagger \ket{u}$. It turns out that the matrix representation of $A^\dagger$ is exactly the complex conjugate of the transpose of $A$, so this is taken as the definition of the Hermitian Adjoint of a matrix. As an example observe the following matrix $A$ and its Hermitian adjoint:
\begin{align*}
	A = \left[\begin{matrix}
		1 & 1+i\\
		0 & 1
	\end{matrix}\right]
	&&&
	A^\dagger = \left[\begin{matrix}
		1 & 0\\
		1-i & 1
	\end{matrix}\right]
\end{align*}

Note now that in the vector space $\mathbb{C}^2$ the Hermitian adjoint of a vector has the same behaviour as a co-vector:
\begin{align*}
	\ket{u}^\dagger\ket{v}
	=&\ 
	\left[\begin{matrix}
		a\\
		b
	\end{matrix}\right]^\dagger
	\left[\begin{matrix}
		c\\
		d
	\end{matrix}\right]
	\\=&\ 
	\left[\begin{matrix}
		a^*&b^*
	\end{matrix}\right]
	\left[\begin{matrix}
		c\\
		d
	\end{matrix}\right]
	\\=&\ a^*c+b^*d
	\\=&\ \braket{u}{v}
\end{align*}
This generalizes to any co-vector in any Hilbert space.

Associativity of the matrix product also lets us represent the outer product $\ket{u}\bra{v}$ using the Hermitian adjoint, as the matrix product $\ket{u}\ket{v}^\dagger$. This can be used to prove that
$(\ket{u}\bra{v})^\dagger = \ket{v}\bra{u}$

Now we move on to the eigenvector problem, which is the problem of finding a scalar $\lambda$ and non-zero vector $\ket{v}$ so that $A\ket{v}=\lambda\ket{v}$. Such a $\lambda$ is called an eigenvalue of $A$, and such a $\ket{v}$ is called the corresponding eigenvector of $A$. For example if A has a non-trivial null-space, then $\lambda=0$ will be an eigenvalue of $A$, and any vector in the null-space of $A$ will be a corresponding eigenvector of $A$: $A\ket{v} = 0\ket{v}$.

If $\ket{v}$ is normalized, then the matrix $B=\lambda\ket{v}\bra{v}$ will also satisfy the eigenvalue problem $B\ket{v}=\lambda\ket{v}$, and any vector orthogonal to $\ket{v}$ will be in the null-space of $B$. This means that if all of the eigenvectors of $A$ are orthogonal to eachother, then we can write $A$ as a sum of such $B$ vectors. This structure $A = \sum_i \lambda_i \ket{v_i}\bra{v_i}$ tells us that $A$ acts like a diagonal matrix on its eigenvalues, which we call the diagonal representation of $A$, and for this reason call any such $A$ `diagonalizable'.

As an example, the following matrix $Z$ is already diagonal and so has a straight-forward diagonal representation:
\begin{align*}
	Z = \left[\begin{matrix}
		1&0\\
		0&-1
	\end{matrix}\right] = 1\ket{0}\bra{0} - 1\ket{1}\bra{1}
\end{align*}

Having orthogonal eigenvectors is closely related to the Hermitian adjoint through a class of matrices called `normal' matrices, where a matrix $A$ is normal if it satisfies $A^\dagger A = AA^\dagger$. A major result of linear algebra called the spectral theorem of normal matrices, is that a matrix is diagonalizable if and only if it is normal. What follows is a list of different classes of normal operator, each defined by a property of the matrix, and by an equivalent property of all of its eigenvalues:
\begin{align*}
	\text{Hermitian matrix:\ }&&& A^\dagger = A & \iff& \lambda \in \mathbb{R} \\
	\text{Unitary matrix:\ }&&& A^\dagger = A^{-1} & \iff& \ord{\lambda}=1 \\
	\text{Positive normal:\ }&&& \bra{v}A\ket{v} \in \mathbb{R}, \geq 0 & \iff& \lambda \in \mathbb{R}, \geq 0 \\
	\text{Positive definite normal:\ }&&& \bra{v}A\ket{v} \in \mathbb{R}, > 0 & \iff& \lambda \in \mathbb{R}, > 0 \\
	\text{Projection matrix:\ }&&& A^2 = A & \iff& \lambda \in \{0, 1\} \\
\end{align*}

The most relevant of these is the unitary matrix, whose defining property can also be written $AA^\dagger = I$, which is simply the matrix equation corresponding to the fact that the columns of $A$ form an orthonormal basis. In addition to the above, another characterisation of unitary matrices is that they preserve inner products, which follows naturally from the matrix interpretation of inner products:
\[(A\ket{u}, A\ket{v}) = (A\ket{u})^\dagger A\ket{v} = \bra{u}AA^\dagger \ket{v} = \braket{u}{v}\]
This means in particular that a unitary matrix will preserve $\ell^2$ norms, and will therefore induce an invertible operation on any (complex, centre at origin) $n$-sphere $\{\ket{v}\ |\ \braket{v}{v} = r^2\}$.

Finally, if $P$ is a polynomial then we can evaluate it on a square matrix $A$ as well, and it is easy to show that this polynomial `applies itself' to the eigenvalues of $A$, i.e.\ if $A\ket{v} = \lambda\ket{v}$ then $P(A)\ket{v} = P(\lambda)\ket{v}$.

Now any analytic function will be a limit of some polynomials, most notably the exponential:
\[e^x = \sum_n^\infty \frac{x^n}{n!}\]
This motivates us to apply such analytic functions directly to normal matrices, simply by applying it to each eigenvalue:
\begin{align*}
	f\left(\sum_i \lambda_i\ket{i}\bra{i}\right)
	&= \lim_{L \to \infty} \sum_{n=0}^L P_i\left(\sum_i \lambda_i\ket{i}\bra{i}\right)
	\\&= \lim_{L \to \infty} \sum_{n=0}^L \sum_i P_i(\lambda_i)\ket{i}\bra{i}
	\\&= \sum_i \lim_{L \to \infty} \left(\sum_{n=0}^L  P_i(\lambda_i)\right)\ket{i}\bra{i}
	\\&= \sum_i f(\lambda_i)\ket{i}\bra{i}
\end{align*}

This can be very useful for solving certain matrix equations such as $A^2 = B$ having the solution $A = \sqrt{B}$. It is also useful for mapping between Hermitian and unitary matrices with the map $U = e^{iH}$.

\subsection{Kronecker Products}
We have made explicit the way in which vectors, co-vectors, and linear operators are represented as matrices, that is as arrays of complex numbers. An operation that is useful for all of these objects is the tensor product, and while the tensor product can be described in the abstract as an operation between Hilbert spaces, the only cases of interest here are complex-valued matrices, whose tensor products are described by a much more concrete operation called the Kronecker product.

The Kronecker product has a fairly simple definition, if $A$ is an $m_1$ by $n_1$ matrix with elements $a_{ij}$, and $B$ is an $m_2$ by $n_2$ matrix with elements $b_{ij}$ then the Kronecker product $A \otimes B$ will be an $m_1m_2$ by $n_1n_2$ matrix and in block matrix form will look like the following:
\begin{align*}
	A &= \left[\begin{matrix}
		a_{00} & a_{01} & \dots & a_{0n_1}\\
		a_{10} & a_{11} & \dots & a_{1n_1}\\
		\vdots & \vdots & \ddots & \vdots\\
		a_{m_10} & a_{m_11} & \dots & a_{m_1n_1}
	\end{matrix}\right]
	\\\implies A\otimes B &= \left[\begin{matrix}
		a_{00}B & a_{01}B & \dots & a_{0n_1}B\\
		a_{10}B & a_{11}B & \dots & a_{1n_1}B\\
		\vdots & \vdots & \ddots & \vdots\\
		a_{m_10}B & a_{m_11}B & \dots & a_{m_1n_1}B
	\end{matrix}\right]
\end{align*}
Written more compactly, if $A\otimes B = C$ has elements $c_{m_2i_1 + i_2,n_2j_1+j_2}$, where $i_1$ is less than $m_1$, etc.\ then these elements of $C$ are exactly the products of elements of $A$ and $B$:
\[c_{m_2i_1 + i_2,n_2j_1+j_2} = a_{i_1j_1}b_{i_2j_2}\]

If $A$ is a vector $\ket{u}$ and $B$ is a co-vector $\bra{v}$ then their Kronecker product will be exactly the matrix product $AB$ which is exactly the outer product $\ket{u}\bra{v}$:
\[
A \otimes B =
\left[\begin{matrix}
	a_0\\a_1\\\vdots\\a_m
\end{matrix}\right]
\otimes
\left[\begin{matrix}
	b_0&b_1&\dots&b_n
\end{matrix}\right]
=
\left[\begin{matrix}
	a_0b_0 & a_0b_1 & \dots & a_0b_n\\
	a_1b_0 & a_1b_1 & \dots & a_1b_n\\
	\vdots & \vdots & \ddots & \vdots\\
	a_mb_0 & a_mb_1 & \dots & a_mb_n
\end{matrix}\right]
\]
On the other hand if $A$ and $B$ are both vectors, in $\mathbb{C}^{m_1}$ and $\mathbb{C}^{m_2}$ respectively, then their Kronecker product will be a vector in the larger space $\mathbb{C}^{m_1\times m_2}$, and in particular if $A$ and $B$ are canonical basis vectors $\ket{j_1}$ and $\ket{j_2}$ then their tensor product will be another canonical basis vector, $\ket{m_2j_1 + j_2}$. This allows the whole space $\mathbb{C}^{m_1 \times m_2}$ to be spanned by Kronecker products of $\mathbb{C}^{m_1}$ and $\mathbb{C}^{m_2}$.

The Kronecker product satisfies the following algebraic properties, all of which follow directly from their relevant definitions:
\begin{itemize}
	\item $(A\otimes B)(C \otimes D) = (AC) \otimes (BD)$
	\item in particular $(A\otimes B)(\ket{u} \otimes \ket{v}) = (A\ket{u})\otimes (B\ket{v})$
	\item $I_{n_1} \otimes I_{n_2} = I_{n_1\times n_2}$
	\item $(A\otimes B)^\dagger = A^\dagger \otimes B^\dagger$
	\item in particular $(\ket{u}\otimes \ket{v})^\dagger = \bra{u} \otimes \bra{v}$
	\item if $\lambda$ is a scalar then $\lambda (A \otimes B) = (\lambda A) \otimes B = A \otimes (\lambda B)$
	\item in particular $\lambda (\ket{u} \otimes \ket{v}) = (\lambda \ket{u}) \otimes \ket{v} = \ket{u} \otimes (\lambda \ket{v})$
\end{itemize}
With these properties we can show that the Kronecker product of two unitary matrices will be unitary:
\begin{align*}
	(A \otimes B)(A \otimes B)^\dagger
	&= (A \otimes B)(A^\dagger \otimes B^\dagger)
	\\&= AA^\dagger \otimes BB^\dagger
	\\&= I \otimes I
	\\&= I
\end{align*}

Considering the degrees of freedom involved, we expect that in general there are many $m_1m_2 \times n_1n_2$ matrices that are not the Kronecker product of an $m_1 \times n_1$ matrix with an $m_2 \times n_2$ matrix. Written in block matrix form it is easy to tell whether or not a matrix $C$ is a Kronecker product $A \otimes B$, by checking that every block $B_{ij}$ is a scalar multiple of any non-zero block $B$. The entries of $A$ are simply these scalar multiples, solving $B_{ij} = a_{ij}B$, and hence $C = A \otimes B$. (and if there are no non-zero blocks then $C = 0 = 0 \otimes 0$)
\subsection{The Unitary Group}
So far we have discussed a number of useful algebraic properties exhibited by unitary matrices, but the most basic properties of unitary matrices are of course their forming a group:
\begin{itemize}
	\item Product of unitaries is unitary: ${(AB)}^\dagger = B^\dagger A^\dagger = B^{-1}A^{-1} = {(AB)}^{-1}$
	\item Matrix products are always associative (since composition of any operator will be associative)
	\item The identity matrix is unitary: $I^\dagger = I = I^{-1}$
	\item $A^{-1} = A^\dagger$ always exists and is unitary
\end{itemize}

Quantum computation deals extensively with unitary operators, and so subgroups formed by particular sets of unitary operators will be a frequent point of discussion. A subgroup of a group is simply a subset that is still a group when equipped with the same group operation. Our group operation is matrix multiplication, and so the group of $n \times n$ unitaries written $U(n)$ is actually a subgroup of the much larger general linear group $GL(n, \mathbb{C})$, of $n \times n$ matrices with non-zero determinant.

A further subgroup of the set of unitary matrices $U(n)$ is the symmetric group $\mathcal{S}_n$, which we represent as the set of $n \times n$ permutation matrices. A permutation is an invertible function mapping a set to itself, and given a permutation on the canonical basis $\{\ket{i}\ |\ 0 \leq i < n\} \to \{\ket{i}\ |\ 0 \leq i < n\}$, we can extend this linearly to an invertible matrix $\mathbb{C}^n \to \mathbb{C}^n$ whose columns are all canonical basis vectors. We call such a matrix a permutation matrix, and note that the adjoint of its outer product form $P = \sum_i \ket{\sigma(i)}\bra{i}$ will be $P^{\dagger} = \sum_i \ket{i}\bra{\sigma{i}}$, giving the following:
\begin{align*}
	PP^\dagger = \sum_{i,j} \ket{\sigma(i)}\braket{i}{j}\bra{\sigma(j)} = \sum_i \ket{\sigma(i)}\bra{\sigma(i)} = I
\end{align*}

The symmetric group of any finite set is always finite, and is typically the first group considered when exploring finite subgroups of $U(n)$.
\subsection{Group Quotients, Normal Subgroups, Normalisers}
[perhaps an example of a group and its normalizer, group isomorphisms, and semidirect products. excessive definition stuff can go in axiomata appendix]

Given an equivalence relation $\sim$ on a set $S$ it is often useful to consider equivalence classes, the subsets $[x] = \{y\ |\ x \sim y\}$, since these will be equal exactly when their representatives are equivalent, i.e. $[x] = [y] \iff x \sim y$. The set of all such equivalence classes is called the set quotient, and is written $S/\sim$. It provides a concrete object with the same structural properties that would come from `identifying' $x$ with $y$ whenever $x \sim y$. If $S$ is actually a group $G$ then its group operation sometimes induces an operation in the set quotient $[x][y] = [xy]$, but if this is well defined then we immediately find $[e]$ is a subgroup of $S$:
\[e \sim x \sim y \implies [e] = [e][e] = [x][y] = [xy] \implies e \sim xy\]
\[e \sim x \implies [x^{-1}] = [e][x^{-1}] = [x][x^{-1}] = [xx^{-1}] = [e] \implies e \sim x^{-1}\]
Further if $x \in [e]$ and $z \in S$ then $zxz^{-1} \in [e]$ as well:
\[e \sim x \implies [z][x][z^{-1}] = [z][e][z^{-1}] = [e] \implies e \sim zxz^{-1}\]

In general if a subgroup $H$ of $G$ satisfies this condition, that for any $h \in H$ and $g \in G$, $ghg^{-1} \in H$, then $H$ is said to be a normal subgroup of $G$. It turns out that not only is the equivalence class $[e]$ a normal subgroup, but whenever $H$ is a normal subgroup of $G$, the equivalence relation $x \sim y \iff xy^{-1} \in H$ gives a well defined group operation on the set quotient $G/\sim$. We call this group the group quotient $G/H$. When available, this is a powerful tool for understanding the structure of groups, since the group quotient $G/H$ may have convenient algebraic properties emerging from its corresponding equivalence relation.

As an example of a normal subgroup, take $G \subset GL(n, \mathbb{R})$ to be any group formed by matrix multiplication, and $H$ to be the set of scalars in $G$, that is the set $\{\lambda I\ |\ \lambda \in \mathbb{C}\} \cap G$. Since scalars are commutative, it is straight-forward that $g\lambda I g^{-1} = \lambda gg^{-1} = \lambda I \in H$.

The subgroup relation and the normal subgroup relation can be thought of as a partial order, and are often written as $\leq$ and $\trianglelefteq$ respectively, since both are transitive, reflexive, and anti-symmetric. While these relations are each transitive, we must be careful when mixing them; if $H$ is a normal subgroup of $N$, and $N$ is a subgroup of $G$, $H$ is not necessarily a normal subgroup of $G$, which is surprising when stated more succinctly as ``$H$ is normal in $N$ but is not normal in $G$''.

With this subtlety in mind we can find exactly such a group $N$, given any subgroup $H$ of $G$. We call this the normaliser of $H$ with respect to $G$, defined as the set $N_G(H) = \{g\ |\ g \in G,\ gHg^{-1} \subseteq H\}$. This is the maximal such $N$, any other $N'$ with $H$ normal will sit inside $N_G(H)$. Clearly the group quotient $N_G(H)/H$ will exist for any subgroup $H$ of $G$, which will be useless in the case that $N_G(H) = H$, but otherwise can be an interesting group, and can even inform interesting structure about the normaliser itself, making normalisers a useful and novel tool for exploratory algebraic work.

\section{Quantum Mechanics}

The theory of Quantum mechanics is foundational in contemporary understanding of physical systems, and features phenomena radically different to that of the classical world when the objects under consideration are able to achieve `coherence', e.g.\ when sufficiently small or low in temperature.

Quantum mechanics is deliberately `incomplete' in the sense that it doesn't make specific predictions about the behaviour of atoms, electrons, or anything else, but rather poses constraints and structure that other physical theories such as quantum electrodynamics can adopt.

Quantum computation is the study of how quantum phenomena can be used to effect computation. When designing an individual experiment or programmable quantum computer it is necessary to use a concrete, empirically verified model of physical phenomena, but the specific algorithms that can be run on a quantum computer turn out to be described sufficiently by the general theory of quantum mechanics alone. Since this thesis is concerned with algorithms, we shall summarize the description of quantum mechanics given in the ubiquitous textbook "Quantum Computation and Quantum Information" by Nielsen and Chuang \cite{textbook}. This will provide a foundation for explaining certain techniques and algorithms prominent in quantum computing, and their relationship to the questions we aim to explore.
\subsection{State Space}
The first postulate of quantum mechanics is that any [closed?] quantum system can be modeled by some complex Hilbert space. While many of these Hilbert spaces are of infinite dimension, we can easily create finite dimensional Hilbert spaces by either trapping or ignoring the positions and other such properties of individual particles. Once we have an $n$-dimensional Hilbert space $\mathbb{H}$ modeling a system, we can substitute this for the equivalent Hilbert space $\mathbb{C}^n$, allowing us to model the system directly with this set.

We call the specific elements of $\mathbb{C}^n$ state vectors whenever they represent a state that a quantum computer can take, and we shall see soon that the state vectors are exactly the vectors of unit length in $\mathbb{C}^n$. This allows us to model quantum computation itself as an operation on these unit length state vectors, and to describe the initial, final, and intermediate states of any given computation as an individual state vector.

The simplest possible quantum system is an isolated quantum bit, or qubit, which can be modelled with the 2-dimensional Hilbert space $\mathbb{C}^2$:
\[
\mathbb{C}^2 = \left\{\left[\begin{matrix}
a\\
b
\end{matrix}\right]\ \middle|\ a, b \in \mathbb{C}\right\}
\]
\subsection{Evolution}
The second postulate of quantum mechanics can be stated in three equivalent ways: Given a closed quantum system with initial state vector $\ket{\phi}$ these 3 equivalent equations hold:
\[-i\hbar\frac{d}{dt}\ket{\phi} = H\ket{\phi}\]
\[\ket{\phi} = \exp\left(i \frac{tH}{\hbar}\right)\ket{\phi_0}\]
\[\ket{\phi} = U\ket{\phi_0}\]
Where $\hbar$ is the reduced Planck constant, which is a known scalar constant, $H$ is the Hamiltonian operator, a Hermitian operator associated with the energy present in the system, and $U$ is the unitary operator arising from fixing $t$ before exponentiating $itH/\hbar$.

A quantum computer, then, is simply a device that is able to manipulate the Hamiltonian $H$ of a quantum system, for specific durations of time $t$. When a quantum computer is capable of implementing a particular $H$, $t$ pair, we call the corresponding unitary matrix $U$ an elementary gate of the quantum computer. Composing these elementary gates will always give some overall unitary matrix $U = U_1U_2\dots U_k$, so the core of quantum computation then, is to understand how computationally interesting unitary matrices can be implemented using elementary gates of a quantum computer.
\subsection{Measurement}
Although the quantum systems we are discussing are represented as having a continuous state space of possible state vectors, a fundamental (and titular!) peculiarity of quantum mechanics is that when observed, a quantum object will always appear to be in states that are elements of some orthonormal basis of the corresponding Hilbert space. What is more peculiar, and more well known of quantum mechanics, is that the state that is measured becomes the new state of the system.

It is impossible to perform any computationally useful algorithm without measuring the result at some point in that algorithm, so in any theoretical discussion it is required that at least one procedure is possible for measuring the system. Whichever procedure is used, we call the corresponding orthonormal basis the `computational basis`. When identifying the given Hilbert space $\mathbb{H}$ with the vector space $\mathbb{C}^n$, we make sure to also identify the computational basis of $\mathbb{H}$ with the canonical basis $\{\ket{i}\}$ of $\mathbb{C}^n$, so that measurement can always be done in this basis.

Suppose we have a quantum system with state vector $\ket{\phi}$, and let $\ket{\phi}$ equal the following:
\[\ket{\phi} = \sum_{i=0}^n a_i\ket{i}\]

When measured this system will appear to be in state $\ket{i}$ with probability $\ord{a_i}^2$, and after measurement will be in the state $a_i'\ket{i}$, where $a_i'$ is the normalized complex number $\frac{a_i}{\ord{a_i}}$.

In more algebraic terms this can be stated as follows:
When measured this system will appear to be in state $\ket{i}$ with probability $\ord{\braket{i}{\phi}}^2$, and after measurement will be in the state that comes from applying the projection matrix $\ket{i}\bra{i}$ and normalizing.

Given that the coordinates of a state vector are now interpreted as being related to probabilities, it is important that state vectors have unit length:
\[\norm{\ket{\phi}} = \sum_{i=0}^n \ord{a_i}^2 = 1\]

A defining feature of unitary matrices is that they preserve vector magnitudes, meaning this assumption will be preserved at all points in time in a quantum algorithm.

As an example consider the $\ket{-}$ state:
\[\ket{-}=\frac{1}{\sqrt{2}}\ket{0} - \frac{1}{\sqrt{2}}\ket{1}\]
When measured, this will have probability $1/2$ of appearing to be in the $\ket{0}$ state, after which it will in fact be in the $\ket{0}$ state, and probability $1/2$ of appearing to be in the $\ket{1}$ state, although in actual fact it will be in the $-\ket{1}$ state.

In physical quantum experiments it is often possible to measure in multiple different bases, for example in the $\ket{+}$/$\ket{-}$ basis. While this may have practical advantages, the map between these bases will be unitary anyway so such non-computational measurements can be added or removed as needed by implementing the appropriate unitary instead, and by doing so we can get away with only using the computational basis, making for a more general theory.
\subsection{Composite State Space}
Suppose that we have two quantum systems, which when closed and isolated from each other would be modeled as $\mathbb{C}^m$ and $\mathbb{C}^n$ respectively. When we allow these to interact with each-other, and form a closed composite system, we will need some distinct Hilbert space with which to model them. Intuitively if we measure both systems we should find the first in some computational basis state $\ket{i}$ and the second in some second computational basis state $\ket{j}$, giving a total of $m \times n$ different measurements. The measurement postulate described for individual systems will apply just as well to this coupled system, and so we have to extend these $m \times n$ different pairs of states to the Hilbert space $\mathbb{C}^{m \times n}$, whose computational basis is now given by Kronecker products $\ket{i} \otimes \ket{j}$.

In general if we had of modeled the systems in states $\ket{v_1}$ and $\ket{v_2}$ then we can embed these states in the coupled system as the state $\ket{v_1} \otimes \ket{v_2}$. We shall see in the section on composite measurement that these Kronecker products are postulated to behave in the same way as the individual states when measured. As an example of these embedded states, consider the following pair of qubit states:
\begin{align*}
\ket{1} = \left[\begin{matrix}0\\1\end{matrix}\right]
&&&
\ket{+} = \left[\begin{matrix}\frac{1}{\sqrt{2}}\\\frac{1}{\sqrt{2}}\end{matrix}\right]
\end{align*}
When taking two such qubit objects as a coupled system, the corresponding Kronecker state would look like this:
\[
\ket{1}\otimes \ket{+} = \left[\begin{matrix}0\\1\end{matrix}\right]
\otimes
\left[\begin{matrix}\frac{1}{\sqrt{2}}\\\frac{1}{\sqrt{2}}\end{matrix}\right]
=
\left[\begin{matrix}0\\0\\\frac{1}{\sqrt{2}}\\\frac{1}{\sqrt{2}}\end{matrix}\right]
\]

Note that when taking the Kronecker product of multiple vectors or multiple co-vectors, we can suppress the $\otimes$ operator, simply writing it as a concatenated sequence of kets: $\ket{v_1}\ket{v_2} = \ket{v_1}\otimes\ket{v_2}$. This does not introduce any ambiguity since the matrix product of two vectors is not defined.

In classical computation we represent integers and other data as a sequence of small (typically binary) digits and in the same way we can represent integers as computational basis vectors $\ket{i_1}\ket{i_2}\dots\ket{i_n}$ in some large system made from composing individual `quantum digits', binary or otherwise. In other contexts it is common to take the corresponding sequence of digits as the identifier for a single ket $\ket{i_1i_2\dots i_n}$ but we shall refrain from doing this here, since it poses no significant advantage when dealing with the most foundational aspects of quantum computation.

As was described in our original discussion of Kronecker products, not all quantum states in a composite system $\mathbb{C}^m\otimes\mathbb{C}^n$ are of the form $\ket{u}\otimes\ket{v}$. States that can't be represented as a single Kronecker product $\ket{u}\otimes\ket{v}$ are called entangled states. Entanglement is a very important feature of quantum mechanics and quantum computation that will be discussed on its own later in this document. For now though observe the Bell state:

\[\frac{1}{\sqrt{2}}\ket{0}\otimes\ket{0} + \frac{1}{\sqrt{2}}\ket{1}\otimes\ket{1} = \left[\begin{matrix}
\frac{1}{\sqrt{2}}\\
0\\
0\\
\frac{1}{\sqrt{2}}
\end{matrix}\right]\]
We can see this has two non-zero blocks, $1/\sqrt{2}\ket{0}$ and $1/\sqrt{2}\ket{1}$ respectively, which are not multiples of each-other.

[entanglement as basis independent property?]

Seeing as classical systems aren't treated as having any form of superposition, there is no classical analogy for entangled states. It is worth noting that in classical systems a composite system would be understood to have a state space that is the Cartesian product of two individual state spaces, and that the Cartesian product of an $m$-dimensional vector space with an $n$-dimensional vector space (equipped with point-wise vector addition say, giving the direct sum of two vector spaces) would have dimension $m+n$ rather than the $m\times n$ dimensional space given by composite quantum systems.
\subsection{Composite Evolution}
Just as with individual systems, an isolated composite system evolves according to some unitary operator after any discrete time step. In the composite system $\mathbb{C}^{mn}$ this could be any $mn$ by $mn$ unitary matrix, and once again we call any such operation an elementary gate of a quantum computer whenever it can be reliably produced by that computer. Typically the elementary gates that can be implemented for individual systems will be implemented in the same way in composite systems, again using the Kronecker product to embed the corresponding operation into the composite model:
\[U \otimes I = \sum_i \ket{u_i}\otimes\ket{v_i} \mapsto \sum_i (U\ket{u_i})\otimes \ket{v_i}\]

For example the operation $I \otimes H$ acting on the 2-qubit system $\mathbb{C}^{2\times 2}$, which applies $H$ to the second qubit and leaves the first qubit unchanged, will have the following matrix representation:
\[
\left[\begin{matrix}
	1&0\\
	0&1
\end{matrix}\right]
\otimes
\left[\begin{matrix}
	\frac{1}{\sqrt{2}}&\frac{1}{\sqrt{2}}\\
	\frac{1}{\sqrt{2}}&-\frac{1}{\sqrt{2}}
\end{matrix}\right]
=
\left[\begin{matrix}
	\frac{1}{\sqrt{2}}&\frac{1}{\sqrt{2}}&0&0\\
	\frac{1}{\sqrt{2}}&-\frac{1}{\sqrt{2}}&0&0\\
	0&0&\frac{1}{\sqrt{2}}&\frac{1}{\sqrt{2}}\\
	0&0&\frac{1}{\sqrt{2}}&-\frac{1}{\sqrt{2}}
\end{matrix}\right]
\]

If all of the elementary gates available to a quantum computer can be represented as the Kronecker product of individual gates, and the initial state of the computer can be represented as the Kronecker product of individual states, then the computer will never be able to create entangled states, and will essentially be two distinct quantum computers running in parallel. This means that it is crucial for a quantum computer to have at least one operation that is not the Kronecker product of individual operations, and the most common of these elementary gates is the controlled increment gate, which acts as a permutation on the computational basis $\ket{i}\ket{j} \mapsto \ket{i}\ket{i+j \mod n}$.

The controlled increment operation in binary systems is called the controlled-not operation, which will act mod 2, and therefore simply swap the $\ket{1}\ket{0}$ state with the $\ket{1}\ket{1}$ state. In this case we would call the first qubit the `control' object, and the second qubit the `target' object. This can not be represented as a Kronecker product of two individual operations, which can be seen in the following matrix representation:
\[
\left[\begin{matrix}
1&0&0&0\\
0&1&0&0\\
0&0&0&1\\
0&0&1&0
\end{matrix}\right]
\]
(In fact no controller increment operation will be a Kronecker product of individual operations, since their matrix representation will always have an $I$ block, and an $X$ block, which are not proportional.)

This notion of controlled operations is useful to generalize to any unitary operation acting on a smaller quantum system, which will be discussed in more detail in the context of both binary and ternary quantum systems later.

It is useful to note that the Kronecker product commutes with the outer product:
\[(\ket{u'}\bra{u})\otimes(\ket{v'}\bra{v}) = \ket{u'}\ket{v'}\bra{u}\bra{v}\]
One consequence of this is that as with state vectors, the Kronecker product of canonical basis operators $\ket{i'}\bra{i}$ and $\ket{j'}\bra{j}$ will give all of the canonical basis operators $\ket{i'}\ket{j'}\bra{i}\bra{j}$ on $\mathbb{C}^{m \times n}$.
\subsection{Composite Measurement}
When measuring an individual system we gave the algebraic definition that state $\ket{\phi}$ would be measured in state $\ket{i}$ with probability $\ord{\braket{i}{\phi}}^2$, and would then be in the state that comes from applying the projection $\ket{i}\bra{i}$ and normalizing.

This postulate stated as is would allow us to measure all of the individual components of the composite system at the same time, but it is actually much more powerful to measure one component by itself without otherwise disturbing the rest of the system. To that end suppose we have a composite quantum system whose state vector is the following:

\[\ket{\phi} = \sum_{0 \leq i < m}\sum_{0 \leq j < n} a_{ij}\ket{i}\ket{j}\]

Either the $\mathbb{C}^m$ or the $\mathbb{C}^n$ component of this system could be measured. Without loss of generality we shall describe measurement of the first state. The full statement of the measurement postulate [need to go through and make the postulates more explicit in the structure of this discussion] is that this system will be in state $\ket{i}$ with probability:
\[P(\ket{i}) = \sum_{j=0}^n\ord{a_{ij}}^2\]
After measuring the system will be in the state that comes from applying the projection operator $(\ket{i}\bra{i})\otimes I$ and then normalizing.

For example observe the following 2-qubit state vector $\ket{+}\ket{+}$:
\[\ket{+}\ket{+}=\frac{1}{2}\left(\ket{00}+\ket{01}+\ket{10}+\ket{11}\right)=\frac{1}{2}\left[\begin{matrix}
1\\
1\\
1\\
1
\end{matrix}\right]\]

If we measured the first qubit then we would observe $\ket{0}$ with probability $1/4+1/4 = 1/2$, and similarly $\ket{1}$ with probability $1/2$. Upon measuring $\ket{0}$ the overall state would be $\ket{0}\ket{+}$, whereas upon measuring $\ket{1}$ the overall state would be $\ket{1}\ket{+}$.

While this is straight-forward for states that are simply the Kronecker product of two states, measuring entangled states can be quite interesting. For example suppose we measured the first component of the Bell state $1/\sqrt{2}(\ket{00}+\ket{11})$. As before we would find it in state $\ket{0}$ with probability $1/2$, and $\ket{1}$ with probability $1/2$, but after measurement it would be in states $\ket{00}$ and $\ket{11}$ respectively. Measuring the first component of the system changed the second one! This is the basis of a number of novel features of quantum computing, quantum information theory, and quantum cryptography. [chuck in quantum teleportation]

[note somewhere that more than two systems continue naturally from these subsections]

\section{Quantum Computation}

With quantum mechanics and the associated concept of a quantum algorithm in place, we can begin to reason about the techniques of quantum computation. To this end we shall briefly discuss the computer science of probabilistic vs deterministic algorithms, as well as the group theory of different sets of unitary matrices, important concepts that lay the foundation of technical discussions in the field of quantum computation. With this in place we shall be prepared to discuss the state of the field when it comes to quantum computation on objects with more than two basis states.

\subsection{Probabilistic Algorithms}
In computer science there is a distinction between deterministic algorithms, and randomized/probabilistic/non-deterministic algorithms. In summary a deterministic algorithm is a sequence of exact steps that can be executed in order to compute a result, whereas a probabilistic algorithm is permitted to rely on some source of random information to determine its control flow, meaning that the same input could result in many different outputs.

The advantage of probabilistic algorithms is that they can often avoid the worst-case performance associated with certain input states in deterministic algorithms; for example, many implementations of sorting algorithms will take much longer than usual to sort a list in ascending order if it is initially in descending order. At an intuitive level such worst-case input states tend to exploit the specific order in which an algorithm explores its possible solutions, and so since a randomized algorithm has no single order in which it might explore solutions, such worst-case inputs do not exist.

On classical computers deterministic algorithms are often more natural, and even when a randomized algorithm is used it will likely use a pseudo-random number generator in place of a true random source of information. This is heavily contrasted with quantum computation and quantum physics more generally, where measuring the state of a quantum object is inherently probabilistic, and physical implementations of quantum algorithms often introduce physical sources of error as well, as a result the probabilistic quantum algorithm is taken as the norm, with the exception of algorithms that only measure computational basis states, which when simulated theoretically will act deterministically. In physical quantum computers to date noise has been of primary concern in the real world execution of algorithms, and so even these algorithms that are deterministic when simulated end up being non-deterministic in practice.

One important consequence of this is that a quantum algorithm that appears to perform well might still perform better on a classical computer with a source of randomness; when comparing quantum with classical, we must consider the probabilistic algorithms of both.

A further distinction in computer science is made between Las Vegas and Monte Carlo algorithms, the former being algorithms that always yield a result, but have random run-time, and the latter being algorithms that yield some result in fixed run-time, but have a random chance of failing or producing a result that is incorrect. Often an algorithm in one of these classes can be converted into the other, Las Vegas algorithms that repeatedly search for a solution can be modified to yield a false negative after a fixed number of attempts, and Monte Carlo algorithms that may produce a false negative can check their solution using a deterministic algorithm, and retry until a valid solution is found. This means that in classical contexts this distinction is less important than that of deterministic vs probabilistic algorithms, since both classes can achieve similar things with the same resources.

In quantum contexts however, algorithms are generally assumed to be Monte Carlo algorithms, since the sources of error discussed are sources of incorrect output, rather than sources of increased run-time. Further when a physical quantum computer is run, it is run repeatedly, often thousands of times, in order to determine the probability of each output, so that one can infer which output is the correct one.

When constructing unitary matrices out of some elementary gate set, it becomes possible to use approximate constructions, since all quantum algorithms are Monte Carlo by default, and small changes in a state's coordinates tend not to significantly affect the overall probability distribution of the algorithm's outputs. Often quantum algorithms and their associated unitary matrices are implemented asymptotically, where the desired accuracy of the algorithm is taken as a parameter, and as this parameter gets smaller a longer sequence of elementary gates is constructed to achieve this level of accuracy.

\subsection{Classical Computation in Quantum Algorithms}

In classical computation we can imagine an algorithm or circuit mapping some $M$ discrete states to some $N$ discrete states, according to some function $f: \{0\dots M-1\} \to \{0\dots N-1\}$. (say $M = 2^m,\ N = 2^n$ where $m$ and $n$ are the number of physical wires leading in and out of the circuit) For example we could define the logical conjunction or AND map which maps pairs of bits $\{00, 01, 10, 11\}$ to single bits $\{0, 1\}$: 
\begin{align*}
\text{AND}(x) = \begin{cases}
1 & \text{if\ } x = 11\\
0 & \text{otherwise}
\end{cases}
\end{align*}

Given such a map $f$ we can define a linear operator by extending linearly on the computational basis: $A_f\ket{i} = \ket{f(i)}$, giving a matrix whose columns are all computational basis vectors. For example our logical conjunction becomes:

\[
A_\text{AND} = \left[\begin{matrix}
1&1&1&0\\
0&0&0&1
\end{matrix}\right]
\]

In quantum computation we require that all operations be reversible, unitary operations. This means that a matrix $A_f$ representing a classical computation $f$ will be available as a unitary operator if and only if $M=N$ and $f$ is invertible, i.e.\ if $f$ is a permutation. When $M = N = 2$ we only have two such permutation matrices:
\begin{align*}
I = \left[\begin{matrix}
1&0\\
0&1
\end{matrix}\right]
&&&
X = \left[\begin{matrix}
0&1\\
1&0
\end{matrix}\right]
\end{align*}

Maps that don't satisfy $M = N$ or $f$ invertible can still be represented as a permutation matrix, acting on $\mathbb{C}^{M\times N}$ as follows:
\[B_f(\ket{i}\otimes\ket{j}) = \ket{i}\otimes\ket{j+f(i) \mod N}\]

The inverse of this matrix is simply $\ket{i}\ket{j} \mapsto \ket{i}\ket{j-f(i) \mod N}$.

In the case of the binary AND map, $B_{\text{AND}}$ will be a $8\times8$ permutation matrix known as the Toffoli gate. It can be shown that compositions of Toffoli gates acting on some number of bits can be used to implement any $n$-qubit permutation matrix as an algorithm on some larger number of qubits $n+m$, so long as the extra $m$ qubits are initialized to known values.

Permutation matrices by themselves might not seem interesting, since they exclusively represent calculations that can be done in classical contexts, but in fact are crucial for many quantum algorithms, since they will act linearly on superposition states. For example if we consider the map $f(x) = x^2\mod 16$, then the constructed matrix $B_f$ acting on 8 qubits will of course distribute linearly over any linear combination of basis states including the following:
\begin{align*}
B_f(\ket{2}\ket{0}+\ket{5}\ket{0}) 
= B_f\ket{2}\ket{0} +& B_f\ket{5}\ket{0}
\\= \ket{2}\ket{4} +& \ket{5}\ket{9}
\end{align*}

Further manipulations or measurement of the result of such a transformation can enable many powerful quantum algorithms including Shor's period finding algorithm. This means that permutation matrices are an important topic in quantum computation, and a good deal of research has been and continues to be done to better understand how permutation matrices can be decomposed into efficient quantum algorithms.

\subsection{Phase and Amplitude}
Since we assume a very specific space $\mathbb{C}^n$ with a canonical/computational basis available, we can and often do talk directly about the coordinates of vectors in this space. We call each complex coordinate an \emph{amplitude}, and importantly in a composite space the amplitudes of a Kronecker product are the coordinates \emph{after} expanding the product not before. For example the Kronecker product $\ket{+} \otimes \ket{-}$ as a vector looks like the following:
\[\ket{+} \otimes \ket{-} = \left[\begin{matrix}
\frac{1}{2} \\
-\frac{1}{2} \\
\frac{1}{2} \\
-\frac{1}{2} \\
\end{matrix}\right]\]
This would have amplitudes $1/2$, $-1/2$, $1/2$, and $-1/2$ in the computational basis.

When comparing different states or coordinates within a state we often find that their coordinates are equal in magnitude, but have different complex argument. In such cases we can talk about the phase differences between the states or the coordinates of each state, to understand the state better. If two states are multiples of each other, specifically a unit complex multiple, e.g.\ $\ket{\psi} = e^{i\theta}\ket{\phi}$, then we say that they differ by a global phase factor $e^{i\theta}$, and if two amplitudes of states are unit complex multiples of each other, e.g. $\ket{\psi} = \ket{0}/\sqrt{2} + e^{i\theta}\ket{1}/\sqrt{2}$ vs. $\ket{\phi} = \ket{0}/\sqrt{2} + \ket{1}/\sqrt{2}$ then we say their $\ket{1}$ amplitudes differ by a relative phase factor $e^{i\theta}$.

Interestingly, global phase does not have any physical meaning, and exists only within the model, since the only predictions made by the model are the probabilities of each measurement, and global phase differences cannot affect the magnitude of any amplitude in either the state in question or any resultant states after performing computation. For example if we have two states $\ket{+}$ and $-\ket{+}$ and apply the Hadamard matrix $H$ to each we would get $\ket{0}$ and $-\ket{0}$ respectively, both before and after they have identical behaviour when measured.

Relative phase differences on the other hand, while not directly affecting measurement, can affect the course of calculation, producing states that are very different when measured. For example $\ket{+}$ and $\ket{-}$ differ only by relative phase $-1$ in the $\ket{1}$ amplitude, but after applying $H$ we get $\ket{0}$ and $\ket{1}$ respectively, different computational basis vectors, which we have assumed to be distinct when measured.

Amplitude and relative phase differences can be generalized to arbitrary basis if one needs, whereas global phase differences are basis independent. Global phase differences are much more relevant to us than relative phase anyway, so we won't need any generalized notion of amplitude or relative phase. [in fact I don't think we use the basic version either!]

By ignoring global phase differences, we can remove a degree of freedom from the state space being discussed, but we can also remove a degree of freedom from any sets of unitary matrices being explored, since any scalar multiple of the identity will have no meaningful effect on computation, so for example $X$ and $-X$ can be considered equivalent:
\[-X = -\left[\begin{matrix}0 & 1 \\ 1 & 0\end{matrix}\right]= \left[\begin{matrix}0 & -1 \\ -1 & 0\end{matrix}\right]\]
Given a subgroup of the unitary matrices $G \leq U(n)$ we can make this equivalence explicit, by identifying $U(1)$ with scalar multiples of the identity in $U(n)$, and writing $G/U(1)$, which is understood to be the group quotient $G/(G \cap U(1))$, where $G \cap U(1)$ must be a normal subgroup of $G$ since it consists only of scalars. This lets us state formally that in the above example $X$ and $-X$ would belong to the same coset in $U(n)/U(1)$.
\subsection{Pauli Matrices}
A useful family of matrices are the Pauli matrices, which in the $2\times2$ qubit case are the $X$ and $Z$ matrices discussed in previous examples:
\begin{align*}
X = \left[\begin{matrix}
0&1\\
1&0
\end{matrix}\right]
&&&
Z = \left[\begin{matrix}
1&0\\
0&-1
\end{matrix}\right]
\end{align*}

These can be generalized to $n\times n$ matrices with the following effect on the computational basis:
\begin{align*}
X_n\ket{i} = \ket{i+1\mod n}
&&&
Z_n\ket{i} = \omega_n^i\ket{i}
\end{align*}

Powers of $X_n$ form a cyclic group of permutation matrices of order $n$. Visualizing the computational basis vectors in a circle, powers of $X_n$ rotate the circle, whereas powers of $Z_n$ resemble harmonics on that circle.

Observe that these matrices do not commute, but that $Z_nX_n = \omega_nX_nZ_n$:
\begin{align*}
	Z_nX_n\ket{i}
	=& Z_n\ket{i+1\mod n}
	\\=& \omega_n^{i+1}\ket{i+1\mod n}
	\\=& \omega_n^{i+1}X_n\ket{i}
	\\=& \omega_nX_nZ_n\ket{i}
\end{align*}

From this it can be seen that all elements of the Pauli group will be of the form $\omega_n^iX_n^jZ_n^k$, giving the Pauli group an order of $n^3$.

This group is an interesting finite group in quantum computation, first brought to attention since $X_2$ and $Z_2$ along with $Y  = iX_2Z_2$ have physical meaning in quantum mechanical description of electron spin. Further, these 3 matrices each transform the state space of a qubit in a very geometrically convenient way, each along a different pair of axes in the Bloch sphere. [either describe the Bloch sphere specifically for this statement, or make this statement more vague] The Pauli group of generalized Pauli matrices provide similar algebraic power, being generated by 2 gates that might also have physical meaning, while still providing some level of transformation of quantum states along all axes.
[this is a huge part of my intuition for why we start with the Pauli group, but this explanation is a mess. I don't want to get into the full details of how unitary operators form a sphere in Mat(n, C), although I could, if I brought back the ``displacement operators as basis for general operators'', I think $U(n)$ is literally the unit sphere in this basis? It's a big aside for an intuition that isn't directly used here, since what we tend to talk about instead is the power of $H_2$ and $H_3$.]

The convenience of the $Y_2$ matrix along with the identity $D_{x, z}^k = D_{kx,kz}$ motivate the definition of the displacement operators as seen in \cite{pi-over-eight}:
\[D_{x, z} = \omega^{xz}_{2n}X^xZ^z\]
These displacement operators are equivalent to the generalized Pauli matrices up to global phase, but generate a group of twice the size called the Weyl-Heisenberg group:
\[H(n) = \{\omega_{2n}^iX^jZ^k\}\]

The normaliser of the Weyl-Heisenberg group is called the Clifford group, and must be at least as powerful as the Pauli group, while also providing, among others, the discrete Fourier transform:
\[H_n = \frac{1}{\sqrt{n}}\sum_{i,j} \omega_n^{ij}\ket{i}\bra{j}\]
The Clifford group contains arbitrary scale factors, making it uncountably infinite, but when removing these scale factors it turns out to be another powerful finite group, making its generator a very useful place to start when considering possible elementary gate sets for a quantum computer.

[describe the SWAP operation, which is always Clifford]


\section{Notation for Mixed Systems}
[move to preliminaries and make sure that I am actually using this notation]

A very ergonomic notation for gates in quantum algorithms is to subscript the gate with relevant information about that gate such as the dimension of the object or system on which it acts, the index of the individual object or part of the composite system on which it acts, or the computational basis states that it affects.

Many discussions of quantum algorithms vary only one property of a given quantum gate at a time, making this notation unambiguous. Unfortunately we must have gates acting on distinct objects of a composite system, each with a dimension that could be different, which requires us to take extra care. Lest we lose the ergonomics of putting all the necessary information in small and descriptive subscripts, we continue to place variables in these subscripts, but avoid numerals unless the meaning is clear. We will now enumerate in advance the exact index sets that could be used for a variable in a subscript, a preferred letter in the alphabet for representing such a variable, and what that subscript means when this variable is used.

In composite quantum systems we define $N$ to be the number of computational basis states, and either $n$ to be the number of individual objects in the composite system, or $n$ to be the number of qubits and $m$ to be the number of qutrits. For example we could use these variables to index the Quantum Fourier Transform $\text{QFT}_N$ or a general unitary matrix $U_N$ acting on $N$ computational basis states, or $H^{\otimes n}$ for the Kronecker product of the $H$ matrix with itself $n$ times.

In quantum systems of a single finite-dimensional object we instead call the dimension of this object $n$, and index each of the computational basis states $0$ through to $n-1$, represented by variables such as $i$, $j$, and $k$. For example we will frequently distinguish between $X_2$, $X_3$, and generally $X_n$ in the context of a single quantum object. This is the one case where we will use both numerals and variables for subscripts, since we very often want to say specific things about the $n=2$ and $n=3$ cases.

Note that we change the meaning of $n$ based on context, for the sole purpose of reminding ourselves of the cyclic group formed by the set $\mathbb{Z}_n = \mathbb{Z}/n\mathbb{Z}$ equipped with addition, a very familiar object which appears repeatedly in finite groups of quantum gates acting on single quantum objects.
[do we actually talk about single objects outside of the preliminary discussion of Pauli matrices? If the previous subsection counts then we should move this even earlier!]

When we want to talk about an individual object in a composite system, we index each object $1$ through to $n$, (or $1$ through to $n+m$) and use variables such as $p$, $q$, and $r$ to represent such an index. We then define $d_p$ to be the dimension of the object indexed by $p$. This allows us to extend some unitary matrix $G_d$ acting on a single quantum object of dimension $d$ to a unitary matrix $G_p$ acting on some composite system satisfying $d_p = d$. Written as a Kronecker product, $G_p$ looks like the following:
\[G_p = I_{d_1}\otimes \dots \otimes I_{d_{p-1}} \otimes G_d \otimes I_{d_{p+1}} \otimes \dots \otimes I_{d_n}\]

When indexing computational basis states in a composite system, we use the integers $0$ through to $N-1$ much like in single systems, [unless I removed that paragraph/notation] and variables such as $i$, $j$, and $k$ to represent such indeces. This is particularly useful for the special class of permutation matrices that represent a transposition, where we can write $S_{i,j}$ to represent the matrix that exchanges the computational basis states $\ket{i}$ and $\ket{j}$, while leaving all others unchanged.

When we introduce a variable such as $i$, satisfying $0 \leq i \leq N-1$, we assume implicitly that the sequence of digits $i_1i_2\dots i_p \dots i_n$ exists and that each variable $i_p$ is understood to refer to the corresponding digit in this sequence. We also identify all such integers $0\dots N-1$ with their corresponding sequences of digits whenever we discuss the computational basis states.

When we have a unitary matrix $U$ acting on a composite system with $N$ computational basis states, and a set $c \subseteq \{0\dots N-1\}$, our final subscript notation will be to define the controlled operation $C_c(U)$ which has the following action on the computational basis:
\[C_c(U)\ket{i} = \begin{cases}
	U\ket{i} & \text{if\ }i \in c \\
	\ket{i} & \text{if\ }i \notin c
\end{cases}\]
Note that in order for $C_c(U)$ to be unitary, it is sufficient that the image $U\ket{i}$ of any computational basis vector $\ket{i}$ satisfying $i \in c$ be a linear combination of basis vectors $\ket{j}$ themselves satisfying $j \in c$. For example if we have a gate $G_p$ acting only on object $p$ of the system, then $C_c(G_p)$ will be unitary whenever $c$ is a set whose defining predicate $i_1\dots i_n \in c$ does not depend on the digit $i_p$.

When we have a condition 

Finally when $G_d$ is a unitary matrix acting on a $d$-dimensional object, and $G_p$ is the corresponding matrix acting on object $p$ of a composite system, it is sometimes useful for the set $c$ to impose a constraint on all but $d$ of the computational basis states of the composite system, in which case we can write $C_c(G_d)$ instead of $C_c(G_p)$.
% chap2.tex (Chapter 2 of the thesis)
\chapter[TERNARY ARITHMETIC]{Ternary Arithmetic}
The QArC Group at Microsoft have written a number of papers describing logic in ternary quantum computers, including a full description of integer addition and Shor's factoring algorithm in ternary quantum computers \cite{arithmetics, shor-qutrit}, paying particular mind to a fault tolerant/elementary gate set called the metaplectic basis, described in \cite{universal-anyon, topological-anyon-thing}. While the algorithms for addition are deliberately designed to be generic to any ternary quantum computer, the specific analysis of elementary gates required is specific to the class of quantum computer being considered, so it is unclear whether these analyses pose any relevance to our discussion of mixed logic. We shall describe the algorithms given for addition, and demonstrate how they can be ported to a mixed qubit-qutrit quantum computer without any modification. This will inform our exploration of theoretical quantum gates that might be powerful to have available in such mixed contexts. We also discuss algebraic techniques used in \cite{arithmetics} that fail to generalize to mixed contexts, suggesting some novel complexity present in mixed contexts that does not exist in a purely binary or purely ternary context.

\section{Ternary Addition With Minimal Width}
\ [also what part of it was `improved'? I think leveraging the binary nature of some of the variables was actually the main advantage!]

The first algorithm laid out in \cite{arithmetics} was a reversible algorithm for adding the ternary expansion of two integers, a simple generalization of an operation fundamental to computation in integers. In order to implement addition in some base $d$ one must implement the sum of three digits $a_i + b_i + c_i$ to get a number between $0$ and $3d-3$, a number which is itself represented as two separate digits,
\[c_{i+1} = \left\lfloor \frac{a_i + b_i + c_i}{d} \right\rfloor, \text{\ and\ }
\]\[s_i = a_i + b_i + c_i \mod d.\]
$s_i$ is easily implemented as a pair of SUM operations, which is significantly simpler than $c_{i+1}$ which is an almost arbitrary map $\{0,\dots,d-1\}^3 \to \{0,\dots,d-1\}$. A significant simplification made in \cite{arithmetics} is to note that when $d=3$, the only way $c_{i+1}$ can be 2 is if all $a_i$, $b_i$, and $c_i$ before it are also 2. Since $c_0$ is assumed to be 0 this means $c_i$ is always either 0 or 1, meaning we only need to implement a map from $\{0,1,2\}^2\times\{0,1\} \to \{0, 1\}$. Then with careful analysis of the required map they presented the following circuit:

\begin{quantikz}
	\lstick{$c_i$} & \qw & \qw & \octrl{2} & \octrl{2} & \gate{S_{0,1}} & \swap{1} & \qw & \qw \rstick ? \\
	\lstick{$a_i$} & \gate[wires=2]{S_{00,22}} & \octrl{1} & \qw & \qw & \qw & \targX{} & \swap{1} & \qw \rstick ?\\
	\lstick{$b_i$} & \qw & \targ{} & \targ{} & \targ{} & \phase{0} \vqw{-2} & \qw & \targX{} & \qw \rstick{$c_{i+1}$}\\
\end{quantikz}

This carry circuit is then used to add two $m$-qutrit integers by composing it $m$ times, calculating $c_m$, which is written into a separate output register $s_m$. Then since these carry operations are reversible, they are reversed one by one, overwriting $b_i$ with each $s_i$ in the process. Note that $c_{i+1}$ is swapped with the $b_i$ register once it is calculated, making presentation cleaner later on, and also simulating the fact that qutrits in a physical quantum computer may not be able to interact with each other arbitrarily, they might only be able to interact with adjacent qutrits in some physical layout. They do all of this while only requiring two SWAP operations, which considered significantly cheaper in fault tolerant contexts since they are Clifford operations, unlike the $S_{00,22}$ or $C(S_{0,1})$ operations.

We note that the core of this improvement was to treat $c_i$ like a qubit! This suggests that the whole algorithm for ternary arithmetic will map very cleanly into a mixed quantum computer. If we modify the carry circuit so that $c_i$ actually is a qubit, we get the following circuit:

\begin{quantikz}
	\lstick{$c_i$} & \qw & \qw & \phase{1} \vqw{2} & \targ{} & \qw \rstick{$c_{i+1}$} \\
	\lstick{$a_i$} & \gate[wires=2]{S_{00,22}} & \octrl{1} & \qw & \qw & \qw \rstick ?\\
	\lstick{$b_i$} & \qw & \targ{} & \gate{X_3^{-1}} & \phase{0} \vqw{-2} & \qw \rstick ?\\
\end{quantikz}

The pair of Clifford SUM operations become a single non-Clifford\footnote{One can compute $C(X)X^jZ^kC(X^{-1})$ as we have shown in \ref{}, but we will see in \ref{} that \emph{no} operation that acts dependently on a qubit-qutrit pair can be Clifford.} controlled decrement, the inverse of a controlled increment. Further the controlled transposition simply becomes a controlled increment mod 2. Finally we cannot swap $c_{i+1}$ with $b_i$ since one is a qubit and the other is a qutrit. We see very clearly that this algorithm would be very sensitive to connectivity between qubits and qutrits if implemented in a mixed system. We have turned a carry operation with two non-Clifford operations into one with three non-Clifford operations. In addition to this, the calculation $s_i = a_i + b_i + c_i \mod 3$ becomes

\begin{quantikz}
	\lstick{$c_i$} & \qw & \phase{1} \vqw{2} & \qw \rstick{$c_{i}$} \\
	\lstick{$a_i$} & \octrl{1} & \qw & \qw \rstick{$a_i$}\\
	\lstick{$b_i$} & \targ{} & \targ{} & \qw \rstick{$s_i$}\\
\end{quantikz}

This means that in total to add two $m$-qutrit integers we have gone from $2m$ non-Clifford operations to $4m$ non-Clifford operations, which may be a significant loss, but also isn't strictly meaningful at this point in time, since this metric is specific to the class of quantum computer described in \cite{topological-anyon-thing}, whereas we do not have a proposed fault tolerant model of mixed quantum computation with which to judge this outcome. We have used two qubit-qutrit gates, $C_2(X_3)$ [didn't end up defining $C_d(U)$... do that!] and $C_3(X_2)$, gates that we will end up investigating directly in \ref{finitely-generated}.

\section{Ternary Addition With Minimal Depth}
The second algorithm presented by \cite{arithmetics} aims to parallelise the calculation of the carry trits by observing three cases of $a_i + b_i + c_i$:
\begin{itemize}
	\item $a_i = b_i = 0 \implies c_{i+1} = 0$, denoted $C[i, i+1] = 0$
	\item $a_i + b_i = 1 \implies c_{i+1} = c_i$, denoted $C[i, i+1] = 2$
	\item $a_i + b_i \geq 2 \implies c_{i+1} = 1$, denoted $C[i, i+1] = 1$
\end{itemize}
These three cases can be calculated up front and stored in trit registers, without knowing any carry digits. This allows the calculations to be done in parallel requiring $m$ Clifford gates, but only taking the amount of time required to compute one or two of these calculations in sequence.

Next a merge algorithm is defined, if $C[i, j] = C[j, k] = 2$ then $c_i = c_j = c_k$, so $C[i, k] = 2$ also. In fact whenever $C[j, k] = 2$ we will have $C[i, k] = C[i, j]$. On the other hand, if $C[j, k] \neq 2$ then $c_k$ will be equal to $C[j, k]$ itself, either 0 or 1, and in either case $C[i, k]$ is simply $C[j, k]$. This calculation of $C[i,k]$ in terms of $C[i, j]$ and $C[j, k]$ is another classical operation, which can be implemented as a permutation. Now different combinations of $C[i, k]$ are merged in $\left\lfloor \log l \right\rfloor + \left\lfloor \log \frac{n}{3} \right\rceil + 2$ parallel `layers', requiring $3l - 2\omega(l) - 2\left\lfloor \log l \right\rfloor - 1$ non-Clifford operations across all of these layers, where $\omega(l)$ is the number of 1s in the binary expansion of $l$. The algorithm far from minimizes width however, taking $l - \omega(l) - \left\lfloor \log l \right\rfloor$ additional qutrits on top of the $2l$ needed for the input of the algorithm.

Here a key feature of the algorithm is to observe that $C[i, j]$ is intrinsically ternary, but interestingly the addition of two binary integers would have the same three carry cases, so if we can implement the base case $C[i, i+1]$ as yielding a qutrit based on two qubits, then we would have turned another ternary arithmetic algorithm into a mixed algorithm, but this time using qutrit carry information to add a series of qubits! Let $a_i, b_i \in \{0, 1\}$, then $c_{i+1}$ will be 1 if $a_i = b_i = 1$, 0 if $a_i = b_i = 0$, and $c_{i}$ if $a_i \neq b_i$. This can be implemented as follows

\begin{quantikz}
\lstick{0} & \targ{} & \targ{} & \gate{S_{1,2}} & \qw \rstick{$C[i,i+1]$} \\
\lstick{$a_i$} & \phase{1} \vqw{-1} & \qw & \qw & \qw \rstick{$a_i$}\\
\lstick{$b_i$} & \qw & \phase{1} \vqw{-2} & \qw & \qw \rstick{$s_i$}\\
\end{quantikz}

Here $S_{1,2}$ is a potentially cheap Clifford operation that allows us to calculate $C[i, i+1]$ in terms of $a_i + b_i$. This is much like the $S_{0,1}$ used in the original $C[i, i+1]$ circuit in \cite{arithmetics}. Much like the SWAP operations in their previous carry gates, Clifford operations are being used liberally to turn quantities that are easy to calculate into quantities that are more convenient to design algorithms for. The next step of the algorithm is to merge the different qutrit quantities $C[i, k]$, which is unchanged since there are no qubits involved.

Now in order to calculate the resulting qubits we must obviously diverge from \cite{arithmetics} once more. A key component of their algorithm was to calculate $C[0, 1]$ separately, since $c_0 = 0$ we can replace the $C[0, 1] = 2$ case with $C[0, 1] = 0$, which is another binary quantity. $C[0, 1]$ is 1 if both $a_0$ and $b_0$ are 1, which is implemented with a single Toffoli gate $C_{a_0 = 1, b_0 = 1}(X_{C[0, 1]})$. Now since we have eliminated the $C[0, 1] = 2$ case, what we are really calculating is $c_1$. Next $\cite{arithmetics}$ would have used the merging formula to calculate each $c_k$ implicitly by merging $C[0, j] = c_j$ with $C[j, k]$. Since we want $c_k$ to be a qubit we will replace these specific merges with our own circuit:

\begin{quantikz}
\lstick{$c_j$} & \qw & \phase{1} \vqw{1} & \qw \rstick{$c_j$} \\
\lstick{$C[j, k]$} & \phase{1} \vqw{1} & \phase{2} \vqw{1} & \qw \rstick{$C[j, k]$} \\
\lstick{$0$} & \targ{} & \targ{} & \qw \rstick{$c_k$} \\
\end{quantikz}

Now with all $c_k$ calculated we can of course calculate $s_k = a_i + b_i + c_i \mod 2$ with two SUM operations, i.e.\ two $C(X_2)$ operations. The circuits for calculating $c_k$ used gates with two control objects, so the performance of this part of the algorithm would depend primarily on how these operations are implemented in the given mixed quantum computer. We have now generalized both algorithms from being purely qutrit based to using a mixture of qubits and qutrits, and in doing so have repeatedly encountered the controlled increment operations $C_2(X_3)$ and $C_3(X_2)$.

\section{Speculation on Mixed Coding Schemes}
At this point we make the small aside that while it makes sense to use physical qubits to encode logical qubits, the primary advantage of doing this is that it allows one to avoid thinking about qutrits altogether. As soon as we start considering a mixture of qubits and qutrits we could develop a coding scheme that encodes these objects into a system that is itself mixed, but we should plausibly be able to encode them into a system that is itself a pure ternary system as well, or even a pure binary system. For example both of the algorithms given in \cite{arithmetics} used many binary registers as is, by simply not using the $\ket{2}$ state of the relevant qutrits. Taking the simplified $\{\ket{0}\ket{0}\ket{0}, \ket{1}\ket{1}\ket{1}\}$ example from \ref{}, and generalising it to $\{\ket{0}\ket{0}\ket{0}, \ket{1}\ket{1}\ket{1}, \ket{2}\ket{2}\ket{2}\}$ in order to implement a logical qutrit, we could then implement a logical qubit on the same computer by simply treating all $\ket{2}$ states in that block as errors that need to be corrected.

In actual fact coding schemes can do quite a bit better than detecting bit flip errors with $3\times$ redundancy, but simultaneously detect bit flip and sign flip errors with $7\times$ redundancy with an encoding scheme called the Steane code \cite{steane-code}, which raises the question of what a qubit-in-qutrit Steane code might look like. Further, if a logical qubit and a logical qutrit are both implemented in such a manner on the same ternary system, then another natural question would be what a fault tolerant implementation of $C_2(X_3)$ and $C_3(X_2)$ would look like. All of these arguments apply equally to encoding qutrits in a qubit system, which might be the more compelling option, since the major quantum computers to date have been based on physical qubits. [do I need to cite this? it's kind of the premise of this thesis that we currently mainly understand qubits... It shouldn't be hard to cite though, I'll see what I can find.]

A standard milestone for any proposed quantum computer is to show that a set of fault tolerant/elementary gates [do I need the term elementary if I keep saying fault tolerant?] can be used to approximate any unitary operation $U \in U(N)$. In \ref{universal-computation} we will generalize the approach laid out in \cite{universal-qubit} to show that two different sets of seven [check that it is still seven!] gates can be used to make such an approximation. Both of these sets contain $C_3(X_2)$, $C_2(X_3)$, and one of them also contains $C_2(X_3)$, providing further theoretical motivation for fault tolerant implementation of these gates, in addition to their use in out generalisation of the algorithms in \cite{arithmetics}.

\subsection{Supermetaplectic Basis}
[do I leave this here right after our big discussion of the same paper? I think it really belongs after discussion of \cite{pi-over-eight}, but maybe we could move that here?]

In the above discussion and related papers the QArC group have paid particular attention to a ``Metaplectic'' system of computation which is fault tolerant, and implements an elementary gate set called the ``metaplectic basis'' that can generate all Clifford operations, along with the more expensive implementation of a single non-Clifford operation
\[Y = \begin{bmatrix}
1 & 0 & 0 & \\
0 & 1 & 0 & \\
0 & 0 & -1 & \\
\end{bmatrix}.\]

In \cite{arithmetics} an alternative fault tolerant gate set was proposed, using $P_9$ in place of $Y$, the principal gate satisfying $P_9^3 = Z_3$:
\[P_9 = \begin{pmatrix}
1 & 0 & 0 \\
0 & \omega_9 & 0 \\
0 & 0 & \omega_9^2 \\
\end{pmatrix}\]
Both gate sets can be used to approximate any unitary $U \in U(3^n)$, but \cite{arithmetics} showed that this latter gate set can exactly implement any permutation matrix. They refer to this latter gate set as the ``supermetaplectic basis'', which seems to refer to this increase in power when dealing with permutation matrices.

In order to show that this generator set exactly implements the permutation matrices in $U\left(3^n\right)$, i.e. all of $\mathcal{S}_{N}$, they decompose these matrices in 3 steps.
\begin{enumerate}
	\item It seems to have been taken implicitly that with an implementation of $C_{i_1 = 2}(X)$, other permutations acting on more than 2 qutrits can be decomposed in a manner similar to the qubit process described previously.
	\item Various permutations acting on 2 or 3 qutrits that are not in $\mathcal{C}_2$ were shown to be pairwise equivalent, including arbitrary transpositions of pairs of states.
	\item Two different diagonal matrices were constructed using $P_9$, whose Fourier transforms were qutrit permutations including $C_{i_1=2}(X)$ itself, meaning the permutations themselves are in this basis.
\end{enumerate}

Both of these explicit steps were derived through analysis of the permutation and diagonal matrices as polynomial expressions, first interpreting permutation matrices as acting on the computational basis, e.g.
\[C_{i_1=0}(X)(\ket{i}\ket{j}) = \ket{i}\ket{j+1-i^2 \mod 3}\]
[using $i,j$ instead of $p,q$... but I like $i,j$ here] then interpreting diagonal matrices as acting on the phase factors of the computational basis, e.g.
\[C_{i_1=0}(Z)(\ket{i}\ket{j}) = \omega_{3}^{j(1-i^2)}\ket{i}\ket{j}\]
This allows us to compose permutation matrices and diagonal matrices respectively, and simplify the resulting polynomials mod 3. This technique is clearly powerful, given its success in proving the above claims in \cite{arithmetics}, but unfortunately when we try to do this in a mixed system, we get terms evaluated mod 3, nested inside terms evaluated mod 2, and vice versa. For example in a system with a qubit and a qutrit we could write
\[X_3\ket{i}\ket{j} = \ket{i}\ket{j+1\mod 3}.\]
Further, by using a control value of 1 we can leverage that $2 \mod 2 = 0$ and write
\[C_{j=1}(X_2)\ket{i}\ket{j} = \ket{i+j \mod 2}\ket{j}.\]
Composing these we get
\[C_{j=1}(X_2)X_3\ket{i}\ket{j} = \ket{i+(j+1 \mod 3)\mod 2}\ket{j + 1 \mod 3},\]
which does not simplify, defeating the purpose of this analysis. 

Another approach might be to inject our bits and trits into $\mathbb{Z}_6$, which may work with the correct construction, but if we respect the structure used in \cite{arithmetics} and represent $X_d$ as mapping $i \mapsto i+1 \mod 6$, then we need to identify $i$ with $i+d$ in order for $X_d^d$ to be equivalent to $I$. At this point the problem of incompatibility between  qubits and qutrits immediately reappears, since a trit will identify 0 with 3 for example, which can not be added to a qubit in a well defined way. If we represent $X_3$ as mapping $i \mapsto i+2\mod 6$ to achieve $X_3^3=I$ directly, then we still have to identify the 0 trit with some odd integer, so the same conflict occurs. As a result there is no obvious way to represent operations acting on qubits and qutrits directly as polynomials, which suggests that at an intuitive level mixed systems have the potential for novel structural behaviour not present in non-mixed systems. Whether this novel structure  exists, and whether it benefits or hinders practical computation remains to be seen.
% chap3.tex (Chapter 3 of the thesis)
\chapter[UNIVERSAL COMPUTATION]{Universal Computation}\label{universality}

A foundational result in quantum computation is that of universal computation, that certain combinations of quantum gate can be used to implement any quantum algorithm to some accuracy, given sufficient circuit depth. The resulting circuits are generally too long to use in practice, compared to compilation techniques that rely on specific properties of the algorithm in question, but the result is still useful since it proves that it's not impossible, i.e.\ its necessary and sufficient conditions provide a starting point for designing and using quantum computers in practice.

\section{Universal Computation in Qubit Contexts}
The two most widely useful results about universality of qubit computers are the result of \cite{cnot-decomposition}, that a quantum computer with arbitrary operations from $U(2)$ on each individual qubit, and the controlled not $C(X)$ operation, one can exactly implement any unitary $U \in U(N)$, and the result of \cite{universal-qubit}, which ports this result to fault tolerant computation by showing that with only two fault tolerant elementary gates one can approximate any single-qubit operation in $U(2)$, and hence with the addition of $C(X)$, which is also fault tolerant, one can fault tolerantly approximate any operation in $U(N)$. We shall describe the former of these results, and in doing so generalize it to the following:

\begin{theorem}\label{mixed-universal}
	In any mixed quantum computer with at least one qubit, one can achieve universal computation with either:
\begin{itemize}
	\item Arbitrary qubit operations and arbitrary controlled increments $C_{r_i=q_i}(X_{d_j})$
	\item Arbitrary qubit operations and arbitrary controlled transpositions $C_{r_i=q_i}(S_{p_i,p'_i})$
\end{itemize}
\end{theorem}

In the case of a computer with only qubits, this theorem is equivalent to the result of \cite{cnot-decomposition}. We shall now outline the series of techniques presented in the textbook \cite{textbook}, which collect the relevant techniques from \cite{cnot-decomposition} and its predecessors into a continuous sequence of increasingly powerful proofs of universal computation. We will treat this as a single proof, with each step of the proof decomposing an arbitrary unitary $U \in U(N)$ into a smaller set of basic operations. The first decomposition is of $U$ into
\[U = \prod_{p=0}^{N-2}\prod_{q=p+1}^{N-1}U_{p,q},\]
giving $(N-1)(N-2)/2$ unitary matrices $U_{p,q}$, one for each distinct pair $p = p_n\dots p_1$, $q = q_n\dots q_1$, $p < q$. Specifically $U_{p,q}$ will be `two level' unitaries, in that they only act on the two computational basis vectors $\ket{p}$ and $\ket{q}$, meaning there are some complex $a, b, c, d$ so that
\[U_{p,q} = I + (a-1)\ket{p}\bra{p} + b\ket{p}\bra{q}+ c\ket{q}\bra{p}+ (c-1)\ket{q}\bra{q}.\]
For example in a system of two qubits, with $p = 1$ and $q = 2$, we have
\[U_{1,2} = \begin{bmatrix}
1 & 0 & 0 & 0 \\
0 & a & b & 0 \\
0 & c & d & 0 \\
0 & 0 & 0 & 1
\end{bmatrix}.\]
This operation will have no effect on computational basis states $\ket{0}\ket{0}$ or $\ket{1}\ket{1}$, but will act on $\ket{p_2}\ket{p_1}$ and $\ket{q_2}\ket{q_1}$ in a similar manner to
\[U_2 = \begin{bmatrix}
a & b \\
c & d
\end{bmatrix}.\]

The proof that any unitary can be represented as a product of such two-level unitaries amounts to a kind of row reduction on the lower left triangle of the unitary, choosing the $c$ value of each $U_{p,q}$ in order to eliminate each element one at a time. We won't present the exact details here, since this part requires no change in the case of a mixed quantum computer. The full procedure can of course be found in \cite{textbook}.

These two-level unitaries can then be implemented as a series of controlled operations $\{C_c(U_2)\}$, which can in turn be decomposed into `basic' operations, $C(X)$ along with arbitrary $U_2 \in U(2)$. This decomposition is done using a variety of techniques presented in \cite{cnot-decomposition}. This set of techniques provide an excellent starting point for reasoning about quantum computation at the level of physical qubits, but in order to work with logical qubits one can go a step further and approximately implement all of $U(2)$ using only two basic gates with known fault tolerant implementations. This result was shown in \cite{universal-qubit}, and shall inform our later discussion in \autoref{analyse-perms} of minimal gate sets in quantum computers consisting only of qubits and qutrits.

\section{Representing Two-Level Unitaries With Control Operations}
Once we have decomposed a unitary into two-level unitaries $U_{p,q}$, acting on computational basis states $\ket{p}$ and $\ket{q}$, our next goal will be to represent this two-level unitary as a concrete quantum circuit consisting of various controlled operations. First, we must choose any qubit in the system, which will be indexed by an integer $j$ satisfying $d_j = 2$. Now define $C_c(U_j)$ to be the control operation applying 
\[U_2 = \begin{bmatrix}
a & b \\
c & d
\end{bmatrix}\]
to qubit $j$, so long as every other object in the quantum system is in state $q_i$. In our notation this means
\[c = \{\ket{r_n}\dots\ket{r_1}\ |\ r_i = q_i \text{\ whenever\ } i \neq j\}.\]

Now $C_c(U_j)$ will also be a two-level unitary, acting on
\[p' = q_n \dots q_{j+1} 0 q_{j-1} \dots q_1,\]
\[q' = q_n \dots q_{j+1} 1 q_{j-1} \dots q_1.\]
Formally,
\[C_c(U_j) = I + (a-1)\ket{p'}\bra{p'} + b\ket{p'}\bra{q'}+ c\ket{q'}\bra{p'}+ (c-1)\ket{q'}\bra{q'}.\]

Consider our $U_{1,2}$ example. We could choose $j = 1$. Then $p'_2 = q'_2 = q_2 = 1$, so $p'$ and $q'$ would have binary expansion $10$ and $11$, the integers 2 and 3 respectively. This gives
\[C_c(U_j) = C_{r_2=1}(U_j) = \begin{bmatrix}
	1 & 0 & 0 & 0 \\
	0 & 1 & 0 & 0 \\
	0 & 0 & a & b \\
	0 & 0 & c & d
\end{bmatrix}.
\]

At this point we can fairly easily see how to implement $U_{1,2}$ as a concrete quantum circuit, so long as we can map $\ket{p}=\ket{1}$ to $\ket{p'} = \ket{2}$ and $\ket{q}=\ket{2}$ to $\ket{q'}=\ket{3}$. There are two permutation matrices that will do this:
\[P = \begin{bmatrix}
	1 & 0 & 0 & 0 \\
	0 & 0 & 0 & 1 \\
	0 & 1 & 0 & 0 \\
	0 & 0 & 1 & 0
\end{bmatrix}, \begin{bmatrix}
	0 & 0 & 0 & 1 \\
	1 & 0 & 0 & 0 \\
	0 & 1 & 0 & 0 \\
	0 & 0 & 1 & 0
\end{bmatrix}.\]

We choose the latter of these, which happens to map any $\ket{r}$ to $\ket{r+1 \mod 4}$, and can be represented by the following circuit:

\begin{quantikz}
\lstick{$\ket{r_2}$} & \qw & \targ{} & \qw \rstick[wires=2]{$\ket{r+1 \mod 4}$} \\
\lstick{$\ket{r_1}$} & \gate{X_2} & \phase{0} \vqw{-1} & \qw  \\
\end{quantikz}

As a matrix expression this is $P = (I \otimes X_2)C_{r_1=0}(X_2 \otimes I)$. Since each term in this circuit is self-inverse, we can reverse the circuit to implement $P^{-1}$ as well. Then in order to apply $U_{p,q}$ to $\ket{r}$, we first apply $P$, then $C_c(U_j)$, then $P^{-1}$, which is the following similarity transformation:
\[\begin{bmatrix}
	1 & 0 & 0 & 0 \\
	0 & a & b & 0 \\
	0 & c & d & 0 \\
	0 & 0 & 0 & 1
\end{bmatrix} = \begin{bmatrix}
0 & 1 & 0 & 0 \\
0 & 0 & 1 & 0 \\
0 & 0 & 0 & 1 \\
1 & 0 & 0 & 0
\end{bmatrix}
\begin{bmatrix}
	1 & 0 & 0 & 0 \\
	0 & 1 & 0 & 0 \\
	0 & 0 & a & b \\
	0 & 0 & c & d
\end{bmatrix}
\begin{bmatrix}
0 & 0 & 0 & 1 \\
1 & 0 & 0 & 0 \\
0 & 1 & 0 & 0 \\
0 & 0 & 1 & 0
\end{bmatrix}.\]
Reading from right to left we can get a concrete quantum circuit for this operation.

\begin{quantikz}
\lstick[wires=2]{$\ket{\phi}$} & \qw & \targ{} & \phase{1} \vqw{1} & \targ{} & \qw & \qw \rstick[wires=2]{$U_{p,q}\ket{\phi}$} \\
 & \gate{X_2} & \phase{0} \vqw{-1} & \gate{U_2} & \phase{0} \vqw{-1} & \gate{X_2} & \qw
\end{quantikz}

To generalize this we must describe a process for generating and implementing $P$ in any qubit computer. The process given in \cite{textbook} is to implement the transposition $S_{p,p'}$ mapping $\ket{p}$ to $\ket{p'}$. For them this will do as a permutation $P$ since they assume $q'_j = q_j$, as opposed to our assumption that $q'_j = 1$. Consider as an extreme example among cases $n=3$, and $j=1$, where all three bits need to be inverted, i.e.\ $p_3 \neq q_3$, $p_2 \neq q_2$, $p_1 = q_1$.

\begin{quantikz}
	\lstick{$\ket{r_3}$} & \targ{} & \phase{q_3}\vqw{1} & \phase{q_3}\vqw{1} & \phase{q_3}\vqw{1} & \targ{} & \qw \\
	\lstick{$\ket{r_2}$} & \phase{p_2}\vqw{-1} & \targ{} & \phase{q_2}\vqw{1} & \targ{} & \phase{p_2}\vqw{-1} & \qw \\
	\lstick{$\ket{r_1}$} & \phase{p_1}\vqw{-1} & \phase{p_1}\vqw{-1} & \targ{} & \phase{p_1}\vqw{-1} & \phase{p_1}\vqw{-1} & \qw
\end{quantikz}

In the circuit for $PC_c(U_j)P^{-1}$ the latter half of the above sequence of operations would have no effect, since it would be transforming the action of $C_c(U_2)$ on basis vectors that it doesn't do anything to, so in fact the full circuit simplifies to a circuit that looks much like the above.

\begin{quantikz}
	\lstick{$\ket{r_3}$} & \targ{} & \phase{q_3}\vqw{1} & \phase{q_3}\vqw{1} & \phase{q_3}\vqw{1} & \phase{q_3}\vqw{1} & \phase{q_3}\vqw{1} & \targ{} & \qw \\
	\lstick{$\ket{r_2}$} & \phase{p_2}\vqw{-1} & \targ{} & \phase{q_2}\vqw{1} & \phase{q_2}\vqw{1} & \phase{q_2}\vqw{1} & \targ{} & \phase{p_2}\vqw{-1} & \qw \\
	\lstick{$\ket{r_1}$} & \phase{p_1}\vqw{-1} & \phase{p_1}\vqw{-1} & \targ{} & \gate{U_2} & \targ{} & \phase{p_1}\vqw{-1} & \phase{p_1}\vqw{-1} & \qw
\end{quantikz}

The first half of this is essentially the $P$ we originally wanted, mapping $\ket{p}$ to $\ket{p'}$ and $\ket{q}$ to $\ket{q'}$. This process is fine for showing universal computation in the abstract, but is hard to generalize to mixed contexts, and is generally very wasteful. Instead we shall do something much simpler, using only $C(X)$. The first step shall be to choose a smaller permutation $P_1$ so that $p_j$ maps to 0 and $q_j$ maps to 1. There are four cases to consider:
\begin{itemize}
	\item $p_j = 0$, $q_j = 1$ already,
	\item $p_j = 1$, $q_j = 0$,
	\item $p_j = q_j = 0$,
	\item $p_j = q_j = 1$.
\end{itemize}
	In the first case we can set $P_1 = I$, and do nothing, and in the second case we can set $P = X_j$, but in the last two cases we must pick some $k$ so that $p_k \neq q_k$. Then if we are in the third case we set $P = C_{r_k = q_k}(X_j)$, and in the fourth case $P = C_{r_k = p_k}(X_j)$. Now $P_1\ket{q} = \ket{q'}$, so all that remains is to map each remaining $p_i$ to $q_i$ without changing $\ket{q}$. This is simple to do with $C_{r_j=0}(X_i)$, repeated once for each $i\neq j$ with $p_i \neq q_i$. Applied to $\ket{p}$ this will change $p_j$ to $0$, setting the control value, so that the remaining bits can be set to $\ket{q_i}$, and of course applied to $\ket{q}$ this will change $q_j$ to $1$, making no other changes since the control bit has been set incorrectly. We can apply this process to our extremal three-qubit example to get a circuit that is relatively straight forward.

\begin{quantikz}
	\lstick{$\ket{r_3}$} & \qw & \qw & \targ{} & \phase{q_2}\vqw{1} & \targ{} & \qw & \qw & \qw \\
	\lstick{$\ket{r_2}$} & \phase{p_2}\vqw{1} & \targ{} & \qw & \phase{q_2}\vqw{1} & \qw & \targ{} & \phase{p_2}\vqw{1} & \qw \\
	\lstick{$\ket{r_1}$} & \targ{} & \phase{0}\vqw{-1} & \phase{0}\vqw{-2} & \gate{U_2} & \phase{0}\vqw{-2} & \phase{0}\vqw{-1} & \targ{} & \qw
\end{quantikz}

In this case our circuit is the same length, or even longer if more optimization had been applied, but the simplification from three Toffoli gates to three $C(X)$ gates is dramatic once we start representing Toffoli gates in terms of $C(X)$, and as the number of bits increases this difference will increase quadratically, so it is a significant improvement. Of course arguments about universal computation are not intended to be efficient anyway, and the real purpose of this approach is to generalize to mixed systems. The way that we do this is straight forward. We still choose $j$ to be some qubit, meaning $d_j = 2$ still. $P_1$ is chosen by the same process as before as well, but if $k$ needs to be chosen it can be an arbitrary control digit whether binary or otherwise. Now to change the remaining digits $p_i$ to $q_i$, we still use controlled operations, but can choose whether we use $C_{r_j=0}(S_{p_i,q_i})$ to change $p_i$, or $C_{r_j=0}(X_{d_i})^{q_i-p_i}$, depending on which basis of \autoref{mixed-universal} we are aiming to use.

We have now written $U_{p,q}$ as a combination of control operations, and so next is to use the techniques described in \cite{cnot-decomposition} to decompose these into operations with only a single control object. Of course with the above technique most of our operations are already in this form, but this doesn't eliminate any potential cases, since we still have an arbitrary $C_c(U_j)$ in the middle of the circuit.

\section{Decomposing Control Operations}
We would like to decompose $C_c(U_j)$ into elements of $U(2)$ and operations of the form $C(X_{d_i})$ or $C(S_{p_i,p'_i})$. In a binary computer this is done in three steps. The first is to introduce/require an additional $n-3$ auxiliary qubits to the quantum computer that are not intended to be affected by the original unitary in $U(N)$ or the control operation in question, and to inductively reduce the operation $C_c(U_j)$ into a series of gates with two control qubits. We index the new qubits $n+1$ through $2n-2$, which begin and end this process in the computational basis state $\ket{0}$. The key to this process is the Toffoli gate, which we recall is simply a controlled $X$ operation with two control bits, $C_{r_k=q_k,r_l=q_l}(X_j)$.

Take $I$ to be the set of indices $i$ for which our control operation $C_c(U_j)$ has a constraint $r_i = q_i$, so that
\[c = \{\ket{r_1}\dots\ket{r_n}\ |\ r_i = q_i \text{\ whenever} i \in I\}\]
Then we shall perform induction on the size of $I$, implementing any control operation as $2\ord{I}-4$ Toffoli gates, sandwiching a final single- or double-controlled operation $C_{r_k=q_k, r_l=q_l}(U_j)$. We draw this as a circuit equation, for reference.

\begin{quantikz}
\lstick{$\ket{0}$}  & \qw & \qw \midstick[6,brackets=none]{=}& \targ{} & \phase{1}\vqw{4} & \targ{} & \qw \rstick{$\ket{0}$}\\
\lstick{$\ket{r_k}$}& \phase{q_k}\vqw{1} & \qw& \phase{q_k}\vqw{-1} & \qw & \phase{q_k}\vqw{-1} & \qw \rstick{$\ket{r_k}$} \\
\lstick{$\ket{r_l}$}& \phase{q_l}\vqw{2} & \qw& \phase{q_l}\vqw{-1} & \qw & \phase{q_l}\vqw{-1} & \qw \rstick{$\ket{r_l}$} \\
\lstick{$\vdots$}& & & & & & \rstick{$\vdots$} \\
                 & \ctrl{1} & \qw& \qw & \ctrl{1} & \qw & \qw  \\
\lstick{$\ket{r_j}$}& \gate{U_j} & \qw& \qw & \gate{U_j} & \qw & \qw
\end{quantikz}

In total this implements an operation with $\ord{I}$ control bits in terms of one with $\ord{I}-1$ control bits. Formally we have introduced an auxiliary qubit $r_i$, and reduced $I$ to
\[I' = (I \backslash \{k, l\})\cup \{i\}\]
Correspondingly, set $q_i = 1$ and reduce the condition set $c$ to
\[c' = \{\ket{r_1}\dots\ket{r_n}\ |\ r_i = q_i \text{\ whenever} i \in I'\}.\]
This returns us to the form we started in, with an operation $C_{c'}(U_j)$, but with one less control qubit. Eventually we will reach a base case where $\ord{I} = 2$, in which case $C_c(U_j)$ is already a double-controlled gate sandwiched by $2\ord{i}-4 = 0$ Toffoli gates. This means that inductively we end up with the result that we wanted.

This decomposition into Toffoli gates acting on auxiliary qubits generalizes to mixed logic without any modification, so long as auxiliary qubits are available. Alternatively one can copy \cite{multi-valued-logic} which introduces auxiliary objects of dimension $d>2$, and uses singly-controlled permutations gates, though controlled increments will do as well. We will assume $d=3$ but it is easy to optimize the number of auxiliary objects when $d$ is larger. Again we draw this transformation as a circuit equation for reference.

\begin{quantikz}
	\lstick{$\ket{0}$}  & \qw & \qw \midstick[6,brackets=none]{=}& \gate{S_{0,1}} & \gate{S_{1,2}} & \phase{2}\vqw{4} & \gate{S_{2,1}} & \gate{S_{1,0}} & \qw \rstick{$\ket{0}$}\\
	\lstick{$\ket{r_k}$}& \phase{q_k}\vqw{1} & \qw& \phase{q_k}\vqw{-1} & \qw & \qw & \qw & \phase{q_k}\vqw{-1} & \qw \rstick{$\ket{r_k}$} \\
	\lstick{$\ket{r_l}$}& \phase{q_l}\vqw{2} & \qw& \qw & \phase{q_l}\vqw{-2} & \qw & \phase{q_l}\vqw{-2} & \qw & \qw \rstick{$\ket{r_l}$} \\
	\lstick{$\vdots$}& & & & & & & & \rstick{$\vdots$} \\
	& \ctrl{1} & \qw& \qw & \qw & \ctrl{1} & \qw & \qw & \qw  \\
	\lstick{$\ket{r_j}$}& \gate{U_j} & \qw& \qw & \qw & \gate{U_j} & \qw & \qw & \qw
\end{quantikz}

$S_{0,1}$ in the above diagram can be replaced with $X_3$ and $S_{1,0}$ with $X^{-1}$ to achieve the same effect, depending on the desired basis. If we have arbitrarily many qutrits or higher available, then we can ignore the Toffoli gates altogether, and only use the $C(X)$ or $C(S_{p_i,p'_i})$ gates available to us to reduce $C_c(U_k)$ to a singly-controlled $C_{r_i=2}(U_j)$, but for generality we shall continue as if Toffoli gates need to be used as well. Given $V_j^2 = U_j$, \cite{cnot-decomposition} presents a circuit which we can use to implement $C_{r_k=q_k, r_l=q_l}(U_j)$ acting only on qubits.

\begin{quantikz}
	\lstick{$\ket{r_k}$}& \phase{q_k}\vqw{1} & \qw\midstick[3,brackets=none]{=}& \qw & \phase{q_k}\vqw{1} & \qw & \phase{q_k}\vqw{1} & \phase{q_k}\vqw{2} & \qw \rstick{$\ket{r_k}$} \\
	\lstick{$\ket{r_l}$}& \phase{q_l}\vqw{1} & \qw& \phase{q_l}\vqw{1} & \targ{} & \phase{q_l}\vqw{1} & \targ{} & \qw & \qw \rstick{$\ket{r_l}$} \\
	\lstick{$\ket{r_j}$}& \gate{U_j} & \qw& \gate{V_j} & \qw & \gate{V_j^{-1}} & \qw & \gate{V_j} & \qw
\end{quantikz}

This decomposes our many doubly-controlled gates into singly-controlled gates, but interestingly all of the Toffoli gates acting on auxiliary qubits will become $C(\sqrt{X_2})$\footnote{If one wants a square root of $X$, then $(H^{-1}\sqrt{Z}H)^2 = H^{-1}ZH = X$ is a simple example, with $\sqrt{Z}\ket{i} = \omega_4^i\ket{i}$.} rather than $C(X)$. So long as object $j$ is still a qubit, this proof will generalise to a mixed quantum computer by simply inserting the controlled transpositions available to us in place of the controlled increments.

\begin{quantikz}
	\lstick{$\ket{r_k}$}& \phase{q_k}\vqw{1} & \qw\midstick[3,brackets=none]{=}& \qw & \phase{q_k}\vqw{1} & \qw & \phase{q_k}\vqw{1} & \phase{q_k}\vqw{2} & \qw \rstick{$\ket{r_k}$} \\
	\lstick{$\ket{r_l}$}& \phase{q_l}\vqw{1} & \qw& \phase{q_l}\vqw{1} & \gate{S_{q_l,q_l+1}} & \phase{q_l}\vqw{1} & \gate{S_{q_l+1,q_l}} & \qw & \qw \rstick{$\ket{r_l}$} \\
	\lstick{$\ket{r_j}$}& \gate{U_j} & \qw& \gate{V_j} & \qw & \gate{V_j^{-1}} & \qw & \gate{V_j} & \qw
\end{quantikz}

Once again $S_{q_l,q_l+1}$ and $S_{q_l+1,q_l}$ can be replaced with $X_{d_l}$ and $X_{d_l}^{-1}$ depending on the desired basis.

In fact \cite{cnot-decomposition} generalizes this implemention of doubly-controlled gates directly to arbitrary $C_c(U_j)$, so this approach could be directly generalized to a mixed context by replacing all $X_2$ operations with increment/decrement respectively, avoiding the discussion of auxiliary qubits or qutrits altogether, but requiring greater circuit depth in exchange for the narrower circuit width. This gives a total of three distinct ways of implementing $C_c(U_j)$ in terms of singly-controlled operations $C_{r_k=q_k}(U_j)$, acting on qubits. In all three cases we now need only demonstrate how to implement these in terms of single-qubit unitaries and $C_{r_k=q_k}(X_2)$, which in this case is the same operation as that notated by $C_{r_k=q_k}(S_{0,1})$. The qubit construction referenced in \cite{cnot-decomposition} works for this purpose without modification.

\begin{quantikz}
\lstick{$\ket{r_k}$} & \phase{q_k}\vqw{1} & \qw\midstick[2,brackets=none]{=}& \qw & \phase{q_k}\vqw{1} & \qw & \phase{q_k}\vqw{1} & \qw & \qw \rstick{$\ket{r_k}$}\\
\lstick{$\ket{r_j}$} & \gate{U_j} & \qw & \gate{A} & \targ{} & \gate{B} & \targ{} & \gate{C} & \qw\rstick{$C(U_j)\ket{r_j}$}
\end{quantikz}

Here $A,B,C \in U(2)$ are chosen so that $ABC=I$ and $AXBXC = U_j$. This can be done in general, via the spherical geometry of $U(2)/U(1)$, as shown in \cite{cnot-decomposition}. Combining all of the steps just described we have proven \autoref{mixed-universal}, generalizing the universality result of \cite{cnot-decomposition} and \cite{textbook} to mixed quantum computers with at least one qubit, and with significantly fewer operations than \cite{textbook} seems to have used, even in the case of a computer with only qubits available. $\square$
\section{Analysing Small Permutations}\label{analyse-perms}
We have generated $U(N)$ with a finite set of two-object operations along with arbitrary qubit operations from $U(2)$. We can then apply the result of \cite{universal-qubit}, that any element of $U(2)$ can be approximated using the operations

\begin{align*}
	H_2 = \frac{1}{\sqrt{2}}\left[\begin{matrix}
	1 & 1 \\
	1 & -1
\end{matrix}\right],
&&&
T = \left[\begin{matrix}
	1 & 0 \\
	0 & \frac{1}{\sqrt{2}}\left(1+ i\right)
\end{matrix}\right].
\end{align*}

We will come back in \autoref{infinite-order} to talk briefly about how this is done, since it involves an aspect of algebraic number theory that will be useful to us. The result was then used to argue that by the decompositions described in \cite{cnot-decomposition}, the gate set $H_2$, $T$, and $C_2(X_2)$ is universal, a significant result since this gate set can also be implemented fault tolerantly. We can directly map this result to mixed quantum computers via \autoref{mixed-universal}, to get two alternative finite gate sets, $\{H_2, T, C_{r_k=q_k}(X_j)\}$, and $\{H_2, T, C_{r_k=q_k}(S_{p_j,p'_j})\}$. We will now discuss ways that this operator set can be reduced further, with particular mind to the simplest case of quantum computers that only have qubits and qutrits. First we can fix $q_k=d_k-1$, and generate other control values via a circuit equivalence.

\begin{quantikz}
	& \phase{q_k}\vqw{1} & \qw\midstick[2,brackets=none]{=}& \gate{X_k^{q'_k-q_k}} & \phase{q'_k}\vqw{1} & \gate{X_k^{q_k-q'_k}} & \qw\\
	& \gate{U_j} & \qw & \qw & \gate{U_j} & \qw
\end{quantikz}

This introduces each $X_d$ to our generator set, but removes each $C_{r_k=q_k}(\dots)$ apart from $q_k=d_k-1$. Further we already have $X_2 = H^{-1}T^4H$, so we only need $X_d$ for $d > 2$. Additionally, we can fix $p_j$ to 0 by a similar equivalence.

\begin{quantikz}
	& \phase{q_k}\vqw{1} & \qw\midstick[2,brackets=none]{=}& \qw & \phase{q_k}\vqw{1} & \qw & \qw\\
	& \gate{S_{p_j,p'_j}} & \qw & \gate{X_k^{-p_j}} & \gate{S_{0,p'_j-p_j}} & \gate{X_k^{p_j}} & \qw
\end{quantikz}

where the difference $p'_j-p_j$ is evaluated mod $d_j$. Finally, we can optionally exchange $p_j$ and $p'_j$ to sure that the difference $p'_j - p_j$ is always at most $d_j/2$. For example $S_{2,1}$ acting on a qutrit is equivalent to $S_{1,2}$, which by the above can be reduced to $X_3S_{1,2}X_3^{-1}$. This means ultimately our generator sets are $\{H_2, T, X_{d_i}, C_{r_k=d_k-1}(X_k)\}$ and $\{H_2, T, X_{d_i}, C_{r_k=d_k-1}(S_{0,p'_k})\}$, with $d_i > 2$, $p'_k \leq d_k/2$. In the qubit-qutrit case these will both have seven elements, the increment basis
\[\{H_2, T, X_3, C_2(X_2), C_2(X_3), C_3(X_2), C_3(X_3)\},\]
and the transposition basis
\[\{H_2, T, X_3, C_2(X_2), C_2(S_{0, 1}), C_3(X_2), C_3(S_{0, 1})\}.\]

We will now show how these two bases are directly equivalent to each other, though generating the increment basis from the transposition basis will be much less expensive than the other way around. Five of the operations are common between the two bases, so the only difference is that one contains $C_2(X_3)$ and $C_3(X_3)$, where the other contains $C_2(S_{0,1})$ and $C_3(S_{0,1})$. In general $C_d(A)C_d(B) = C_d(AB)$, so setting $A = S_{0,1}$ and $B = S_{1,2}$ we get $C_d(X_3)$. We can write this as an explicit quantum circuit in the transposition basis.

\begin{quantikz}
	& \phase{d_k-1}\vqw{1} & \qw\midstick[2,brackets=none]{=}& \qw & \phase{d_k-1}\vqw{1} & \qw & \phase{d_k-1}\vqw{1} & \qw\\
	& \targ{} & \qw & \gate{X_3^{-1}} & \gate{S_{0,1}} & \gate{X_k} & \gate{S_{0,1}} & \qw
\end{quantikz}

Since $d_k$ is arbitrary in this construction, we have implemented both of the controlled increment operations $C_2(X_3)$ and $C_3(X_3)$. The reverse is more contrived; we will later find by brute force that $X_2$, $X_3$, $C_2(X_3)$, and $C_3(X_2)$ generate all permutations on an $N=6$ composite system, and in particular $C_2(S_{0,1})$ becomes a convoluted sequence of controlled increments.

\begin{quantikz}
	& \phase{1}\vqw{1} & \qw\midstick[2,brackets=none]{=}& \targ{} & \phase{1}\vqw{1} & \targ{} & \phase{1}\vqw{1} & \phase{1}\vqw{1} & \targ{} & \qw\\
& \gate{S_{0,1}} & \qw & \phase{2}\vqw{-1} & \targ{} & \phase{2}\vqw{-1} & \targ{} & \targ{} & \phase{2}\vqw{-1} & \qw
\end{quantikz}

We cannot directly generate $C_3(S_{0,1})$ using $C_3(X_3)$ due to parity, since $C_3(S_{0,1})$ is an odd number of transpositions and $C_3(X_3)$ is an even number of transpositions. One thing we can do is introduce an auxiliary qubit, and use $C_2(S_{0,1})$ as above:

\begin{quantikz}
	\lstick{$\ket{0}$} & \qw & \qw\midstick[3,brackets=none]{=}& \targ{} & \phase{1}\vqw{2} & \targ{} & \qw \rstick{$\ket{0}$} \\
	& \phase{2}\vqw{1} & \qw& \phase{2}\vqw{-1} & \qw & \phase{2}\vqw{-1} & \qw\\
	& \gate{S_{0,1}} & \qw & \qw & \gate{S_{0,1}} & \qw & \qw
\end{quantikz}

Another option we have is to introduce the Clifford SWAP and SUM operations, and appeal to the equivalences shown in \cite{arithmetics}. $S_{01,10}$ in a two qutrit system was shown to be implementable by five $C_3(X_3)$ gates and a SWAP.

\begin{quantikz}
& \gate[2]{S_{01,10}} & \qw\midstick[2,brackets=none]{=} & \phase{2}\vqw{1} & \targ{} & \phase{2}\vqw{1} & \targ{} & \phase{2}\vqw{1} & \swap{1} & \qw \\
& \qw & \qw & \targ{} & \phase{2}\vqw{-1} & \targ{} & \phase{2}\vqw{-1} & \targ{} & \targX{} & \qw \\
\end{quantikz}

Then $C_{r_2=0}(S_{0,1})$ is identical to $S_{00,01}$, which is a reflection in $\mathbb{C}^9$ just like $S_{01,10}$, and in fact in \cite{arithmetics} these operations were transformed to each other using Clifford operations $X_3$ and SUM.

\begin{quantikz}
	& \phase{0}\vqw{1} & \qw\midstick[2,brackets=none]{=} & \octrl{1} & \gate{X} & \gate[2]{S_{01,10}} & \gate{X^{-1}} & \octrl{1} & \octrl{1} & \qw\\
	& \gate{S_{0,1}} & \qw & \targ{} & \gate{X} & \qw & \gate{X^{-1}} & \targ{} & \targ{} & \qw\\
\end{quantikz}

These equivalences are theoretically interesting, but at this algebraic level, the transposition basis appears to be significantly more efficient than the increment basis. Despite this, it is hard to speculate on what a good basis would be without properly considering the physical constraints and trade-offs of a specific quantum computer. For example, it would not be surprising if $C_2(X_3)$ were more favourable to implement fault tolerantly than $C_2(S_{0, 1})$, due to its resemblance to the SUM operation between two qutrits, but it is also possible that neither end up being implemented exactly, and that a different basis is used which approximately implements both of the bases we have presented.

Another decomposition that we have not discussed until now is the implementation of $X_3$ in terms of $C_2(X_3)$ and $X_2$.

\begin{quantikz}
	\lstick{$\ket{0}$} & \qw & \qw\midstick[2,brackets=none]{=}& \gate{X_2} & \phase{1}\vqw{1} & \gate{X_2} & \phase{1}\vqw{1} & \qw\\
	& \gate{X_3} & \qw& \qw & \gate{X_3} & \qw & \gate{X_3} & \qw
\end{quantikz}

We do not use the abbreviated $\oplus$ notation in this case to make it clear what is happening in this circuit -- we are applying $X_3$ conditionally, once for every possible condition, which has the same effect as unconditionally applying $X_3$. In theory this reduces both of our generator sets to only need six gates, but in practice this is absurd, since $X_3$ is a Pauli operation which will be one of the first and simplest operations to implement on a fault tolerant quantum computer, and we have implemented it in terms of a controlled operation with no known fault tolerant implementation.

In any case we have a set of seven gates that appear useful to implement and work with. All of the examples we have seen of universal computation for non-mixed systems of qubits or qutrits have involved some number of Clifford gates, and a single non-Clifford gate, but in order to generalize the standard binary model of quantum computation we find we require four control gates, only one of which is Clifford, together with the non-Clifford $T$, for a total of four non-Clifford gates. An interesting avenue of future research would be the question of what the minimum number of non-Clifford gates is for achieving universal computation in mixed systems like this.

% chap4.tex (Chapter 4 of the thesis)

\chapter[FINITELY GENERATED UNITARY GROUPS]{FINITELY GENERATED UNITARY GROUPS}

\section{Finite Gate Sets in Higher Dimensions}
In the previous chapter we outlined the Pauli matrices, and defined the related Weyl-Heisenberg and Clifford groups, including the matrix $H$ satisfying $H^{-1}ZH = X$. We saw above that $H$ was one of 3 gates needed for universal computation in a binary quantum computer, but the other two are also directly related to the Pauli matrices. $C(X)$ is of course the controlled version of the $X$ gate, and $T$ when applied four times is equivalent to $Z$:
\[
T^4 = \left[\begin{matrix}
	1 & 0 \\
	0 & \frac{1}{\sqrt{2}}(1+i)
\end{matrix}\right]^4 = \left[\begin{matrix}
	1 & 0 \\
	0 & -1
\end{matrix}\right] = Z\]

It turns out that generalizing the Pauli matrices to $d$-dimensional quantum objects as we have done previously proves very useful in describing higher dimensional systems as well, and the relationships that different quantum gates have with each other in these systems.

First of all, the Clifford group formed by taking the normaliser of the Weyl-Heisenberg group lets us generalize the algebraic relationships of $H$ as well as $D = T^2$ to higher dimensions, and looking at \cite{tolar-clifford} will tell us how this group will behave in quantum computers with multiple objects of different dimension.

Additionally the $T$ gate, which does not directly appear in the Clifford group, proves essential to the generalization of universal computation to other quantum systems. In the qubit case the produce $TH^{-1}TH$ is an infinite order matrix, which was proven in \cite{pi-over-eight}, so we outline the technique used to show this, since this means that $T$ and $H$ generate [define this] an infinite group, even with scale factors removed, a property that is necessary for universal computation, and that distinguishes $T$ from the rest of the Clifford group. Additionally we discuss a paper \cite{pi-over-eight} which generalizes this gate $T$ to higher dimensions based on other algebraic relationships it has with the Clifford group.
\subsection{On Clifford Groups}
The paper \cite{tolar-clifford} describes the Clifford group associated with quantum systems with just a single object, as well as the Clifford group associated with arbitrary composite quantum systems, with multiple objects each of different dimension, and in particular for a system with two objects of different dimension. This last case is of particular interest since, as with the $C(X_2)$ gate used in binary quantum computers, gates that only interact with 1 or 2 objects are often easier to implement than those that interact with more. [and might generate the rest anyway, don't know how to justify that though]

Tolar removes the scalar factors from both the Weyl-Heisenberg group, and the Clifford group, not just by taking the group quotient $G/U(1)$, but also by looking at the group conjugation action, mapping the Weyl-Heisenberg group to itself under the map
\[\text{Ad}_A(X^jZ^k) = AX^jZ^kA^{-1}\]
\ [note Tolar uses $\text{Ad}_X$ and uses $Q$ and $P$ for clock and shift respectively] These conjugation maps form a group under function composition, and are isomorphic to the corresponding quotient groups $H(n)/U(1)$ and $N(H(n))/U(1)$. When applied to the Weyl-Heisenberg group associated with a single quantum object of arbitrary dimension, it was shown that $N(H(n))/U(1)$ was isomorphic to
\[(\mathbb{Z}_n\times\mathbb{Z}_n)\rtimes\text{SL}(2,\mathbb{Z}_n)\]

This isomorphism by itself is not new, [can I reference something older then?] and can be used to generate the Clifford group (up to global phase) with the four matrices $X_d$, $Z_d$, $D_d$, and $H_d$, [apparently I'm using $d$ now] defined by:
[insert definition... maybe put these generators up in preliminaries, with an older reference]

The more significant result of the paper however, was the extension to Clifford groups for arbitrary composite systems. The Weyl-Heisenberg group of a composite system $C^{d_1}\otimes \dots \otimes C^{d_n}$ [have we defined Kronecker products of spaces?] is taken to be the product of Weyl-Heisenberg groups acting on each individual system:
\[H(d_1, \dots, d_n) = \{A_1 \otimes \dots \otimes A_n\ |\ A_i \in H(d_i)\} \cong H(d_1) \times \dots \times H(d_n)\]
\ [this isn't the notation used in Tolar, it's our own.]

While this composite Weyl-Heisenberg group consists only of Kronecker products, [I want to use the word dependent, need to mention that in the Kronecker product section] its normalizer can contain operations that are not Kronecker products, for example
\[\text{CNOT} = C(X) = \left[\begin{matrix}
	1 & 0 & 0 & 0 \\
	0 & 1 & 0 & 0 \\
	0 & 0 & 0 & 1 \\
	0 & 0 & 1 & 0
\end{matrix}\right] \in N(H(2, 2))\]

The Clifford groups of non-mixed composite systems have been explored elsewhere, [have they? where?] but here Tolar showed a number of results for mixed systems, the most important of these for us was that in a composite system $\mathbb{C}^m \otimes \mathbb{C}^n$ with $gcd(m, n) = 1$, the quotient of the Clifford group is simply the direct product of the corresponding single-object Clifford groups:
\[N(H(m, n))/U(1) \cong N(H(m))/U(1) \times N(H(n))/U(1) \]
This means that the Clifford group is \emph{not} capable of acting dependently between quantum objects, making it much less powerful than in non-mixed systems.

The same paper also showed that when $m$ and $n$ have square or cubic divisors in common, that the corresponding Clifford groups are novel in a way unlike any non-mixed quantum system. This is interesting in the broader context of mixed logic, but will not be relevant to our later discussion of $\mathbb{C}^2 \otimes \mathbb{C}^3$.

This is the only paper we found that looks at quantum systems with objects of differing dimension.

\subsection{Infinite-Order Gates}
In our outline of how universal computation is achieved for binary quantum computers, the last step was to decompose unitary operations acting on one qubit, into a sequence of $H$ and $T$ operations. The technique for doing this was described in \cite{universal-qubit}, and involved constructing a pair of operations which each rotate the Bloch sphere by irrational portions of a full $2\pi$ rotation each in a different, orthogonal plane. By iterating one of these rotations, one can approximate any angle in the unit circle, and thus with each of these rotations implemented, one can implement any rotation of the Bloch sphere as 3 successive rotations made using these gates.

One of the facts that makes the Clifford group algebraically and theoretically significant is that up to scale factors it is finite, whereas any group containing both $T$ and $H$ must not be finite, or else the rotations that were generated with them would be finite order, i.e.\ a rational fraction of a $2\pi$ rotation in their respective planes. This means the technique by which these rotations were shown to be irrational/infinite order are significant in both of these contexts, in working out alternate conditions for universal computation in different quantum systems, and in analysing the order of different subgroups of $U(n)$ in search of finite groups.

The technique they use for showing this is to generate an expression for the angle of rotation, $2\pi \theta$ say, then to show that the unit complex number $e^{i2\pi \theta}$ is not a root of unity, i.e. that $\theta$ is rational. This can be done by leveraging the algebraic number theory of cyclotomic polynomials, with the following theorem proven in an appendix of their paper:

[capital T theorem] For any $\theta \in \mathbb{R}$, $\theta$ is rational if and only if $e^{i2\pi \theta}$ is the root of a minimal [vs irreducible?] polynomial with rational coefficients.

This analysis works well in the Bloch sphere all unitary matrices will represent a single rotation in some plane, but when dealing with more complicated geometries we can still apply this kind of analysis if we look at the eigenvalues of a given matrix instead. Given a unitary matrix $A$, we already know that its eigenvalues are unit complex numbers $e^{i 2\pi\theta}$, so then

[capital P proposition?] the following are equivalent:
\begin{itemize}
	\item $A$ is finite order
	\item Every eigenvalue of $A$ is finite order under multiplication, i.e.\ is a root of unity
	\item Every eigenvalue of $A$ is a root of a cyclotomic polynomial
\end{itemize}

Now the eigenvalues of $A$ will be roots of the characteristic polynomial $det(A - \lambda I)$, so if the characteristic polynomial has rational coefficients then we can work directly with this to show that $A$ is of finite or infinite order.

Now $e^{i 2\pi\theta}$ and $e^{-i 2\pi\theta}$ are both eigenvalues of the rotation
\[R(2\pi \theta) = \left[\begin{matrix}
	\cos(2\pi\theta) & -\sin(2\pi\theta) \\
	\sin(\theta) & \cos(\theta)
\end{matrix}\right]\]
so in this way \cite{universal-qubit} was already dealing with eigenvalues, and we simply generalize this. That said \cite{universal-qubit} deals with rotations of the Bloch sphere, which means the eigenvalues of the original operation being considered might be different to this. Further, the Bloch sphere is designed to remove global phase factors, and so an operation like $e^{i2\pi\theta}I$ will act trivially on the Bloch sphere, whereas under our eigenvalue analysis this operation might have infinite order despite being trivial in practice. This means that if we want to show that a matrix has infinite order, then we must show that some scalar multiple of it has some eigenvalues which are roots of unity, and others which are not. For example let $\theta$ be irrational, so that the unit complex number $e^{i2\pi\theta}$ is not a root of unity, then
\[A = \left[\begin{matrix}
	e^{i2\pi\theta} & 0 \\
	0 & e^{-i2\pi\theta}
\end{matrix}\right]\]
will be infinite order, but so will any non-zero scalar multiple $\lambda A$, since if $\lambda e^{i2\pi\theta}$ is a root of unity, then $\lambda = e^{i2\pi(\phi - \theta)}$ with $\phi$ rational, meaning that $\lambda e^{-i2\pi\theta}$ still won't be a root of unity, and vice versa.
\subsection{Qudit versions of the qubit ``pi-over-eight'' gate}
In searching for a generalization of the $T$ gate from binary to higher dimensions, there are a number of properties that could be used to find such gates. For example one could naively define the family of gates:
\[T_n = Z_n^{1/4} = \left[\begin{matrix}
	1 & 0 & 0 & \dots & 0 \\
	0 & \omega_{4n} & 0 & \dots & 0 \\
	0 & 0 & \omega_{4n}^2 & \dots & 0 \\
	\vdots & \vdots & \vdots & \ddots & \vdots \\
	0 & 0 & 0 & \dots & \omega_{4n}^{n-1} \\
\end{matrix}\right]\]

This takes the defining property of $T$ and generalizes it, which is exactly what we do to define the rest of the objects defined in this thesis. In this case however, it is much more useful to generalize two other properties that $T$ has, properties which are not unique to $T$, but which are essential for its power in discussions of universal computation. The first property is that $T$ is diagonal. The equation $T^4 = Z$ seems to be a coincidence of low dimension, rather than a general property of the diagonal matrices of interest to us. The second property is that $T$ is in the second level of the Clifford hierarchy, as defined in \cite{clifford-hierarchy}:

[capital D definition] The Clifford hierarchy is an infinite sequence of sets $\mathcal{C}_i$ where $\mathcal{C}_1$ is the Pauli group, $\mathcal{C}_2$ is the Clifford group, and generalizing this
\[\mathcal{C}_{i+1} = \{A\ |\ A\mathcal{C}_1A^{-1} \subseteq \mathcal{C}_i\}\]
Note that only the first two levels of this hierarchy form groups, the rest are simply different sets of unitary matrices.
[do we have cosets and double cosets?]

We see that for qubits, $T \in \mathcal{C}_3$, and so to generalize $T$ to any prime dimension $p$, the paper \cite{pi-over-eight} finds the full (finite) solution set $\{U_v\}$ satisfying $U_v \in \mathcal{C}_3$, $U_v$ diagonal. Global phase is also removed by assuming that $U_v\ket{0} = \ket{0}$. They also show that this solution set forms a group, analyze this group to determine its structure and minimum number of generators. Additionally they explore the geometric properties of these generalized gates, and use this to argue for the strength of these generalized gates as being optimally resistant to multiple forms of noise in physical implementations.

When exploring the group theoretic structure of these solution sets, they found that $p = 2$ and $p = 3$ gave groups that were generated by $1$ and $2$ elements respectively, whereas for $p \geq 5$ there were always $3$ generator elements. This also appears to be a coincidence of low dimension akin to the relation $T^4 = Z$. In addition to both of these properties of low dimension, the paper \cite{arithmetics} shows that $P_9 \in \mathcal{C}_3$ along with the rest of the Clifford group $\mathcal{C}_2$ can be used to exactly generate \emph{all} of the permutation matrices in a system made of multiple qutrits, and interestingly $P_9$ can be scaled to satisfy the relation $P_9^3 = Z_3$.
\subsection{Supermetaplectic Basis}
The set of matrices generated by $\{X_3, H_3, D_3, \text{SUM}, P_9\}$, [define SUM, give as an example of a Clifford gate in any dimension, alongside SWAP] or equivalently by $\mathcal{C}_2 \cup \{P_9\}$, asymptotically [define this in the universal computation discussions] cover all of the matrices in $U\left(3^n\right)$, but were shown in \cite{arithmetics} to exactly implement all permutation matrices, as opposed to the metaplectic basis demonstrated in \cite{topological-anyon-thing}, [if this paper is the one that proves that $\mathcal{C}_3 + 1$ is always universal for qutrits (or more?) then that could be relevant, since what we show later is an example of $\mathcal{C}_3 + 2$ being universal for qubit qutrit hybrids, which seems worse, suggesting optimization] which was universal but is conjectured to implement some permutation matrices asymptotically but not exactly. Due to its increased coverage of matrices this new basis was called the supermetaplectic basis, a highly decorated name.

In order to show that this generator set exactly implements the permutation matrices in $U\left(3^n\right)$ they decompose these matrices in 3 steps.
\begin{enumerate}
	\item It seems to have been taken implicitly that with an implementation of $C_{i_1 = 2}(X)$, other permutations acting on more than 2 qutrits can be decomposed in a manner similar to the qubit process described previously.
	\item Various permutations acting on 2 or 3 qutrits that are not in $\mathcal{C}_2$, including arbitrary transpositions on qutrit pairs, were shown to be equivalent to each other
	\item Two of these qutrit permutations including $C_{i_1=2}(X)$ itslef, were each shown to be the Fourier transform of diagonal matrices in the supermetaplectic basis, meaning the permutations themselves are in this basis.
\end{enumerate}

Both of these explicit steps were derived through analysis of the permutation and diagonal matrices as polynomial expressions, first interpreting permutation matrices as acting on the computational basis, e.g.
\[C_{i_1=0}(X)(\ket{i}\ket{j}) = \ket{i}\ket{j+1-i^2 \mod 3}\]
then interpreting diagonal matrices as acting on the phase factors of the computational basis, e.g.
\[C_{i_1=0}(Z)(\ket{i}\ket{j}) = \omega_{3}^{j(1-i^2)}\ket{i}\ket{j}\]
While the exact details of these proofs are interesting, the only thing relevant to say in this context is that this technique will not map to mixed contexts in any straight forward way. The key to making each of these formulations work is that $1 - i^2 \mod 3$ is 0 if $i$ is \emph{either} 1 or 2. In a mixed context where say $i$ is binary and $j$ is ternary, we can write expressions like $\ket{i}\ket{j} \mapsto \ket{i}\ket{i + j \mod 3}$, but we cannot compose these polynomial expressions meaningfully. [It's been a while since I have thought about this topic, can't remember the details. not sure where zero divisors came up exactly. would be good just to get a couple examples down, maybe more detail on what \cite{arithmetics} actually needed, and then a concise example of how that goes wrong in mixed contexts. I think the argument is: 1. you can't work mod 2 or mod 3 because of composition, 2. you can't work mod 6 because of zero divisors... if the reasoning is this drawn out it seems more like a secondary result, so I think I'll end up putting it in an appendix.]


%\chapter[CALCULATION IN BINARY-TERNARY]{Calculation in Binary-Ternary}
[seems like the chapters we really want are universal computation, finite groups, and then maybe the ternary arithmetic/magic state/fault tolerant stuff?? might be a big restructure, and hides the fact that understanding the background material is not just a means to an end in this document, but an end of this document in itself.]

[explain the context we infer from the previous discussion: interesting work on non-binary computation, esp. ternary, but very small amounts on mixed computation, meaning we ought to start with the absolute basics... maybe also speculate on what the trade-offs of the different number systems seem to be?]

\section{Programmatic Search for Finite Groups}
Normally the Clifford group is an excellent tool for reasoning about quantum systems, since it is finite, (up to scalar factors [I should be careful about my previous statements about generators of the Clifford group, maybe define what a `finite Clifford group' is]) and cheap to implement, but somehow maximal in that one can add just a single gate to achieve universal computation. We saw in \cite{tolar-clifford}, however, that the Clifford group only introduces operations that act dependently between objects when the dimensions of those objects have a common factor, which 2 and 3 do not share, essentially splitting a mixed binary-ternary quantum computer into two separate computers, with two separate Clifford groups, only able to communicate via non-Clifford operations. There is no reason why this can't be done, but there isn't a clear advantage either, so in this section we will take the generators of the Clifford group,
\[X_2, Z_2, H_2, D_2, X_3, Z_3, H_3, D_3,\]
[have I defined $H_d$ and $D_d$?]
and different combinations of permutation matrix, starting with $C_2(X_3)$ and $C_3(X_2)$, to create different generators of subgroups of $U(6)$. We implement a breadth first search in the groups generated by these sets using the C programming language, and find some combinations which yield finite groups, and others that cannot be fully enumerated without crashing due to integer overflow, which we will present and discuss.

The first group we will discuss is the one generated by removing $H_2$ and $H_3$ from the generator set, and adding both $C_2(X_3)$ and $C_3(X_2)$ in their place. All of these generators sit inside of the generalized symmetric group
\[\{AB\ |\ A\text{\ diagonal}, A^6 = I, A \in U(6), B \in \mathcal{S}_6\}\]
[define in general, and prove this is a group, presumably in the groups discussion where we needed examples of normal subgroups anyway]
which means that they generate some subgroup of this.

In fact $X_2$, $X_3$, $C_2(X_3)$ and $C_3(X_2)$ will turn out to generate the whole symmetric group on 6 elements, so having them all would be quite powerful for computation, but now we run into a conjugate problem, which is that our group has a way of doing nontrivial things with entangled states, but no longer any way of creating states other than simple multiples of the computational basis! This generalized permutation group provides essential operations, but isn't much more powerful than the Pauli group without $H_2$ or $H_3$, so our primary focus will be understanding the trade-offs between different combinations of $H_2$, $H_3$, $C_3(X_2)$, and $C_2(X_3)$.
\subsection{Representing Complex Matrices in C}
It would be standard to use some algebraic program or package such as Magma to solve this problem, but because our goal is simple enough to implement in a handful of sittings, it was just as easy to implement it from scratch in C, yielding a program that is simple, high performance, and easy to customize. The full source code of this program is given in Appendix C, and was tested using the Tiny C Compiler, but run using the Clang compiler with optimization level 2. We will now outline the techniques used, since they are of independent interest. 

The first obstacle in not using an algebraic package is the exact representation of numbers whose binary expansion never terminates or repeats, for example the sixth root of unity
\[\omega_6 = e^{\frac{i\pi}{3}} = \frac{\sqrt{3}}{2} + \frac{1}{2}i\]
To solve this we aim to represent the field extension
\begin{align*}
	\text{Num} = \mathbb{Q}[\sqrt{-1}, \sqrt{2}, \sqrt{3}]
	&= \left\{ \sum_{j,k,l = 0}^1 n_{jkl}\ i^j\left(\sqrt{2}\right)^k\left(\sqrt{3}\right)^l
	\ \middle|\ n_{jkl} \in \mathbb{Q}\right\}
\end{align*}
using an array of 32-bit integers, eight signed integers representing the numerator in $\mathbb{Z}[\dots]$, and one unsigned ineger for the denominator.

We have a few basic operations to define in Num, the most significant of which are multiplication, and inverses. We implement these algorithms naively, and since these are the most fundamental building blocks of our program, this will lead to very bad performance without optimization, (in fact we found the program runs four times worse, 20 seconds vs 5 seconds) but when the search is going to take numerous seconds anyway, the additional second or two of optimization is less painful than it would be in other projects.

For multiplication in Num, we implement multiplication in $\mathbb{Z}[X, Y, Z]$, [oh no do I have to define field extensions? Might as well throw it in the appendix I guess] iterating 6 variables over the range ${0, 1}$ according to the equality
\begin{align*}
	\left(\sum_{j,k,l = 0}^1 a_{jkl}X^jY^kZ^l\right)\left(\sum_{j,k,l = 0}^1 b_{jkl}X^jY^kZ^l\right)
	\\= \sum_{j_1,k_1,l_1=0}^1\sum_{j_2,k_2,l_2=0}^1 a_{j_1k_1l_1}b_{j_2k_2l_2}X^{j_1+j_2}Y^{k_1+k_2}Z^{l_1+l_2}
\end{align*}
, then we make three reductions, according to the rules $x^2 = -1$, $y^2 = 2$, and $z^2 = 3$, returning us to Num. When calculating the product of two elements of Num we generally also divide out any common factors of the nine integers in our representation, so that each number has a unique 36 byte representation in memory.

To calculate the inverse of $a \in \text{Num}$, we set $x_0 = a$ and $y_0 = 1$, make repeated modifications of the form $x_{j+1} = x_jc_j$, $y_{j+1} = y_jc_j$. In this way we eventually get $x_k = 1$, while preserving the identity $x_j = ay_j$, meaning $y_k = a^{-1}$. We set $c_0$ to the denominator of $x_0$ so that $x_1 \in \mathbb{Z}[i, \sqrt{2}, \sqrt{3}]$, then remove each radical by setting $c_k$ to the conjugate of $x_k$. That is $c_1$ is $x_1$ but with $i$ replaced by $-i$, so that $x_2 \in \mathbb{Z}[\sqrt{2}, \sqrt{3}]$. Repeating for $\sqrt{2}$ gives $x_3 \in \mathbb{Z}[\sqrt{3}]$, and then repeating for $\sqrt{3}$ gives $x_4 \in \mathbb{Z}$. Now we can set $c_4 = 1/x_4$ to get $x_5 = 1$ and hence $y_5 = a^{-1}$. For example if $a = \omega_{6} = (\sqrt{3} + i)/2$, then the process looks like this:
\begin{align*}
	x_0 &= (\sqrt{3} + i)/2 & y_0 &= 1 & c_0 &= 2 \\
	x_1 &= \sqrt{3} + i & y_1 &= 2 & c_1 &= \sqrt{3}-i \\
	x_2 &= 4 & y_2 &= 2\sqrt{3}-2i & c_2 &= 4 \\
	x_3 &= 16 & y_3 &= 8\sqrt{3}-8i & c_3 &= 16 \\
	x_4 &= 256 & y_4 &= 128\sqrt{3}-128i & c_4 &= 1/256 \\
	x_5 &= 1 & y_5 &= (\sqrt{3}-i)/2 & \\
\end{align*}
In practice we could have stopped earlier, in fact for any root of unity it will be enough to multiply by the complex conjugate, but checking for these conditions for early termination is very likely to be slower for a CPU than to simply finish the calculation as we have above.

Next we represent a complex matrix as a $6\times 6$ array of Num, taking a total of 1296 bytes per matrix! Matrix multiplication is defined in the usual way, by setting $c_{ik} = \sum_j a_{ij}b_{jk}$, but reduced matrix multiplication was also defined, where the matrix is scaled so that the first non-zero entry of the matrix is exactly 1. We also represent a permutation in $\mathcal{S}_6$ as 6 bytes, representing a lookup table of a map $\{1\dots 6\} \to \{1 \dots 6\}$.

\subsection{The Search Algorithm}
We now have an implementation of two groups, $GL(6, \text{Num})$, [define $GL$] and $\mathcal{S}_6$, so now we can implement the search itself. The search has six non-optional parameters:
\begin{enumerate}
	\item \verb`gen_len` the amount of `letters' in the generator set
	\item \verb`gen`, the array of letters, each of type \verb`void*`
	\item \verb`names`, string data to print when describing new words that were found
	\item \verb`elem_size`, the size in memory of an element of the group
	\item \verb`compose`, a binary operation acting on group elements
	\item \verb`print_elem`, a procedure to display the group element (matrix or permutation) that has been generated
\end{enumerate}
The search then generates an array of \verb`PathNode`s, (named after the geometric interpretation of groups as `Cayley graphs' [I could change the source code so that I don't need to explain the weird names!]) which are tuples containing
\begin{itemize}
	\item the word length of the corresponding element generated
	\item a pointer to an earlier \verb`PathNode`, called the predecessor; may be \verb`NULL`
	\item the leftmost letter of the word, which when added to the predecessor gives the current word in question
	\item a \verb`void*` pointer to the unique group element that was found, called \verb`result`
\end{itemize}
The procedure also prints each object as it generates them, for example:
\begin{verbatim}
	X3 D2 X2:
	0, 0, 0, 0, 0, -i, 
	0, 0, 0, -i, 0, 0, 
	0, 0, 0, 0, -i, 0, 
	0, 0, 1, 0, 0, 0, 
	1, 0, 0, 0, 0, 0, 
	0, 1, 0, 0, 0, 0, 
\end{verbatim}

The algorithm itself is a breadth first search, where the array of \verb`PathNode`s is also used as the queue of elements not yet searched. The array is initialized to contain the letters themselves, as words of length 1 with no predecessor, and a pointer \verb`curr` is initialized to the start of the array. Then new group elements are generated via \verb`compose(gen[i], curr->result)`, checked for uniqueness, and if they are unique they are appended to the array. Once \verb`curr` reaches the end of the array, we will have proven that the group is finite, and have enumerated all of its elements.

Checking uniqueness becomes quite slow, the biggest group that was exhaustively generated had 165888 matrix elements, meaning the minimum number of matrix comparisons needed would have been 13759331328, which is like comparing 16 terabytes of data, one kilobyte at a time. In order to overcome this we also implement an open addressed hash map, using a technique called Robin Hood hashing. [quick reference for this I guess!] The hash map takes the \verb`void*` corresponding to the group element in question, and generates a hash by literally shuffling bits around using the XOR and SHIFT operations offered in C. The hash is then used as an offset into an additional array, called the hash table, where another pointer to the group element will be stored. When a new group element is calculated, rather than search for that group element within a whole megabyte of existing elements, we simply calculate its hash, and check if the element we are looking for is in the correct location in the hash table.

Things are complicated when two group elements correspond to the same location in the hash table, which is where the lookup algorithm must `probe' for other locations near the offset associated with the hash, to store the new entries there instead. Robin hood hashing/probing is a small optimization on this concept, allowing entries to be moved again after being written the first time, to make room for other entries that would otherwise end up too far from their original offset; minimizing this `probe distance' from the original offset of a hash entry makes lookups faster, especially as the hash table becomes full. Another technical detail is the fact that since we never remove entries from the hash table, and only run the program once before exiting, we don't need to allocate, reallocate, ore remove entries from the hash table; we simply use a global block of memory, and implement lookup and insert procedures, and we are done.

With this technology in place, all that remains is to wire it together, with definitions for \verb`compose`, and \verb`print_elem`. For permutations, we compose $x$ and $y$ by setting \verb`xy[i] = x[y[i]]`, and for matrices we multiply, and divide out the first non-zero scale factor, essentially implementing the quotient group $U(6)/U(1)$. We also check the order of the matrix, as a simple heuristic for detecting infinite order groups; If we can calculate the order without causing an integer overflow then we continue generating the group, but if any of the 324 integers in our matrix representation are above some threshold, we assume the order is infinite, represented by \verb`-1`, and print this fact, to indicate that the search is unlikely to terminate. We actually continue regardless of the order, since it might be worth knowing which words of similar size have infinite order as well.

\subsection{Results}
Call the six generators of the Clifford group other than $H_2$ and $H_3$ the reduced generator set $R$. [we haven't defined generators at all!] Then $\langle R, H_2, H_3 \rangle$ is a finite Clifford group, and $\langle R, C_2(X_3), C_3(X_2) \rangle$ is a subgroup of the generalized symmetric group and therefore finite as described previously, but additionally, $\langle R, H_2, C_3(X_2)\rangle$ and $\langle R, H_3, C_2(X_3)\rangle$ are finite. [suddenly realizing that the orbits of these groups would be easier to analyze than the groups themselves, since the orbits implicitly quotient out permutations on the computational basis, and by removing global phase matrices will have no effect on their eigenvectors as well]

The last two generators we could consider, $\langle R, H_2, C_2(X_3)\rangle$, and $\langle R, H_3, C_3(X_2)\rangle$, both trigger an integer overflow when calculating the order of the products $C_2(X_3)H_2$, and $C_3(X_2)H_3$.

The exact values of these matrices are
\begin{align*}
	C_2(X_3)H_2 &=
	\frac{1}{\sqrt{2}}
	\begin{bmatrix}
		1&0&0&0&0&0\\
		0&1&0&0&0&0\\ 
		0&0&1&0&0&0\\ 
		0&0&0&0&0&1\\ 
		0&0&0&1&0&0\\ 
		0&0&0&0&1&0
	\end{bmatrix}
	\begin{bmatrix}
		1&0&0&1&0&0\\ 
		0&1&0&0&1&0\\ 
		0&0&1&0&0&1\\ 
		1&0&0&-1&0&0\\ 
		0&1&0&0&-1&0\\ 
		0&0&1&0&0&-1
	\end{bmatrix}
	\\&=
	\frac{1}{\sqrt{2}}
	\begin{bmatrix}
		1&0&0&1&0&0\\ 
		0&1&0&0&1&0\\ 
		0&0&1&0&0&1\\ 
		0&0&1&0&0&-1\\ 
		1&0&0&-1&0&0\\ 
		0&1&0&0&-1&0
	\end{bmatrix}
\end{align*}
For a $6\times6$ matrix this seems fairly innocuous, and the other doesn't look much worse
\begin{align*}
	C_3(X_2)H_3 &=
	\frac{1}{\sqrt{3}}
	\begin{bmatrix}
		1&0&0&0&0&0\\
		0&1&0&0&0&0\\ 
		0&0&0&0&0&1\\ 
		0&0&0&1&0&0\\ 
		0&0&0&0&1&0\\ 
		0&0&1&0&0&0
	\end{bmatrix}
	\begin{bmatrix}
		1&1&1&0&0&0\\ 
		1&\omega_3&\omega_3^2&0&0&0\\ 
		1&\omega_3^2&\omega_3&0&0&0\\ 
		0&0&0&1&1&1\\ 
		0&0&0&1&\omega_3^2&\omega_3\\ 
		0&0&0&1&\omega_3&\omega_3^2
	\end{bmatrix}
	\\&=
	\frac{1}{\sqrt{3}}
	\begin{bmatrix}
		1&0&0&0&0&0\\
		0&1&0&0&0&0\\ 
		0&0&0&0&0&1\\ 
		0&0&0&1&0&0\\ 
		0&0&0&0&1&0\\ 
		0&0&1&0&0&0
	\end{bmatrix}
	\begin{bmatrix}
		1&1&1&0&0&0\\ 
		1&\frac{-1 +\sqrt{3}i}{2}&\frac{-1 -\sqrt{3}i}{2}&0&0&0\\ 
		1&\frac{-1 -\sqrt{3}i}{2}&\frac{-1 +\sqrt{3}i}{2}&0&0&0\\ 
		0&0&0&1&1&1\\ 
		0&0&0&1&\frac{-1 +\sqrt{3}i}{2}&\frac{-1 -\sqrt{3}i}{2}\\ 
		0&0&0&1&\frac{-1 -\sqrt{3}i}{2}&\frac{-1 +\sqrt{3}i}{2}
	\end{bmatrix}
	\\&=
	\frac{1}{\sqrt{3}}
	\begin{bmatrix}
		1&1&1&0&0&0\\ 
		1&\omega_3&\omega_3^2&0&0&0\\
		0&0&0&1&\omega_3^2&\omega_3\\ 
		0&0&0&1&1&1\\ 
		0&0&0&1&\omega_3&\omega_3^2\\
		1&\omega_3^2&\omega_3&0&0&0
	\end{bmatrix}
\end{align*}
[keeping the middle entry for now for Symbolab]

But if we raise the simpler of these to the fourth power, it becomes quite chaotic:
\[\left(C_2(X_3)H_2\right)^4 =
\frac{1}{4}
\begin{bmatrix}
	2&-1&3&0&-1&1\\ 
	3&2&-1&1&0&-1\\ 
	-1&3&2&-1&1&0\\ 
	-1&1&0&3&-1&2\\ 
	0&-1&1&2&3&-1\\ 
	1&0&-1&-1&2&3
\end{bmatrix}
\]

The second power is $C_2(X_3)H_2C_2(X_3)H_2$, which resembles the infinite order term from \cite{universal-qubit}, a non-Clifford matrix, times the Fourier transform of that matrix, [I haven't explained why I call it the Fourier transform of the matrix] and when we ask Wolfram Mathematica for its characteristic polynomial we get
\[1/4 (\lambda - 1)^2 (4 \lambda^4 + 2 \lambda^3 - 3 \lambda^2 + 2 \lambda + 4)\]
This rational polynomial has an irreducible factor that is cyclotomic and another that is non-cyclotomic, which by the reasoning used in \cite{universal-qubit} tells us that as we expected, one of its eigenvalues is a root of unity, and another is not, meaning no scale multiple of this matrix is finite order, and so $\langle R, H_2, C_2(X_3) \rangle$ really is an infinite group. Further $C_2(X_3)H_2$ is also infinite order, since two of its eigenvalues will square to roots of unity, and are therefore also roots of unity, whereas the other four will square to non-roots of unity, and are therefore also not roots of unity.

Inspired by this, we could calculate
\[C_3(X_2) H_3^{-1} C_3(X_2) H_3 =
\frac{1}{3} \begin{bmatrix}
	2 & \omega_6 & \omega_6^2 & 1 & \omega_6^4 & \omega_6^5 \\
	\omega_6^2 & 2 & \omega_6 & \omega_6^5 & 1 & \omega_6^4 \\
	\omega_6^4 & \omega_6^5 & 1 & \omega_6 & \omega_6^2 & 2 \\
	1 & \omega_6^4 & \omega_6^5 & 2 & \omega_6 & \omega_6^2 \\
	\omega_6^5 & 1 & \omega_6^4 & \omega_6^2 & 2 & \omega_6 \\
	\omega_6 & \omega_6^2 & 2 & \omega_6^4 & \omega_6^5 & 1
\end{bmatrix}
\]
[a. I still need Mathematica up before I can get a polynomial for this b. this doesn't necessarily have the novelty of the $H_2$ case? I don't know the geometry]

There are 720 different permutations in $\mathcal{S}_6$, 12 of which are already in the Clifford group, and we have shown manually that one of the remaining 708 is infinite order when multiplied with $H_2$. It is quite difficult to prove that a matrix is infinite order using this method, especially when it isn't sparse the way $H_2$ is, and often the characteristic polynomial has $\sqrt{2}$ or $\sqrt{3}$ coefficients which would require deeper number theory than we have presented here, so we return now to the search for matrices that we can prove are finite order. We can write another C program extending the existing codebase to exhaustively generate all $6^6$ tables of integers, and generate matrices for the $720$ that are permutations. Then we attempt to calculate the order of $P H_2$ and $P H_3$, and report when either or both have a finite order that we could calculate. Other than the 12 Clifford permutations, the only $P$ for which both $P H_2$ and $P H_3$ were finite order were the 12 matrices of the form
\[X_2^a C_2(S_{p,q}) X_2^b\text{\ where\ }a, b \in \{0, 1\}, p, q \in \{0, 1, 2\}, p \neq q.\]
This seems promising, but in fact such $P$ will still have $\langle P, H_2, H_3\rangle$ infinite order, since $P H_3^2 = P S_{1,2}$ will not be in the above form, meaning $\langle P, H_2, H_3 \rangle$ is still infinite order, and so the only time where $\langle P, H_2, H_3 \rangle$ can be finite is if $P \in \mathcal{C}_2$ anyway, meaning $P$ is simply the Kronecker product of a qubit operation and a qutrit operation. [need to emphasise this a little more, maybe just theorem it, something I'm going to do anyway]

The final result of this program is that we can use it to check that
\[\langle X_2, C_3(X_2), X_3, C_2(X_3) \rangle = \mathcal{S}_6,\]
This is how we know that
\[C_2(S_{p,q}) \in \langle X_2, C_3(X_2), X_3, C_2(X_3) \rangle\]
which is where we found the word for $C_2(S_{p,q})$ given in \ref{}. Additionally since we have $C_2(X_3) = C_2(S_{0,1})X_3C_2(S_{0,1})X_3$, we can infer that
\[\langle X_2, C_3(X_2), X_3, C_2(S_{0,1})\rangle = \mathcal{S}_6\]
also. These results are interesting but it is important not to forget about the qubit-qubit and qutrit-qutrit control gates which were essential for universal computation, which are concealed by this myopic discussion of $U(6)$.

\subsection{Non-Clifford States}
[I should be talking about magic states sooner as a part of fault tolerance computation and as a motivation for having finite gate sets, I could write some stuff about the following circuits but I need to know what I've actually set up at this point]

\begin{quantikz}
	\lstick{qubit $\ket{0}$} & \gate{H} & \gate{T} & \targ{} & \qw \rstick{$\ket{0}+\omega_4\ket{1}$}\\
	\lstick{qubit $\ket{\phi}$} & \qw & \qw & \phase{0}\vqw{-1} & \qw \rstick{$\omega_4^3T\ket{\phi}$}
\end{quantikz}

\begin{quantikz}
	\lstick{qubit $\ket{0}$} & \gate{H_2} & \phase{1} \vqw{1} & \gate{H_2} & \meter{0/1} & \qw\rstick{$\ket{0}$ or $\ket{1}$}\\
	\lstick{qutrit $\ket{0}$} & \qw & \targ{} & \qw & \qw & \qw \rstick{$\ket{+_{0,1}}$ or $\ket{-_{0,1}}$}
\end{quantikz}

\begin{quantikz}
	\lstick{qubit $\ket{0}$} & \qw & \targ{} & \meter{0/1} & \qw \rstick{$\ket{0}$ or $\ket{1}$} \\
	\lstick{qutrit $\ket{0}$} & \gate{H_3} & \phase{2} \vqw{-1} & \qw & \qw \rstick{$\ket{+_{0,1}}$ or $\ket{2}$}
\end{quantikz}
\appendix % switches to appendix mode. Do this instead of % app0.tex (file to switch to appendix mode)
% No need to alter this file...
\appendix

% app1.tex (will be Appendix A)

\chapter[OTHER DEFINITIONS AND PROPOSITIONS]{Other Definitions and Propositions}\label{formalities}

\section{Algebraic Structures}

The definitions and propositions in this section and also in \autoref{finite-dim} are assumed knowledge, and are thus not explained in great detail, but are given anyway, so as to make this thesis more complete and less ambiguous.

\begin{define}[Binary Operation] A \emph{binary operation} is a map of the form $A \times B \to C$, where $A$, $B$, and $C$, are some sets.
\\In particular, when we say $\cdot$ is a \emph{binary operation on $A$} we mean that it is a map of the form $A \times A \to A$. We write $x \cdot y$ infix to mean the image of $(x, y)$ under the map.
\end{define}

\begin{define}[Associativity] A binary operation $\cdot$ on a set $A$ is \emph{associative} if, for every $a, b, c \in A$, $a \cdot (b \cdot c) = (a \cdot b) \cdot c$.
\end{define}

The point of associativity, of course, is to identify $a \cdot (b \cdot c)$ with $(a \cdot b) \cdot c$, writing both $a \cdot b \cdot c$.

\begin{define}[Commutativity] A binary operation $\cdot$ on a set $A$ is \emph{commutative} if, for every $a, b \in A$, $a \cdot b = b \cdot a$.
\end{define}

\begin{define}[Identity] Given a binary operation $\cdot$ on a set $A$ we say that $e \in A$ is an \emph{identity} if, for every $a \in A$, $a \cdot e = e \cdot a = a$.
\end{define}

\begin{prop}[Uniqueness of Identities] If $e_1$ and $e_2$ are identities of a binary operation $\cdot$ on $A$, then $e_1 = e_2$.
\end{prop}
\begin{proof}
By definition of $e_2$, $e_1 = e_1\cdot e_2$, then by definition of $e_1$, $e_1\cdot e_2 = e_2$. By transitivity this gives $e_1 = e_2$.
\end{proof}

This proposition tells us that when a set has an identity, it follows that it has only one identity, which we can give a name. We will not dwell on any other foundational propositions of abstract algebra like this.

\begin{define}[Group] A pair $(A, \cdot)$ is a \emph{group} if $\cdot$ is an associative binary operation on $A$, with some identity $e$, and, for every $a \in A$ there is a unique inverse $y$, satisfying $x\cdot y = y \cdot x = e$. If $\cdot$ is commutative then $(A, \cdot)$ is said to be a \emph{commutative group}, or an \emph{Abelian group}.
\end{define}

\begin{define}
	Given a positive integer $n$, the \emph{symmetric group} $\mathcal{S}_n$ is the set of bijections, or \emph{permutations}, that map $\{0,1,\dots,n-1\}\to\{0,1,\dots,n-1\}$.
\end{define}
\begin{prop}
	For any positive integer $n$, $(\mathcal{S}_n, \circ)$ is in fact a group.
\end{prop}
\begin{proof}
	Function composition $\circ$ is always associative.\\
	Define id to be the map $i \mapsto i$. Then $x \circ \text{id} = \text{id} \circ x = x$, giving us an identity element.\\
	Finally since $x \in \mathcal{S}_n$ is bijective, we can use the function inverse $y = x^{-1}$ to get $xy = yx = \text{id}$.
\end{proof}

From this point on if we talk about a set $A$ which has only one relevant way of being interpreted as a group, or more generally has only one way of being interpreted as an algebraic structure $(A, \alpha_1, \dots, \alpha_n)$, then we will treat $A$ as if it is itself the group, etc. For example if we talk about the group $\mathbb{Z}$, we mean $(\mathbb{Z}, +)$, not any other group which happens to be well defined.

\begin{define}\label{induced-subgroup}[Induced operations] Given a set $A$ which is a subset of a set $B$, and a binary operation $\cdot: B \times B \to B$, we say that $\cdot$ \emph{induces a binary operation in $A$} if for every $a_1, a_2 \in A$, the product $a_1 \cdot a_2$ is also in $A$.
\end{define}
The induced operation being referred to in this phrase is $\cdot$ with the restricted domain $\cdot|_A: A \times A \to B$, since it will now be a well defined operation on $A$ in that $\cdot|_A: A \times A \to A$.

\begin{define}[Subgroups] Given a group $(B, \cdot)$, or more generally just a binary operation $\cdot$ on a set $B$, and a set $A$ which is a subset of $B$, we say that $A$ is a \emph{subgroup} of $B$ under $\cdot$, if $\cdot$ induces a binary operation on $A$ and this binary operation forms a group $(A, \cdot)$.
\end{define}

\begin{define}[Distributivity] A binary operation $\cdot$ on a set $A$ is said to \emph{distribute over} a binary operation $+$ on the same set $A$, if for every $a, b, c \in A$, the identities
\[a \cdot (b + c) = a\cdot b +a\cdot c,\]
and
\[(b + c) \cdot a = b \cdot a + c \cdot a\]
both hold. More generally if $\cdot$ is a binary operation $A \times B \to C$, and $+_A, +_B, +_C$ are binary operations on $A$, $B$, and $C$ respectively, then $\cdot$ \emph{distributes over} these $+$ operations if for every $a_1, a_2 \in A$, $b_1, b_2 \in B$, the identities
\[a_1 \cdot (b_1 +_B b_2) = a_1 \cdot b_1 +_C a_2 \cdot b_2,\]
and
\[(a_1+_A a_2)\cdot b_1 = a_1 \cdot b_1 +_C a_2 \cdot b_1\]
both hold.
\end{define}

A simple example of a distributive operation is multiplication $\mathbb{R} \times \mathbb{R} \to \mathbb{R}$, distributing over addition. An example of the more general distributive operator $A \times B \to C$ is matrix multiplication, taking an $l \times m$ matrix and an $m \times n$ matrix, and producing an $l \times n$ matrix.

\begin{define}[Additivity, Multiplicativity] To aid intuition, and to guide notation, when $\cdot: A \times B \to C$ distributes over $+_A, +_B, +_C$, and these binary operations form commutative groups $(A, +_A)$, $(B, +_B)$, $(C, +_C)$, then we call these groups \emph{additive}, and call $\cdot$ \emph{multiplicative}.
\end{define}

We denote binary operations as $+$ whenever it is obvious that they form an additive group, and omit subscripts such as $+_A$ whenever clear. We also denote the identity of an additive group as $0$, and the inverse of $x$ as $-x$. We can then extend summation notation to the additive groups, writing $a_1 + a_2 + \dots + a_n$ as $\sum_{i=1}^n a_n$. Further if $\cdot$ is an associative binary operation on $A$, and is multiplicative, or simply \emph{not} additive, then we can write products $a_1 \cdot a_2 \cdot \dots \cdot c_n$ as $a_1a_2 \dots a_n$, or even $\prod_{i=1}^n a_i$. Further, if $(A, \cdot)$ is in fact a group, then we can write the inverses of $x$ as $x^{-1}$.

\begin{prop}Given an additive group $(A, +)$, summations can be swapped in the sense that
	\[\sum_i\sum_j a_{ij} = \sum_j\sum_i a_{ij}\]
\end{prop}
This can be proven using induction, but when you unpack the notation these two sums are clearly just reordered versions of each other. This allows us to identify these two sums, and optionally combine the quantifiers as $\sum_{i,j} a_{ij}$.

\begin{prop}Given a multiplicative operation $\cdot: A \times B \to C$, distribution can be generalized to
\[(\sum_i a_i)\cdot (\sum_j b_j) = \sum_{i,j} a_ib_j\]
\end{prop}
This is once again proven inductively.

One important special case of multiplicative operations is that of rings.
\begin{define}[Rings] A triple $(R, \cdot, +)$ is said to be a \emph{ring} if $(R, +)$ is a commutative group, $\cdot$ is an associative binary operation $R \times R \to R$, and $\cdot$ distributes over $+$.
\end{define}
Clearly whenever $(R, \cdot, +)$ is a ring, then $\cdot$ will be multiplicative and $(R, +)$ will be an additive group, and so all rings inherit the notation of additive identities 0 and additive inverses $-x$.

\begin{define}[Field, Multiplicative group] A ring $(F, \cdot, +)$ is said to be a \emph{field} if $\cdot$ is commutative and $F \backslash \{0\}$ is a subgroup of $F$ under $\cdot$. In this case we call $(F\backslash\{0\}, \cdot)$ the \emph{multiplicative group} of $F$, which will of course be a commutative group.
\end{define}

Central examples of fields include $\mathbb{Q}$, $\mathbb{R}$, and $\mathbb{C}$. In all of these examples, the identity of the corresponding multiplicative group is 1, so just as we denote all additive identities as 0, we will denote the multiplicative identity of any field $F$ as 1.

\begin{define}[Subfield] Given a field $(F, \cdot, +)$, and a set $A$ which is a subset of $F$, $A$ is said to be a \emph{subfield} of $F$ under $\cdot$ and $+$ if $\cdot$ and $+$ both induce binary operations on $A$ and $(A, \cdot, +)$ is a field.
\end{define}

\begin{define}[Linear Groups]
	Given a field $(F, \cdot, +)$, and a positive integer $n$, the \emph{general linear group} $GL(n, F)$ is the set of $n \times n$ matrices with determinant not equal to 0. The \emph{special linear group} $SL(n, F)$ is the set of $n \times n$ matrices with determinant equal to 1.
\end{define}

\begin{prop}
	The general linear group $GL(n, F)$ is in fact a group under matrix multiplication, and the special linear group $SL(n, F)$ is a subgroup of this.
\end{prop}
This follows from the multiplicative property of the determinant.

Related to fields are another important case of a multiplicative operation, the vector space.
\begin{define}[Vector Space] Given a field $(F, \cdot_F, +_F)$, a triple $(V, \cdot_V, +_V)$ is said to be a \emph{vector space over the field $F$} if all of the following hold:
\begin{itemize}
	\item $(V, +_V)$ is a commutative group,
	\item $\cdot_V: F \times V \to V$ distributes over $+_F$ and $+_V$
	\item for any $a, b \in F$, $v \in V$, the identity $a \cdot_V (b \cdot_V v) = (a \cdot_F b) \cdot_V v$
	\item for any $v \in V$, $1 \cdot_V v = v$.
\end{itemize}
We also refer to elements of $V$ as \emph{vectors}, elements of $F$ as \emph{scalars}, and products of the form $a\cdot_V v$ as \emph{scalar multiplication}.
\end{define}
Since $\cdot_F$ and $\cdot_V$ are `associative' in the above sense, we can write $abv$ or more broadly $a_1a_2\dots a_nv$ without ambiguity, once again identifying all of the possible interpretations of these expressions with each other. Further, since $\cdot_V$ and $\cdot_F$ are of different types, and $+_V$ and $+_V$ are also of different types, we drop all of these subscripts without ambiguity.

\begin{prop}
For any non-negative integer $n$, and field $(F, \cdot, +)$, the set of column vectors $F^n$ forms a vector space.
\end{prop}
Specifically, $F^n$ is the set
\[\left\{\begin{bmatrix}a_0\\a_1\\\vdots\\a_{n-1}\end{bmatrix}\ \middle|\ a_i \in F\right\},\]
and vector addition and scalar multiplication are defined in the usual way
\[\begin{bmatrix}
	a_0 \\ a_1 \\ \vdots \\ a_{n-1}
\end{bmatrix} + \begin{bmatrix}
b_0 \\ b_1 \\ \vdots \\ b_{n-1}
\end{bmatrix} = \begin{bmatrix}
a_0 + b_0 \\ a_1 + b_1 \\ \vdots \\ a_{n-1} + b_{n-1}
\end{bmatrix}\]
\[
a\begin{bmatrix}
	b_0 \\ b_1 \\ \vdots \\ b_{n-1}
\end{bmatrix} = \cdot \begin{bmatrix}
	a b_0 \\ a b_1 \\ \vdots \\ a b_{n-1}
\end{bmatrix}.
\]
Proving that these operations have the required properties is a simple matter of applying the field properties of $F$ coordinate-wise to each of the elements of the column vectors defined by these operations. Note also that $F^0$ will be the set of empty column vectors, but since we inherit the language of additive groups, we can denote this as vector space as the trivial group $\{0\}$.

\section{Finite Dimensional Vector Spaces}\label{finite-dim}
For all of the following definitions suppose that $(F, \cdot, +)$ is an arbitrary field, and $(V, \cdot, +)$ is an arbitrary vector space over $F$.
\begin{define}
Given a finite set of vectors $S = \{u_0, \dots, u_{n-1}\}$ with $u_i \in V$, and a set of scalars $a_0, \dots, a_{n-1}$, the sum
\[v = \sum_i a_i u_i.\]
is said to be a \emph{linear combination} of $u_0, \dots u_{n-1}$. Then the \emph{span} of $S$ is the set of all such linear combinations, written $\vecspan S$ or $\vecspan \{u_0, \dots, u_{n-1}\}$. If $\vecspan S = V$ then $S$ is said to be \emph{spanning}.
\end{define}

\begin{define}
A finite set of vectors $S = \{u_0, \dots, u_{n-1}\}$ is said to be \emph{linearly independent} if the only linear combination of its elements
\[v = \sum_i a_i u_i.\]
with $v = 0$ is the trivial $0 = a_0 = a_1 = \dots = a_{n-1}$. If $S$ is both spanning and linearly independent, then $S$ is said to be a \emph{basis} of $V$.
\end{define}

\begin{prop}
	If $S = \{u_i\}$ is linearly independent, and two linear combinations $\sum_i a_iu_i$ and $\sum_i b_iu_i$ are equal, then each $a_i$ is equal to the corresponding $b_i$.
\end{prop}
\begin{proof}
	We have
	\[\sum_i a_iu_i = \sum_i b_iu_i,\]
	and so by subtracting one side from the other,
	\[\sum_i a_iu_i-b_iu_ = 0.\]
	Now $S$ is linearly independent, so since this linear combination gives the zero vector, it must be the trivial linear combination
	\[(a_i - b_i) = 0 \forall i,\]
	but undoing the subtraction gives
	\[a_i = b_i \forall i.\]
\end{proof}

Now, if every linear combination of a set of vectors will be the unique such linear combination, then we can extract the corresponding scalars in a well defined way.

\begin{define}
If $S$ is a finite set of vectors $\{u_0, \dots, u_{n-1}\}$, then the \emph{coordinates} of a vector $v \in \vecspan S$ are the unique scalars $a_i$ that give
\[v = \sum_i a_i u_i.\]
In particular if $S$ is a basis of $V$ then every vector $v \in V$ will have unique coordinates.
\end{define}

A crucial example of all of the above concepts is $F^n$, equipped with a simple and natural basis.
\begin{define}\label{canonical-basis}
	The \emph{canonical basis} of the set of column vectors $F^n$ is the set of vectors $u_i$, whose $i$th entry is 1 and the rest are 0.\footnote{Later in \autoref{dirac} these will be written $\ket{i}$}
	\[u_0 = \begin{bmatrix}
		1 \\ 0 \\ \vdots \\ 0
	\end{bmatrix}, u_1 = \begin{bmatrix}
		0 \\ 1 \\ \vdots \\ 0
	\end{bmatrix}, u_{n-1} = \dots, \begin{bmatrix}
		0 \\ 0 \\ \vdots \\ 1
	\end{bmatrix}\]
\end{define}

\begin{prop}
	The canonical basis of $F^n$ is in fact a basis of $F^n$.
\end{prop}
\begin{proof}
	We can prove $\{u_i\}$ is spanning by reading the coordinates directly out of $v \in F^n$,
	\[v = \begin{bmatrix}
		a_0 \\ a_1 \\ \vdots \\ a_{n-1}
	\end{bmatrix} = \sum_i a_i u_i.\]
	Then since this formula is arbitrary in the scalars $a_i$, we can read it in reverse to see that every linear combination of $u_i$ will be of the form
	\[v = \sum_i a_i u_i = \begin{bmatrix}
		a_0 \\ a_1 \\ \vdots \\ a_{n-1}
	\end{bmatrix},\]
	so if $v = 0$, then
	\[\begin{bmatrix}
		0 \\ 0 \\ \vdots \\ 0
	\end{bmatrix} = \begin{bmatrix}
		a_0 \\ a_1 \\ \vdots \\ a_{n-1}
	\end{bmatrix},\]
	meaning $a_i$ are all 0 and the linear combination was in fact trivial. This means $\{u_i\}$ is linearly independent, and spanning, which by definition makes it a basis of $F^n$.
\end{proof}

Bases tell us something very important about the structure of $V$, as evidenced by an established fact about bases of vector spaces.
\begin{prop}[Dimension Theorem for Finite Vector Spaces]
	Whenever finite sets $S_1$ and $S_2$ are both bases of $V$, they must have the same size.
\end{prop}

\begin{define}
	When $V$ has at least one finite set $S = \{u_0,\dots,u_{n-1}\}$ that is a basis for $V$, we say that $V$ is \emph{finite-dimensional}, and define the \emph{dimension} $\dim(V)$ to be the unique size of any basis of $V$. Since this includes $S$ this means $\dim(V) = \ord{S}$.
\end{define}

\section{Inner Products, Hilbert Spaces}\label{inner-products}
An important operation in the geometry of vectors is the inner product, which simultaneously indicates the magnitude of vectors, and the angle that different vectors take to each other. While vector spaces are easy to generalize to any field $F$, for inner products we require specific operations defined on $\mathbb{C}$, which restricts generality a lot. Note that we write the complex conjugate of $a \in \mathbb{C}$ as $a^*$.

\begin{define}
	Given a vector space $V$ over a subfield $\mathbb{F}$ of $\mathbb{C}$, a map $(,): V \times V \to F$ is called an \emph{inner product} if for every $u, v, w \in V$, and every $a, b \in \mathbb{F}$, the following identities hold:
	\begin{itemize}
		\item conjugate symmetry: $(u, v) = (v, u)^*$ (where $a^*$ is the complex conjugate of $a$),
		\item right linearity: $(u, av+bw) = a(u, v) + b(u, w)$,
		\item positive definite: $(v, v)$ real, and strictly positive whenever $v \neq 0$.
	\end{itemize}
	We also say that $(V, \cdot, +, (,))$ is an \emph{inner product space} under $\mathbb{F}$.
\end{define}

\begin{prop}
	Every inner product $(,)$ is left conjugate linear, i.e.\ for every $u, v, w \in V$, and every $a, b \in \mathbb{F}$ the identity
	\[(au+bv,w) = a^*(u,w) + b^*(v,w).\]
\end{prop}
\begin{proof}
	\begin{align*}
		(au+bv,w)
		&= (w,au+bv)^*
		\\&= (a(w,u)+b(w,v))^*
		\\&= a^*(w,u)^*+b^*(w,v)^*
		\\&= a^*(u,w)+b^*(v,w).
	\end{align*}
\end{proof}

\begin{prop}
	If for every scalar $a \in \mathbb{F}$ we have $a^* \in \mathbb{F}$, then the map
\[
\left(
\left[\begin{matrix} a_0\\a_1\\\vdots\\a_n\end{matrix}\right]
,
\left[\begin{matrix} b_0\\b_1\\\vdots\\b_n\end{matrix}\right]
\right)
=
\left[\begin{matrix} a_0^*&a_1^*&\cdots&a_n^*\end{matrix}\right]
\left[\begin{matrix} b_0\\b_1\\\vdots\\b_n\end{matrix}\right]
= \sum_{i=0}^{n-1} a_i^*b_i
\]
is an inner product in $\mathbb{F}^n$.
\end{prop}
Once again, this is straight-forward to prove using the algebraic properties of $\mathbb{C}$. This is not the only inner product one can define for $\mathbb{F}^n$, but it is the one we will be using.

\begin{define}[Norm]
	Given an inner product space $(V, \cdot, + , (,))$ over a field $\mathbb{F}$, the \emph{norm} or \emph{length} of a vector $v \in V$ is the non-negative real number
	\[\norm{v} = \sqrt{(v,v)}.\]
\end{define}

\begin{prop}\label{orthog-zero}
	For any vector $v$ in an inner product space $V$, the inner product with zero is $(v, 0) = 0$.
\end{prop}
\begin{proof}
	By basic group arithmetic we have $0 = 0 + 0$, which means
	\begin{align*}
		(v, 0)
		&= (v, 0+0)
		&= (v,0)+(v,0),
	\end{align*}
	subtracting $(v,0)$ from both sides gives $(v,0) = 0$.
\end{proof}

\begin{prop}\label{norm-zero}
	 The length of a vector $v$ in an inner product space $V$ is 0 if an only if the vector is itself 0.
\end{prop}
\begin{proof}
	We have $\norm{0} = \sqrt{(0,0)} = 0$ by \autoref{orthog-zero}. and inversely we have $\norm{v} = \sqrt{(v,v)} > 0$ whenever $v \neq 0$.
\end{proof}

\begin{define}[Orthogonality, Orthonormality]
	In an inner product space $(V, \cdot, +, (,))$ over a field $\mathbb{F}$, we say that a pair of non-zero vectors $u_0, u_1 \in V$ are \emph{orthogonal} if $(u,v) = 0$, and we say that a set of non-zero vectors $u_0, \dots u_k \in V$ are \emph{orthogonal} if every distinct pair of vectors is orthogonal. Further we say this set is orthonormal if every vector in the set has length 1.
\end{define}

\begin{prop}\label{orthog-independent}
	If a set $S$ of non-zero vectors $u_0,\dots,u_k \subset V$ is orthogonal then $S$ is also linearly independent.
\end{prop}
\begin{proof}
	Suppose that $v$ is a linear combination $\sum_i a_iu_i$. Then for each $0 \leq j \leq k$,
	\begin{align*}
		(u_j,v)
		&= \left(u_j,\sum_i a_iu_i\right)
		\\&= \sum_i a_i\left(u_j,u_i\right)
		\\&= 0 + 0 + \dots + 0 + a_j\left(u_j,u_j\right) + 0 + \dots + 0
		\\&= a_j\norm{u_j}^2
	\end{align*}
	If we let $v = 0$ then by \autoref{orthog-zero} we get $0 = a_j\norm{u_j}^2$, and by \autoref{norm-zero} we know that $\norm{u_j}^2$ is non-zero, so every $a_j$ will be zero.
\end{proof}

The converse does not hold, but we can always use the Gram–Schmidt process to construct an orthonormal set from a linearly independent set.

\begin{prop}[Gram–Schmidt]
	For every finite dimensional vector space $(V, \cdot, +)$ over the field $\mathbb{R}$ or $\mathbb{C}$ with inner product $(,)$, there is a basis which is orthonormal in the inner product space $(V, \cdot, +, (,))$.
\end{prop}
This process leads to great generality in ones discussion of inner product spaces in the abstract, but in practice many inner product spaces already have an obvious orthonormal basis, one that may be more algebraically convenient than one that comes out of the Gram-Schmidt process anyway.

\begin{prop}\label{coords-inner-product}
	In an inner product space $(V, \cdot, +, (,))$ over the field $\mathbb{F}$ with orthonormal basis $S={u_0, \dots, u_{n-1}}$, the inner products $(u_i, v)$ will be exactly the coordinates of $v$ in the basis $S$, meaning $v = \sum_{i=0}^{n-1} (u_i,v)u_i$.
\end{prop}
\begin{proof}
	Let the coordinates of $v$ be $a_i$, so that
	\[v = \sum_{i=0}^{n-1}a_iv.\]
	Then calculate
	\begin{align*}
		(u_i, v)
		&= \left(u_i, \sum_j a_ju_j\right)
		\\&= \sum_j a_j\left(u_i, u_j\right)
		\\&= 0 + \dots + 0 + a_i\left(u_i, u_i\right) + 0 + \dots + 0
		\\&= a_i\norm{u_i}^2
		\\&= a_i.
	\end{align*}
\end{proof}

\begin{define}
	A map between two inner product spaces, $\phi: U \to V$ is said to \emph{preserve inner products} if for every $u_1, u_2 \in U$, $(u_1, u_2) = (\phi(u_1), \phi(u_2))$.
\end{define}

Such maps essentially show that $U$ exists embedded inside of $V$. If they are surjective then they show that $U$ is structurally equivalent to $V$.

\begin{prop}\label{vectors-tuples}
	Given an inner product space $(V, \cdot, +, (,))$ over a field $\mathbb{F}$ with an orthonormal basis $S = {u_0, u_1, \dots, u_{n-1}}$, the map $\phi:V \to \mathbb{F}^n$, 
	\[\phi(v) = \left[\begin{matrix}
		(u_0, v)\\
		(u_1, v)\\
		\cdots\\
		(u_n, v)
	\end{matrix}\right]\]
preserves inner products.
\end{prop}
\begin{proof}
	From \autoref{coords-inner-product} we have $u = \sum_i (u_i, u)$ and $v = \sum_j (u_i, v)$, which gives
	\begin{align*}
		(u, v)
		&= \left(\sum_{i=0}(u_i, u)u_i, \sum_{j=0} (u_i,v)u_i\right)
		\\&= \sum_{i=0}(u_i, u)^*\left(u_i, \sum_{j=0} (u_j, v)u_j\right)
		\\&= \sum_{i=0}(u_i, u)^*\sum_{j=0} (u_j, v)\left(u_i, u_j\right)
		\\&= \sum_{i=0}(u_i, u)^*\left(0 + \dots + 0 + (u_i, v)\left(u_i, u_i\right) + 0 + \dots + 0\right)
		\\&= \sum_{i=0}(u_i, u)^*(u_i, v)
		\\&= \left(
		\left[\begin{matrix}
			(u_0, u)\\
			(u_1, u)\\
			\cdots\\
			(u_n, u)
		\end{matrix}\right]
		,
		\left[\begin{matrix}
			(u_0, v)\\
			(u_1, v)\\
			\cdots\\
			(u_n, v)
		\end{matrix}\right]
		\right)
		\\&= (\phi(u), \phi(v))
	\end{align*}
\end{proof}

\begin{define}[Hilbert Spaces]
	In an inner product space $V$, an infinite sequence of vectors $v_i$ is said to be \emph{Cauchy} if for every $\epsilon > 0$, no matter how small, there is some $N$ after which any $n, m > N$ will satisfy $\norm{v_n - v_m} < \epsilon$. The sequence is said to be convergent if there is some vector $v \in V$ so that again, for every $\epsilon > 0$ there is an $N$ after which any $n > N$ will satisfy $\norm{v_n - v} < \epsilon$. If every Cauchy sequence in $V$ is convergent then $(V, \cdot, +, (,))$ is said to be a \emph{complete} inner product space, also known as a \emph{Hilbert} space.
\end{define}

\begin{prop}
	For every non-negative integer $n$, the inner product spaces $\mathbb{R}^n$ and $\mathbb{C}^n$ are complete.
\end{prop}
The proof amounts to showing that a Cauchy sequence in either of these spaces is coordinate-wise Cauchy, then finding a limit of each of these coordinates using the completeness of $\mathbb{R}$.

\section{Group Definitions}\label{group-defs}

\begin{define}
	A relation $\sim$ between elements of a set $S$ is called an \emph{equivalence relation} if it satisfies the following three properties, for any $a, b, c \in S$:
	\begin{itemize}
		\item Reflexivity: $a \sim a$,
		\item Symmetry: $a \sim b \iff b \sim a$,
		\item Transitivity: $a \sim b, b \sim c \implies a \sim c$.
	\end{itemize}
\end{define}

Equivalence relations are a powerful way of identifying different elements of a set with each other. The set quotient is an even more powerful way of exhibiting the structure of the equivalence relation as a concrete set.

\begin{define}[Set Quotient]
	Given an equivalence relation $\sim$ on a set $S$, and an element $x \in S$, the \emph{equivalence class} of $x$ is the subset of $S$ written
	\[[x] = \{y\ |\ x \sim y\}\]
	The \emph{set quotient} of $S$ with respect to $\sim$ is the set of equivalence classes
	\[S/\sim \ = \{[x]\ |\ x \in S\}.\]
\end{define}

\begin{prop}
	For any equivalence relation $\sim$ on a set $S$, $S/\sim$ partitions $S$.
\end{prop}

\begin{prop}
	In an equivalence relation $\sim$ on a set $S$, any two elements $x, y \in S$ will satisfy $x \sim y$ exactly when $[x] = [y]$.
\end{prop}

Sometimes set quotients of algebraic structures will themselves have an algebraic structure.
\begin{define}
	Given an equivalence relation $\sim$ on a group $G$, we say that $G$ \emph{induces a binary operation on $G/\sim$} if the binary operation
	\[[x]\cdot[y] \mapsto [xy]\]
	is well defined, i.e. if this product does not depend on the choice of $x$ and $y$, so that whenever $x' \in [x]$ and $y' \in [y]$ we have $[xy] = [x'y']$.
\end{define}

\begin{prop}
	If a group $G$ induces a binary operation on $G/\sim$, then $(G/\sim, \cdot)$ is also a group.
\end{prop}
\begin{proof}
We have associativity by $([x][y])[z] = [xyz] = [x]([y][z])$.\\
We have an identity by $[e][x] = [x][e] = [x]$.\\
We have inverses by $[x][x^{-1}] = [xx^{-1}] = [e]$.
\end{proof}

\begin{define}\label{subgroup}
	A subgroup $H$ of a group $G$ is said to be a \emph{normal subgroup} if, for every $x \in H$, $z \in G$, the group product $xzx^{-1} \in G$.
\end{define}

As an example of a normal subgroup, take $G$ to be any subgroup of $GL(n, \mathbb{R})$, and $H$ to be the set of `scalars' in $G$, that is the set $\{\lambda I\ |\ \lambda \in \mathbb{C}\} \cap G$. Since scalars are commutative, it is straight-forward that $g\lambda I g^{-1} = \lambda gg^{-1} = \lambda I \in H$.

\begin{prop}
	If a group $G$ induces a binary operation on some set quotient $G/\sim$, then the equivalence class $[e]$ is a normal subgroup.
\end{prop}
\begin{proof}
	First, suppose that $e\sim x$ and $e\sim y$, then 
	\[[e] = [e][e] = [x][y] = [xy],\]
	meaning $e \sim xy$ as well, so the binary operation of $G$ is induced (in the sense of \autoref{induced-subgroup}) in $[e]$.
	
	Now by reflexivity we have $e \sim e$, which gives the identity element $e \in [e]$, and if $e \sim x$ then
	\[[x^{-1}] = [e][x^{-1}] = [x][x^{-1}] = [xx^{-1}] = [e],\]
	meaning $e \sim x^{-1}$, which gives inverses $x^{-1}$ in $[e]$. This makes $[e]$ a subgroup of $G$.
	
Finally, given $x \in [e]$ and $z \in S$ then
\[[zxz^{-1}] = [z][x][z^{-1}] = [z][e][z^{-1}] = [zez^{-1}] = [e],\]
meaning $zxz^{-1} \in [e]$ as well, making $[e]$ normal.
\end{proof}

Not only is $[e] \in G/\sim$ a normal subgroup of $G$, but this set allows us to characterize $\sim$ even further. In group theory one can write any equation $g_1=g_2$ as $g_1g_2^{-1} = e$, which in the set quotient $G/\sim$ becomes $[xy^{-1}] = [e]$, i.e. $xy^{-1} \sim e$. We now draw this out formally, in terms of $[e]$ as a set.
\begin{prop}
	If a group $G$ induces a binary operation on some set quotient $G/\sim$, and $x, y \in G$, then $x \sim y$ exactly when $xy^{-1} \in [e]$.
\end{prop}
\begin{proof}
	If we have $x \sim y$, then equivalently $[x] = [y]$, which we can now manipulate algebraically. Clearly $[x][y^{-1}] = [y][y^{-1}]$, meaning $[xy^{-1}] = [e]$, and hence $xy^{-1} \sim e$. By symmetry we have $e \sim xy^{-1}$, and so $xy^{-1} \in [e]$.
	
	Conversely if $xy^{-1} \in [e]$, then $[xy^{-1}] = [e]$, and in much the same way
	\[[x] = [xy^{-1}y] = [xy^{-1}][y] = [e][y] = [y],\]
	giving $x \sim y$.
\end{proof}

This result motivates us to reverse the formulation we have given, by constructing a set quotient out of a normal subgroup rather than constructing a normal subgroup out of a set quotient. First we must check that we have an equivalence relation.
\begin{prop}\label{normal-equiv}
	If $H$ is a normal subgroup of the group $G$, or in fact just a subgroup of $G$, then $x \sim y \iff xy^{-1} \in H$ is an equivalence relation.
\end{prop}
\begin{proof}
	$xx^{-1} = e \in H$ gives reflexivity.\\
	If $xy^{-1} \in H$ then its inverse $yx^{-1}$ will be in $H$ as well, giving symmetry.\\
	If $xy{-1}, yz^{-1} \in H$, then their product $xz^{-1}$ will be as well, giving transitivity.
\end{proof}

Now if $H$ is in fact normal, we can induce a binary operation.
\begin{prop}
	If $H$ is a normal subgroup of $G$ then $G$ induces a binary operation on the set quotient $G/\sim$ given by the equivalence relation
	\[x \sim y \iff xy^{-1} \in H\].
\end{prop}
\begin{proof}
	Suppose that $x' \in [x]$ and $y' \in [y]$, i.e. that $xx'^{-1} \in H$ and $yy'^{-1} \in H$. We would like to show that $[xy] = [x'y']$.
	
	Since $H$ is normal, and $yy'^{-1}$ is in $H$, $x'\left(yy'^{-1}\right)x'^{-1}$ will be in $H$ as well. Now the product of $xx'^{-1}$ with this will be $xyy'^{-1}x'^{-1}$, which must be in $H$ also. This is exactly $xy\left(x'y'\right)^{-1}$, meaning $xy \sim x'y'$, and so $[xy] = [x'y']$ as required.
\end{proof}

We have now characterised a special case of set quotients directly in terms of the concepts of group theory. Given the foundational power of set quotients, it might be little surprise that this characterisation we have presented is the basis of many other group theoretic concepts and techniques, and as such is given a name and taken as the standard way of building a group out of a set quotient.
\begin{define}
	If $H$ is a normal subgroup of $G$ then the \emph{quotient group} $G/H$ is the group $(G/\sim, \cdot)$, where $\sim$ is the equivalence relation given in \autoref{normal-equiv}, and $\cdot$ is the binary operation induced by $G$ on $G/\sim$.
\end{define}

\begin{define}[Cosets]
	Given a subgroup $H$ of a group $G$, and elements $g_1, g_2 \in G$, we define the \emph{two-sided coset} $g_1Hg_2$ to be the set
	\[\{g_1hg_2\ |\ h \in H\}.\]
	When $g_1 = e$ we say that $eHg_2 = Hg_2$ is a \emph{right coset} of $H$, and when $g_2 = e$ we say that $g_1He = g_1H$ is a \emph{left coset} of $H$.
\end{define}

\begin{prop}
	If $H$ is a normal subgroup of $G$, and $g \in G$, then every left coset $gH$ is equal to the corresponding right coset $Hg$, and the group quotient $G/H$ consistes exactly of these cosets of $H$.
\end{prop}

As an example of a group with a normal subgroup, consider the following finite group generalising the set of permutation matrices:
\begin{define}[Generalised Symmetric Group]\label{generalised-perm}
	For each pair of positive integers $m, n$, define $S(m, n)$ to be the set of unitary matrices $P\in U(n)$ satisfying
	\[Pu_i = \exp\left(\frac{i2\pi r_i}{m}\right)u_{\sigma(i)}\]
	for every canonical basis $u_i$, where $\sigma$ is some permutation in $\mathcal{S}_n$, and $r_0,\dots,r_{n-1}$ are some set of integers.
\end{define}

\begin{prop}
	For any positive integers $m, n$, $S(m, n)$ is a subgroup of $U(n)$.
\end{prop}
\begin{proof}
	Clearly $I \in S(m, n)$ by setting $\sigma = \text{id}$, $r_i = 0$.
	\\If $P_1 \in S(m, n)$ via $\sigma_1$ and $r_i$, and $P_2 \in S(m, n)$ via $\sigma_2$ and $s_i$, then
	\[P_1P_2u_i = \exp\left(\frac{i2\pi s_i}{m}\right)P_1u_{\sigma_2(i)} = \exp\left(\frac{i2\pi (r_{\sigma_2(i)} + s_i)}{m}\right)u_{\sigma_1(\sigma_2(i))},\]
	so set $r'_i = r_{\sigma_2(i)} + s_i$, and $\sigma = \sigma_1 \circ \sigma_2$, giving $P_1P_2 \in S(m, n)$. Additionally if we take the equation
	\[P_1u_i = \exp\left(\frac{i2\pi r_i}{m}\right)u_{\sigma_1(i)},\]
	and rearrange, we get
	\[P_1^{-1}u_{\sigma_1(i)} = \exp\left(\frac{-i2\pi r_i}{m}\right)u_i\]
	So set $r'_i = -r_{\sigma^{-1}_1(i)}$, and $\sigma = \sigma_1^{-1}$ to get $P_1^{-1} \in S(m, n)$.
\end{proof}

\begin{prop}
	The set $\Delta(m, n)$ of diagonal matrices in $S(m, n)$ form a normal subgroup of $S(m, n)$.
\end{prop}
\begin{proof}
	If $P_1, P_2 \in S(m, n)$ are diagonal, then $\sigma_1 = \sigma_2 = \text{id}$, so by the above formulas we get $\sigma = \sigma_1\circ \sigma_2 = \text{id}$ or $\sigma = \sigma_1^{-1} = \text{id}$, giving $P_1P_2$ and $P_1^{-1}$ diagonal. This tells us that $\Delta(m, n)$ is a subgroup of $S(m, n)$.
	
	Further if $P$ is some other matrix in $S(m, n)$ via $\sigma$ and $q_i$, and $u_i$ is a canonical basis vector in $\mathbb{C}^n$, then
	\begin{align*}
		P^{-1}P_1Pu_i
		&= \exp\left(\frac{i2\pi\left(q_i\right)}{m}\right)P^{-1}P_1u_{\sigma(i)}
		\\&= \exp\left(\frac{i2\pi\left(q_i+r_{\sigma(i)}\right)}{m}\right)P^{-1}u_{\sigma(i)}
		\\&= \exp\left(\frac{i2\pi\left(r_{\sigma(i)}\right)}{m}\right)u_i,
	\end{align*}
clearly a diagonal matrix, so $\Delta(m, n)$ is normal.
\end{proof}

This group also contains $\mathcal{S}_n$ as a subgroup, of course, but this will not be normal.

If $H$ is a normal subgroup of $N$ and $N$ is a subgroup of $G$, we might say ``$H$ is normal in $N$'' and ``$N$ is in $G$''. Despite this terminology, $H$ is not necessarily a normal subgroup in $G$. With this subtlety in mind we can find exactly such a group $N$, given any subgroup $H$ of $G$.
\begin{define}
Given a subgroup $H$ of a group $G$, the \emph{normaliser} of $H$, written $N_G(H)$ or simply $N(H)$ is the set
\[\{g\ |\ g \in G,\ ghg^{-1} \in H \forall h \in H\}.\]
\end{define}
\begin{prop}
	Given a subgroup $H$ of a group $G$, the normaliser $N_G(H)$ is the maximal subgroup of $G$ with $H$ as a normal subgroup.
\end{prop}

As an example, consider the normaliser of $\mathcal{S}_n$ in the generalised symmetric group.

\begin{prop}
	The normaliser $N_{S(m, n)}(\mathcal{S}_n)$, containing $\mathcal{S}_n$ and contained in $S(m, n)$, is exactly the set of scalar multiples of permutation matrices, \[\{\exp\left(\frac{i2\pi s}{m}\right)P\ |\ s \in \mathbb{Z}, P \in \mathcal{S}_n\}.\]
\end{prop}

We can also define equivalence classes \emph{between} groups, based on whether they have the same structure under group multiplication.
\begin{define}
	Given two groups $G_1$ and $G_2$, a \emph{homomorphism} is a map $\phi: G_1 \to G_2$ satisfying $\phi(x)\phi(y) = \phi(xy)$, for any $x, y \in G_1$. If $\phi$ is bijective then it is also called an \emph{isomorphism}, and if there is at least one such isomorphism then $G_1$ and $G_2$ are said to be \emph{isomorphic}, represented infix as $G_1 \cong G_2$.
\end{define}

\begin{prop}
	The relation $\cong$ between groups, of being isomorphic, is an equivalence relation.
\end{prop}

\begin{prop}\label{inj-hom-thm}
	If $\phi$ is a homomorphism between $G_1$ and $G_2$, then the image $\phi(G_1)$ is a subgroup of $G_2$, and if $\phi$ is injective then $G_1 \cong \phi(G_1)$.
\end{prop}

\begin{define}
	If $F$ is a field and $n$ is a positive integer, then a matrix $P \in GL(n, F)$ is said to be a \emph{permutation matrix} if there is a permutation $\sigma \in \mathcal{S}_n$ so that
	\[Pu_i = u_{\sigma(i)},\]
	where $u_i$ is any of the $n$ canonical basis vectors defined in \autoref{canonical-basis}.
\end{define}

By \autoref{inj-hom-thm} the set of permutation matrices form a group isomorphic to the symmetric group $\mathcal{S}_n$, which allows us to identify these two groups, treating $\mathcal{S}_n$ as a subgroup of $GL(n, F)$.

\begin{define}
	If $G$ is a group and $\phi$ is an isomorphism from $G$ to itself, then $\phi$ is said to be an \emph{automorphism}. The set of automorphisms is written Aut$(G)$.
\end{define}
\begin{prop}
	For any group $G$, $(Aut(G), \circ)$ is a group.
\end{prop}
\begin{define}
	Given a group $(G_1, \cdot)$, and a subgroup $G_2$ of $Aut(G)$, the \emph{semi-direct product} $G_1 \rtimes G_2$ is the group $(G_1 \times G_2, \cdot)$ with the product
	\[(g_1, f_1)\cdot (g_2, f_2) = (g_1 \cdot f_1(g_2), f_1 \circ f_2).\]
\end{define}

\begin{define}
	We say that a subgroup $H$ of a group $G$ is \emph{generated by} a set $S \subset H$, if every $h \in H$ is of the form
	\[h = \prod_{i=1}^k s_i,\]
	for some choice of $k$ and $s_1,\dots,s_k$, where each $s_i$ is either an element of $S$ or the inverse of an element of $S$. Clearly the set of elements of $G$ of this form is a subgroup of $G$, which we call \emph{the subgroup generated by} $S$, and denote $\langle S \rangle$. If $S$ is a finite set $\{s_1,\dots,s_n\}$ then we can also write $\langle s_1,\dots,s_n\rangle$ directly, and say that any group $H$ generated by such a set is \emph{finitely generated}.
\end{define}
% app2.tex (will be Appendix B)

\chapter[SOURCE CODE]{Source Code}
%% app2.tex (will be Appendix C)

\section{Appendix C, Source Code for Group Search}
%\bibliographystyle{style} \bibliography{bibFIle} 
% Add in the bibliography
% Set the text size to be small
\small
% Set the bibliography style to be plain.
\bibliographystyle{plain}
% List all citations in thesis.bib file
% Comment this out for final printing
% if you have un-cited items in your
% bib database
\nocite{*}
% refs.bib is the name of our database
% and we load it with the following command
\bibliography{refs}

%% A small file that prints the index in the main document.
% No need to alter this file...
\printindex


\end{document}
