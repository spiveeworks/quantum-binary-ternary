% thesis.tex (starting point of a UTas mathematics thesis)

% Note that the following defaults are also contained in the 'report' class
% (this is not exhaustive...see appropriate references for all options)
% 'oneside' mode...overide with 'twoside' to force output for two-sided printing.
% 'final' mode...overide with 'draft' to see linebreak malfunctioning.
% For example, to use two-sided printing one would declare the 'documentclass'
% as follows...
% '\documentclass[11pt,a4paper,twoside]{report}'
%
% Specific Mathematical options are as follows
% 'leqno' to force equation numbering on the left side of the page.
% 'fleqn' to force formulas to flush left (centered is the default).
% If 'fleqn' is specified then the left indent is controlled with '\mathindent'.
% i.e. '\setlength{\mathindent}{2.5cm}'

% Declare overall type of document (use 11pt report class on A4 paper).
\documentclass[12pt,a4paper]{report}

\usepackage{amsmath}
\usepackage{amssymb}

\usepackage{tikz}
\usetikzlibrary{quantikz}

\renewcommand\thesection{}
\renewcommand\thesubsection{}

% went the whole year without tikz, had to uncomment this when I started using it, worked without any further modification!
%\newcommand{\bra}[1]{\langle #1 |}
%\newcommand{\ket}[1]{| #1 \rangle}
%\newcommand{\braket}[2]{\langle #1 | #2 \rangle}

\newcommand{\ord}[1]{\left| #1 \right|}
\newcommand{\norm}[1]{\left\Vert #1 \right\Vert}

% Include the style file which contains all the required formatting
% information that is set out in the Research Higher Degrees Resource
% Handbook (2003 version). NOTE: This file uses the following packages
% 'graphicx' for graphics manipulation
% 'fancyhdr' for nice headers and footers.
% 'makeidx' for generating the index
% 'tocbibind' for adding table of contents entries for bibliography, index etc.
% 'sectsty' for generating stylised chapter and section headings.
% You will need to make sure your LaTeX installation has these packages
% installed...else it wont work :(
\usepackage{MathPhysHonoursThesis}

%% newcommands.tex (new command definitions)

% Here you would include any additional packages that you want to use.
% You should make sure they don't clash with the above packages that
% are in use in the style file.
% If you want to call in some style files or new packages, put them here
\usepackage{undertilde}
%\usepackage[left=2cm,right=2cm,top=2cm,bottom=2cm]{geometry}
\usepackage[a4paper]{geometry}
\usepackage[latin1]{inputenc}
\usepackage{amsmath, latexsym, color, graphicx, amssymb, here}
\usepackage{amsfonts}
\usepackage{epsf, epsfig, pifont,tikz}
\usepackage{graphics, calrsfs}
%\usepackage{tangocolors}
\usepackage{times}
\usepackage{fancybox,calc}
\usepackage{hyperref}
\usepackage{pgfplots}
\usepackage{verbatim}

% Some examples (yours may be different):
\newtheorem{theorem}{Theorem}[section]
\newtheorem{lemma}[theorem]{Lemma}
\newcommand{\bfx}{{\ensuremath{\mathbf{x}}}}

\newcommand{\A}{{\bf A}}
\newcommand{\B}{{\bf B}}
\newcommand{\T}{{\bf T}}
\newcommand{\C}{{\bf C}}
\newcommand{\N}{{\bf N}}
\newcommand{\R}{{\mathbb R}}
\newcommand{\Z}{{\mathbb Z}}
\newcommand{\n}{{\bf n}}
\renewcommand{\v}{{\bf v}}
\renewcommand{\r}{{\bf r}}
\renewcommand{\a}{{\bf a}}

\newcommand{\uniti}{{\hat{\mbox{\boldmath $\imath$}}}}
\newcommand{\unitj}{{\hat{\mbox{\boldmath $\jmath$}}}}
\newcommand{\unitk}{{\hat{\mbox{\boldmath $\mathit{k}$}}}}
\newcommand{\unitn}{{\hat{\mbox{\boldmath $\mathit{n}$}}}}
\newcommand{\unite}{{\hat{\mbox{\boldmath $\mathit{e}$}}}}
\newcommand{\unitu}{{\hat{\mbox{\boldmath $\mathit{u}$}}}}
\newcommand{\ie}{{\em i.e.} \/}
\newcommand{\eg}{{\em e.g.} \/}
\newcommand{\etc}{{\em etc.} \/}
\newcommand{\etal}{{\em et al. }}
\newcommand{\mathbi}[1]{\textbf{\em #1}}
\newcommand{\bcdot}{\mbox{\boldmath $\, \cdot \, $}}
\newcommand{\vect}[1]{{\mbox{\boldmath $\utilde{\mathit{#1}}$}}}
\newcommand{\xyplane}{$x$-$y$ plane \/}
\newcommand{\xzplane}{$x$-$z$ plane \/}
\newcommand{\yzplane}{$y$-$z$ plane \/}
\newcommand{\dint}{\int \! \! \int}
\newcommand{\tint}{\int \! \! \int \! \! \int}
\newcommand{\doint}{\bigcirc \! \! \! \! \! \! \! \! \int \! \! \! \! \!  \int}
\newcommand{\inlinedoint}{\circ\!\!\!\! \! \int \!\!\!\! \int}
%\newcommand{\deloperator}[3]{{\frac{\partial{#1}}{\partial x} \, \uniti \frac{\partial{#2}}{\partial y} \, \unitj + \frac{\partial{#3}}{\partial z} \, \unitk}}
\newcommand{\deloperator}{{\frac{\partial}{\partial x} \, \uniti + \frac{\partial}{\partial y} \, \unitj + \frac{\partial}{\partial z} \, \unitk}}
\newcommand{\laplaceoperator}{{\frac{\partial^2}{\partial x^2} + \frac{\partial^2}{\partial y^2} + \frac{\partial^2}{\partial z^2}}}
%\newcommand{\grad}[1]{{\frac{\partial {#1}}{\partial x} \, \uniti + \frac{\partial {#1}}{\partial y} \, \unitj + \frac{\partial {#1}}{\partial z} \, \unitk}}
\newcommand{\gradCylindrical}[1]{{\frac{\partial {#1}}{\partial \rho} \, \unite_\rho \ + \ \frac{1}{\rho} \frac{\partial {#1}}{\partial \phi} \, \unite_\phi \ + \ \frac{\partial {#1}}{\partial z} \, \unite_z} }
\newcommand{\gradSpherical}[1]{{\frac{\partial {#1}}{\partial r} \, \unite_r \ + \ \frac{1}{r} \frac{\partial {#1}}{\partial \theta} \, \unite_\theta \ + \ \frac{1}{r \sin(\theta)}\frac{\partial {#1}}{\partial \phi} \, \unite_\phi} }
\newcommand{\divSpherical}[3]{{\frac{1}{R^2}\, \frac{\partial}{\partial R}\left(R^2\, {#1}\right) \ + \  \frac{1}{R \sin(\theta)} \frac{\partial}{\partial \theta} \left(\sin(\theta)\, {#2}\right) \ + \ \frac{1}{R \sin(\theta)}\frac{\partial {#3}}{\partial \phi} \  }}
%\newcommand{\laplacian}[1]{{\frac{\partial^2 {#1}}{\partial x^2} + \frac{\partial^2 {#1}}{\partial y^2} + \frac{\partial^2 {#1}}{\partial z^2}}}
\newcommand{\posvect}[1]{{#1}_1 \, \uniti + {#1}_2 \, \unitj + {#1}_3 \, \unitk}
\newcommand{\posvectr}{x \, \uniti + y \, \unitj + z \, \unitk}
\newcommand{\posvectcyl}{\rho \, \unite_\rho \, + z \, \unite_z}
\newcommand{\posvectsph}{r \, \unite_\r}
\newcommand{\arbvect}[3]{{#1} \, \uniti \, + \, {#2} \, \unitj \, + \, {#3} \, \unitk}
\newcommand{\genvect}[5]{{#1} \, \uniti \, {#2}\,  {#3} \, \unitj \, {#4} \, {#5} \, \unitk}
\newcommand{\parametricposvectr}{\vect{r}(t) \ = \ x(t) \, \uniti + y(t) \, \unitj + z(t) \, \unitk}
\newcommand{\divergence}[1]{\nabla \bcdot \vect{#1}}
%\newcommand{\curl}[1]{\nabla \times \vect{#1}}
\newcommand{\magnitude}[1]{ \| \vect{#1} \| }
\newcommand{\drCart}{d x \, \uniti + d y \, \unitj + d z \, \unitk}
\newcommand{\drCyl}{d \rho \, \unite_\rho + \rho \, d \phi \, \unite_\phi + d z \, \unite_z}
\newcommand{\drSph}{d r \, \unite_r + r \, d \theta \, \unite_\theta + r \, \sin(\theta) \, d \phi \, \unite_\phi }
\newcommand{\coordvect}[3]{{#1} \, \uniti \, + \, {#2} \, \unitj \, + {#3} \, \unitk}
\newcommand{\cylcoordvect}[3]{{#1} \, \unite_\rho \, + \, {#2} \, \unite_\phi \, + {#3} \, \unite_z}
\newcommand{\sphcoordvect}[3]{{#1} \, \unite_\r \, + \, {#2} \, \unite_\theta \, + {#3} \, \unite_\phi}

\newcommand{\ul}[1]{\underline{#1}}

\newsavebox{\fmbox}
\newenvironment{eqnframe}[1]     
{
	\begin{center} 
	\begin{lrbox}{\fmbox}
	\begin{minipage}{#1}
}     
{
	\end{minipage}
	\end{lrbox}\fbox{\usebox{\fmbox}}
	\end{center}
}

\newcommand\Tpad{\rule[4.5ex]{0pt}{0pt}}
\newcommand\Bpad{\rule[-3.75ex]{0pt}{0pt}}

\renewcommand{\labelenumi}{\textbf{\arabic{enumi}}.}
\renewcommand{\labelenumii}{\textbf{(\roman{enumii})}}
\renewcommand{\labelenumiii}{\textbf{(\alph{enumiii})}}

\newcommand{\parD}[2]{\frac{\partial #1}{\partial #2}}
\newcommand{\parDD}[2]{\frac{\partial^2 #1}{\partial #2 ^2}}
\newcommand{\laplacian}{\Delta}
\renewcommand{\div}{\nabla\cdot}
\newcommand{\grad}{\nabla}
\newcommand{\divp}{\nabla^\prime\cdot}
\newcommand{\gradp}{\nabla^\prime}
\newcommand{\curl}{\nabla\times}
\newcommand{\cross}{\times}
\renewcommand{\dot}{\cdot}
% define some colors
\definecolor{cBlue}{rgb}{.255,.41,.884} % RoyalBlue of svgnames
\definecolor{cRed}{rgb}{1, 0, 0} % Red of svgnames


%% newcommands.tex (new command definitions)

% Here you would include any additional packages that you want to use.
% You should make sure they don't clash with the above packages that
% are in use in the style file.
% If you want to call in some style files or new packages, put them here
%\usepackage{undertilde}
%\usepackage[left=2cm,right=2cm,top=2cm,bottom=2cm]{geometry}
\usepackage[a4paper]{geometry}
%\usepackage[latin1]{inputenc}
\usepackage{amsmath, latexsym, color, graphicx, amssymb, here}
\usepackage{amsfonts}
\usepackage{epsf, epsfig, pifont,tikz}
\usepackage{graphics, calrsfs}
%\usepackage{tangocolors}
\usepackage{times}
\usepackage{fancybox,calc}
\usepackage{hyperref}
\usepackage{pgfplots}
\usepackage{verbatim}
\usepackage{esint} 
\usepackage{amsthm}


%% environment name, counter, text
\newtheorem{theorem}{Theorem}[section]
\newtheorem{lemma}[theorem]{Lemma}
\newtheorem{definition}[theorem]{Definition}
%% Use like:
% \begin{definition}
% Blah blah blah
% \end{definition}



%% Brackets
\newcommand{\lb}{\left(}
\newcommand{\rb}{\right)}
%% Square Brackets
\newcommand{\lbs}{\left[}
\newcommand{\rbs}{\right]}

%Bolded letters
\newcommand{\A}{{\bf A}}
\newcommand{\B}{{\bf B}}
\newcommand{\T}{{\bf T}}
\newcommand{\C}{{\bf C}}
\newcommand{\N}{{\bf N}}
\newcommand{\n}{{\bf n}}
\newcommand{\bfx}{{\ensuremath{\mathbf{x}}}}
%Bolded letters, which previously had a command
\renewcommand{\v}{{\bf v}}
\renewcommand{\r}{{\bf r}}
\renewcommand{\a}{{\bf a}}

%Set notation. Needs to be in a maths environment (equation or $$)
\newcommand{\R}{{\mathbb R}}
\newcommand{\Z}{{\mathbb Z}}

%Unit vectors. Needs to be in a maths environment (equation or $$)
\newcommand{\uniti}{{\hat{\mbox{\boldmath $\imath$}}}}
\newcommand{\unitj}{{\hat{\mbox{\boldmath $\jmath$}}}}
\newcommand{\unitk}{{\hat{\mbox{\boldmath $\mathit{k}$}}}}
\newcommand{\unitn}{{\hat{\mbox{\boldmath $\mathit{n}$}}}}
\newcommand{\unite}{{\hat{\mbox{\boldmath $\mathit{e}$}}}}
\newcommand{\unitu}{{\hat{\mbox{\boldmath $\mathit{u}$}}}}


%Shorthand for commonly used expressions with cursive font
\newcommand{\ie}{{\em i.e.} \/}
\newcommand{\eg}{{\em e.g.} \/}
\newcommand{\etc}{{\em etc.} \/}
\newcommand{\etal}{{\em et al. }}


%Bolded, cursive strings of characters
\newcommand{\mathbi}[1]{\textbf{\em #1}}


%Bolded dot symbol
\newcommand{\bcdot}{\mbox{\boldmath $\, \cdot \, $}}

%bolds and puts in a cursive underline for a vector
\renewcommand{\vec}[1]{{\mbox{\boldmath $\utilde{\mathit{#1}}$}}}


%Shorthand commands for writing xy,xz,yz planes
\newcommand{\xyplane}{$x$-$y$ plane\/}
\newcommand{\xzplane}{$x$-$z$ plane\/}
\newcommand{\yzplane}{$y$-$z$ plane\/}
\newcommand{\abplane}[2]{${#1}$-${#2}$ plane\/}

%Full derivative of A wrt B. Needs to be in a maths environment (equation or $$)
\newcommand{\FullDif}[2]{\dfrac{d {#1}}{d {#2}}}

%Partial derivative of A wrt B. Needs to be in a maths environment (equation or $$)
\newcommand{\ParDif}[2]{\dfrac{\partial {#1}}{\partial {#2}}}
%Second partial derivative of A wrt B. Needs to be in a maths environment (equation or $$)
\newcommand{\DblParDif}[2]{\frac{\partial^2 #1}{\partial #2 ^2}}

%Material derivative of A wrt B. Needs to be in a maths environment (equation or $$)
\newcommand{\MatDif}[2]{\dfrac{D {#1}}{D {#2}}}


%Writes the cartesian gradient of a function. Leave as {} to write the operator. Needs to be in a maths environment (equation or $$)
\newcommand{\grad}[1]{\ParDif{#1}{x}  \uniti+ \ParDif{#1}{y} \, \unitj + \ParDif{#1}{z} \, \unitk}


%Writes the cartesian Laplacian of a function. Leave as {} to write the operator. Needs to be in a maths environment (equation or $$)
\newcommand{\laplacian}[1]{\DblParDif{{#1}}{x} + \DblParDif{{#1}}{y}+ \DblParDif{{#1}}{z}}

%Writes the cylindrical gradient of a function. Leave as {} to write the operator. Needs to be in a maths environment (equation or $$)
\newcommand{\gradCylindrical}[1]{\ParDif{{#1}}{\rho} \, \unite_\rho \ + \ \dfrac{1}{\rho} \ParDif{{#1}}{\phi} \, \unite_\phi \ + \ \ParDif{{#1}}{z} \, \unite_z}



%Writes the spherical gradient of a function. Leave as {} to write the operator. Needs to be in a maths environment (equation or $$)
\newcommand{\gradSpherical}[1]{\ParDif{{#1}}{r} \, \unite_r \ + \ \dfrac{1}{r} \ParDif{{#1}}{\theta} \, \unite_\theta \ + \ \dfrac{1}{r \sin(\theta)}\ParDif{{#1}}{\phi} \, \unite_\phi}

%Writes the spherical divergence of a function. Leave as {}{}{} to write the operator, otherwise #1 is r component, #2 is theta component and #3 is phi component. Needs to be in a maths environment (equation or $$)
\newcommand{\divSpherical}[3]{\dfrac{1}{R^2}\, \ParDif{}{R}\lb(R^2\, {#1}\rb) \ + \  \dfrac{1}{R \sin(\theta)} \ParDif{}{\theta} \lb(\sin(\theta)\, {#2}\rb) \ + \ \dfrac{1}{R \sin(\theta)}\ParDif{{#3}}{\phi} \  }

%Creates a cartesian position vector, argument is the vector field in question. Needs to be in a maths environment (equation or $$)
\newcommand{\posvect}[1]{{#1}_x \, \uniti + {#1}_y \, \unitj + {#1}_z \, \unitk}

%Creates the cartesian generic position vector r. Needs to be in a maths environment (equation or $$)
\newcommand{\posvectr}{x \, \uniti + y \, \unitj + z \, \unitk}
%Creates the cylindrical generic position vector r. Needs to be in a maths environment (equation or $$)
\newcommand{\posvectcyl}{\rho \, \unite_\rho \, + z \, \unite_z}
%Creates the spherical generic position vector r. Needs to be in a maths environment (equation or $$)
\newcommand{\posvectsph}{r \, \unite_\r}


%Shorthand for a parametrised position vector. Needs to be in a maths environment (equation or $$)
\newcommand{\parametricposvectr}{\vec{r}(t) \ = \ x(t) \, \uniti + y(t) \, \unitj + z(t) \, \unitk}

%Creates the divergence of a specified vector. Needs to be in a maths environment (equation or $$)
\renewcommand{\div}[1]{\nabla \bcdot \vec{#1}}

%Creates the curl of a specified vector. Needs to be in a maths environment (equation or $$)
\newcommand{\curl}[1]{\nabla \times \vec{#1}}

%Creates the magnitude of a specified vector. Needs to be in a maths environment (equation or $$)
\newcommand{\magnitude}[1]{ \| \vec{#1} \| }



%Creates a differential distance in Cartesian. Needs to be in a maths environment (equation or $$)
\newcommand{\drCart}{d x \, \uniti + d y \, \unitj + d z \, \unitk}
%Creates a differential distance in Cylindrical. Needs to be in a maths environment (equation or $$)
\newcommand{\drCyl}{d \rho \, \unite_\rho + \rho \, d \phi \, \unite_\phi + d z \, \unite_z}
%Creates a differential distance in Spherical. Needs to be in a maths environment (equation or $$)
\newcommand{\drSph}{d r \, \unite_r + r \, d \theta \, \unite_\theta + r \, \sin(\theta) \, d \phi \, \unite_\phi }


%Creates arbitrary vectors with 3 arguments. Needs to be in a maths environment (equation or $$)
\newcommand{\coordvect}[3]{{#1} \, \uniti \, + \, {#2} \, \unitj \, + {#3} \, \unitk}
\newcommand{\cylcoordvect}[3]{{#1} \, \unite_\rho \, + \, {#2} \, \unite_\phi \, + {#3} \, \unite_z}
\newcommand{\sphcoordvect}[3]{{#1} \, \unite_\r \, + \, {#2} \, \unite_\theta \, + {#3} \, \unite_\phi}


%Creates a box around a given equation. Must be in a maths environment.
\newsavebox{\fmbox}
\newenvironment{eqnframe}[1]     
{
	\begin{center} 
	\begin{lrbox}{\fmbox}
	\begin{minipage}{#1}
}     
{
	\end{minipage}
	\end{lrbox}\fbox{\usebox{\fmbox}}
	\end{center}
}


%These make differently aligned boxes, with width {#1} and height {#2}
\newcommand\Tpad[2]{\rule[4.5ex]{{#1}pt}{{#2}pt}}
\newcommand\Bpad[2]{\rule[-3.75ex]{{#1}pt}{{#2}pt}}

%These make slightly different shaped zeros.
\renewcommand{\labelenumi}{\textbf{\arabic{enumi}}.}
\renewcommand{\labelenumii}{\textbf{(\roman{enumii})}}
\renewcommand{\labelenumiii}{\textbf{(\alph{enumiii})}}

%Lets you use \cross instead of times. Needs to be in a maths environment (equation or $$)
\newcommand{\cross}{\times}


% define some colors
\definecolor{cBlue}{rgb}{.255,.41,.884} % RoyalBlue of svgnames
\definecolor{cRed}{rgb}{1, 0, 0} % Red of svgnames


%Shorthand for various useful things
%Argument 1, with argument 2 as a subscript
\newcommand{\An}[2]{{#1}_{{#2}}}
%Define the pair of co-ordinates
\newcommand{\AnBm}[4]{(\An{{#1}}{{#2}},\An{{#3}}{{#4}})}

%integral from x0 to x1
\newcommand{\intxoxi}{\int_{\An{x}{0}}^{\An{x}{1}}}

%f(x,y)
\newcommand{\fxy}{f(x,y)}
%g(x,y)
\newcommand{\gxy}{g(x,y)}

%%f(x,y,z)
\newcommand{\fxyz}{f(x,y,z)}
%%g(x,y,z)
\newcommand{\gxyz}{g(x,y,z)}

%% Variant epsilon
\newcommand{\ep}{\varepsilon}

%%Yprime squared
\newcommand{\yp}{(y^\prime)^{2}}

%%Coordinates ~extra style~
\newcommand{\cord}{co$\ddot{\text{o}}$rdinates\ }


% If you want to generate an index you should include the following command
% which puts a makeindex command in the preamble. Additionally, you need to un-comment
% the file 'index' in the '\includeonly' command below and also include the 'index'
% file in the main document.
\makeindex

% During thesis writing you might only want to see the pdf of the current chapter rather than the entire 
% thesis. However, you will want to ensure page numbers, chapter numbers, and referencing is maintained 
% for the entire thesis. 
% In the main section below, between \begin{document} and \end{document} you can use the \include{}
% command to specify all of the components (other .tex files) of the thesis. 
% Prior to \begin{document} you can use \includeonly{} to specify which of those components you want to 
% see. You can comment out files that you will include later or have already finished to speed
% up TeX processing. Comment out the \includeonly{} command to see the entire thesis.

%\includeonly{prelude % Contains all the relevant candidate information (name, degrees, abstract etc)
%,chap1  % The first chapter
%,chap2  % The second chapter
%,chap3  % The third chapter
%,chap4  % The fourth chapter
%%,app0   % Needed to switch to appendix mode
%,app1   % The first appendix
%,app2   % The second appendix
%,biby   % Makes the bibliography from the BibTeX database
%,index  % Places the index in the thesis
%}

% Begin the thesis
\begin{document}

% Include all the pieces of your thesis in here.

% prelude.tex (specification of which features in `mathphdthesis.sty' you
% are using, your personal information, and your title & abstract)

% Specify features of `mathphdthesis.sty' you want to use:
\titlepgtrue % main title page (required)
\signaturepagetrue % page for declaration of originality (required)
\copyrighttrue % copyright page (required)
\abswithesistrue % abstract to be bound with thesis (optional)
\acktrue % acknowledgments page (optional)
\tablecontentstrue % table of contents page (required)
\tablespagefalse % table of contents page for tables (required only if you have tables)
\figurespagefalse % table of contents page for figures (required only if you have figures)

\title{GENERALISING FOUNDATIONS OF QUANTUM COMPUTATION TO MIXED BINARY-TERNARY SYSTEMS} % use all capital letters
\author{Jarvis Carroll} % use mixed upper & lower case
\prevdegrees{B.Sc. (Tas)} % Used to specify your previous degrees...use mixed upper & lower case
\advisor{Doctor Michael Cromer, and Doctor Jeremy Sumner.} % example: Professor Lawrence K. Forbes 
\dept{Mathematics} % your academic department
\submitdate{November 2020} % month & year of your thesis submission

\newcommand{\abstextwithesis}
{
Quantum computation is the study of how quantum mechanics can be utilised to reliably compute or simulate results that are significantly harder to compute when done by classical means. Existing quantum computation research focuses overwhelmingly on binary quantum logic, the logic of the `qubit', due to this being both the simplest and the most familiar number system for performing computation. Quantum computation does not have to operate with binary quantum data, however, and calculation is theoretically possible with ternary `qutrits', or higher, or even with quantum systems that utilise a mixture of binary and ternary, or other quantum logic systems simultaneously. These systems might store data more efficiently, might scale more rapidly and might present richer or more novel techniques for calculation, or might prove detrimental; we need to develop a theory of non-binary quantum logic in order to resolve these possibilities.

In this thesis we present a detailed description of what a quantum algorithm is and the basic operations available to a qubit computer, before discussing existing literature surrounding proposed devices and algorithms for computation on ternary or `qutrit' computers. We then convert these existing ternary algorithms into algorithms that act on mixed quantum computers, with access to both qubits and qutrits. After this we describe a basic set of gates already known to be capable of performing generic or universal computation on binary quantum computers, and generalise these to mixed quantum computers as well. After this we present the results of a basic algebraic search implemented in the C programming language, for finding finite and infinite groups of quantum operations, and finally discuss the many novel implications of what was found and the most obvious questions for future research prompted by these findings.
}

\newcommand{\acknowledgement}
{
I would of course like to thank my supervisors Michael Cromer and Jeremy Sumner for their support in this project, who were very generous in taking this project on, especially Michael, who helped me learn everything I know about quantum computation, despite only being a guest at the university.

Further, I would like to thank the friends and peers with whom I have shared an office and classes this year, primarily Mae, Luke, Cassady, Georgia, and Larissa, for sharing in the journey we have just been through, of writing a thesis during a pandemic, making the year much less alienating than it would have been otherwise.

Finally, I would like to thank all of my lecturers from this year and past years for encouraging me and bringing me to this point in my education. This thesis is ultimately the result of five years of interaction with this university and its wonderful staff.
}

% Take care of things in `mathphdthesis.sty' or 'MathPhysHonoursThesis.sty'
% behind the scenes.
% Basically just does a check of all the fields that have been activated
% above and fills out the appropriate pages and adds them to the thesis.
\beforepreface
\afterpreface


% Set page numbering to arabic the first time we commence a chapter.
% This is required to get the page numbering font correct.
\pagenumbering{arabic}

% chap1.tex (Chapter 1 of the thesis)

% Note that the text in the [] brackets is the one that will
% appear in the table of contents, whilst the text in the {}
% brackets will appear in the main thesis.

%[script style reduce fraction size]

%[quantikz]

%\section{Quantum Mechanics}

\chapter*{Introduction}

Quantum computation is a contemporary paradigm of computation based on manipulating the quantum behaviour of particles, which leverages phenomena such as superposition and entanglement of `qubits', quantum analogues to the classical bit, and could theoretically compute useful results that would, for example, take millenia for a classical computer to calculate. Increasingly powerful quantum computers are being realized in practice, and the theory of quantum algorithms and quantum information is growing as well; both of these things are necessary for real life quantum computation to achieve results not already available in classical contexts.

One emerging topic within the field of quantum computation is the exploration of devices and algorithms that utilise logic systems other than the binary system familiar to classical computer science, since these alternative number systems present themselves readily within the quantum mechanics of certain physical systems. A related topic that is newer still is the question of what devices and algorithms might utilise logic systems that \emph{mix} binary, ternary, and higher quantum data, the simplest of which is the mixed binary-ternary quantum system. Mixing binary and ternary is of particular interest since we might be able to combine the strengths of both systems, using binary data and the relatively simple binary algorithms when convenient, but utilising the increased novelty and complexity of ternary logic whenever there is an obvious way of capitalising on this novelty.

In \autoref{preliminaries} we go to great lengths to establish the foundational linear algebra, quantum mechanics, and quantum computation that are already understood and used to construct basic algorithms in qubit contexts, primarily restating the ideas described in the textbook \cite{textbook}. We then describe the basic notation and behaviour of ternary and higher analogues to these operations, all things which are in common usage in the literature. This discussion is long, and in \autoref{formalities} one can find further details and definitions for the basic concepts upon which linear algebra is itself defined. The reader may not need to follow all of this discussion, in which case they can skip familiar content, until as late as \autoref{notation}, where we describe the specific ways that we need to notate quantum operations and quantum circuits in order to keep track of the many indeces that can vary independently in a mixed quantum computer.

With this in place we continue in \autoref{ternary} to describe known results of ternary quantum logic, and `qutrit' computers, centering on the paper \cite{arithmetics} which presents two algorithms for implementing integer addition of trits in a quantum computer. We describe the algorithms, and then present original circuits which perform the same calculations, but using qubits instead of qutrits where possible. We do this without having a particular physical implementation in mind, which blinds us to the performance characteristics that the original algorithms were better able to optimise and discuss. We do find, however, that the resulting circuits are simpler than the original qutrit circuits, suggesting that mixed logic is in fact capable of combining the strengths of individual logic systems. We also discuss analytical techniques presented in \cite{arithmetics} for representing ternary gates as polynomials mod 3, which do not easily generalise to mixed systems, suggesting that there is hidden structure exclusive to mixed logic that does not exist in ternary logic, which may either be beneficial or detrimental.

Then, in \autoref{universality}, we discuss the topic of universal computation, which explores sufficient conditions for a quantum computer to be capable of executing arbitrary quantum algorithms. We follow the results of \cite{textbook} and its sources \cite{cnot-decomposition, universal-qubit} for achieving universal computation in a qubit computer with standard gate sets, and provide original but straight-forward generalisations of these results to mixed computers, finding that tightly analogous sets of elementary operations are sufficient in mixed contexts as long as at least one qubit is available.

After this, in \autoref{finite-gen} we describe how an important set of operations called the Clifford operations do not provide any way of transmitting data between a qubit and a qutrit, dramatically changing what can be done with such Clifford operations. We discuss how one property of the Clifford operations -- that they form a finite group -- might apply to other sets of operations as well, including sets that can transmit data between qubits and qutrits, and describe the way that \cite{universal-qubit} applied algebraic number theory to prove results about operations that we can use to show they form infinite groups.

We then perform a programmatic search for groups of matrices to see which are finite and which are infinite. The search is implemented in the C programming language, without any external dependencies, using integer/rational arithmetic to keep track of the algebraic roots of unity that occur in the binary and ternary Clifford groups. We describe the algorithms used, including a summary of the open addressed hash map that we implement, before stating the results of the search, including novel operations that appear in groups which we either prove or conjecture to be infinite.

At the end of all of this in \autoref{conclusion} we summarise the key points that we found in our original research, and point to the vast possibilities for future research in mixed quantum systems directly following from the discoveries and literature reviewed in this thesis.

\chapter[PRELIMINARIES]{Preliminaries}\label{preliminaries}
\section{Algebra of Unitary Matrices}
In quantum mechanics it will turn out crucial to have a strong theory of unitary operations acting linearly on complex-valued objects, so to that end we shall define these concepts and their notation here.
\subsection{Hilbert Spaces}
Cartesian coordinates provide a powerful abstract way of reasoning about physical space as the combination of 3 variables, or conversely a way of visualizing combinations of variables as planes or volumes within a physical space. Hilbert spaces are a description which abstracts the Cartesian coordinate system even further, describing a much larger class of mathematical objects with similar geometric properties, including the space of possible states that a quantum object can take.

Hilbert spaces are defined explicitly in \autoref{inner-products}, they are a class of vector space with an inner product, that is closed under limits. Naturally the field defining the coordinates of a complete vector space ought to be either $\mathbb{R}$ or $\mathbb{C}$, and in this thesis we will always use $\mathbb{C}$. In \autoref{qm} we will see that quantum computation most commonly acts on finite dimensional Hilbert spaces, which are structurally equivalent to $\mathbb{C}^n$, as outlined in \autoref{coords-inner-product}. $\mathbb{C}^n$ is the set of $n$-dimensional column vectors
\[\left\{\left[\begin{matrix} a_0\\a_1\\\vdots\\a_n\end{matrix}\right]\ \middle|\ a_0, a_1, \dots a_n \in \mathbb{C}\right\},\]
equipped with the usual definitions of addition and scaling,
\[
c_0\left[\begin{matrix} a_0\\a_1\\\vdots\\a_n\end{matrix}\right]
+
c_1\left[\begin{matrix} b_0\\b_1\\\vdots\\b_n\end{matrix}\right]
=
\left[\begin{matrix} c_0a_0+c_1b_0\\c_0a_1+c_1b_1\\\vdots\\c_0a_n+c_1b_n\end{matrix}\right],
\]
along with the simplest possible inner product,
\[
\left(
\left[\begin{matrix} a_0\\a_1\\\vdots\\a_n\end{matrix}\right]
,
\left[\begin{matrix} b_0\\b_1\\\vdots\\b_n\end{matrix}\right]
\right)
=
\left[\begin{matrix} a_0^*&a_1^*&\cdots&a_n^*\end{matrix}\right]
\left[\begin{matrix} b_0\\b_1\\\vdots\\b_n\end{matrix}\right]
= \sum_{i=0}^{n-1} a_i^*b_i.
\]

One caveat to this equivalence is the concept of basis-independence, that many concepts in linear algebra are made useful by the fact that they give the same result regardless of which basis is used for the space in question. To show that something is basis independent, it is sufficient and often convenient to define it directly in terms of the inner product and vector operations, rather than the coordinate system induced by a given basis. As an example the $\ell^2$ norm in $\mathbb{C}^n$ does not depend on the basis used:
\[\norm{v}_2 = \sqrt{(v, v)} = \sqrt{\sum_{i=0}^{n-1}\ord{v_i}^2}\]
This cannot be done for any other $\ell^p$ norm.

\subsection{Dirac Notation}\label{dirac}
There are a number of notations available for reasoning about linear algebra and vector spaces, but we shall see that in order to reason about quantum algorithms we will rely heavily on the following techniques:
\begin{itemize}
	\item linear operators defined in terms of inner products
	\item change of basis via unitary operators
	\item change of index set when summing vectors
\end{itemize}
For this collection of techniques the Dirac notation for vectors and linear operators is particularly well suited.

The main feature of Dirac notation is the ket, where a bar and angle bracket are used to distinguish a symbol as representing a vector: $\ket{v}$, $\ket{*}$, $\ket{0}$, $\ket{i}$, etc.

The most common vectors used are the canonical basis vectors, and in Dirac notation it is natural to use the index of each basis vector as the symbol \textit{for the vector itself}:
\begin{align*}
	\ket{0} = \left[\begin{matrix}
		1\\
		0
	\end{matrix}\right]
	&&&
	\ket{1} = \left[\begin{matrix}
		0\\
		1
	\end{matrix}\right]
\end{align*}

The inner product of two vectors $(\ket{u}, \ket{v})$ is normally written $\braket{u}{v}$, called a `bra-ket'. This is inspired by the notation $\langle u, v\rangle$, used before Dirac, but allows an elegant representation of co-vectors. Co-vectors map a vector to a scalar, and in an inner product space every vector has a natural co-vector defined using the inner product. In Dirac notation we can simply omit the ket from an inner product to notate a co-vector:
\begin{align*}
	\bra{u} &: \mathbb{C}^n \to \mathbb{C}
	\\\bra{u} &\equiv \ket{v} \mapsto \braket{u}{v}
\end{align*}

\subsection{Linear Operators}
If $U$ and $V$ are Hilbert spaces, (or generally vector spaces) then a map $A: U \to V$ is a linear operator if it satisfies the following in general:
\[A(a\ket{u} + b\ket{v}) = aA\ket{u} + bA\ket{v}\]
If we have vectors $\ket{u} \in U$ and $\ket{v} \in V$ then we can define a simple linear operator using the inner product in $U$:
\[\ket{v}\bra{u} \equiv \bra{u'} \mapsto \braket{u}{u'}\ket{v}\]
The fact that this is linear is a consequence of the inner product being linear in its second argument, one of the defining properties of Hilbert spaces.

Note that if we reverse the notation for scaling a vector we get the expression $\ket{v}\braket{u}{u'}$, which looks very similar to the application $\ket{v}\bra{u}(\ket{u'})$, again equivocating the map with the object it produces, just as was done in defining co-vectors.

As we have discussed, when dealing with any finite dimensional Hilbert space we can assume the space is equivalent to some complex vector space $\mathbb{C}^n$, in which case matrix representations of a linear operator become available. We derive matrix representation explicitly, since the relevant technology is readily available from Dirac notation. First observe that linear operators $A$ say, are determined uniquely by the image of each basis vector $\ket{j}$, say $A\ket{j} = \ket{v_j}$. First evaluate an arbitrary application of $A$:
\begin{align*}
	A\left(\sum_{j=0}^{n-1}u_j\ket{j}\right)
	&= \sum_{j=0}^{n-1}u_jA\ket{j}
	\\&= \sum_{j=0}^{n-1}u_j\ket{v_j}
\end{align*}
Then this turns out to be the same result that we get from applying the following:
\begin{align*}
	\left(\sum_{j=0}^{n-1}\ket{v_j}\bra{j}\right)\left(\sum_{i=0}^{n-1}u_i\ket{i}\right)
	&= \sum_{i=0}^{n-1}\sum_{j=0}^{n-1}u_i\braket{j}{i}\ket{v_j}
	\\&= \sum_{j=0}^{n-1}u_j\ket{v_j}
\end{align*}
Since these two maps are equal on their whole domain, they are the same:
\[A = \sum_{j=0}^{n-1} \ket{v_j}\bra{j}\]
Further if the codomain is also finite dimensional then we can repeat this process. First note that the outer product notation $\ket{v}\bra{u}$ is linear on $\ket{v}$, regardless of which vector $\ket{u'}$ it is applied to:
\[(a\ket{v_1}+b\ket{v_2})\braket{u}{u'} = a\ket{v_1}\braket{u}{u'}+b\ket{v_2}\braket{u}{u'}\]
Now suppose the image vectors $\ket{v_j}$ are in $\mathbb{C}^m$ and have coordinates themselves, $\ket{v_j} = \sum_i^{m-1} a_{ij}\ket{i}$, then:
\begin{align*}
	A = \sum_{i,j} a_{ij}\ket{i}\bra{j}
\end{align*}
That is, all linear operators are a linear combination of outer products between basis vectors, making these outer products a basis for the (Hilbert) space of linear operators between finite dimensional Hilbert spaces. We have derived a way of representing linear operators $\mathbb{C}^n \to \mathbb{C}^m$ as an array of $n \times m$ scalar coordinates, which is the exact structure we want from a matrix representation, and matrix multiplication appears directly from composing the corresponding maps:
\[\left(\sum_{i,j}a_{ij}\ket{i}\bra{j}\right)\left(\sum_{j',k}b_{j'k}\ket{j'}\bra{k}\right) = \sum_{i,k}\left(\sum_j a_{ij}b_{jk}\right)\ket{i}\bra{k}\]

For example any linear operator $A: \mathbb{C}^2 \to \mathbb{C}^2$ can be written as follows:
\begin{align*}
	A = \left[\begin{matrix}
		a&b\\
		c&d
	\end{matrix}\right]
	=&\ 
	a\left[\begin{matrix}
		1&0\\
		0&0
	\end{matrix}\right]
	+
	b\left[\begin{matrix}
		0&1\\
		0&0
	\end{matrix}\right]
	+
	c\left[\begin{matrix}
		0&0\\
		1&0
	\end{matrix}\right]
	+
	d\left[\begin{matrix}
		0&0\\
		0&1
	\end{matrix}\right]
	\\=&\  a\ket{0}\bra{0}+b\ket{0}\bra{1}+c\ket{1}\bra{0}+d\ket{1}\bra{1}
\end{align*}

When defining a linear operator $L: \mathbb{C}^n \to \mathbb{C}^n$ we can define it as the `linear extension' of a more succinct map $\phi: \{\ket{i}\ |\ 0 \leq i < n\} \to \mathbb{C}^n$, by first defining $\ket{v_j} = \phi(\ket{j})$ and then setting $L = \sum_{j} \ket{v_j}\bra{j}$. This is simply a matrix whose columns are the vectors $\ket{v_j}$.
\subsection{The Hermitian Adjoint and Diagonalisable Matrices}\label{diagonalisable}
An aspect of linear algebra essential to quantum computation is diagonalisation. We shall describe a number of properties that a square matrix can have, in terms of both its Hermitian Adjoint, and its eigenvalues, but first we define these concepts.

The Hermitian Adjoint of a linear operator $A$ (square or otherwise) is the unique linear operator $A^\dagger$ that satisfies $\braket{u}{v'} = \braket{u'}{v}$ whenever $\ket{v'} = A\ket{v}$, and $\ket{u'} = A^\dagger \ket{u}$. It turns out that the matrix representation of $A^\dagger$ is exactly the complex conjugate of the transpose of $A$, so this is taken as the definition of the Hermitian Adjoint of a matrix. As an example observe the following matrix $A$ and its Hermitian adjoint:
\begin{align*}
	A = \left[\begin{matrix}
		1 & 1+i\\
		0 & 1
	\end{matrix}\right]
	&&&
	A^\dagger = \left[\begin{matrix}
		1 & 0\\
		1-i & 1
	\end{matrix}\right]
\end{align*}

Note now that in the vector space $\mathbb{C}^2$ the Hermitian adjoint of a vector has the same behaviour as a co-vector:
\begin{align*}
	\ket{u}^\dagger\ket{v}
	=&\ 
	\left[\begin{matrix}
		a\\
		b
	\end{matrix}\right]^\dagger
	\left[\begin{matrix}
		c\\
		d
	\end{matrix}\right]
	\\=&\ 
	\left[\begin{matrix}
		a^*&b^*
	\end{matrix}\right]
	\left[\begin{matrix}
		c\\
		d
	\end{matrix}\right]
	\\=&\ a^*c+b^*d
	\\=&\ \braket{u}{v}
\end{align*}
This generalizes to any co-vector in any Hilbert space.

Associativity of the matrix product also lets us represent the outer product $\ket{u}\bra{v}$ using the Hermitian adjoint, as the matrix product $\ket{u}\ket{v}^\dagger$. This can be used to prove that
$(\ket{u}\bra{v})^\dagger = \ket{v}\bra{u}$

Now we move on to the eigenvector problem, which is the problem of finding a scalar $\lambda$ and non-zero vector $\ket{v}$ so that $A\ket{v}=\lambda\ket{v}$. Such a $\lambda$ is called an eigenvalue of $A$, and such a $\ket{v}$ is called the corresponding eigenvector of $A$. For example if A has a non-trivial null-space, then $\lambda=0$ will be an eigenvalue of $A$, and any vector in the null-space of $A$ will be a corresponding eigenvector of $A$: $A\ket{v} = 0\ket{v}$.

If $\ket{v}$ is normalized, then the matrix $B=\lambda\ket{v}\bra{v}$ will also satisfy the eigenvalue problem $B\ket{v}=\lambda\ket{v}$, and any vector orthogonal to $\ket{v}$ will be in the null-space of $B$. This means that if all of the eigenvectors of $A$ are orthogonal to eachother, then we can write $A$ as a sum of such $B$ vectors. This structure $A = \sum_i \lambda_i \ket{v_i}\bra{v_i}$ tells us that $A$ acts like a diagonal matrix on its eigenvalues, which we call the diagonal representation of $A$, and for this reason call any such $A$ `diagonalisable'.

As an example, the following matrix $Z$ is already diagonal and so has a straight-forward diagonal representation:
\begin{align*}
	Z = \left[\begin{matrix}
		1&0\\
		0&-1
	\end{matrix}\right] = 1\ket{0}\bra{0} - 1\ket{1}\bra{1}
\end{align*}

Having orthogonal eigenvectors is closely related to the Hermitian adjoint through a class of matrices called `normal' matrices, where a matrix $A$ is normal if it satisfies $A^\dagger A = AA^\dagger$. A major result of linear algebra called the spectral theorem of normal matrices, is that a matrix is diagonalisable if and only if it is normal. What follows is a list of different classes of normal operator, each defined by a property of the matrix, and by an equivalent property of all of its eigenvalues:
\begin{align*}
	\text{Hermitian matrix:\ }&&& A^\dagger = A & \iff& \lambda \in \mathbb{R} \\
	\text{Unitary matrix:\ }&&& A^\dagger = A^{-1} & \iff& \ord{\lambda}=1 \\
	\text{Positive normal:\ }&&& \bra{v}A\ket{v} \in \mathbb{R}, \geq 0 & \iff& \lambda \in \mathbb{R}, \geq 0 \\
	\text{Positive definite normal:\ }&&& \bra{v}A\ket{v} \in \mathbb{R}, > 0 & \iff& \lambda \in \mathbb{R}, > 0 \\
	\text{Projection matrix:\ }&&& A^2 = A & \iff& \lambda \in \{0, 1\} \\
\end{align*}

The most relevant of these is the unitary matrix, whose defining property can also be written $AA^\dagger = I$, which is simply the matrix equation corresponding to the fact that the columns of $A$ form an orthonormal basis. In addition to the above, another characterisation of unitary matrices is that they preserve inner products, which follows naturally from the matrix interpretation of inner products:
\[(A\ket{u}, A\ket{v}) = (A\ket{u})^\dagger A\ket{v} = \bra{u}AA^\dagger \ket{v} = \braket{u}{v}\]
This means in particular that a unitary matrix will preserve $\ell^2$ norms, and will therefore induce an invertible operation on any (complex, centre at origin) $n$-sphere $\{\ket{v}\ |\ \braket{v}{v} = r^2\}$.

Finally, if $P$ is a polynomial then we can evaluate it on a square matrix $A$ as well, and it is easy to show that this polynomial `applies itself' to the eigenvalues of $A$, i.e.\ if $A\ket{v} = \lambda\ket{v}$ then $P(A)\ket{v} = P(\lambda)\ket{v}$.

Now any analytic function will be a limit of some polynomials, most notably the exponential:
\[e^x = \sum_n^\infty \frac{x^n}{n!}\]
This motivates us to apply such analytic functions directly to normal matrices, simply by applying it to each eigenvalue:
\begin{align*}
	f\left(\sum_i \lambda_i\ket{i}\bra{i}\right)
	&= \lim_{L \to \infty} \sum_{n=0}^L P_i\left(\sum_i \lambda_i\ket{i}\bra{i}\right)
	\\&= \lim_{L \to \infty} \sum_{n=0}^L \sum_i P_i(\lambda_i)\ket{i}\bra{i}
	\\&= \sum_i \lim_{L \to \infty} \left(\sum_{n=0}^L  P_i(\lambda_i)\right)\ket{i}\bra{i}
	\\&= \sum_i f(\lambda_i)\ket{i}\bra{i}
\end{align*}

This can be very useful for solving certain matrix equations such as $A^2 = B$ having the solution $A = \sqrt{B}$. It is also useful for mapping between Hermitian and unitary matrices with the map $U = e^{iH}$.

\subsection{Kronecker Products}
We have made explicit the way in which vectors, co-vectors, and linear operators are represented as matrices, that is as arrays of complex numbers. An operation that is useful for all of these objects is the tensor product, and while the tensor product can be described in the abstract as an operation between Hilbert spaces, the only cases of interest here are complex-valued matrices, whose tensor products are described by a much more concrete operation called the Kronecker product.

The Kronecker product has a fairly simple definition, if $A$ is an $m_1$ by $n_1$ matrix with elements $a_{ij}$, and $B$ is an $m_2$ by $n_2$ matrix with elements $b_{ij}$ then the Kronecker product $A \otimes B$ will be an $m_1m_2$ by $n_1n_2$ matrix and in block matrix form will look like the following:
\begin{align*}
	A &= \left[\begin{matrix}
		a_{00} & a_{01} & \dots & a_{0n_1}\\
		a_{10} & a_{11} & \dots & a_{1n_1}\\
		\vdots & \vdots & \ddots & \vdots\\
		a_{m_10} & a_{m_11} & \dots & a_{m_1n_1}
	\end{matrix}\right]
	\\\implies A\otimes B &= \left[\begin{matrix}
		a_{00}B & a_{01}B & \dots & a_{0n_1}B\\
		a_{10}B & a_{11}B & \dots & a_{1n_1}B\\
		\vdots & \vdots & \ddots & \vdots\\
		a_{m_10}B & a_{m_11}B & \dots & a_{m_1n_1}B
	\end{matrix}\right]
\end{align*}
Written more compactly, if $A\otimes B = C$ has elements $c_{m_2i_1 + i_2,n_2j_1+j_2}$, where $i_1$ is less than $m_1$, etc.\ then these elements of $C$ are exactly the products of elements of $A$ and $B$:
\[c_{m_2i_1 + i_2,n_2j_1+j_2} = a_{i_1j_1}b_{i_2j_2}\]

If $A$ is a vector $\ket{u}$ and $B$ is a co-vector $\bra{v}$ then their Kronecker product will be exactly the matrix product $AB$ which is exactly the outer product $\ket{u}\bra{v}$:
\[
A \otimes B =
\left[\begin{matrix}
	a_0\\a_1\\\vdots\\a_m
\end{matrix}\right]
\otimes
\left[\begin{matrix}
	b_0&b_1&\dots&b_n
\end{matrix}\right]
=
\left[\begin{matrix}
	a_0b_0 & a_0b_1 & \dots & a_0b_n\\
	a_1b_0 & a_1b_1 & \dots & a_1b_n\\
	\vdots & \vdots & \ddots & \vdots\\
	a_mb_0 & a_mb_1 & \dots & a_mb_n
\end{matrix}\right]
\]
On the other hand if $A$ and $B$ are both vectors, in $\mathbb{C}^{m_1}$ and $\mathbb{C}^{m_2}$ respectively, then their Kronecker product will be a vector in the larger space $\mathbb{C}^{m_1\times m_2}$, and in particular if $A$ and $B$ are canonical basis vectors $\ket{j_1}$ and $\ket{j_2}$ then their tensor product will be another canonical basis vector, $\ket{m_2j_1 + j_2}$. This allows the whole space $\mathbb{C}^{m_1 \times m_2}$ to be spanned by Kronecker products of $\mathbb{C}^{m_1}$ and $\mathbb{C}^{m_2}$.

The Kronecker product satisfies the following algebraic properties, all of which follow directly from their relevant definitions:
\begin{itemize}
	\item $(A\otimes B)(C \otimes D) = (AC) \otimes (BD)$
	\item in particular $(A\otimes B)(\ket{u} \otimes \ket{v}) = (A\ket{u})\otimes (B\ket{v})$
	\item $I_{n_1} \otimes I_{n_2} = I_{n_1\times n_2}$
	\item $(A\otimes B)^\dagger = A^\dagger \otimes B^\dagger$
	\item in particular $(\ket{u}\otimes \ket{v})^\dagger = \bra{u} \otimes \bra{v}$
	\item if $\lambda$ is a scalar then $\lambda (A \otimes B) = (\lambda A) \otimes B = A \otimes (\lambda B)$
	\item in particular $\lambda (\ket{u} \otimes \ket{v}) = (\lambda \ket{u}) \otimes \ket{v} = \ket{u} \otimes (\lambda \ket{v})$
\end{itemize}
With these properties we can show that the Kronecker product of two unitary matrices will be unitary:
\begin{align*}
	(A \otimes B)(A \otimes B)^\dagger
	&= (A \otimes B)(A^\dagger \otimes B^\dagger)
	\\&= AA^\dagger \otimes BB^\dagger
	\\&= I \otimes I
	\\&= I
\end{align*}

Considering the degrees of freedom involved, we expect that in general there are many $m_1m_2 \times n_1n_2$ matrices that are not the Kronecker product of an $m_1 \times n_1$ matrix with an $m_2 \times n_2$ matrix. Written in block matrix form it is easy to tell whether or not a matrix $C$ is a Kronecker product $A \otimes B$, by checking that every block $B_{ij}$ is a scalar multiple of any non-zero block $B$. The entries of $A$ are simply these scalar multiples, solving $B_{ij} = a_{ij}B$, and hence $C = A \otimes B$. (and if there are no non-zero blocks then $C = 0 = 0 \otimes 0$)
\subsection{The Unitary Group}
So far we have discussed a number of useful algebraic properties exhibited by unitary matrices, but the most basic properties of unitary matrices are of course their forming a group:
\begin{itemize}
	\item Product of unitaries is unitary: ${(AB)}^\dagger = B^\dagger A^\dagger = B^{-1}A^{-1} = {(AB)}^{-1}$
	\item Matrix products are always associative (since composition of any operator will be associative)
	\item The identity matrix is unitary: $I^\dagger = I = I^{-1}$
	\item $A^{-1} = A^\dagger$ always exists and is unitary
\end{itemize}

Quantum computation deals extensively with unitary operators, and so subgroups formed by particular sets of unitary operators will be a frequent point of discussion. A subgroup of a group is simply a subset that is still a group when equipped with the same group operation. Our group operation is matrix multiplication, and so the group of $n \times n$ unitaries written $U(n)$ is actually a subgroup of the much larger general linear group $GL(n, \mathbb{C})$, of $n \times n$ matrices with non-zero determinant.

A further subgroup of the set of unitary matrices $U(n)$ is the symmetric group $\mathcal{S}_n$, which we represent as the set of $n \times n$ permutation matrices. A permutation is an invertible function mapping a set to itself, and given a permutation on the canonical basis $\{\ket{i}\ |\ 0 \leq i < n\} \to \{\ket{i}\ |\ 0 \leq i < n\}$, we can extend this linearly to an invertible matrix $\mathbb{C}^n \to \mathbb{C}^n$ whose columns are all canonical basis vectors. We call such a matrix a permutation matrix, and note that the adjoint of its outer product form $P = \sum_i \ket{\sigma(i)}\bra{i}$ will be $P^{\dagger} = \sum_i \ket{i}\bra{\sigma{i}}$, giving the following:
\begin{align*}
	PP^\dagger = \sum_{i,j} \ket{\sigma(i)}\braket{i}{j}\bra{\sigma(j)} = \sum_i \ket{\sigma(i)}\bra{\sigma(i)} = I
\end{align*}

The symmetric group of any finite set is always finite, and is typically the first group considered when exploring finite subgroups of $U(n)$.

\section{Quantum Mechanics}\label{qm}

The theory of Quantum mechanics is foundational in contemporary understanding of physical systems, and features phenomena radically different to that of the classical world when the objects under consideration are able to achieve `coherence', e.g.\ when sufficiently small or low in temperature.

Quantum mechanics is deliberately `incomplete' in the sense that it doesn't make specific predictions about the behaviour of atoms, electrons, or anything else, but rather poses constraints and structure that other physical theories such as quantum electrodynamics can adopt.

Quantum computation is the study of how quantum phenomena can be used to effect computation. When designing an individual experiment or programmable quantum computer it is necessary to use a concrete, empirically verified model of physical phenomena, but the specific algorithms that can be run on a quantum computer turn out to be described sufficiently by the general theory of quantum mechanics alone. Since this thesis is concerned with algorithms, we shall summarize the description of quantum mechanics given in the ubiquitous textbook "Quantum Computation and Quantum Information" by Nielsen and Chuang \cite{textbook}. This will provide a foundation for explaining certain techniques and algorithms prominent in quantum computing, and their relationship to the questions we aim to explore.
\subsection{State Space}
The first postulate of quantum mechanics is that any quantum system can be modeled by some complex Hilbert space. While many of these Hilbert spaces are of infinite dimension, we can easily create finite dimensional Hilbert spaces by either trapping or ignoring the positions and other such properties of individual particles. Once we have an $n$-dimensional Hilbert space $\mathbb{H}$ modeling a system, we can substitute this for the equivalent Hilbert space $\mathbb{C}^n$, allowing us to model the system directly with this set.

We call the specific elements of $\mathbb{C}^n$ state vectors whenever they represent a state that a quantum computer can take, and we shall see soon that the state vectors are exactly the vectors of unit length in $\mathbb{C}^n$. This allows us to model quantum computation itself as an operation on these unit length state vectors, and to describe the initial, final, and intermediate states of any given computation as an individual state vector.

The simplest possible quantum system is an isolated quantum bit, or qubit, which can be modelled with the 2-dimensional Hilbert space $\mathbb{C}^2$:
\[
\mathbb{C}^2 = \left\{\left[\begin{matrix}
a\\
b
\end{matrix}\right]\ \middle|\ a, b \in \mathbb{C}\right\}
\]
\subsection{Evolution}
The second postulate of quantum mechanics can be stated in three equivalent ways: Given a closed quantum system with initial state vector $\ket{\phi}$ these 3 equivalent equations hold:
\[-\frac{ih}{2\pi}\frac{d}{dt}\ket{\phi} = H\ket{\phi}\]
\[\ket{\phi} = \exp\left(\frac{i 2\pi tH}{h}\right)\ket{\phi_0}\]
\[\ket{\phi} = U\ket{\phi_0}\]
Where $\hbar$ is the reduced Planck constant, which is a known scalar constant, $H$ is the Hamiltonian operator, a Hermitian operator associated with the energy present in the system, and $U$ is the unitary operator arising from fixing $t$ before exponentiating $i2\pi tH/h$.

A quantum computer, then, is simply a device that is able to manipulate the Hamiltonian $H$ of a quantum system, for specific durations of time $t$. For example, suppose there is a qubit system such as an electron that can be in one of two spin states, or an ion that can be in ground or excited states, and there is an external electromagnetic field that can be activated on command which affects one of these qubit states without affecting the other, giving a Hamiltonian like
\[H = \begin{bmatrix}
0 & 0 \\ 0 & c
\end{bmatrix},\]
then by setting $t = h/2c$ we get
\begin{align*}
U &= \exp\left(\frac{i 2\pi tH}{h}\right)
\\&= \exp\left(\begin{bmatrix}
0 & 0 \\
0 & \frac{i 2\pi tc}{h}
\end{bmatrix}\right)
\\&= \exp\left(\begin{bmatrix}
0 & 0 \\
0 & i\pi
\end{bmatrix}\right)
\\&= \begin{bmatrix}
1 & 0 \\
0 & -1
\end{bmatrix}
\end{align*}
When a quantum computer is capable of implementing a particular $H$, $t$ pair, we call the corresponding unitary matrix $U$ an elementary evolution of the quantum computer. Composing these elementary evolutions will always give some overall unitary matrix $U = U_1U_2\dots U_k$, so the core of quantum computation then, is to understand how computationally interesting unitary matrices can be implemented using elementary evolutions of a quantum computer.
\subsection{Measurement}
Although the quantum systems we are discussing are represented as having a continuous state space of possible state vectors, a fundamental (and titular!) peculiarity of quantum mechanics is that when observed, a quantum object will always appear to be in states that are elements of some orthonormal basis of the corresponding Hilbert space. What is more peculiar, and more well known of quantum mechanics, is that the state that is measured becomes the new state of the system.

It is impossible to perform any computationally useful algorithm without measuring the result at some point in that algorithm, so in any theoretical discussion it is required that at least one procedure is possible for measuring the system. Whichever procedure is used, we call the corresponding orthonormal basis the `computational basis`. When identifying the given Hilbert space $\mathbb{H}$ with the vector space $\mathbb{C}^n$, we make sure to also identify the computational basis of $\mathbb{H}$ with the canonical basis $\{\ket{i}\}$ of $\mathbb{C}^n$, so that measurement can always be done in this basis.

Suppose we have a quantum system with state vector $\ket{\phi}$, and let $\ket{\phi}$ equal the following:
\[\ket{\phi} = \sum_{i=0}^n a_i\ket{i}\]

When measured this system will appear to be in state $\ket{i}$ with probability $\ord{a_i}^2$, and after measurement will be in the state $a_i'\ket{i}$, where $a_i'$ is the normalized complex number $\frac{a_i}{\ord{a_i}}$.

In more algebraic terms this can be stated as follows:
When measured this system will appear to be in state $\ket{i}$ with probability $\ord{\braket{i}{\phi}}^2$, and after measurement will be in the state that comes from applying the projection matrix $\ket{i}\bra{i}$ and normalizing.

Given that the coordinates of a state vector are now interpreted as being related to probabilities, it is important that state vectors have unit length:
\[\norm{\ket{\phi}} = \sum_{i=0}^n \ord{a_i}^2 = 1\]

A defining feature of unitary matrices is that they preserve vector magnitudes, meaning this assumption will be preserved at all points in time in a quantum algorithm.

As an example consider the $\ket{-}$ state:
\[\ket{-}=\frac{1}{\sqrt{2}}\ket{0} - \frac{1}{\sqrt{2}}\ket{1}\]
When measured, this will have probability $1/2$ of appearing to be in the $\ket{0}$ state, after which it will in fact be in the $\ket{0}$ state, and probability $1/2$ of appearing to be in the $\ket{1}$ state, although in actual fact it will be in the $-\ket{1}$ state.

In physical quantum experiments it is often possible to measure in multiple different bases, for example in the $\ket{+}$/$\ket{-}$ basis. While this may have practical advantages, the map between these bases will be unitary anyway so such non-computational measurements can be added or removed as needed by implementing the appropriate unitary instead, and by doing so we can get away with only using the computational basis, making for a more general theory.
\subsection{Composite State Space}
Suppose that we have two quantum systems, which when closed and isolated from each other would be modeled as $\mathbb{C}^m$ and $\mathbb{C}^n$ respectively. When we allow these to interact with each-other, and form a closed composite system, we will need some distinct Hilbert space with which to model them. Intuitively if we measure both systems we should find the first in some computational basis state $\ket{i}$ and the second in some second computational basis state $\ket{j}$, giving a total of $m \times n$ different measurements. The measurement postulate described for individual systems will apply just as well to this coupled system, and so we have to extend these $m \times n$ different pairs of states to the Hilbert space $\mathbb{C}^{m \times n}$, whose computational basis is now given by Kronecker products $\ket{i} \otimes \ket{j}$.

In general if we had of modeled the systems in states $\ket{v_1}$ and $\ket{v_2}$ then we can embed these states in the coupled system as the state $\ket{v_1} \otimes \ket{v_2}$. We shall see in the section on composite measurement that these Kronecker products are postulated to behave in the same way as the individual states when measured. As an example of these embedded states, consider the following pair of qubit states:
\begin{align*}
\ket{1} = \left[\begin{matrix}0\\1\end{matrix}\right]
&&&
\ket{+} = \left[\begin{matrix}\frac{1}{\sqrt{2}}\\\frac{1}{\sqrt{2}}\end{matrix}\right]
\end{align*}
When taking two such qubit objects as a coupled system, the corresponding Kronecker state would look like this:
\[
\ket{1}\otimes \ket{+} = \left[\begin{matrix}0\\1\end{matrix}\right]
\otimes
\left[\begin{matrix}\frac{1}{\sqrt{2}}\\\frac{1}{\sqrt{2}}\end{matrix}\right]
=
\left[\begin{matrix}0\\0\\\frac{1}{\sqrt{2}}\\\frac{1}{\sqrt{2}}\end{matrix}\right]
\]

Note that when taking the Kronecker product of multiple vectors or multiple co-vectors, we can suppress the $\otimes$ operator, simply writing it as a concatenated sequence of kets: $\ket{v_1}\ket{v_2} = \ket{v_1}\otimes\ket{v_2}$. This does not introduce any ambiguity since the matrix product of two vectors is not defined.

In classical computation we represent integers and other data as a sequence of small (typically binary) digits and in the same way we can represent integers as computational basis vectors $\ket{i_1}\ket{i_2}\dots\ket{i_n}$ in some large system made from composing individual `quantum digits', binary or otherwise. In other contexts it is common to take the corresponding sequence of digits as the identifier for a single ket $\ket{i_1i_2\dots i_n}$ but we shall refrain from doing this here, since it poses no significant advantage when dealing with the most foundational aspects of quantum computation.

As was described in our original discussion of Kronecker products, not all quantum states in a composite system $\mathbb{C}^m\otimes\mathbb{C}^n$ are of the form $\ket{u}\otimes\ket{v}$. States that can't be represented as a single Kronecker product $\ket{u}\otimes\ket{v}$ are called entangled states. Entanglement is a very important feature of quantum mechanics and quantum computation that will be discussed on its own later in this document. For now though observe the Bell state:

\[\frac{1}{\sqrt{2}}\ket{0}\otimes\ket{0} + \frac{1}{\sqrt{2}}\ket{1}\otimes\ket{1} = \left[\begin{matrix}
\frac{1}{\sqrt{2}}\\
0\\
0\\
\frac{1}{\sqrt{2}}
\end{matrix}\right]\]
We can see this has two non-zero blocks, $1/\sqrt{2}\ket{0}$ and $1/\sqrt{2}\ket{1}$ respectively, which are not multiples of each-other.

Seeing as classical systems aren't treated as having any form of superposition, there is no classical analogy for entangled states. It is worth noting that in classical systems a composite system would be understood to have a state space that is the Cartesian product of two individual state spaces, and that the Cartesian product of an $m$-dimensional vector space with an $n$-dimensional vector space would have $m+n$ degrees of freedom rather than the $m\times n$ degrees of freedom given by the vector spaces associated with composite quantum systems.
\subsection{Composite Evolution}
Just as with individual systems, an isolated composite system evolves according to some unitary operator after any discrete time step. In the composite system $\mathbb{C}^{mn}$ this could be any $mn$ by $mn$ unitary matrix, and once again we call any such operation an elementary gate of a quantum computer whenever it can be reliably produced by that computer. Typically the elementary gates that can be implemented for individual systems will be implemented in the same way in composite systems, again using the Kronecker product to embed the corresponding operation into the composite model:
\[U \otimes I = \sum_i \ket{u_i}\otimes\ket{v_i} \mapsto \sum_i (U\ket{u_i})\otimes \ket{v_i}\]

For example the operation $I \otimes H$ acting on the 2-qubit system $\mathbb{C}^{2\times 2}$, which applies $H$ to the second qubit and leaves the first qubit unchanged, will have the following matrix representation:
\[
\left[\begin{matrix}
	1&0\\
	0&1
\end{matrix}\right]
\otimes
\left[\begin{matrix}
	\frac{1}{\sqrt{2}}&\frac{1}{\sqrt{2}}\\
	\frac{1}{\sqrt{2}}&-\frac{1}{\sqrt{2}}
\end{matrix}\right]
=
\left[\begin{matrix}
	\frac{1}{\sqrt{2}}&\frac{1}{\sqrt{2}}&0&0\\
	\frac{1}{\sqrt{2}}&-\frac{1}{\sqrt{2}}&0&0\\
	0&0&\frac{1}{\sqrt{2}}&\frac{1}{\sqrt{2}}\\
	0&0&\frac{1}{\sqrt{2}}&-\frac{1}{\sqrt{2}}
\end{matrix}\right]
\]

If all of the elementary gates available to a quantum computer can be represented as the Kronecker product of individual gates, and the initial state of the computer can be represented as the Kronecker product of individual states, then the computer will never be able to create entangled states, and will essentially be two distinct quantum computers running in parallel. This means that it is crucial for a quantum computer to have at least one operation that is not the Kronecker product of individual operations, and the most common of these elementary gates is the controlled increment gate, which acts as a permutation on the computational basis $\ket{i}\ket{j} \mapsto \ket{i}\ket{i+j \mod n}$. Since these operations have a different effect on each object in the system, dependent on the state of another object in the system, we call any operation that cannot be factored as a Kronecker product acting on each individual object in the quantum system \emph{dependent}, and otherwise call them \emph{independent}.

The controlled increment operation in binary systems is called the controlled-not operation, which will act mod 2, and therefore simply swap the $\ket{1}\ket{0}$ state with the $\ket{1}\ket{1}$ state. In this case we would call the first qubit the `control' object, and the second qubit the `target' object. This can not be represented as a Kronecker product of two individual operations, which can be seen in the following matrix representation:
\[
\left[\begin{matrix}
1&0&0&0\\
0&1&0&0\\
0&0&0&1\\
0&0&1&0
\end{matrix}\right]
\]
(In fact no controller increment operation will be a Kronecker product of individual operations, since their matrix representation will always have an $I$ block, and an $X$ block, which are not proportional.)

This notion of controlled operations is useful to generalize to any unitary operation acting on a smaller quantum system, which will be discussed in more detail in the context of both binary and ternary quantum systems later.

It is useful to note that the Kronecker product commutes with the outer product:
\[(\ket{u'}\bra{u})\otimes(\ket{v'}\bra{v}) = \ket{u'}\ket{v'}\bra{u}\bra{v}\]
One consequence of this is that as with state vectors, the Kronecker product of canonical basis operators $\ket{i'}\bra{i}$ and $\ket{j'}\bra{j}$ will give all of the canonical basis operators $\ket{i'}\ket{j'}\bra{i}\bra{j}$ on $\mathbb{C}^{m \times n}$.
\subsection{Composite Measurement}
When measuring an individual system we gave the algebraic definition that state $\ket{\phi}$ would be measured in state $\ket{i}$ with probability $\ord{\braket{i}{\phi}}^2$, and would then be in the state that comes from applying the projection $\ket{i}\bra{i}$ and normalizing.

This postulate stated as is would allow us to measure all of the individual components of the composite system at the same time, but it is actually much more powerful to measure one component by itself without otherwise disturbing the rest of the system. To that end suppose we have a composite quantum system whose state vector is the following:

\[\ket{\phi} = \sum_{0 \leq i < m}\sum_{0 \leq j < n} a_{ij}\ket{i}\ket{j}\]

Either the $\mathbb{C}^m$ or the $\mathbb{C}^n$ component of this system could be measured. Without loss of generality we shall describe measurement of the first state. The full statement of the measurement postulate is that this system will be in state $\ket{i}$ with probability:
\[P(\ket{i}) = \sum_{j=0}^n\ord{a_{ij}}^2\]
After measuring the system will be in the state that comes from applying the projection operator $(\ket{i}\bra{i})\otimes I$ and then normalizing.

For example observe the following 2-qubit state vector $\ket{+}\ket{+}$:
\[\ket{+}\ket{+}=\frac{1}{2}\left(\ket{00}+\ket{01}+\ket{10}+\ket{11}\right)=\frac{1}{2}\left[\begin{matrix}
1\\
1\\
1\\
1
\end{matrix}\right]\]
If we measured the first qubit then we would observe $\ket{0}$ with probability $1/4+1/4 = 1/2$, and similarly $\ket{1}$ with probability $1/2$. Upon measuring $\ket{0}$ the overall state would be $\ket{0}\ket{+}$, whereas upon measuring $\ket{1}$ the overall state would be $\ket{1}\ket{+}$.

While this is straight-forward for states that are simply the Kronecker product of two states, measuring entangled states can be quite interesting. For example suppose we measured the first component of the Bell state $1/\sqrt{2}(\ket{00}+\ket{11})$. As before we would find it in state $\ket{0}$ with probability $1/2$, and $\ket{1}$ with probability $1/2$, but after measurement it would be in states $\ket{00}$ and $\ket{11}$ respectively. Measuring the first component of the system changed the second one! This is the basis of a number of novel features of quantum computing, quantum information theory, and quantum cryptography.

All of the above discussion generalise naturally to systems with more than two components, so long as one component is measured at a time. In practice there is no requirement that only one object be measured at a time, or even that every object can be measured individually, but for the sake of simplicity and generality we will focus on the generic quantum computer with multiple objects each individually measurable in their computational basis.

\section{Quantum Computation}

With quantum mechanics and the associated concept of a quantum algorithm in place, we can begin to reason about the techniques of quantum computation. To this end we shall briefly discuss the computer science of probabilistic vs deterministic algorithms, as well as the group theory of different sets of unitary matrices, important concepts that lay the foundation of technical discussions in the field of quantum computation. With this in place we shall be prepared to discuss the state of the field when it comes to quantum computation on objects with more than two computational basis states, and our generalisations of these findings to quantum computers that mix objects with two and more than two basis states.

\subsection{Classical Computation in Quantum Algorithms}
In classical computation we can imagine an algorithm or circuit mapping some $M$ discrete states to some $N$ discrete states, according to some function $f: \{0\dots M-1\} \to \{0\dots N-1\}$. (say $M = 2^m,\ N = 2^n$ where $m$ and $n$ are the number of physical wires leading in and out of the circuit) For example we could define the logical conjunction or AND map which maps pairs of bits $\{00, 01, 10, 11\}$ to single bits $\{0, 1\}$: 
\begin{align*}
	\text{AND}(x) = \begin{cases}
		1 & \text{if\ } x = 11\\
		0 & \text{otherwise}
	\end{cases}
\end{align*}

Given such a map $f$ we can define a linear operator by extending linearly on the computational basis: $A_f\ket{i} = \ket{f(i)}$, giving a matrix whose columns are all computational basis vectors. For example our logical conjunction becomes:

\[
A_\text{AND} = \left[\begin{matrix}
	1&1&1&0\\
	0&0&0&1
\end{matrix}\right]
\]

In quantum computation we require that all operations be reversible, unitary operations. This means that a matrix $A_f$ representing a classical computation $f$ will be available as a unitary operator if and only if $M=N$ and $f$ is invertible, i.e.\ if $f$ is a permutation. When $M = N = 2$ we only have two such permutation matrices:
\begin{align*}
	I = \left[\begin{matrix}
		1&0\\
		0&1
	\end{matrix}\right]
	&&&
	X = \left[\begin{matrix}
		0&1\\
		1&0
	\end{matrix}\right]
\end{align*}

Maps that don't satisfy $M = N$ or $f$ invertible can still be represented as a permutation matrix, acting on $\mathbb{C}^{M\times N}$ as follows:
\[B_f(\ket{i}\otimes\ket{j}) = \ket{i}\otimes\ket{j+f(i) \mod N}\]

The inverse of this matrix is simply $\ket{i}\ket{j} \mapsto \ket{i}\ket{j-f(i) \mod N}$.

In the case of the binary AND map, $B_{\text{AND}}$ will be a $8\times8$ permutation matrix known as the Toffoli gate. It can be shown that compositions of Toffoli gates acting on some number of bits can be used to implement any $n$-qubit permutation matrix as an algorithm on some larger number of qubits $n+m$, so long as the extra $m$ qubits are initialized to known values.

Permutation matrices by themselves might not seem interesting, since they exclusively represent calculations that can be done in classical contexts, but in fact are crucial for many quantum algorithms, since they will act linearly on superposition states. For example if we consider the map $f(x) = x^2\mod 16$, then the constructed matrix $B_f$ acting on 8 qubits will of course distribute linearly over any linear combination of basis states including the following:
\begin{align*}
	B_f(\ket{2}\ket{0}+\ket{5}\ket{0}) 
	= B_f\ket{2}\ket{0} +& B_f\ket{5}\ket{0}
	\\= \ket{2}\ket{4} +& \ket{5}\ket{9}
\end{align*}

Further manipulations or measurement of the result of such a transformation can enable many powerful quantum algorithms including Shor's period finding algorithm. This means that permutation matrices are an important topic in quantum computation, and a good deal of research has been and continues to be done to better understand how permutation matrices can be decomposed into efficient quantum algorithms.

\subsection{Quantum Circuits}
We have described quantum algorithms as a sequence of unitary transformations and measurements on quantum states represented using Kronecker products, but as the number of quantum objects gets large the relevant vectors and matrices grow exponentially, which is a problem because these exponentially large operations are precisely the ones that we can't calculate on a classical computer, so they are of the most interest. In order to notate these operations which could become quite large, we introduce the quantum circuit diagram.

The diagram of a quantum circuit features multiple horizontal lines representing distinct quantum objects, and with squares on these lines representing operations that should be performed on the corresponding objects over time:

\begin{quantikz}
\lstick{$\ket{\phi}$} & \gate{U_1} & \gate{U_3} & \qw \rstick{$\ket{\phi'}$} \\
\lstick{$\ket{0}$} & \gate{U_2} & \qw & \qw \rstick{$\ket{\psi}$}
\end{quantikz}

This would represent two quantum objects, initially in states $\ket{\phi}$ and $\ket{0}$, i.e.\ the whole system was in state $\ket{\phi}\otimes\ket{0}$. Then $U_1\otimes U_2$ is applied to this state, giving
\[(U_1 \otimes U_2)(\ket{\phi} \otimes \ket{0}) = (U_1\ket{\phi})\otimes (U_2\ket{\phi})\]
and then $U_3 \otimes I$, giving overall
\[(U_3 \otimes I)((U_1\ket{\phi})\otimes (U_2\ket{\phi})) = (U_3U_1\ket{\phi})\otimes (U_2\ket{0})\]
We also label these final states $\ket{\phi'} = U_3U_1\ket{\phi}$, and $\ket{\psi} = U_2\ket{0}$.

These diagrams are inspired by classical circuit diagrams, and as such the horizontal lines are typically referred to as `wires', but unlike classical circuits there is no such wire in practice. The horizontal axis of a quantum circuit is strictly chronological, as a single object evolves over time, usually while sitting still in space. Additionally wires can't split or merge, reflecting the fact that all quantum circuits are reversible between measurements. Measurements can still be included in a quantum circuit, however. For example measuring the $\ket{+}$ state can be written:

\begin{quantikz}
\lstick{$\ket{+}$} & \meter{} & \qw \rstick{$\ket{0}$ or $\ket{1}$}
\end{quantikz}

In general measurement can be quite flexible, with different bases of measurement, and with conditional execution of gates depending on the result of measurement, but we only need this simplest case of measurement in the computational basis.

Finally when something other than a Kronecker product of single-object matrices is applied, we can have boxes spanning multiple wires:

\begin{quantikz}
	\lstick{$\ket{i}$} & \gate[2]{B_f} & \qw \rstick{$\ket{i}$}\\
	\lstick{$\ket{j}$} & & \qw \rstick{$\ket{j + f(i) \mod 2}$}
\end{quantikz}
\subsection{Probabilistic Algorithms}
In computer science there is a distinction between deterministic algorithms, and randomized/probabilistic/non-deterministic algorithms. In summary a deterministic algorithm is a sequence of exact steps that can be executed in order to compute a result, whereas a probabilistic algorithm is permitted to rely on some source of random information to determine its control flow, meaning that the same input could result in many different outputs.

The advantage of probabilistic algorithms is that they can often avoid the worst-case performance associated with certain input states in deterministic algorithms; for example, many implementations of sorting algorithms will take much longer than usual to sort a list in ascending order if it is initially in descending order. At an intuitive level such worst-case input states tend to exploit the specific order in which an algorithm explores its possible solutions, and so since a randomized algorithm has no single order in which it might explore solutions, such worst-case inputs do not exist.

On classical computers deterministic algorithms are often more natural, and even when a randomized algorithm is used it will likely use a pseudo-random number generator in place of a true random source of information. This is heavily contrasted with quantum computation and quantum physics more generally, where measuring the state of a quantum object is inherently probabilistic, and physical implementations of quantum algorithms often introduce physical sources of error as well, as a result the probabilistic quantum algorithm is taken as the norm, with the exception of algorithms that only measure computational basis states, which when simulated theoretically will act deterministically. In physical quantum computers to date noise has been of primary concern in the real world execution of algorithms, and so even these algorithms that are deterministic when simulated end up being non-deterministic in practice.

One important consequence of this is that a quantum algorithm that appears to perform well might still perform better on a classical computer with a source of randomness; when comparing quantum with classical, we must consider the probabilistic algorithms of both.

A further distinction in computer science is made between Las Vegas and Monte Carlo algorithms, the former being algorithms that always yield a result, but have random run-time, and the latter being algorithms that yield some result in fixed run-time, but have a random chance of failing or producing a result that is incorrect. Often an algorithm in one of these classes can be converted into the other, Las Vegas algorithms that repeatedly search for a solution can be modified to yield a false negative after a fixed number of attempts, and Monte Carlo algorithms that may produce a false negative can check their solution using a deterministic algorithm, and retry until a valid solution is found. This means that in classical contexts this distinction is less important than that of deterministic vs probabilistic algorithms, since both classes can achieve similar things with the same resources.

In quantum contexts however, algorithms are generally assumed to be Monte Carlo algorithms, since the sources of error discussed are sources of incorrect output, rather than sources of increased run-time. Further when a physical quantum computer is run, it is run repeatedly, often thousands of times, in order to determine the probability of each output, so that one can infer which output is the correct one.

When constructing unitary matrices out of some elementary gate set, it becomes possible to use approximate constructions, since all quantum algorithms are Monte Carlo by default, and small changes in a state's coordinates tend not to significantly affect the overall probability distribution of the algorithm's outputs. Often quantum algorithms and their associated unitary matrices are implemented asymptotically, where the desired accuracy of the algorithm is taken as a parameter, and as this parameter gets smaller a longer sequence of elementary gates is constructed to achieve this level of accuracy.

\subsection{Error Correction}\label{error-codes}
We have described how by controlling the external forces acting on a quantum system, we can implement some set of elementary evolutions on that system. In practice however, no matter how much engineering work we do to control an environment, the forces we deliberately impose are not the only forces present. This means there is always some (hopefully small) quantity of noise affecting the system, and so the longer that a quantum computer is left running, the more deviation there will be between what state the quantum computer is in, and what theoretical state our model would predict it to be in. This is a major source of error in quantum computation, and since these errors persist across time, and even propagate between different quantum objects as those objects interact with each other, this imposes strict limits on how long a naive algorithm can be before the whole process is drowned in error.

The paper \cite{algos} describes many algorithms and demonstrates their implementation on the 5-qubit \verb`ibmqx4`/\verb`ibmqx5` quantum computers, all of which were implemented at scales that would be trivial to implement classically instead, such as finding the smallest $p$ so that $15^p-1$ is divisible by 11. These implementations did not use any error correction, and so results that should have had a 0.5 chance of being measured in simulation, (among eight possible outcomes) were measured with a frequency of 0.25 in practice. This goes to show that if the algorithms had to compute on larger numbers, and thus got any longer than the ten to thirty operations demonstrated in \cite{algos}, then measurement at the end would have started to become entirely random, making computation impossible.

What makes quantum computation scalable beyond this limit is error correction, where redundant qubits are added, so that if the state of only one qubit changes then it can be reset to the state of the others. The theory of error correction is very sophisticated but it is enough for now to know only that it exists, taking as the only example a partial version of Shor's coding scheme\footnote{Shor's full scheme nests this process inside another process for detecting `phase flip' errors as well.} \cite{shor-encoding}, if we represent the logical state $\ket{0}$ with a physical state $\ket{0}\ket{0}\ket{0}$ in three qubits, and a `bit flip' error occurs, turning this into $\ket{0}\ket{1}\ket{0}$, then we can write a quantum algorithm that detects and corrects this error. Algorithms capable of detecting and correcting errors are of a smaller scale to the algorithms demonstrated in \cite{algos}, so by alternating between calculation and error detection one can reasonably expect to perform calculations at a greater scale than naive implementation of algorithms would allow.

When using these redundant coding schemes, a distinction is made between operations that are called fault tolerant, versus those that are not. Taking the example of a three qubit coding scheme capable of detecting bit flip errors, whose physical state is in $\ket{0}\ket{0}\ket{0}$, before undergoing an error is introduced that transforms the physical state into the invalid state $\ket{0}\ket{1}\ket{0}$. As is the design, this error is possible to detect and correct, but what if this error occurs right before a computation is performed? If we want to map the logical state $\ket{0}$ to $\ket{1}$, then we could imagine this erroneous physical state mapping to $\ket{1}\ket{0}\ket{1}$, which after error correction will in fact represent $\ket{1}$. If we had gotten $\ket{1}\ket{0}\ket{0}$ instead, i.e.\ if the error had propagated between physical qubits, then after error correction we would have mapped $\ket{0}$ to $\ket{0}$. An algorithm that always exhibits the former behaviour is called fault tolerant, and in the same way that we have elementary evolutions that are possible on physical qubits, we will also have an analogy for these acting on logical qubits:

An \emph{elementary gate} is a unitary transformation that can be implemented fault tolerantly on some logical states, as a sequence of elementary evolutions acting on the corresponding physical gates.

Once the technology of fault tolerant computation is set up, we can effectively pretend it doesn't exist, and draw quantum circuits that act on logical bits directly. These logical circuits can then be automatically fleshed out into a physical circuit acting on perhaps seven times as many physical objects, with error correction occupying much of what may have once been a simple algorithm. Typically algorithms are first described in a theoretical, ideal context, with no errors, possible only in classical simulations of these algorithms, and are later run as a naive physical implementation, as in \cite{algos}, and then finally as a scalable, fault tolerant algorithm, supposing that there is a quantum computer with enough qubits to run it. In this way all of the algorithms and techniques described in this thesis are intuitively understood to run at the physical level, in terms of elementary evolutions, but paying deliberate mind to the eventual need for fault tolerant implementation, using a much more restricted set of elementary gates acting on logical quantum objects.

\subsection{Phase, Bloch Sphere}
Since we assume a very specific space $\mathbb{C}^N$ with a canonical/computational basis available, we can and often do talk directly about the coordinates of vectors in this space. In a composite space we can use the coordinates in the Kronecker basis, so for example the Kronecker product $\ket{+}\otimes \ket{+}$ and $\ket{+} \otimes \ket{-}$ are vectors with the following coordinates:
\begin{align*}
\ket{+} \otimes \ket{+} = \left[\begin{matrix}
\frac{1}{2} \\
\frac{1}{2} \\
\frac{1}{2} \\
\frac{1}{2} \\
\end{matrix}\right]
&&&
\ket{+} \otimes \ket{-} = \left[\begin{matrix}
	\frac{1}{2} \\
	-\frac{1}{2} \\
	\frac{1}{2} \\
	-\frac{1}{2} \\
\end{matrix}\right]\end{align*}

If we have two states $\ket{\psi}$ and $\ket{\phi}$ such as the above two states, whose $p$th coordinate have coordinates that have the same complex modulus, and hence are unit complex multiples of each other, then we say they differ by a relative phase factor $e^{i\theta}$ in this coordinate, such as in the above example of $\ket{+}\otimes \ket{+}$ and $\ket{+}\otimes \ket{-}$, where their $\ket{0}\ket{1}$ coordinates differ by a local phase factor of $-1$. A general example in a single qubit is be $\ket{\psi} = \ket{0}/\sqrt{2} + e^{i\theta}\ket{1}/\sqrt{2}$ vs. $\ket{\phi} = \ket{0}/\sqrt{2} + \ket{1}/\sqrt{2}$, whose $\ket{1}$ amplitude differs by a complex phase factor $e^{i\theta}$.

On the other hand, if two states are unit complex multiples of each other overall, i.e.\ $\ket{\psi} = e^{i\theta}\ket{\phi}$, then we say that they differ by a global phase factor $e^{i\theta}$. Interestingly, global phase differences are a peculiarity of the model we are using, and are not understood to have any effect on measurement, since global phase differences cannot affect the size of any coordinate in  a given quantum state, nor can it affect any resultant states after performing computation. For example if we have two states $\ket{+}$ and $-\ket{+}$ and apply the Hadamard matrix $H$ to each we would get $\ket{0}$ and $-\ket{0}$ respectively, both before and after they have identical behaviour when measured.

For comparison, relative phase differences do not affect the probability of getting a certain measurement when measured in the computational basis, but can affect the course of calculation, producing states that are very different when measured. For example $\ket{+}$ and $\ket{-}$ differ only by relative phase $-1$ in the $\ket{1}$ amplitude, but after applying $H$ we get $\ket{0}$ and $\ket{1}$ respectively, different computational basis vectors, which are distinct when measured. Additionally, while global phase differences are a basis independent property of two quantum states, different bases may have different relative phase factors, or simple different coordinates altogether, so in this $\ket{+}$ and $\ket{-}$ example if we are able to directly measure in the $\ket{+}/\ket{-}$ basis then they will give distinct measurements from each other every time.

By ignoring global phase differences, we can remove a degree of freedom from the state space being discussed. In a single qubit we have the state space
\[\left\{\begin{bmatrix}a+ib \\ c+id\end{bmatrix}\ \middle|\ a, b, c, d \in \mathbb{C}, a^2+b^2+c^2+d^2 = 1\right\}.\]
This is essentially the unit sphere in $\mathbb{R}^4$, but by presenting these coordinates in polar form, and making global phase explicit we can present the same set as
\[\left\{e^{i\gamma}\begin{bmatrix}\cos(\theta) \\ e^{i\phi}\sin(\theta)\end{bmatrix}\ \middle|\ \gamma, \theta, \phi \in \mathbb{R}\right\}.\]
This presentation motivates what is called the Bloch sphere, which ignores global phase factors in order to represent a single qubit as a sphere in $\mathbb{R}^3$, a space much more amenable to human intuition. The coordinates are chosen by the map
\[e^{i\gamma}\begin{bmatrix}\cos(\theta) \\ e^{i\phi}\sin(\theta)\end{bmatrix} \mapsto
\begin{bmatrix}\cos(\phi)\sin(2\theta) \\ \sin(\phi)\sin(2\theta) \\ \cos(2\theta)\end{bmatrix}.\]
We directly remove the global phase factor $e^{i\gamma}$, and we also double $\theta$ which removes a global phase factor of $-1$ associated with $\pi \leq \theta < 2\pi$. This means that while in the basic $\mathbb{C}^2$ picture of a qubit, $\ket{0}$ and $\ket{1}$ are orthogonal, in the Bloch sphere they are in fact collinear:
\begin{align*}
\begin{bmatrix}1 \\ 0\end{bmatrix} \mapsto \begin{bmatrix}0\\0\\1\end{bmatrix}
&&&
\begin{bmatrix}0 \\ 1\end{bmatrix} \mapsto \begin{bmatrix}0\\0\\-1\end{bmatrix}.
\end{align*}
The same applies to the orthogonal states $\ket{+}$ and $\ket{-}$:
\begin{align*}
	\frac{1}{\sqrt{2}}\begin{bmatrix}1 \\ 1\end{bmatrix} \mapsto \begin{bmatrix}1\\0\\0\end{bmatrix}
	&&&
	\frac{1}{\sqrt{2}}\begin{bmatrix}1 \\ -1\end{bmatrix} \mapsto \begin{bmatrix}-1\\0\\0\end{bmatrix}.
\end{align*}

We can also remove a degree of freedom from any sets of unitary matrices being explored, since any scalar multiple of the identity will have no meaningful effect on computation, so for example $X$ and $-X$ can be considered equivalent:
\[-X = -\left[\begin{matrix}0 & 1 \\ 1 & 0\end{matrix}\right]= \left[\begin{matrix}0 & -1 \\ -1 & 0\end{matrix}\right]\]
Given a subgroup of the unitary matrices $G \leq U(n)$ we can make this equivalence explicit; the set of matrices $\{e^{i\gamma}I\}$ form a subgroup of $U(n)$ isomorphic to the group of unit complex numbers, $\{e^{i\gamma}\}$. By taking the group quotient $G / (G \cap \{e^{i\gamma}I\})$ we get a group in which for example $X$ and $-X$ would be recognized as equivalent, in the formal sense that they belong to the same coset of $(G \cap \{e^{i\gamma}I\})$. In order to make notation more succinct, we interpret $1 \times 1$ matrices as simply being scalars, giving $\{e^{i\gamma}\} = U(1)$, and based on this write $G / U(1)$ as a shorthand for $G / (G \cap \{e^{i\gamma}I\})$.

In the case of the qubit we can go further and interpret cosets in $U(2)/U(1)$ as rotations of the three dimensional Bloch sphere, i.e.\ rotations in $SO(3)$. For example $X$ corresponds to the half-turn in the $y$-$z$ plane
\[\begin{bmatrix}
1 & 0 & 0 \\
0 & -1 & 0 \\
0 & 0 & -1
\end{bmatrix}.\]
This interpretation is very powerful, and makes quantum computation on single qubits much more intuitive, however, as soon as one moves to having multiple qubits, or even a single qutrit, we lose the capacity to meaningfully visualise the full state space of the system in three dimensional space.

\section{Non-Binary Logic}
In classical computation data is represented in the voltage levels of conductive materials, and logic is implemented using transistors as digital switches, whose behaviour is more responsive and more consistent to the difference between 0 voltage and high voltage, than to the difference between different levels of high voltage. This means that hardware for classical computation deals almost exclusively with Boolean algebra and binary arithmetic.

In contrast to this quantum computation inherits its number system directly from the dimension of the Hilbert space of the particles being used for computation. For example the spin of an electron forms a 2-dimensional space, the net spin of a Nitrogen-14 atom forms a 3-dimensional space, and the energy level of a trapped ion can form a multi-dimensional space depending on the number of energy levels allowed. This means that quantum computation has a much easier time implementing logical and numerical systems other than the binary system used in classical contexts, but despite this most of the quantum computation in theoretical and practical contexts is based on binary logic and arithmetic, since this is the simplest, and inherits a lot of theoretical results directly from classical computation.

In practice the problem of implementing higher logic systems than binary is much more tractable in the quantum case, whereas the problem of creating and maintaining a large number of entangled quantum objects over time, a problem that doesn't exist in the classical case, becomes the primary constraint limiting practical quantum computation. By using logic systems that store more information in a smaller number of particles, we might actually have an easier time developing a quantum computer capable of performing any given scale of computation. This means that the step from binary to higher logic systems has a lot of motivation in the quantum case.

While there is a lot of potential in the topic of non-binary quantum computation, and a lot of novelty already described, even this has the implicit assumption that all quantum objects need to have the same dimension. In theory this isn't necessary either, and it might be possible to have a quantum computer with a mixture of objects of different dimension. Writing algorithms for such a device could allow one to economise on the strengths and weaknesses of each individual number system, potentially requiring even less physical complexity for the same level of computational power. Further, although we won't discuss fault-tolerant computation anywhere else in this thesis, it is worth noting that in order to mitigate noise in real world quantum computers, information-theoretic encoding schemes are used that embed a certain number of `logical' qubits in a larger number of `physical' qubits, so we may one day find quantum computation acting on logically mixed systems, regardless of the dimensions of the physical objects involved.

\subsection{Pauli and Clifford Matrices}\label{pauli}
A useful family of matrices are the Pauli matrices, which in the $2\times2$ qubit case are the $X$ and $Z$ matrices discussed in previous examples:
\begin{align*}
X = \left[\begin{matrix}
0&1\\
1&0
\end{matrix}\right]
&&&
Z = \left[\begin{matrix}
1&0\\
0&-1
\end{matrix}\right]
\end{align*}

These can be generalized to $n\times n$ matrices with the following effect on the computational basis:
\begin{align*}
X_n\ket{i} = \ket{i+1\mod n}
&&&
Z_n\ket{i} = \omega_n^i\ket{i}
\end{align*}

Powers of $X_n$ form a cyclic group of permutation matrices of order $n$. Visualizing the computational basis vectors in a circle, powers of $X_n$ rotate the circle, whereas powers of $Z_n$ resemble harmonics on that circle.

Observe that these matrices do not commute, but that $Z_nX_n = \omega_nX_nZ_n$:
\begin{align*}
	Z_nX_n\ket{i}
	=& Z_n\ket{i+1\mod n}
	\\=& \omega_n^{i+1}\ket{i+1\mod n}
	\\=& \omega_n^{i+1}X_n\ket{i}
	\\=& \omega_nX_nZ_n\ket{i}
\end{align*}

From this it can be seen that all elements of the Pauli group will be of the form $\omega_n^iX_n^jZ_n^k$, giving the Pauli group an order of $n^3$.

This group is an interesting finite group in quantum computation, first brought to attention since $X_2$ and $Z_2$ along with $Y  = iX_2Z_2$ have physical meaning in quantum mechanical description of electron spin. Further, these 3 matrices each transform the state space of a qubit in a very geometrically convenient way, each along a different pair of axes in the Bloch sphere. The Pauli group of generalized Pauli matrices provide similar algebraic power, being generated by two gates that represent very simple geometric angles in their corresponding state spaces, simple enough to be the first two gates one implements fault tolerantly, but with different combinations of $X_N$ and $Z_N$ providing complete generality in the combinations of axes one rotates about.

In order to capture the $Y_2$ matrix as well we need to introduce additional scale factors $\omega_{2N}$, which in the $N=2$ case gives $\omega_4 = i$, and $Y_2 = \omega_{4}X_2Z_2$. This gives a group of twice the size called the Weyl-Heisenberg group, denoted
\[H(n) = \{\omega_{2n}^iX^jZ^k\}.\]
In a quantum system with one object of dimension $N$, we call the normaliser of the Weyl-Heisenberg group the Clifford group, written $\mathcal{C}$. Since $H(n)$ is generated by $X$, $Z$, and $\omega_{2n}$, we can write
\[\mathcal{C} = N(H(n)) = \{A\ |\ A \in U(N), AXA^{-1}, AZA^{-1} \in H(n)\}\]

The Clifford group contains arbitrary scale factors, making it uncountably infinite, but when removing these scale factors it turns out to be another powerful finite group, making its generator a very useful place to start when considering possible elementary gate sets for a quantum computer. In a single-qubit system the Clifford group (with scale factors removed) is commonly known to be generated by
\begin{align*}
D_2 = \sqrt{Z_2} = \begin{bmatrix}
1 & 0 \\
0 & i
\end{bmatrix}
&&&
H_2 = \frac{1}{\sqrt{2}}\begin{bmatrix}
1 & 1 \\
1 & -1
\end{bmatrix},
\end{align*}
and in a single-qutrit system it is generated by
\begin{align*}
S_3 = \begin{bmatrix}
1 & 0 & 0 \\
0 & 1 & 0 \\
0 & 0 & \omega_3
\end{bmatrix}
&&&
H_3 = \frac{1}{\sqrt{3}}\begin{bmatrix}
1 & 1 & 1 \\
1 & \omega_3 & \omega_3^2 \\
1 & \omega_3^2 & \omega_3
\end{bmatrix}
\end{align*}

The matrices $H_2$ and $H_3$ are also called the discrete Fourier transform in two and three dimensions respectively. They can be generalized to arbitrary dimension:
\[H_N = \frac{1}{\sqrt{N}}\sum_{i,j} \omega_N^{ij}\ket{i}\bra{j}\]
When we visualise the Pauli matrices as rotations and harmonics of the circle, the discrete Fourier transform actually maps into a basis where $Z_N$ is a rotation and $X_N^{-1}$ is a harmonic. Algebraically this can be written
\begin{align*}
H_N^{-1}Z_NH_N = X_N, &&& H_N^{-1}X_NH_N = Z_N^{-1},
\end{align*}
which tells us that in general $H_N \in N(H(N))$.

The Clifford group is often easy to implement fault tolerantly, and for example in the Steane code \cite{steane-code} both $H_2$ and $D_2$ can be implemented on a logical qubit by simply applying it to each of the seven physical qubits in the code. Being a group that is easy to implement, easy to algebraically manipulate, and quite wide reaching in its coverage of $U(N)$, the Clifford group comes up very frequently in discussions of quantum computation.

Pauli, Weyl-Heisenberg, and Clifford groups also exist in composite systems. In the Pauli case this group contains arbitrary combinations of $X_{d_i}$ and $Z_{d_i}$ on each object $i$ of the system, so in a system with two objects of dimension $d_1$ and $d_2$ the Pauli group would be
\[\{\omega_{d_1}^a\omega_{d_2}^b(X_{d_1}^cZ_{d_1}^d\otimes X_{d_2}^eZ_{d_2}^f)\ |\ a, b, c, d, e, f \in \mathbb{Z}\}.\]
The scale factors $\omega_{d_i}$ are shared, and can be combined into a single term $\omega_{d'}$ where $d'$ is the greatest common divisor of $d_1$ and $d_2$, but the operations $X_{d_1}Z_{d_1}$ and $X_{d_2}Z_{d_2}$ will not combine. The Clifford group of a composite system, again written $\mathcal{C}$, is simply the normaliser of the corresponding Weyl-Heisenberg group. This means that while the Weyl-Heisenberg group only contains Kronecker products, the Clifford group can contain operations acting on dependently multiple objects. In a composite system with two $d$ level objects the operations SUM and SWAP will always be Clifford:
\begin{align*}
	SUM(\ket{i}\ket{j}) = \ket{i}\ket{i+j \mod d} &&& SWAP(\ket{i}\ket{j}) = \ket{j}\ket{i}
\end{align*}
(extended linearly)

To prove that SWAP is Clifford, observe $SWAP^{-1} = SWAP$, and let $D_1, D_2 \in H_d$, then
\begin{align*}
SWAP (D_1 \otimes D_2) SWAP (\ket{i}\ket{j})
&= SWAP (D_1 \otimes D_2)(\ket{j}\ket{i})
\\&=SWAP (D_1\ket{j} \otimes D_2\ket{i})
\\&=D_2\ket{i} \otimes D_1\ket{j}
\end{align*}
Extended linearly this means $SWAP (D_1 \otimes D_2) SWAP$ is the same linear operation as $D_2 \otimes D_1$, which is in the Weyl-Heisenberg group. For SUM one must manipulate $X$ and $Z$ separately. In the following take all sums to be mod $d$:
\begin{align*}
&SUM^{-1} (X^aZ^b \otimes X^eZ^f) SUM (\ket{i}\ket{j})
\\&=SUM^{-1} (X^aZ^b \otimes X^eZ^f) (\ket{i}\ket{i+j})
\\&=\omega_d^{bi+fi+fj}SUM^{-1}(X^a \otimes X^e) (\ket{i}\ket{i+j})
\\&=\omega_d^{bi+fi+fj}SUM^{-1} (\ket{i+a}\ket{i+j+e})
\\&=\omega_d^{bi+fi+fj}\ket{i+a}\ket{j+e-a}
\\&=\omega_d^{bi+fi+fj}(X^a \otimes X^{e-a})\ket{i}\ket{j}
\\&=(X^aZ^{b+f} \otimes X^{e-a}Z^f)\ket{i}\ket{j}
\end{align*}
Extended linearly this is once again a Weyl-Heisenberg matrix, $X^aZ^{b+f} \otimes X^{e-a}Z^f$.

\subsection{Notation for Mixed Systems}\label{notation}
We have been notating $N \times N$ objects with the $N$ subscript, which distinguishes between objects that perform similar roles in different Hilbert spaces, but in composite systems it is more important to distinguish between operations acting on distinct objects within that system, i.e. to distinguish between $X \otimes I$ and $I \otimes X$. Further, since we will be describing algorithms acting in mixed systems we would like to distinguish between both of these at the same time! In order to prevent ambiguity we adopt strict conventions on which letter is used for different pronumerals, so that for example $X_{d_i}$ is always an operation in the non-composite system, i.e. in $U(d_i)$, whereas $X_i$ is an operation in the composite system, i.e. in $U(d_1d_2\dots d_n)$.

In any quantum system we define $N$ to be the number of computational basis states, and either $n$ or $m+n$ to be the number of individual objects in the composite system, where the first $m$ objects will be somehow distinct from the last $n$. This means the most general operation we could describe in such a system would be an operation in $U(N)$, and such an operation would be denoted $U_N$ to indicate its dimension.

When we want to talk about an individual object within a composite system, we index each object $1$ through to $n$, (or $1$ through to $n+m$) and use variables such as $i$, $j$, and $k$ to represent such an index. We then define $d_i$ to be the dimension of the object indexed by $i$. When the system is a single object we will have $n = 1$, and $N = d_1$, combining the $N$ notation of \cite{tolar-clifford} with the $d$ notation of \cite{multi-valued-logic}.

Now, when indexing computational basis states in a quantum system, we use the integers $0$ through to $N-1$, and variables such as $p$, $q$, and $r$ to represent such indeces, except for \autoref{ternary} where we adopt $i$, $j$, $k$, for this purpose, to make it easier to follow the paper \cite{arithmetics}. This is particularly useful for the special class of permutation matrices that represent a transposition, where we can write $S_{p,q}$ to represent the matrix that exchanges the computational basis states $\ket{p}$ and $\ket{q}$, while leaving all others unchanged. Usually we will introduce $p$ indirectly, by naming a computational basis state $\ket{p_n}\dots\ket{p_1}$, implicitly defining $p$ to be the integer corresponding to this computational basis state, where of course $\ket{p_i}$ will be a computational basis state in the individual system $\mathbb{C}^{d_i}$. Note that we have reversed the order of $p_1 \dots p_n$, so that $p_n$ is the most significant digit, and $p_1$ is the least significant digit.

One thing that these $p$ and $p_i$ indeces allow us to do is to notate transpositions acting on a composite system by writing $S_{p,q} \in U(N)$ and transpositions acting on a single object by writing $S_{p_i,q_i} \in U(d_i)$. Transpositions are permutation matrices that swap two computational basis vectors and leave the rest unchanged, so they will be the linear extension of the following actions:
\[S_{p,q}\ket{r} = \begin{cases}
\ket{q}, & \text{if\ } r = p, \\
\ket{p}, & \text{if\ } r = q, \\
\ket{r}, & \text{otherwise.}
\end{cases}\]

Next, if we have a unitary matrix $G_{d} \in U(d)$, and some object $i$ with dimension $d_i = d$, then we can define $G_i$ to be this operation $G_d$ applied to the $i$th object, i.e. 
\[G_i = I_{d_n}\otimes \dots \otimes I_{d_{i-1}} \otimes G_d \otimes I_{d_{i+1}} \otimes \dots \otimes I_{d_n}\]
When indexing an operation other than $S_{p,q}$ with a bare numeral, such as $X_2$ or $Z_3$, we exclusively interpret this as $X_d$ with $d = 2$, or $Z_d$ with $d = 3$ respectively. We will not use $X_2$ to mean $X_{d_2} \otimes I$, for example. If we talk about an operation such as  $X_2 \in U(N)$ or $X_d \in U(N)$, in a composite system, we mean $X_i$, where $i$ is an \emph{arbitrary} index satisfying $d_i = d$. For example we could suppose that a quantum computer has $X_2$ available as an elementary gate, and what we would mean is that the system has $X_i$ available as an elementary gate for every $i$ satisfying $d_i = 2$.

Take as an example the smallest possible mixed system, with $n = 2$, $d_1 = 3$, $d_2 = 2$, and hence $N = d_1d_2 = 6$. Set $i = 2$, then
\[X_i = X_2 \otimes I_3 = \begin{bmatrix}
0 & 0 & 0 & 1 & 0 & 0 \\
0 & 0 & 0 & 0 & 1 & 0 \\
0 & 0 & 0 & 0 & 0 & 1 \\
1 & 0 & 0 & 0 & 0 & 0 \\
0 & 1 & 0 & 0 & 0 & 0 \\
0 & 0 & 1 & 0 & 0 & 0
\end{bmatrix}\]
On the other hand if $i = 3$ then
\[X_i = I_2 \otimes X_3 = \begin{bmatrix}
	0 & 0 & 1 & 0 & 0 & 0 \\
	1 & 0 & 0 & 0 & 0 & 0 \\
	0 & 1 & 0 & 0 & 0 & 0 \\
	0 & 0 & 0 & 0 & 0 & 1 \\
	0 & 0 & 0 & 1 & 0 & 0 \\
	0 & 0 & 0 & 0 & 1 & 0
\end{bmatrix}\]
Here $d_i = 2$ and $d_i = 3$ are unique, so in fact the notation we have described already identifies $X_2$ with $X_2 \otimes I_3$, and $X_3$ with $I_2 \otimes X_3$. We have 6 computational basis states:
\begin{align*}
	p = 0 &&& \implies \ket{p_2}\ket{p_1} = \ket{0}\ket{0} \\
	p = 1 &&& \implies \ket{p_2}\ket{p_1} = \ket{0}\ket{1} \\
	p = 2 &&& \implies \ket{p_2}\ket{p_1} = \ket{0}\ket{2} \\
	p = 3 &&& \implies \ket{p_2}\ket{p_1} = \ket{1}\ket{0} \\
	p = 4 &&& \implies \ket{p_2}\ket{p_1} = \ket{1}\ket{1} \\
	p = 5 &&& \implies \ket{p_2}\ket{p_1} = \ket{1}\ket{2} \\
\end{align*}

\subsection{Conditional Operations}
A very expressive family of operations in composite systems are the controlled operations. If $U_j \in U(d_j)$ is an operation acting on object $j$ of a composite system, then for each $i \neq j$ we would like to define $C_{r_i=q_i}(U_j)$ to be the linear extension of
\[\ket{r_n}\dots\ket{r_1} \mapsto \begin{cases}
	\ket{r_n}\dots \ket{r_{j+1}} (U_j\ket{r_j})\ket{r_{j-1}}\dots\ket{r_1}, & \text{if\ } r_i = q_i, \\
	\ket{r_n}\dots\ket{r_1}, & \text{otherwise.}
\end{cases}\]
For example a central gate in qubit computation is
\[C_{r_2=1}(X_2) = \begin{bmatrix}
1 & 0 & 0 & 0 \\
0 & 1 & 0 & 0 \\
0 & 0 & 0 & 1 \\
0 & 0 & 1 & 0
\end{bmatrix},\]
often called CNOT or XOR due to its interpretation as conditionally negating, or calculating the exclusive OR of two Boolean values. In an operation $C_{r_i = q_i}(U_j)$ we call object $i$ the control object, or the control digit, and object $j$ the target object, or the target digit. In systems where the base is known these can also be called the control qubit and target qubit respectively, or control qutrit and target qutrit. We also call the value $q_i$ the control value. Just as we can say $X_2$ to mean $X_i$ for an arbitrary qubit $i$, we can say $C_d(U_j)$ to represent $C_{r_i = d_i-1}(X_2)$ an arbitrary control object $i$ satisfying $d = d_i$, with $q_i = d_i - 1$ chosen as a conventional default control value.

Control operations can be quite complicated to write in this form, but become quite simple in circuit form. We write the qubit CNOT operation as:

\begin{quantikz}
	\lstick{$\ket{1}$} & \phase{1} \vqw{1} & \qw \rstick{$\ket{1}$} \\
	\lstick{$\ket{r_1}$} & \gate{X_2} & \qw \rstick{$\ket{r_1+1\mod 2}$}
\end{quantikz}

Generalized to an arbitrary $C_{r_2 = q_2}(U_{d_1})$ operation, i.e. an arbitrary controlled $U_{d_1}$ operation with control object $i=2$ and target object $j=1$ is notated as

\begin{quantikz}
\lstick{$\ket{q_2}$} & \phase{q_2} \vqw{1} & \qw \rstick{$\ket{q_2}$} \\
\lstick{$\ket{r_1}$} & \gate{U_{d_1}} & \qw \rstick{$U_{d_1}\ket{r_1}$}
\end{quantikz}

The controlled $X$ operation is so common that we call it the controlled increment, interpreting $X$ as incrementing the index of the computational basis states by 1. Further rather than write the gate $X_{d_1}$ in a box we can simply write

\begin{quantikz}
	\lstick{$\ket{q_2}$} & \phase{q_2} \vqw{1} & \qw \rstick{$\ket{q_2}$} \\
	\lstick{$\ket{r_1}$} & \targ{} & \qw \rstick{$\ket{r_1+1\mod d_1}$}
\end{quantikz}

In all of the above examples if $\ket{r_2}$ were not $\ket{q_2}$ then we would have simply written $\ket{r_1}$ and $\ket{r_2}$ unchanged at the end of the circuit. One should remember that these operations distribute linearly on computational basis states however. As an example consider the controlled NOT gate acting on $\ket{+}\ket{0}$ to create an entangled state:

\begin{quantikz}
	\lstick{$\ket{+}$} & \phase{1} \vqw{1} & \qw \rstick[wires=2]{$\frac{1}{\sqrt{2}}\ket{0}\ket{0} + \frac{1}{\sqrt{2}}\ket{1}\ket{1}$} \\
	\lstick{$\ket{0}$} & \targ{} & \qw
\end{quantikz}

(This works by expanding the Kronecker $\ket{+}\ket{0}$ to $\frac{1}{\sqrt{2}}\ket{0}\ket{0} + \frac{1}{\sqrt{2}}\ket{1}\ket{0}$, and then applying $C(X)$ to each term.)

When dealing with qubit computers it has been conventional to omit the control value from circuit diagrams, and indicate the control value with either a filled in or hollow dot, for control values 1 and 0 respectively:

\begin{quantikz}
	\lstick{$\ket{r_2}$} & \ctrl{1} & \octrl{1} & \qw \rstick{$\ket{r_2}$} \\
	\lstick{$\ket{r_1}$} & \targ{} & \targ{} & \qw \rstick{$\ket{r_1+1\mod d_1}$}
\end{quantikz}

We cannot use this convention since we have more than two computational basis states to represent, so we represent this more explicitly as

\begin{quantikz}
	\lstick{$\ket{r_2}$} & \phase{1} \vqw{1} & \phase{0} \vqw{1} & \qw \rstick{$\ket{r_2}$} \\
	\lstick{$\ket{r_1}$} & \targ{} & \targ{} & \qw \rstick{$\ket{r_1+1\mod d_1}$}
\end{quantikz}

This makes the open dot notation available for re-appropriation, so we follow \cite{arithmetics} in using this to represent the Clifford SUM operation:

\begin{quantikz}
	\lstick{$\ket{r_2}$} & \octrl{1} & \qw \rstick{$\ket{r_2}$} \\
	\lstick{$\ket{r_1}$} & \targ{} & \qw \rstick{$\ket{r_1+r_2}$}
\end{quantikz}

A qubit convention that we can generalize is to represent the SWAP operation as:

\begin{quantikz}
	\lstick{$\ket{r_2}$} & \swap{1} & \qw \rstick{$\ket{r_1}$} \\
	\lstick{$\ket{r_1}$} & \targX{} & \qw \rstick{$\ket{r_2}$}
\end{quantikz}

\subsection{Multi-control Operations}

One can generalize control operations to control an operation that acts on multiple target bits, and hence define operations of the form $C_{r_3=q_3}(C_{r_2=q_2}(U_{d_1}))$, which has uses, especially when generalizing SUM to `soft-control' operations the way that \cite{arithmetics} does, but we do not use this formulation, and instead directly define $C_{r_3=q_3,r_2=q_2}(U_{d_1})$ to have the same effect, applying $U_{d_1}$ when \emph{all} conditions are satisfied, and doing nothing otherwise. The circuit representation of such a gate is, unsurprisingly:

\begin{quantikz}
	\lstick{$\ket{r_3}$} & \phase{q_3} \vqw{1} & \qw \rstick{$\ket{r_3}$} \\
	\lstick{$\ket{r_2}$} & \phase{q_2} \vqw{1} & \qw \rstick{$\ket{r_2}$} \\
	\lstick{$\ket{r_1}$} & \gate{U_{d_1}} & \qw \rstick{$U_{d_1}\ket{r_1}$}
\end{quantikz}

For the sake of formality we will often work with an index set $I \subset \{1 \dots n\}$ containing the indices of all of the control objects, and write the control operation more succinctly as $C_c(U_j)$, where $c$ is the set of computational basis vectors that $C_c(U_j)$ will apply $U_j$ to, i.e.\ the set \[\{\ket{r_n}\dots\ket{r_1}\ |\ r_i = q_i \forall i \in I\}\]

An operation that is commonly used in qubit contexts is the Toffoli gate, which is simply a controlled $X_2$ operation with two control qubits, $C_{r_{i_1} = 1, r_{i_2} = 1}(X_2)$.

\begin{quantikz}
	\lstick{$\ket{1}$} & \phase{1} \vqw{1} & \qw \rstick{$\ket{1}$} \\
	\lstick{$\ket{1}$} & \phase{1} \vqw{1} & \qw \rstick{$\ket{1}$} \\
	\lstick{$\ket{r_1}$} & \targ{} & \qw \rstick{$\ket{r_1+1\mod 2}$}
\end{quantikz}

In the above examples the target object has always been $i = 1$. This is not necessary, but when it is possible it makes for a much simpler block representation of the resulting $C_c(U_{d_1}) \in U(N)$. If \emph{every} object except for $i = 1$ is a control object, then this will be the following block matrix:
\[\begin{bmatrix}
I & 0 & \cdots & 0 & 0 & 0 & \cdots & 0\\
0 & I & \cdots & 0 & 0 & 0 & \cdots & 0 \\
\vdots & \vdots & \ddots & \vdots & \vdots & \vdots & \cdots & \vdots \\
0 & 0 & \cdots & I & 0 & 0 & \cdots & 0 \\
0 & 0 & \cdots & 0 & U_j & 0 & \cdots & 0\\
0 & 0 & \cdots & 0  & 0 & I & \cdots & 0\\
\vdots & \vdots & \cdots & \vdots & \vdots & \vdots & \ddots & \vdots \\
0 & 0 & \cdots & 0 & 0 & 0 & \cdots & I
\end{bmatrix}\]
where the point along the diagonal at which $U_j$ appears is determined by interpreting the control values as an integer $0 \leq q_n\dots q_2 \leq N/{d_1}$.
% chap2.tex (Chapter 2 of the thesis)

\chapter[MULTI VALUED LOGIC LITERATURE REVIEW]{Multi-Valued Logic Literature Review}
While quantum computation has sparked a lot of theoretical and practical interest in recent decades, and quite a strong understanding is being developed, much of the focus is on binary quantum mechanical systems, that is, systems that are a composite of some number of qubits. These objects are the simplest, and also the most familiar object of computation, with certain numerical algorithms coming directly from classical computer science without much modification, however this means we inherit the lack of flexibility of binary computation, without any of the advantages or needs that motivated it in the classical case.

In classical computation data is represented as the voltage of a circuit, and the fundamental operations of computation come from transistors which essentially multiply voltages. If we wish to represent two states then we can choose voltages that will be closed under this multiplication, essentially the order 2 group $\{0, 1\}$. As soon as we add any positive number to this set it will stop being closed under multiplication, which makes physical implementation of a ternary or greater logic system drastically more complicated than binary, while at the same time being much more sensitive to error. Quantum objects on the other hand are fundamentally poly-dimensional, and a very deliberate effort is made to choose objects that are merely binary and not ternary or greater. This suggests that alternate models of computation, while still being more complicated to implement and use than binary models, aren't more complicated to the same degree as in classical computation.

Another distinction between the trade-offs of classical and quantum computation is that storage of bits is not an inherent bottleneck of classical computation, so long as one has the materials and power supply, one can create a second body of latches for storing additional data, whereas in quantum computation we must always retain access to quantum entanglement in order to effectively use the algorithms that make quantum computation powerful. This means that not only are qubits not essential, but they are the \textit{least} efficient at achieving what is in fact the largest bottleneck of quantum computation.

These tradeoffs suggest a lot of promise in the topic of non-binary quantum computation, but the decision isn't a matter of picking a base for doing all computation. In theory each object in the system could have a different base, allowing costs and benefits to be chosen based on the role that each object has in an algorithm. Computers that have a variety of objects in a variety of bases could turn out to be the most cost effective way of providing high amounts of quantum information capacity, without requiring the redundancy of purely ternary or higher valued systems in the situations where an algorithm demands binary calculation.

To that end the latter half of this document will give an overview of some of the literature that makes up the current understanding of quantum computation, with a specific focus on ternary or higher forms of logic, as well as what little is currently known about mixed systems with a combination of qubits, qutrits, and higher objects.

\subsection{Textbook}
The textbook ``Quantum Computation and Quantum Information''\cite{textbook} is ubiquitous and foundational in its fields. It demands little prerequisite knowledge except for familiarity with the practice of algebra itself, and builds up through linear algebra and group theory, through Quantum/Schrödinger mechanics, up to the fundamental techniques of quantum computation and quantum information theory. A good deal of attention is also paid to the historical and conceptual context of the field, making for a read that is as interesting and intuitive as it is informative.

This textbook does not address ternary, n-ary, or mixed logic systems beyond the general treatment of quantum mechanics and linear algebra that was echoed in the previous chapter of this document, [at least I think it doesn't!] however one cannot get far without these things, and without an understanding of what the techniques of quantum computation look like in its simplest case of binary digits.

\subsection{Quantum Algorithm Implementations for Beginners}
``Quantum Algorithm Implementations for Beginners''\cite{algos} is a very recent paper coming from a large number of authors that summarises a large number of quantum algorithms to be understood by a general computer science audience. This is motivated by the trend that quantum computation is making, with more powerful quantum computers becoming accessible to larger cohorts of people, it is becoming needful to increase the number of people capable of using these devices. The paper starts with a more brief overview of the required quantum mechanics and linear algebra, before diving into a conceptual and technical overview of 20 distinct quantum algorithms.

Most of these algorithms are implemented on various 5-qubit computers available via IBM, (QX4, QX5, ESSEX, VIGO) making for a powerful practical demonstration of a topic that can be quite overwhelmingly theoretical. The level of error that comes from short algorithms to solve problems at the scale of 2 bits is significant, and so this paper demonstrates practical motivation for the techniques for detecting and correcting errors that are available in quantum computation, although this paper does not directly talk about those techniques.

This paper focuses entirely on qubit computation, so while it is a very useful and contemporary paper for understanding the algorithms that exist in quantum computation, it does not make any of the further steps into ternary or higher forms of computation.
\subsection{On Clifford Groups}
The first paper studying higher forms of computation that we shall look at is the paper ``On Clifford groups in Quantum Computing''\cite{tolar-clifford}, which aims to understand an important group of unitary matrices -- the Clifford group. This group is generally defined as being the normalizer of the Weyl-Heisenberg group, (or sometimes the Pauli group, though this excludes some useful operations) that is the set of all unitary matrices $A$ with the property that the two conjugations $AX_nA^{-1}$ and $AZ_nA^{-1}$ are both in the Weyl-Heisenberg group.

The first thing that Tolar does is develop a way of proving a known isomorphism between Clifford groups on single objects, and finite groups acting on ring modules. Specifically he analyses conjugation of the Pauli group as a group action of the Clifford group. Since this group action is unaffected by scalar factors, it turns out to be a very algebraically powerful, in particular the Pauli group is Abelian up to scale factors, forming a convenient Abelian subgroup of these Ad-actions.

In any case the technique turns out to be amenable to analysing systems of multiple objects as well, allowing a general understanding of Clifford groups in composite quantum systems in terms of the same kinds of groups that single object Clifford groups are isomorphic to.

In particular, for single object Clifford groups the isomorphism was to the group $(\mathbb{Z}_n\times\mathbb{Z}_n)\rtimes\text{SL}(2,\mathbb{Z}_n)$, which is a finite group generated by less than $4$ group elements; for systems with $k$ objects of the same dimension $n$ the Clifford group was further isomorphic to $\mathbb{Z}_n^{2k}\rtimes\text{SL}(2k, \mathbb{Z}_n)$. For pairs of objects with co-prime dimension the Clifford group was essentially the Clifford group of single objects with the same dimension, but for systems with an object of dimension $ap^k$ and an object of dimension $bp^{k+r}$ for some $a, b, p, k, r$ positive and $p$ prime, no decomposition of this form was found, an interesting place left to explore by this analysis.

This paper thus presents a technique for analyzing Clifford groups of arbitrary and mixed dimension, a presentation of how many Clifford groups are isomorphic to subsets of different ring modules, and finally a novel unexplored possibility within mixed-level objects of particular dimensions.

This is the only paper we found that looks at quantum systems with objects of differing dimension.

\subsection{Multi-valued Logic Gates}
Up until this point our discussion has been largely abstract, but since our stated motivations are practical, it is important to pay attention to practical concerns, such as the kinds of hardware that might implement ternary or higher forms of quantum computation, and how those implementations might in principle perform arbitrary quantum algorithms. A paper that is very directly concerned with these questions is ``Multi-valued logic gates for quantum computation''\cite{multi-valued-logic}, which looks at how quantum algorithms acting on multiple objects that are all of dimension $d$ might be performed using a common technique known as the linear ion trap.

The first half of the paper is focused on an intuitive scheme for decomposing an arbitrary unitary matrix based on its spectral decomposition, where objects analogous to the generalized $X$ and $Z$ Pauli matrices are used to implement gradually larger classes of unitary operator, until eventually all unitary operators have been implemented using only these $X$ and $Z$ like objects acting on one or two digits at a time.

The $X$ and $Z$ like gates are not really individual gates, but families of gates, so this result is not general to any quantum computer; typically a theoretical proof of arbitrary (or universal) quantum computation relies on a very small set of individual gates, such as $\{H, \text{Toffoli}\}$, whereas this paper relies on families of gates with $2d$ real-valued parameters, and a similar number of degrees of freedom in how each of these families might be chosen, making for a computational basis that is easy to work with, but vague to implement.

The paper gets away with this by proposing a physical implementation of this family of gates, through direct control of physical parameters of the device, relating the implementation of these basic gates to known problems of quantum device control in linear trapped ion systems, where the $n$-dimensional computational basis corresponds to the various excitation levels of the ion.

This paper is a very promising theoretical and practical foundation for working with quantum systems with multiple objects all of one arbitrary dimension.

\{Qudit versions of the qubit ``pi-over-eight'' gate\}
\begin{itemize}
	\item A standard basis for asymptotically universal quantum computation in binary contexts is the Hadamard gate, and the ``pi-over-eight'' phase gate, which is a unitary operation that is diagonal in the computational basis
	\item this paper generalizes the latter gate to a family of gates with similar properties, acting on any quantum object with a prime number of basis states.
	\item first the basic properties of the phase gate are described, and solved algebraically, giving a discrete family of diagonal unitary matrices in the same level of the Clifford hierarchy as the phase gate in 2 dimensions.
	\item this family of gates is then understood as a finite Abelian group, and is therefore reduced to the direct product of cyclic groups
	\item the different number of states gave different group structure, depending on whether there were 2, 3, or more states. These 3 cases were each given a small generator of diagonal phase-like gates.
	\item these gates turn out to share important geometric properties with the phase gate, in addition to being diagonal, they are maximally distinct from any Clifford gate, which has implications for how accurate they will be in the presence of noise.
	\item \ [something something magic state distillation]
\end{itemize}

\subsection{Improved Ternary Arithmetics}
\begin{itemize}
	\item Opens with a description of two implementations of ternary addition
	\item First is a "modified ripple-carry adder" which uses only a single ancillary qutrit to compute a sum in-place
	\item Next is a "carry look-ahead adder" which implements a divide-and-conquer algorithm of addition in order to implement addition with many more ancillary qubits, but logarithmic circuit depth through parallelism
	\item both of these algorithms are fairly unsurprising, but their implementation takes advantage of the specifics of qutrits in order to minimize the memory and computational overhead as described
	\item addition is foundational to implementing modular arithmetic, which the paper discusses in the context of Shor's period finding algorithm
	\item modular addition, subtraction, and integer comparison are discussed, including the positive or negative effects they have on the computational cost of the circuit
	\item the carry look-ahead adder also demonstrates a non-numerical variable which takes 3 values, well leveraged in this algorithm, and very interesting in the discussion of different number schemes and their relative trade-offs.
	\item the technology that was used to optimize each algorithm is given, where gates that permute the computational basis are understood as polynomials acting on the digits of this basis, and algebraic manipulation is used to understand which gates are equivalent to which
	\item further this technology was used to construct important gates exactly (as opposed to asymptotically) using a diagonal unitary gate described in ``Qudit versions of the qubit ``pi-over-eight'' gate.''
	\item the technology of this paper is very well attuned to the goal of understanding quantum computation in different number systems, both from algorithmic perspective of efficiently implementing useful computations, and from the more foundational perspective of choosing a basis/generator in which to do these computations.
	\item further the clear communication of which kinds of non-clifford gate are used allows one to apply the algorithms to any other basis, finding a more suitable compilation of these operations. This was done well in that the gates required were all well justified, and their equivalence to each-other was made clear.
\end{itemize}


% chap3.tex (Chapter 3 of the thesis)
\chapter[UNIVERSAL COMPUTATION]{Universal Computation}

A foundational result in quantum computation is that of universal computation, that certain combinations of quantum gate can be used to implement any quantum algorithm to some accuracy, given sufficient circuit depth. The resulting circuits are generally too long to use in practice, compared to compilation techniques that rely on specific properties of the algorithm in question, but the result is still useful since it proves that it's not impossible, i.e.\ its necessary and sufficient conditions provide a starting point for designing and using quantum computers in practice.

\section{Universal Computation in Qubit Contexts}
The two most widely useful results about universality of qubit computers are the result of \cite{cnot-decomposition}, that a quantum computer with arbitrary operations from $U(2)$ on each individual qubit, and the controlled not $C(X)$ operation, one can exactly implement any unitary $U \in U(N)$, and the result of \cite{universal-qubit}, which ports this result to fault tolerant computation by showing that with only two fault tolerant elementary gates one can approximate any single-qubit operation in $U(2)$, and hence with the addition of $C(X)$, which is also fault tolerant, one can fault tolerantly approximate any operation in $U(N)$. We shall describe the former of these results, and in doing so generalize it to the following:

[Theorem] In any mixed quantum computer with at least one qubit, one can achieve universal computation with either:
\begin{itemize}
	\item Arbitrary qubit operations and arbitrary controlled increments $C_{r_i=q_i}(X_{d_j})$
	\item Arbitrary qubit operations and arbitrary controlled transpositions $C_{r_i=q_i}(S_{p_i,p'_i})$
\end{itemize}
[I define $S_{p,q}$ after this point so... should move those definitions up to the notation section]

In the case of a computer with only qubits, this theorem is equivalent to the result of \cite{cnot-decomposition}. We shall now outline the series of techniques presented in the textbook \cite{textbook}, which collect the relevant techniques from \cite{cnot-decomposition} and its predecessors into a continuous sequence of increasingly powerful proofs of universal computation. We will treat this as a single proof, with each step of the proof decomposing an arbitrary unitary $U \in U(N)$ into a smaller set of basic operations. The first decomposition is of $U$ into
\[U = \prod_{p=0}^{N-2}\prod_{q=p+1}^{N-1}U_{p,q},\]
giving $(N-1)(N-2)/2$ unitary matrices $U_{p,q}$, one for each distinct pair $p = p_n\dots p_1$, $q = q_n\dots q_1$, $p < q$. [need to define these digital expansions instead of the $\ket{p_n}\dots\ket{p_1}$ thing I thought I would use] Specifically $U_{p,q}$ will be `two level' unitaries, in that they only act on the two computational basis vectors $\ket{p}$ and $\ket{q}$, meaning there are some complex $a, b, c, d$ so that
\[U_{p,q} = I + (a-1)\ket{p}\bra{p} + b\ket{p}\bra{q}+ c\ket{q}\bra{p}+ (c-1)\ket{q}\bra{q}.\]
For example in a system of two qubits, with $p = 1$ and $q = 2$:
\[U_{1,2} = \begin{bmatrix}
1 & 0 & 0 & 0 \\
0 & a & b & 0 \\
0 & c & d & 0 \\
0 & 0 & 0 & 1
\end{bmatrix}\]
This operation will have no effect on computational basis states $\ket{0}\ket{0}$ or $\ket{1}\ket{1}$, but will act on $\ket{p_2}\ket{p_1}$ and $\ket{q_2}\ket{q_1}$ in a similar manner to
\[U_2 = \begin{bmatrix}
a & b \\
c & d
\end{bmatrix}\]

The proof that any unitary can be represented as a product of such two-level unitaries amounts to a kind of row reduction on the lower left triangle of the unitary, choosing the $c$ value of each $U_{p,q}$ in order to eliminate each element one at a time. We won't present the exact details here, since this part requires no change in the case of a mixed quantum computer. The full procedure can of course be found in \cite{textbook}.

These two-level unitaries can then be implemented as a series of controlled operations $\{C_c(U_2)\}$, which can in turn be decomposed into `basic' operations, $C(X)$ along with arbitrary $U_2 \in U(2)$. This decomposition is done using a variety of techniques presented in \cite{cnot-decomposition}. This set of techniques provide an excellent starting point for reasoning about quantum computation at the level of physical qubits, but in order to work with logical qubits one can go a step further and approximately implement all of $U(2)$ using only two basic gates with known fault tolerant implementations. This result was shown in \cite{universal-qubit}, and shall inform our later discussion in \ref{} of minimal gate sets in quantum computers consisting only of qubits and qutrits.

\section{Representing Two-Level Unitaries With Control Operations}
Once we have decomposed a unitary into two-level unitaries $U_{p,q}$, acting on computational basis states $\ket{p}$ and $\ket{q}$, our next goal will be to represent this two-level unitary as a concrete quantum circuit consisting of various controlled operations. First, we must choose any qubit in the system, which will be indexed by an integer $j$ satisfying $d_j = 2$. Now define $C_c(U_j)$ to be the control operation applying 
\[U_2 = \begin{bmatrix}
a & b \\
c & d
\end{bmatrix}\]
to qubit $j$, so long as every other object in the quantum system is in state $q_i$. In our notation this means
\[c = \{\ket{r_n}\dots\ket{r_1}\ |\ r_i = q_i \text{\ whenever\ } i \neq j\}\]

Now $C_c(U_j)$ will also be a two-level unitary, acting on
\[p' = q_n \dots q_{j+1} 0 q_{j-1} \dots q_1,\]
\[q' = q_n \dots q_{j+1} 1 q_{j-1} \dots q_1.\]
Formally,
\[C_c(U_j) = I + (a-1)\ket{p'}\bra{p'} + b\ket{p'}\bra{q'}+ c\ket{q'}\bra{p'}+ (c-1)\ket{q'}\bra{q'}.\]

Consider our $U_{1,2}$ example. We could choose $j = 1$. Then $p'_2 = q'_2 = q_2 = 1$, so $p'$ and $q'$ would have binary expansion $10$ and $11$, the integers 2 and 3 respectively. This gives
\[C_c(U_j) = C_{r_2=1}(U_j) = \begin{bmatrix}
	1 & 0 & 0 & 0 \\
	0 & 1 & 0 & 0 \\
	0 & 0 & a & b \\
	0 & 0 & c & d
\end{bmatrix}.
\]

At this point we can fairly easily see how to implement $U_{1,2}$ as a concrete quantum circuit, so long as we can map $\ket{p}=\ket{1}$ to $\ket{p'} = \ket{2}$ and $\ket{q}=\ket{2}$ to $\ket{q'}=\ket{3}$. There are two permutation matrices that will do this:
\[P = \begin{bmatrix}
	1 & 0 & 0 & 0 \\
	0 & 0 & 0 & 1 \\
	0 & 1 & 0 & 0 \\
	0 & 0 & 1 & 0
\end{bmatrix}, \begin{bmatrix}
	0 & 0 & 0 & 1 \\
	1 & 0 & 0 & 0 \\
	0 & 1 & 0 & 0 \\
	0 & 0 & 1 & 0
\end{bmatrix}\]

We choose the latter of these, which happens to map any $\ket{r}$ to $\ket{r+1 \mod 4}$, and can be represented by the following circuit:

\begin{quantikz}
\lstick{$\ket{r_2}$} & \qw & \targ{} & \qw \rstick[wires=2]{$\ket{r+1 \mod 4}$} \\
\lstick{$\ket{r_1}$} & \gate{X_2} & \phase{0} \vqw{-1} & \qw  \\
\end{quantikz}

As a matrix expression this is $P = (I \otimes X_2)C_{r_1=0}(X_2 \otimes I)$. Since each term in this circuit is self-inverse, we can reverse the circuit to implement $P^{-1}$ as well. Then in order to apply $U_{p,q}$ to $\ket{r}$, we first apply $P$, then $C_c(U_j)$, then $P^{-1}$, which is the following similarity transformation:
\[\begin{bmatrix}
	1 & 0 & 0 & 0 \\
	0 & a & b & 0 \\
	0 & c & d & 0 \\
	0 & 0 & 0 & 1
\end{bmatrix} = \begin{bmatrix}
0 & 1 & 0 & 0 \\
0 & 0 & 1 & 0 \\
0 & 0 & 0 & 1 \\
1 & 0 & 0 & 0
\end{bmatrix}
\begin{bmatrix}
	1 & 0 & 0 & 0 \\
	0 & 1 & 0 & 0 \\
	0 & 0 & a & b \\
	0 & 0 & c & d
\end{bmatrix}
\begin{bmatrix}
0 & 0 & 0 & 1 \\
1 & 0 & 0 & 0 \\
0 & 1 & 0 & 0 \\
0 & 0 & 1 & 0
\end{bmatrix}.\]
As a quantum circuit this is:

\begin{quantikz}
\lstick[wires=2]{$\ket{\phi}$} & \qw & \targ{} & \phase{1} \vqw{1} & \targ{} & \qw & \qw \rstick[wires=2]{$U_{p,q}\ket{\phi}$} \\
 & \gate{X_2} & \phase{0} \vqw{-1} & \gate{U_2} & \phase{0} \vqw{-1} & \gate{X_2} & \qw
\end{quantikz}

To generalize this we must describe a process for generating and implementing $P$ in any qubit computer. The process given in \cite{textbook} is to implement the transposition $S_{p,p'}$ mapping $\ket{p}$ to $\ket{p'}$. For them this will do as a permutation $P$ since they assume $q'_j = q_j$ as opposed to our assumption that $q'_j = 1$. Consider as an extreme example where $n=3$, and $j=1$, where all three bits need to be inverted, i.e.\ $p_3 \neq q_3$, $p_2 \neq q_2$, $p_1 = q_1$:

\begin{quantikz}
	\lstick{$\ket{r_3}$} & \targ{} & \phase{q_3}\vqw{1} & \phase{q_3}\vqw{1} & \phase{q_3}\vqw{1} & \targ{} & \qw \\
	\lstick{$\ket{r_2}$} & \phase{p_2}\vqw{-1} & \targ{} & \phase{q_2}\vqw{1} & \targ{} & \phase{p_2}\vqw{-1} & \qw \\
	\lstick{$\ket{r_1}$} & \phase{p_1}\vqw{-1} & \phase{p_1}\vqw{-1} & \targ{} & \phase{p_1}\vqw{-1} & \phase{p_1}\vqw{-1} & \qw
\end{quantikz}

In the circuit for $PC_c(U_j)P^{-1}$ the latter half of the above sequence of operations would have no effect, since it would be transforming the action of $C_c(U_2)$ on basis vectors that it doesn't do anything to, so in fact the full circuit simplifies to a circuit that looks much like the above:

\begin{quantikz}
	\lstick{$\ket{r_3}$} & \targ{} & \phase{q_3}\vqw{1} & \phase{q_3}\vqw{1} & \phase{q_3}\vqw{1} & \phase{q_3}\vqw{1} & \phase{q_3}\vqw{1} & \targ{} & \qw \\
	\lstick{$\ket{r_2}$} & \phase{p_2}\vqw{-1} & \targ{} & \phase{q_2}\vqw{1} & \phase{q_2}\vqw{1} & \phase{q_2}\vqw{1} & \targ{} & \phase{p_2}\vqw{-1} & \qw \\
	\lstick{$\ket{r_1}$} & \phase{p_1}\vqw{-1} & \phase{p_1}\vqw{-1} & \targ{} & \gate{U_2} & \targ{} & \phase{p_1}\vqw{-1} & \phase{p_1}\vqw{-1} & \qw
\end{quantikz}

The first half of this is essentially the $P$ we originally wanted, mapping $\ket{p}$ to $\ket{p'}$ and $\ket{q}$ to $\ket{q'}$. This process is fine for showing universal computation in the abstract, but is hard to generalize to mixed contexts, and is generally very wasteful. Instead we shall do something much simpler, using only $C(X)$. The first step shall be to choose a smaller permutation $P_1$ so that $p_j$ maps to 0 and $q_j$ maps to 1. There are four cases to consider:
\begin{itemize}
	\item $p_j = 0$, $q_j = 1$ already,
	\item $p_j = 1$, $q_j = 0$
	\item $p_j = q_j = 0$
	\item $p_j = q_j = 1$
\end{itemize}
In the first case we can set $P_1 = I$, and do nothing, and in the second case we can set $P = X_j$, but in the last two cases we must pick some $k$ so that $p_k \neq q_k$. Then if we are in the third case we set $P = C_{r_k = q_k}(X_j)$, and in the fourth case $P = C_{r_k = p_k}(X_j)$. Now $P_1\ket{q} = \ket{q'}$, so all that remains is to map each remaining $p_i$ to $q_i$ without changing $\ket{q}$. This is simple to do with $C_{r_j=0}(X_i)$, repeated once for each $i\neq j$ with $p_i \neq q_i$. Applied to $\ket{p}$ this will change $p_j$ to $0$, setting the control value, so that the remaining bits can be set to $\ket{q_i}$, and of course applied to $\ket{q}$ this will change $q_j$ to $1$, making no other changes since the control bit has been set incorrectly.

Applied to our extremal three-qutrit example this will look like the following:

\begin{quantikz}
	\lstick{$\ket{r_3}$} & \qw & \qw & \targ{} & \phase{q_2}\vqw{1} & \targ{} & \qw & \qw & \qw \\
	\lstick{$\ket{r_2}$} & \phase{p_2}\vqw{1} & \targ{} & \qw & \phase{q_2}\vqw{1} & \qw & \targ{} & \phase{p_2}\vqw{1} & \qw \\
	\lstick{$\ket{r_1}$} & \targ{} & \phase{0}\vqw{-1} & \phase{0}\vqw{-2} & \gate{U_2} & \phase{0}\vqw{-2} & \phase{0}\vqw{-1} & \targ{} & \qw
\end{quantikz}

In this case our circuit is the same length, or even longer if more optimization had been applied, but the simplification from three Toffoli gates to three $C(X)$ gates is dramatic once we start representing Toffoli gates in terms of $C(X)$, and as the number of bits increases this difference will increase quadratically, so it is a significant improvement. Of course arguments about universal computation are not intended to be efficient anyway, and the real purpose of this approach is to generalize to mixed systems. The way that we do this is straight forward. We still choose $j$ to be some qubit, meaning $d_j = 2$ still. $P_1$ is chosen by the same process as before as well, but if $k$ needs to be chosen it can be an arbitrary control digit whether binary or otherwise. Now to change the remaining digits $p_i$ to $q_i$, we still use controlled operations, but can choose whether we use $C_{r_j=0}(S_{p_i,q_i})$ to change $p_i$, or $C_{r_j=0}(X_{d_i})^{q_i-p_i}$, depending on which basis of \ref{} we are aiming to use.

We have now written $U_{p,q}$ as a combination of control operations, and so next is to use the techniques described in \cite{cnot-decomposition} to decompose these into operations with only a single control object. Of course with the above technique most of our operations are already in this form, but this doesn't eliminate any potential cases, since we still have an arbitrary $C_c(U_j)$ in the middle of the circuit.

\section{Decomposing Control Operations}
We would like to decompose $C_c(U_j)$ into elements of $U(2)$ and operations of the form $C(X_{d_i})$ or $C(S_{p_i,p'_i})$. In a binary computer this is done in three steps. The first is to introduce/require an additional $n-3$ auxiliary qubits to the quantum computer that are not intended to be affected by the original unitary in $U(N)$ or the control operation in question, and to inductively reduce the operation $C_c(U_j)$ into a series of gates with two control qubits. We index the new qubits $n+1$ through $2n-2$, which begin and end this process in the computational basis state $\ket{0}$. The key to this process is the Toffoli gate, which [we recall? I've already been talking about them...] is simply a controlled $X$ operation with two control bits, $C_{r_k=q_k,r_l=q_l}(X_j)$.

Take $I$ to be the set of indices $i$ for which our control operation $C_c(U_j)$ has a constraint $r_i = q_i$, so that
\[c = \{\ket{r_1}\dots\ket{r_n}\ |\ r_i = q_i \text{\ whenever} i \in I\}\]
Then we shall perform induction on the size of $I$, implementing any control operation as $2\ord{I}-4$ Toffoli gates, sandwiching a final single- or double-controlled operation $C_{r_k=q_k, r_l=q_l}(U_j)$. Informally this looks like the following:

\begin{quantikz}
\lstick{$\ket{0}$}  & \qw & \qw \midstick[6,brackets=none]{=}& \targ{} & \phase{1}\vqw{4} & \targ{} & \qw \rstick{$\ket{0}$}\\
\lstick{$\ket{r_k}$}& \phase{q_k}\vqw{1} & \qw& \phase{q_k}\vqw{-1} & \qw & \phase{q_k}\vqw{-1} & \qw \rstick{$\ket{r_k}$} \\
\lstick{$\ket{r_l}$}& \phase{q_l}\vqw{2} & \qw& \phase{q_l}\vqw{-1} & \qw & \phase{q_l}\vqw{-1} & \qw \rstick{$\ket{r_l}$} \\
\lstick{$\vdots$}& & & & & & \rstick{$\vdots$} \\
                 & \ctrl{1} & \qw& \qw & \ctrl{1} & \qw & \qw  \\
\lstick{$\ket{r_j}$}& \gate{U_j} & \qw& \qw & \gate{U_j} & \qw & \qw
\end{quantikz}

In total this implements an operation with $\ord{I}$ control bits in terms of one with $\ord{I}-1$ control bits. Formally we have introduced an auxiliary qubit $r_i$, and reduced $I$ to
\[I' = (I \backslash \{k, l\})\cup \{i\}\]
Correspondingly, set $q_i = 1$ and reduce the condition set $c$ to
\[c' = \{\ket{r_1}\dots\ket{r_n}\ |\ r_i = q_i \text{\ whenever} i \in I'\}\]
This returns us to the form we started in, with an operation $C_{c'}(U_j)$, but with one less control qubit. Eventually we will reach a base case where $\ord{I} = 2$, in which case $C_c(U_j)$ is already a double-controlled gate sandwiched by $2\ord{i}-4 = 0$ Toffoli gates. This means that inductively we end up with the result that we wanted.

This decomposition into Toffoli gates acting on auxiliary qubits generalizes to mixed logic without any modification, so long as auxiliary qubits are available. Alternatively one can copy \cite{multi-valued-logic} which introduces auxiliary objects of dimension $d>2$, and uses singly-controlled permutations gates, though controlled increments will do as well. We will assume $d=3$ but it is easy to optimize the number of auxiliary objects when $d$ is larger. The transformation instead looks like this:

\begin{quantikz}
	\lstick{$\ket{0}$}  & \qw & \qw \midstick[6,brackets=none]{=}& \gate{S_{0,1}} & \gate{S_{1,2}} & \phase{2}\vqw{4} & \gate{S_{2,1}} & \gate{S_{1,0}} & \qw \rstick{$\ket{0}$}\\
	\lstick{$\ket{r_k}$}& \phase{q_k}\vqw{1} & \qw& \phase{q_k}\vqw{-1} & \qw & \qw & \qw & \phase{q_k}\vqw{-1} & \qw \rstick{$\ket{r_k}$} \\
	\lstick{$\ket{r_l}$}& \phase{q_l}\vqw{2} & \qw& \qw & \phase{q_l}\vqw{-2} & \qw & \phase{q_l}\vqw{-2} & \qw & \qw \rstick{$\ket{r_l}$} \\
	\lstick{$\vdots$}& & & & & & & & \rstick{$\vdots$} \\
	& \ctrl{1} & \qw& \qw & \qw & \ctrl{1} & \qw & \qw & \qw  \\
	\lstick{$\ket{r_j}$}& \gate{U_j} & \qw& \qw & \qw & \gate{U_j} & \qw & \qw & \qw
\end{quantikz}

$S_{0,1}$ in the above diagram can be replaced with $X_3$ and $S_{1,0}$ with $X^{-1}$ to achieve the same effect, depending on the desired basis. If we have arbitrarily many qutrits or higher available, then we can ignore the Toffoli gates altogether, and only use the $C(X)$ or $C(S_{p_i,p'_i})$ gates available to us to reduce $C_c(U_k)$ to a singly-controlled $C_{r_i=2}(U_j)$, but for generality we shall continue as if Toffoli gates need to be used as well. Given $V_j^2 = U_j$, \cite{cnot-decomposition} presents the following circuit for any $C_{r_k=q_k, r_l=q_l}(U_j)$:

\begin{quantikz}
	\lstick{$\ket{r_k}$}& \phase{q_k}\vqw{1} & \qw\midstick[3,brackets=none]{=}& \qw & \phase{q_k}\vqw{1} & \qw & \phase{q_k}\vqw{1} & \phase{q_k}\vqw{2} & \qw \rstick{$\ket{r_k}$} \\
	\lstick{$\ket{r_l}$}& \phase{q_l}\vqw{1} & \qw& \phase{q_l}\vqw{1} & \targ{} & \phase{q_l}\vqw{1} & \targ{} & \qw & \qw \rstick{$\ket{r_l}$} \\
	\lstick{$\ket{r_j}$}& \gate{U_j} & \qw& \gate{V_j} & \qw & \gate{V_j^{-1}} & \qw & \gate{V_j} & \qw
\end{quantikz}

This decomposes our many doubly-controlled gates into singly-controlled gates, but interestingly all of the Toffoli gates acting on auxiliary qubits will become $C(\sqrt{X_2})$\footnote{If one wants a square root of $X$, then $(H^{-1}\sqrt{Z}H)^2 = H^{-1}ZH = X$ is a simple example, with $\sqrt{Z}\ket{i} = \omega_4^i\ket{i}$.} rather than $C(X)$. So long as object $j$ is still a qubit, this proof will generalise to a mixed quantum computer as follows:

\begin{quantikz}
	\lstick{$\ket{r_k}$}& \phase{q_k}\vqw{1} & \qw\midstick[3,brackets=none]{=}& \qw & \phase{q_k}\vqw{1} & \qw & \phase{q_k}\vqw{1} & \phase{q_k}\vqw{2} & \qw \rstick{$\ket{r_k}$} \\
	\lstick{$\ket{r_l}$}& \phase{q_l}\vqw{1} & \qw& \phase{q_l}\vqw{1} & \gate{S_{q_l,q_l+1}} & \phase{q_l}\vqw{1} & \gate{S_{q_l+1,q_l}} & \qw & \qw \rstick{$\ket{r_l}$} \\
	\lstick{$\ket{r_j}$}& \gate{U_j} & \qw& \gate{V_j} & \qw & \gate{V_j^{-1}} & \qw & \gate{V_j} & \qw
\end{quantikz}

Once again $S_{q_l,q_l+1}$ and $S_{q_l+1,q_l}$ can be replaced with $X_{d_l}$ and $X_{d_l}^{-1}$ depending on the desired basis.

In fact \cite{cnot-decomposition} generalizes this implemention of doubly-controlled gates directly to arbitrary $C_c(U_j)$, so this approach could be directly generalized to a mixed context by replacing all $X_2$ operations with increment/decrement respectively, avoiding the discussion of auxiliary qubits or qutrits altogether, but requiring greater circuit depth in place of the shorter circuit length. This gives a total of three distinct ways of implementing $C_c(U_j)$ in terms of singly-controlled operations $C_{r_k=q_k}(U_j)$, acting on qubits. In all three cases we now need only demonstrate how to implement these in terms of single-qubit unitaries and $C_{r_k=q_k}(X_2)$, which in this case is the same as $C_{r_k=q_k}(S_{0,1})$. The qubit construction referenced in \cite{cnot-decomposition} works for this purpose without modification:

\begin{quantikz}
\lstick{$\ket{r_k}$} & \phase{q_k}\vqw{1} & \qw\midstick[2,brackets=none]{=}& \qw & \phase{q_k}\vqw{1} & \qw & \phase{q_k}\vqw{1} & \qw & \qw \rstick{$\ket{r_k}$}\\
\lstick{$\ket{r_j}$} & \gate{U_j} & \qw & \gate{A} & \targ{} & \gate{B} & \targ{} & \gate{C} & \qw\rstick{$C(U_j)\ket{r_j}$}
\end{quantikz}

where $A,B,C \in U(2)$ are chosen so that $ABC=I$ and $AXBXC = U_j$. This can be done in general, via the spherical geometry of $U(2)/U(1) \cong SO(3)$[define!], as shown in \cite{cnot-decomposition}. Combining all of the steps just described we have proven \ref{}, generalizing the universality result of \cite{cnot-decomposition} and \cite{textbook} to mixed quantum computers with at least one qubit, and with significantly fewer operations than \cite{textbook} seems to have used, even in the case of a computer with only qubits available. $\square$
\section{Analysing Small Permutations}
We have generated $U(N)$ with a finite set of two-object operations along with arbitrary qubit operations from $U(2)$. We can then apply the result of \cite{universal-qubit}, that any element of $U(2)$ can be approximated using the following operations $H_2$ and $T$:

\begin{align*}
	H_2 = \frac{1}{\sqrt{2}}\left[\begin{matrix}
	1 & 1 \\
	1 & -1
\end{matrix}\right]
&&&
T = \left[\begin{matrix}
	1 & 0 \\
	0 & \frac{1}{\sqrt{2}}\left(1+ i\right)
\end{matrix}\right]
\end{align*}

We will come back in \ref{} to talk briefly about how this is done, since it involves an aspect of algebraic number theory that will be useful to us. The result was then used to argue that by the decompositions described in \cite{cnot-decomposition}, the gate set $H_2$, $T$, and $C_2(X_2)$ is universal, a significant result since this gate set can also be implemented fault tolerantly. We can directly map this result to mixed quantum computers via \ref{}, to get two alternative finite gate sets, $\{H_2, T, C_{r_k=q_k}(X_j)\}$, and $\{H_2, T, C_{r_k=q_k}(S_{p_j,p'_j})\}$. We will now discuss ways that this operator set can be reduced further, with particular mind to the simplest case of quantum computers that only have qubits and qutrits. First we can fix $q_k=d_k-1$, and generate other control values via the circuit equivalence

\begin{quantikz}
	& \phase{q_k}\vqw{1} & \qw\midstick[2,brackets=none]{=}& \gate{X_k^{q'_k-q_k}} & \phase{q'_k}\vqw{1} & \gate{X_k^{q_k-q'_k}} & \qw\\
	& \gate{U_j} & \qw & \qw & \gate{U_j} & \qw
\end{quantikz}

This introduces each $X_d$ to our generator set, but removes each $C_{r_k=q_k}(\dots)$ apart from $q_k=d_k-1$. Further we already have $X_2 = H^{-1}T^4H$, so we only need $X_d$ for $d > 2$. Additionally, we can fix $p_j$ to 0 by a similar equivalence

\begin{quantikz}
	& \phase{q_k}\vqw{1} & \qw\midstick[2,brackets=none]{=}& \qw & \phase{q_k}\vqw{1} & \qw & \qw\\
	& \gate{S_{p_j,p'_j}} & \qw & \gate{X_k^{-p_j}} & \gate{S_{0,p'_j-p_j}} & \gate{X_k^{p_j}} & \qw
\end{quantikz}

where the difference $p'_j-p_j$ is evaluated mod $d_j$. Finally, we can optionally exchange $p_j$ and $p'_j$ to sure that the difference $p'_j - p_j$ is always at most $d_j/2$. For example $S_{2,1}$ acting on a qutrit is equivalent to $S_{1,2}$, which by the above can be reduced to $X_3S_{1,2}X_3^{-1}$. This means ultimately our generator sets are $\{H_2, T, X_{d_i}, C_{r_k=d_k-1}(X_k)\}$ and $\{H_2, T, X_{d_i}, C_{r_k=d_k-1}(S_{0,p'_k})\}$, with $d_i > 2$, $p'_k \leq d_k/2$. In the qubit-qutrit case these will both have seven elements, the increment basis
\[\{H_2, T, X_3, C_2(X_2), C_2(X_3), C_3(X_2), C_3(X_3)\},\]
and the transposition basis
\[\{H_2, T, X_3, C_2(X_2), C_2(S_{0, 1}), C_3(X_2), C_3(S_{0, 1})\}.\]

We will now show how these two bases are directly equivalent to each other, though generating the increment basis from the transposition basis will be much less expensive than the other way around. Five of the operations are common between the two bases, so the only difference is that one contains $C_2(X_3)$ and $C_3(X_3)$, where the other contains $C_2(S_{0,1})$ and $C_3(S_{0,1})$. In general $C_d(A)C_d(B) = C_d(AB)$, so setting $A = S_{0,1}$ and $B = S_{1,2}$ we get $C_d(X_3)$. As an explicit quantum circuit in the transposition basis:

\begin{quantikz}
	& \phase{d_k-1}\vqw{1} & \qw\midstick[2,brackets=none]{=}& \qw & \phase{d_k-1}\vqw{1} & \qw & \phase{d_k-1}\vqw{1} & \qw\\
	& \targ{} & \qw & \gate{X_3^{-1}} & \gate{S_{0,1}} & \gate{X_k} & \gate{S_{0,1}} & \qw
\end{quantikz}

Since $d_k$ is arbitrary in this construction, we have implemented both of the controlled increment operations $C_2(X_3)$ and $C_3(X_3)$. The reverse is more contrived; we will later find by brute force that $X_2$, $X_3$, $C_2(X_3)$, and $C_3(X_2)$ generate all permutations on an $N=6$ composite system, and in particular $C_2(S_{0,1})$ becomes the following:

\begin{quantikz}
	& \phase{1}\vqw{1} & \qw\midstick[2,brackets=none]{=}& \targ{} & \phase{1}\vqw{1} & \targ{} & \phase{1}\vqw{1} & \phase{1}\vqw{1} & \targ{} & \qw\\
& \gate{S_{0,1}} & \qw & \phase{2}\vqw{-1} & \targ{} & \phase{2}\vqw{-1} & \targ{} & \targ{} & \phase{2}\vqw{-1} & \qw
\end{quantikz}

We cannot directly generate $C_3(S_{0,1})$ using $C_3(X_3)$ due to parity, since $C_3(S_{0,1})$ is an odd number of transpositions and $C_3(X_3)$ is an even number of transpositions. One thing we can do is introduce an auxiliary qubit, and use $C_2(S_{0,1})$ as above:

\begin{quantikz}
	\lstick{$\ket{0}$} & \qw & \qw\midstick[3,brackets=none]{=}& \targ{} & \phase{1}\vqw{2} & \targ{} & \qw \rstick{$\ket{0}$} \\
	& \phase{2}\vqw{1} & \qw& \phase{2}\vqw{-1} & \qw & \phase{2}\vqw{-1} & \qw\\
	& \gate{S_{0,1}} & \qw & \qw & \gate{S_{0,1}} & \qw & \qw
\end{quantikz}

Another option we have is to introduce the Clifford SWAP and SUM operations, and appeal to the equivalences shown in \cite{arithmetics}. $S_{01,10}$ in a two qutrit system was shown to be implementable by five $C_3(X_3)$ gates and a SWAP:

\begin{quantikz}
& \gate[2]{S_{01,10}} & \qw\midstick[2,brackets=none]{=} & \phase{2}\vqw{1} & \targ{} & \phase{2}\vqw{1} & \targ{} & \phase{2}\vqw{1} & \gate[2]{\text{SWAP}} \\
& \qw & \qw & \targ{} & \phase{2}\vqw{-1} & \targ{} & \phase{2}\vqw{-1} & \targ{} & \\
\end{quantikz}

Then $C_{r_2=0}(S_{0,1})$ is identical to $S_{00,01}$, which is a reflection in $\mathbb{C}^9$ just like $S_{01,10}$, and in fact \cite{arithmetics} transform these operations to each other using Clifford operations $X_3$ and SUM:

\begin{quantikz}
	& \phase{0}\vqw{1} & \qw\midstick[2,brackets=none]{=} & \octrl{1} & \gate{X} & \gate[2]{S_{01,10}} & \gate{X^{-1}} & \octrl{1} & \octrl{1} & \qw\\
	& \gate{S_{0,1}} & \qw & \targ{} & \gate{X} & \qw & \gate{X^{-1}} & \targ{} & \targ{} & \qw\\
\end{quantikz}

These equivalences are theoretically interesting, but at this algebraic level, the transposition basis appears to be significantly more efficient than the increment basis. Despite this, it is hard to speculate on what a good basis would be without properly considering the physical constraints and trade-offs of a specific quantum computer. For example, it would not be surprising if $C_2(X_3)$ were more favourable to implement fault tolerantly than $C_2(S_{0, 1})$, due to its resemblance to the SUM operation between two qutrits, but it is also possible that neither end up being implemented exactly, and that a different basis is used which approximately implements both of the bases we have presented.

Another decomposition that we have not discussed until now is the implementation of $X_3$ in terms of $C_2(X_3)$ and $X_2$:

\begin{quantikz}
	\lstick{$\ket{0}$} & \qw & \qw\midstick[2,brackets=none]{=}& \gate{X_2} & \phase{1}\vqw{1} & \gate{X_2} & \phase{1}\vqw{1} & \qw\\
	& \gate{X_3} & \qw& \qw & \gate{X_3} & \qw & \gate{X_3} & \qw
\end{quantikz}

We do not use the abbreviated $\oplus$ notation in this case to make it clear what is happening in this circuit -- we are applying $X_3$ conditionally, once for every possible condition, which has the same effect as unconditionally applying $X_3$. In theory this reduces both of our generator sets to only need six gates, but in practice this is absurd, since $X_3$ is a Pauli operation which will be one of the first and simplest operations to implement on a fault tolerant quantum computer, and we have implemented it in terms of a controlled operation with no known fault tolerant implementation.

In any case we have a set of seven gates that appear useful to implement and work with. All of the examples we have seen of universal computation for non-mixed systems of qubits or qutrits have involved some number of Clifford gates, and a single non-Clifford gate, but in order to generalize the standard binary model of quantum computation we find we require four control gates, only one of which is Clifford, together with the non-Clifford $T$, for a total of four non-Clifford gates. [I should probably show that $C_3(X_3)$ is not Clifford..........] An interesting avenue of future research would be the question of what the minimum number of non-Clifford gates is for achieving universal computation in mixed systems like this.

%% chap4.tex (Chapter 4 of the thesis)

\chapter[CONCLUSION]{Conclusion} \label{concl}

\section{New Section}
This better be good if you don't want your PhD reduced to a Masters!

\appendix % switches to appendix mode. Do this instead of \include{app0}
% app1.tex (will be Appendix A)

\chapter[AXIOMATA]{Axiomata}

\section{Group Quotients, Normal Subgroups, Normalisers}
[perhaps an example of a group and its normalizer, group isomorphisms, and semidirect products. excessive definition stuff can go in axiomata appendix]

Given an equivalence relation $\sim$ on a set $S$ it is often useful to consider equivalence classes, the subsets $[x] = \{y\ |\ x \sim y\}$, since these will be equal exactly when their representatives are equivalent, i.e. $[x] = [y] \iff x \sim y$. The set of all such equivalence classes is called the set quotient, and is written $S/\sim$. It provides a concrete object with the same structural properties that would come from `identifying' $x$ with $y$ whenever $x \sim y$. If $S$ is actually a group $G$ then its group operation sometimes induces an operation in the set quotient $[x][y] = [xy]$, but if this is well defined then we immediately find $[e]$ is a subgroup of $S$:
\[e \sim x \sim y \implies [e] = [e][e] = [x][y] = [xy] \implies e \sim xy\]
\[e \sim x \implies [x^{-1}] = [e][x^{-1}] = [x][x^{-1}] = [xx^{-1}] = [e] \implies e \sim x^{-1}\]
Further if $x \in [e]$ and $z \in S$ then $zxz^{-1} \in [e]$ as well:
\[e \sim x \implies [z][x][z^{-1}] = [z][e][z^{-1}] = [e] \implies e \sim zxz^{-1}\]

In general if a subgroup $H$ of $G$ satisfies this condition, that for any $h \in H$ and $g \in G$, $ghg^{-1} \in H$, then $H$ is said to be a normal subgroup of $G$. It turns out that not only is the equivalence class $[e]$ a normal subgroup, but whenever $H$ is a normal subgroup of $G$, the equivalence relation $x \sim y \iff xy^{-1} \in H$ gives a well defined group operation on the set quotient $G/\sim$. We call this group the group quotient $G/H$. When available, this is a powerful tool for understanding the structure of groups, since the group quotient $G/H$ may have convenient algebraic properties emerging from its corresponding equivalence relation.

As an example of a normal subgroup, take $G \subset GL(n, \mathbb{R})$ to be any group formed by matrix multiplication, and $H$ to be the set of scalars in $G$, that is the set $\{\lambda I\ |\ \lambda \in \mathbb{C}\} \cap G$. Since scalars are commutative, it is straight-forward that $g\lambda I g^{-1} = \lambda gg^{-1} = \lambda I \in H$.

The subgroup relation and the normal subgroup relation can be thought of as a partial order, and are often written as $\leq$ and $\trianglelefteq$ respectively, since both are transitive, reflexive, and anti-symmetric. While these relations are each transitive, we must be careful when mixing them; if $H$ is a normal subgroup of $N$, and $N$ is a subgroup of $G$, $H$ is not necessarily a normal subgroup of $G$, which is surprising when stated more succinctly as ``$H$ is normal in $N$ but is not normal in $G$''.

With this subtlety in mind we can find exactly such a group $N$, given any subgroup $H$ of $G$. We call this the normaliser of $H$ with respect to $G$, defined as the set $N_G(H) = \{g\ |\ g \in G,\ gHg^{-1} \subseteq H\}$. This is the maximal such $N$, any other $N'$ with $H$ normal will sit inside $N_G(H)$. Clearly the group quotient $N_G(H)/H$ will exist for any subgroup $H$ of $G$, which will be useless in the case that $N_G(H) = H$, but otherwise can be an interesting group, and can even inform interesting structure about the normaliser itself, making normalisers a useful and novel tool for exploratory algebraic work.

% app2.tex (will be Appendix B)

\section{Appendix B, Porting ``Improved Ternary Arithmetics''}
[this was going to be right after the universality discussion, but I think now I should combine it with the appendix I'll be writing on why the polynomial analysis doesn't work, since they're both related to the same improved ternary arithmetics paper. I might explain a little more in the background chapter then, or at least state that it does work quite easily.]
We have discussed how the algorithms for arithmetic described in \cite{arithmetics} could naturally be ported to a mixed logic system, by using qubits for binary carry bits, or by using 

\begin{quantikz}
	\lstick{$c_i$} & \qw & \qw & \phase{1} \vqw{2} & \gate{S_{0,1}} & \qw \rstick{$c_{i+1}$} \\
	\lstick{$a_i$} & \gate[wires=2]{S_{00,22}} & \octrl{1} & \qw & \qw & \qw \\
	\lstick{$b_i$} & \qw & \targ{} & \gate{X_3^{-1}} & \phase{0} \vqw{-2} & \qw \\
\end{quantikz}

[don't know how to do $\oplus^\dagger$ so ugly box for now.]

have found a useful set of basic gates for mechanized compilation
\\let's look at specific algorithms that have been designed rather than compiled
\\\cite{arithmetics} implements ternary addition as a permutation on computational basis states
\\the ripple carry is designed so that the carry digit is always binary, which could be implemented in a mixed context
\\(possibly draw up a circuit implementing the carry operations)
\\the look ahead carry has ternary carry registers, but could be used for binary arithmetic, again giving a mixed algorithm
\\(possibly draw up circuit for this as well)
\\either of these would have less entropy than the pure ternary case, but many practical tradeoffs may render pure ternary to be superior, such as connectivity requirements, performance costs of mixed logic in general
\\we have access to any method, lot of flexibility! add bits or trits with bit carry or trit carry
\\(possibly mention mixed-string representations, here or in final discussion)
% app2.tex (will be Appendix C)

\section{Appendix C, Source Code for Group Search}
%\bibliographystyle{style} \bibliography{bibFIle} 
\include{biby}
%\include{index}

\end{document}
