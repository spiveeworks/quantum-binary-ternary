% prelude.tex (specification of which features in `mathphdthesis.sty' you
% are using, your personal information, and your title & abstract)

% Specify features of `mathphdthesis.sty' you want to use:
\titlepgtrue % main title page (required)
\signaturepagetrue % page for declaration of originality (required)
\copyrighttrue % copyright page (required)
\abswithesistrue % abstract to be bound with thesis (optional)
\acktrue % acknowledgments page (optional)
\tablecontentstrue % table of contents page (required)
\tablespagefalse % table of contents page for tables (required only if you have tables)
\figurespagefalse % table of contents page for figures (required only if you have figures)

\title{GENERALISING FOUNDATIONS OF QUANTUM COMPUTATION TO MIXED BINARY-TERNARY SYSTEMS} % use all capital letters
\author{Jarvis Carroll} % use mixed upper & lower case
\prevdegrees{B.Sc. (Tas)} % Used to specify your previous degrees...use mixed upper & lower case
\advisor{Doctor Michael Cromer, and Doctor Jeremy Sumner.} % example: Professor Lawrence K. Forbes 
\dept{Mathematics} % your academic department
\submitdate{November 2020} % month & year of your thesis submission

\newcommand{\abstextwithesis}
{
Quantum computation is the study of how quantum mechanics can be utilised to reliably compute or simulate results that are significantly harder to compute when done by classical means. Existing quantum computation research focuses overwhelmingly on binary quantum logic, the logic of the `qubit', due to this being both the simplest and the most familiar number system for performing computation. Quantum computation does not have to operate with binary quantum data, however, and calculation is theoretically possible with ternary `qutrits', or higher, or even with quantum systems that utilise a mixture of binary and ternary, or other quantum logic systems simultaneously. These systems might store data more efficiently, might scale more rapidly and might present richer or more novel techniques for calculation, or might prove detrimental; we need to develop a theory of non-binary quantum logic in order to resolve these possibilities.

In this thesis we present a detailed description of what a quantum algorithm is and the basic operations available to a qubit computer, before discussing existing literature surrounding proposed devices and algorithms for computation on ternary or `qutrit' computers. We then convert these existing ternary algorithms into algorithms that act on mixed quantum computers, with access to both qubits and qutrits. After this we describe a basic set of gates already known to be capable of performing generic or universal computation on binary quantum computers, and generalise these to mixed quantum computers as well. After this we present the results of a basic algebraic search implemented in the C programming language, for finding finite and infinite groups of quantum operations, and finally discuss the many novel implications of what was found and the most obvious questions for future research prompted by these findings.
}

\newcommand{\acknowledgement}
{
I would of course like to thank my supervisors Michael Cromer and Jeremy Sumner for their support in this project, who were very generous in taking this project on, especially Michael, who helped me learn everything I know about quantum computation, despite only being a guest at the university.

Further, I would like to thank the friends and peers with whom I have shared an office and classes this year, primarily Mae, Luke, Cassady, Georgia, and Larissa, for sharing in the journey we have just been through, of writing a thesis during a pandemic, making the year much less alienating than it would have been otherwise.

Finally, I would like to thank all of my lecturers from this year and past years for encouraging me and bringing me to this point in my education. This thesis is ultimately the result of five years of interaction with this university and its wonderful staff.
}

% Take care of things in `mathphdthesis.sty' or 'MathPhysHonoursThesis.sty'
% behind the scenes.
% Basically just does a check of all the fields that have been activated
% above and fills out the appropriate pages and adds them to the thesis.
\beforepreface
\afterpreface
